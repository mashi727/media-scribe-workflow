% 動画などの前処理アプリについて
% LuaLaTeX用ドキュメント
\documentclass[a4paper,10pt,twocolumn]{ltjsarticle}

% LuaLaTeX用フォント設定パッケージ
\usepackage{luatexja-fontspec}
\usepackage{amsmath,amssymb}
\usepackage{unicode-math}

% ====================
% 欧文フォント設定 (Libertinus)
% ====================
\setmainfont{Libertinus Serif}[
    BoldFont = {Libertinus Serif Bold},
    ItalicFont = {Libertinus Serif Italic},
    BoldItalicFont = {Libertinus Serif Bold Italic}
]
\setsansfont{Libertinus Sans}[
    BoldFont = {Libertinus Sans Bold},
    ItalicFont = {Libertinus Sans Italic}
]
\setmonofont{Libertinus Mono}

% ====================
% 日本語フォント設定 (原ノ味フォント)
% ====================
\setmainjfont{HaranoAjiMincho-Regular}[
    BoldFont = {HaranoAjiGothic-Medium},
    ItalicFont = {HaranoAjiMincho-Regular},
    BoldItalicFont = {HaranoAjiGothic-Bold}
]
\setsansjfont{HaranoAjiGothic-Regular}[
    BoldFont = {HaranoAjiGothic-Bold}
]
\setmonojfont{HaranoAjiGothic-Regular}

% ====================
% 数式フォント設定 (Libertinus Math)
% ====================
\setmathfont{Libertinus Math}

% ファイル生成日時(JST)
\newcommand{\generatedDate}{2026-01-07}
\newcommand{\generatedTime}{--:--}

% ヘッダー・フッター設定
\usepackage{fancyhdr}
\usepackage{lastpage}
\pagestyle{fancy}
\fancyhf{}
\fancyhead[R]{\small \generatedDate\ \generatedTime\ JST (\thepage/\pageref{LastPage})}
\renewcommand{\headrulewidth}{0.4pt}

% その他のパッケージ
\usepackage[margin=20mm]{geometry}
\usepackage{hyperref}
\usepackage{booktabs}
\usepackage{array}
\usepackage{enumitem}

\usepackage{listings}
\usepackage{DejaVuSansMono}

\lstset{
  basicstyle=\footnotesize\ttfamily,
  columns=fixed,
  basewidth=0.5em,
  keepspaces=true
}



% ハイパーリンクの色設定
\hypersetup{
    colorlinks=true,
    linkcolor=blue,
    urlcolor=blue,
    citecolor=blue
}

\title{動画などの前処理アプリ(\texttt{VCE})について}
\author{massy\\{}mashi.zzz@gmail.com}
\date{\today}

\begin{document}
\maketitle
\tableofcontents

\thispagestyle{empty}
\newpage

\abstract{
  本アプリは、video-chapter-editorというもので、
  PCってどういうものかと申しますと、入力に対して処理\footnote{計算}を行い、結果を出力するものです。ここにユーザーとの関わりを含めて記述を行いますと、ユーザーが意図したコトを、PCが処理可能なモノにあつらえ、ユーザーが意図した処理を行い、ユーザーが意図した結果をモノとして出力する。ということになるでしょう。

  本稿では、このような基本的な事項からユーザーがギャップを感じることなく、
}


\section{前処理とは}

前処理とは、端末の中にある\textbf{動画}や\textbf{音声ファイル}を自分が扱いやすいようにする処理のことである。

\section{本アプリの概要}

\subsection{どのようなアプリか}

本アプリは、メディアファイルに対して以下の処理を行うツールである:

\begin{itemize}
  \item メディアファイルを\textbf{標準化}(再利用性と相互運用性の確保)する
  \item メディアファイルを編集し、有意義な単位で動画の切り出しを行う
\end{itemize}

\subsection{今後の展望}

最終的には動画の\textbf{文字起こし}を行い、さらなる標準化を実現する。

\section{本アプリの設計上の前提}

\subsection{想定するメディアファイル}

本アプリが処理対象とするメディアファイルは以下の2種類である:

\begin{enumerate}
  \item 必要な部分だけ抽出された(もしくは複数に分割された)動画・音声ファイル
  \item 未編集の長時間動画もしくは音声ファイル
\end{enumerate}

\subsection{編集の定義}

本アプリにおける「編集」とは、メディアファイルのうち:

\begin{itemize}
  \item \textbf{不要な部分}にマーキングを付与する
  \item \textbf{必要な部分}にチャプター名を付与する
\end{itemize}

という処理を指す。

\subsection{想定しているファイル処理}

\begin{itemize}
  \item 複数の画角のカットの同期
  \item 音声のノーマライズ処理
\end{itemize}

\section{本アプリの最終的なアウトプット}

\subsection{出力形式}

本アプリの最終的なアウトプットは、\textbf{チャプター対応付きの動画ファイル}である。

\subsection{互換性}

他のアプリでもチャプター再生が可能である:

\begin{itemize}
  \item VLC media player
  \item mplayer
  \item iPhoneのMusic.app
  \item その他チャプター対応プレイヤー
\end{itemize}

\section{標準化の具体例}

\subsection{Video Chapter Editorによる編集}

本アプリでは、Video Chapter Editor\footnote{本プロジェクトで開発したチャプター編集アプリケーション}を使用して、動画にチャプター情報を付与する。

\subsubsection{主な機能}

\begin{itemize}
  \item 動画ファイルの読み込み
  \item チャプターの追加・編集・削除
  \item タイムスタンプの設定
  \item チャプター情報の埋め込み
\end{itemize}

\subsection{YouTubeへのアップロード}

YouTubeにアップロードすることにより、チャプター再生が可能となる。

\subsubsection{チャプター再生の利点}

\begin{itemize}
  \item 視聴者が目的のセクションに直接ジャンプ可能
  \item コメント欄にタイムスタンプを記載することで検索性向上
  \item 動画の構造が視覚的に把握可能
\end{itemize}

\subsection{リハーサル記録への応用例}

創価大学新世紀管弦楽団定期演奏会リハーサル記録を例として、以下のようなチャプター構成が可能である:

\begin{table}[h]
\centering
\small
\begin{tabular}{@{}ll@{}}
\toprule
タイムスタンプ & 内容 \\
\midrule
00:00:12 & リハーサル概要 \\
00:00:46 & 演奏開始前の準備 \\
00:01:41 & 音量バランスの調整 \\
00:01:55 & 呼吸法の指導 \\
00:02:35 & 楽譜を見ずに指揮を見る \\
00:03:24 & 音の方向性・落とし方 \\
00:03:53 & テンション・緊張感の維持 \\
00:04:32 & 内的聴取の重要性 \\
00:05:39 & テンポの取り方 \\
00:05:41 & 暗い音色の表現 \\
\bottomrule
\end{tabular}
\end{table}

\section{まとめ:標準化の意義}

\textbf{標準化}(再利用性と相互運用性の確保)により、以下の効果が期待できる:

\begin{enumerate}
  \item \textbf{再利用性の向上}:一度作成したコンテンツを様々な場面で活用可能
  \item \textbf{相互運用性の確保}:異なるプラットフォーム・アプリケーション間での互換性
  \item \textbf{検索性の向上}:チャプター情報による構造化で目的の箇所への素早いアクセス
  \item \textbf{アーカイブ価値の向上}:メタデータ付与による長期保存と再発見の容易化
\end{enumerate}

\vspace{1em}
\noindent
\textit{標準化(再利用性×相互運用性の確保)って、なかなかステキでしょ!}

\end{document}
