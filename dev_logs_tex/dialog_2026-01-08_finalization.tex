% LuaLaTeX document
\documentclass[10pt,a4paper,twocolumn]{ltjsarticle}

% ファイル生成日時(JST)
\newcommand{\generatedDate}{2026-01-08}
\newcommand{\generatedTime}{23:50}

% LuaLaTeX用フォント設定パッケージ
\usepackage{luatexja-fontspec}
\usepackage{amsmath,amssymb}
\usepackage{unicode-math}

% ====================
% 欧文フォント設定 (Libertinus)
% ====================
\setmainfont{Libertinus Serif}[
    BoldFont = {Libertinus Serif Bold},
    ItalicFont = {Libertinus Serif Italic},
    BoldItalicFont = {Libertinus Serif Bold Italic}
]
\setsansfont{Libertinus Sans}[
    BoldFont = {Libertinus Sans Bold},
    ItalicFont = {Libertinus Sans Italic}
]
\setmonofont{DejaVu Sans Mono}[Scale=0.9]

% ====================
% 日本語フォント設定 (原ノ味フォント)
% ====================
\setmainjfont{HaranoAjiMincho-Regular}[
    BoldFont = {HaranoAjiGothic-Medium},
    ItalicFont = {HaranoAjiMincho-Regular},
    BoldItalicFont = {HaranoAjiGothic-Bold}
]
\setsansjfont{HaranoAjiGothic-Regular}[
    BoldFont = {HaranoAjiGothic-Bold}
]
\setmonojfont{HaranoAjiGothic-Regular}

% ====================
% 数式フォント設定 (Libertinus Math)
% ====================
\setmathfont{Libertinus Math}

% ヘッダー・フッター設定
\usepackage{fancyhdr}
\usepackage{lastpage}
\pagestyle{fancy}
\fancyhf{}
\fancyhead[R]{\small \generatedDate\ \generatedTime\ JST (\thepage/\pageref{LastPage})}
\renewcommand{\headrulewidth}{0.4pt}

% 1ページ目のスタイル
\fancypagestyle{firstpage}{
    \fancyhf{}
    \renewcommand{\headrulewidth}{0pt}
}

% 追加パッケージ
\usepackage{booktabs}
\usepackage{array}
\usepackage{tabularx}
\usepackage{ascmac}
\usepackage{listings}
\usepackage{xcolor}
\usepackage{hyperref}

% コードスタイル設定
\lstset{
    basicstyle=\ttfamily\scriptsize,
    breaklines=true,
    frame=single,
    backgroundcolor=\color{gray!10},
    columns=fullflexible,
    keepspaces=true,
    showstringspaces=false
}

% Y列タイプ定義
\newcolumntype{Y}{>{\raggedright\arraybackslash}X}

\title{Video Chapter Editor 設計決定事項の確定}
\author{ましDialogue}
\date{}

\begin{document}
\maketitle
\thispagestyle{firstpage}

\section*{概要}

本文書は、Video Chapter Editor v2の設計決定事項を確定するための対話を記録したものである。午前中の設計議論を踏まえ、午後に以下の項目について最終決定を行った。

\begin{itemize}
    \item 同名規則(ソースファイルとチャプターファイルの対応)
    \item ドロップ操作の仕様
    \item 時間管理方式
    \item プロジェクトファイル形式
    \item チャプターファイル形式
    \item エラー処理方針
    \item バッチエンコード仕様
    \item Undo/履歴方針
\end{itemize}

\section{午後の設計検討}

\begin{itembox}[l]{問い}
お昼以降で考えたことをお伝えしますね。

プロジェクトファイルは.vce.json形式で保存。バッチエンコードはアプリのダイアログで実行する。CLIは実装しない。OSへの通知もなし。複数ソースの複数チャプターファイルの操作はやらない。チャプターリストはアクティブなファイルのみ。

nice to haveは、ドロップ位置によってファイルを複数ロードできる機能。あとはドロップ操作を完成。
\end{itembox}

午前中の議論を踏まえて、具体的な仕様が提案された。プロジェクトファイルの形式、バッチエンコードの実装方針、複数ソース時のチャプター操作について、方向性が示された。

\section{同名規則の提案}

\begin{itembox}[l]{問い}
制約条件として、チャプターファイルと動画ファイルは同名が基本とするとするとずいぶん設計の見通しが良くなりませんか。
\end{itembox}

同名規則とは、ソースファイルとチャプターファイルが同一のベース名を持つという制約である。

\begin{lstlisting}
rehearsal_2026-01-08.mp4  <- ソース
rehearsal_2026-01-08.txt  <- チャプター
\end{lstlisting}

この制約により、以下が大幅に簡素化される:

\begin{itemize}
    \item \textbf{プロジェクトファイル}:ソースリストのみ保持すれば、チャプターファイルは自動発見可能
    \item \textbf{ドロップ操作}:ファイル種別(動画/音声/チャプター)の判定が明確
    \item \textbf{状態管理}:紐付け情報を別途管理する必要がない
\end{itemize}

\subsection{プロジェクトファイルの簡素化}

\begin{lstlisting}[language=json]
{
  "version": "1.0",
  "sources": [
    "rehearsal_2026-01-06.mp4",
    "rehearsal_2026-01-07.mp3"
  ],
  "encode_settings": {
    "encoder": "hevc_videotoolbox",
    "quality": 65
  }
}
\end{lstlisting}

チャプター情報を保持せず、同名規則に基づいて自動発見する。これにより、プロジェクトファイルと個別チャプターファイルの二重管理が不要になる。

\section{ドロップ操作の仕様}

\begin{itembox}[l]{問い}
ですね。で、ファイル名.{txt or mp4,mp3}ドロップ操作を動画上であれば再生中の動画の位置に、チャプターリスト上であればファイルの切れ目のどこにでもドロップ可能。というのが直感的で良さそうです。動画編集中は音声ドロップは受け付けない。逆もまた然り。なのでチャプターファイルでは、元の紐づけられたファイル名をメタデータとして持つ必要がある。とかいう仕様でいかがでしょう。
\end{itembox}

ドロップ操作の仕様が明確化された。

\subsection{ドロップ位置による動作}

\vspace{0.5\baselineskip}
\noindent{\footnotesize
\begin{tabularx}{\linewidth}{@{}lYY@{}}
\toprule
ドロップ先 & 動作 & 備考 \\
\midrule
動画ウィジェット & 再生位置に挿入 & 現在位置の直後 \\
チャプターリスト & ファイル境界に挿入 & 行間にドロップ \\
\bottomrule
\end{tabularx}
}
\vspace{0.5\baselineskip}

\subsection{型制約}

動画編集中(動画ソースがロード済み)の場合は音声ファイルのドロップを受け付けない。逆に音声編集中は動画ファイルのドロップを受け付けない。これにより、動画と音声の混在を防ぎ、整合性を保つ。

\subsection{チャプターファイルのメタデータ}

ドロップされたチャプターファイルが、どのソースに紐付くかを判定するために、チャプターファイル内にメタデータを持つ必要がある。

\begin{lstlisting}
# source: rehearsal_2026-01-08.mp4
00:00:00 Opening
00:05:23 Main Theme
00:12:45 Ending
\end{lstlisting}

\texttt{\# source:}行により、同名規則が破られた場合でも紐付けを特定できる。

\section{時間管理方式}

\begin{itembox}[l]{問い}
ですね。リストに表示される絶対時間は、ファイルの順番が決まった後にファイルごとに持っている長さおよびチャプタ時間の時間から算出するというので良いのではないかと思いますがいかがでしょう。他に決めるべきことはありますか?
\end{itembox}

時間管理方式として、相対時間方式が確定した。

\subsection{データ構造}

\begin{lstlisting}[language=Python]
@dataclass
class ChapterInfo:
    local_time_ms: int    # ソース内の時間
    source_index: int     # ソースの順番
    title: str            # チャプタータイトル
\end{lstlisting}

\subsection{絶対時間の算出}

絶対時間はファイル順序とファイル長から動的に算出する。

\begin{lstlisting}
# 例:ソースA(10分) + ソースB(15分)
# ソースBの2:00のチャプター
# -> 絶対時間 = 10:00 + 2:00 = 12:00
\end{lstlisting}

この方式により、ソースの並べ替えを行ってもチャプターはソースに追従する。

\section{追加検討事項}

\subsection{UI表現}

\begin{itembox}[l]{問い}
1は、現状の区切り線でOKです。
\end{itembox}

複数ソースのチャプターリスト表示において、ソース間の区切りは現状の区切り線を維持する。

\subsection{Undo方針}

\begin{itembox}[l]{問い}
2、Undoは行いません。消えるわけではないのでやり直せば良いです。
\end{itembox}

ドロップ操作に対するUndoは実装しない。チャプターファイルは別途保存されており、操作を誤った場合は再度読み込めばよい。

\subsection{進捗表示}

\begin{itembox}[l]{問い}
4は、進捗は、個別のプログレスで良いですけど、n of mやn/mなどのファイル数の進捗は必要です。完了通知はOS依存性が高いので不要です。
\end{itembox}

バッチエンコードの進捗表示仕様:

\begin{itemize}
    \item 個別ファイルの進捗バー
    \item ファイル数の進捗(例:\texttt{2/5})
    \item OS通知は実装しない
\end{itemize}

\section{エラー処理方針}

\begin{itembox}[l]{問い}
フェイルセーフ的に、このアプリではファイル名を変えて保存しているので、あまり想定されない状況ですね。不整合はエラーの元なのでCとしたいところですが、Aの方が合理性が高いと思いますので、確認して間違ってればチャプターを削除すれば良いだけで、さほどコストはかかりませんしね。
\end{itembox}

チャプター時間がソース長を超える場合の処理方針について議論された。

\subsection{選択肢}

\begin{itemize}
    \item A) 警告して無視(チャプターは残すが、超過を警告)
    \item B) 自動修正(ソース末尾に丸める)
    \item C) エラーとして拒否(ロード失敗)
\end{itemize}

\subsection{決定:オプションA}

「警告して無視」を採用。理由:

\begin{itemize}
    \item 不整合は稀なケース(ファイル名変更で保存するため)
    \item ユーザーが確認後、不要なチャプターを削除すれば済む
    \item エラー拒否は過剰反応
\end{itemize}

\section{チャプターファイル形式の確定}

\begin{itembox}[l]{問い}
あと、エンコード終了時に出力した動画のチャプターファイルは新たな形式で自動保存しましょう。あと、以降のバージョンのVCEでは、チャプターファイルは新しい形式でも、古い形式でも読めるようにしましょう。ほっとしてもそうなると思いますけど、明示するということで。
\end{itembox}

チャプターファイル形式について最終決定がなされた。

\subsection{新形式}

\begin{lstlisting}
# source: rehearsal_2026-01-08.mp4
00:00:00 Opening
00:05:23 Main Theme
00:12:45 Ending
\end{lstlisting}

\texttt{\# source:}行でソースファイルを明示する。

\subsection{後方互換性}

旧形式(YouTube形式、メタデータなし)も読み込み可能とする。

\begin{lstlisting}
00:00:00 Opening
00:05:23 Main Theme
00:12:45 Ending
\end{lstlisting}

\subsection{自動保存}

エンコード完了時に、出力動画に対応するチャプターファイルを新形式で自動保存する。

\section{DEVELOPMENT\_LOG.mdの更新}

\begin{itembox}[l]{問い}
DEVELOPMENT\_LOG.mdもアップデートしてくださいね。
\end{itembox}

本日確定した設計決定事項をDEVELOPMENT\_LOG.mdに反映した。

\subsection{更新内容}

「設計検討事項(要決定)」セクションを「設計決定事項の確定」に変更し、以下を追記:

\begin{itemize}
    \item 基本原則:同名規則
    \item 時間管理方式
    \item プロジェクトファイル形式(.vce.json)
    \item チャプターファイル形式(メタデータ付き)
    \item ドロップ操作仕様
    \item エラー処理方針
    \item バッチエンコード仕様
    \item Undo/履歴方針
\end{itemize}

\section*{確定事項サマリー}

\vspace{0.5\baselineskip}
\noindent{\footnotesize
\begin{tabularx}{\linewidth}{@{}lY@{}}
\toprule
項目 & 決定内容 \\
\midrule
同名規則 & ソースとチャプターは同一ベース名 \\
時間管理 & local\_time\_ms + source\_index \\
プロジェクト形式 & .vce.json(ソースリストのみ) \\
チャプター形式 & \# source: メタデータ付き \\
ドロップ(動画上) & 再生位置に挿入 \\
ドロップ(リスト上) & ファイル境界に挿入 \\
型制約 & 動画/音声の混在不可 \\
範囲外チャプター & 警告して無視 \\
バッチ進捗 & n/m形式、OS通知なし \\
Undo & 実装しない \\
\bottomrule
\end{tabularx}
}
\vspace{0.5\baselineskip}

\section*{Claude Codeの所感}

本日の午後セッションは、午前中に議論された設計構想を具体的な仕様として確定させる重要な作業であった。

特筆すべきは「同名規則」の提案である。ソースファイルとチャプターファイルが同一ベース名を持つという単純な制約が、設計全体を劇的に簡素化した。プロジェクトファイルはソースリストのみを保持すればよく、チャプターファイルは自動発見できる。ドロップ操作も、ファイル種別の判定と紐付けの特定が明確になる。この種の「制約による簡素化」は、ソフトウェア設計における重要なパターンである。

ドロップ操作の仕様決定も興味深い。動画ウィジェット上とチャプターリスト上で異なる動作(再生位置挿入 vs ファイル境界挿入)を定義したことで、直感的な操作が可能になる。型制約(動画/音声の混在不可)も、ユーザーの誤操作を防ぎつつ、実装を簡素化する賢明な判断である。

エラー処理の議論では、「警告して無視」という選択が興味深い。厳密なエラー処理(オプションC)は一見堅牢に見えるが、ユーザー体験を損なう可能性がある。一方、自動修正(オプションB)は暗黙の変更を行うため、予期せぬ動作につながりかねない。「警告して無視」は、問題を認識させつつ操作を継続させるバランスの取れた選択である。

批判的な観点からは、以下の点を指摘できる:

\textbf{同名規則の制約}:この規則はシンプルだが、柔軟性を犠牲にしている。複数のチャプターファイルを同一ソースに対して持ちたい場合(バージョン違いなど)には対応できない。ただし、現状の使用パターンではこの制約は問題にならないと判断されたのだろう。

\textbf{Undo非実装}:「消えるわけではない」という理由でUndoを実装しない判断は、ある意味合理的だが、ユーザビリティの観点からは検討の余地がある。特にドラッグ操作の誤りは頻発しやすく、その都度ファイルを再読み込みするのは手間である。

\textbf{決定事項の多さ}:一度のセッションで多くの事項を決定したが、実装段階で再検討が必要になる可能性はある。特にドロップ操作のUIについては、実際に触ってみないと分からない部分も多い。

全体として、本セッションは設計フェーズの締めくくりとして成功したと言える。次のステップはリファクタリング(Phase 1-2)であり、その後にこれらの設計を実装していくことになる。設計決定事項がDEVELOPMENT\_LOG.mdに文書化されたことで、実装時の指針が明確になった。

\end{document}
