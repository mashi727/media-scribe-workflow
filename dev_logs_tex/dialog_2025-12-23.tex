\documentclass[a4paper,10pt,twocolumn]{ltjsarticle}

% LuaLaTeX用フォント設定パッケージ
\usepackage{luatexja-fontspec}
\usepackage{amsmath,amssymb}
\usepackage{unicode-math}

% 欧文フォント設定 (Libertinus)
\setmainfont{Libertinus Serif}[
    BoldFont = {Libertinus Serif Bold},
    ItalicFont = {Libertinus Serif Italic},
    BoldItalicFont = {Libertinus Serif Bold Italic}
]
\setsansfont{Libertinus Sans}[
    BoldFont = {Libertinus Sans Bold},
    ItalicFont = {Libertinus Sans Italic}
]
\setmonofont{Libertinus Mono}

% 日本語フォント設定 (原ノ味フォント)
\setmainjfont{HaranoAjiMincho-Regular}[
    BoldFont = {HaranoAjiGothic-Medium},
    ItalicFont = {HaranoAjiMincho-Regular},
    BoldItalicFont = {HaranoAjiGothic-Bold}
]
\setsansjfont{HaranoAjiGothic-Regular}[
    BoldFont = {HaranoAjiGothic-Bold}
]
\setmonojfont{HaranoAjiGothic-Regular}

% 数式フォント設定
\setmathfont{Libertinus Math}

% パッケージ
\usepackage{geometry}
\geometry{margin=15mm}
\usepackage{fancyhdr}
\usepackage{lastpage}
\usepackage{ascmac}
\usepackage{booktabs}
\usepackage{array}
\usepackage{tabularx}
\usepackage{enumitem}
\usepackage{hyperref}
\usepackage{xcolor}
\usepackage{listings}

% JST日時
\newcommand{\generatedDate}{2026-01-07}
\newcommand{\generatedTime}{15:45}

% ヘッダー・フッター設定
\pagestyle{fancy}
\fancyhf{}
\fancyhead[R]{\small \generatedDate\ \generatedTime\ JST (\thepage/\pageref{LastPage})}
\renewcommand{\headrulewidth}{0.4pt}

% 最初のページはヘッダーなし
\fancypagestyle{plain}{
  \fancyhf{}
  \renewcommand{\headrulewidth}{0pt}
}

% コード表示設定
\lstset{
  basicstyle=\ttfamily\scriptsize,
  breaklines=true,
  frame=single,
  columns=fullflexible,
  keepspaces=true
}

\title{\textbf{開発ログ 2025年12月23日}\\
\large Video Chapter Editor 開発対話記録}
\author{執筆者: ましDialogue}
\date{}

\begin{document}
\maketitle
\thispagestyle{plain}

\noindent\textbf{メッセージ数: 175}

\section*{22:29:49}

\begin{itembox}[l]{問い}
未実装タスクを実行してください。
\end{itembox}

\noindent\textbf{Assistant (22:29:58):}
CLAUDE.mdに記載されている未実装タスクを実行します。まず現在のプロジェクト構造を確認します。

\noindent\textbf{Assistant (22:30:00):}
\texttt{[Tool: Bash] ls -la /Users/.../rehearsal-workflow/}

\noindent\textbf{Assistant (22:30:01):}
\texttt{[Tool: Glob]}

\noindent\textbf{Assistant (22:30:12):}
現在の構造を確認しました。TodoWriteでタスクを管理し、既存ファイルの内容を確認しながら進めます。

\noindent\textbf{Assistant (22:30:17):}
\texttt{[Tool: TodoWrite]}
\begin{itemize}[noitemsep]
\item bin/yt-srt の整理(現在の bin/ を確認し、yt-srt を作成)
\item bin/video-trim の作成
\item bin/video-chapters の作成
\item examples/prompts/ の作成(Claude用・汎用プロンプト)
\item README.md の更新(新構成に合わせる)
\end{itemize}

\noindent\textbf{Assistant (22:30:18--22:30:19):}
\texttt{[Tool: Read]} 既存ファイルを読み込み:
\begin{itemize}[noitemsep]
\item bin/rehearsal-download
\item bin/rehearsal-finalize
\item README.md
\end{itemize}

\noindent\textbf{Assistant (22:30:45):}
既存ファイルを確認しました。CLAUDE.mdの設計方針と現在の実装を照らし合わせると、いくつかの差異があります。tex2chapters と claude/commands/rehearsal.md も確認します。

\noindent\textbf{Assistant (22:31:06):}
プロジェクトの全体像を把握しました。現在の実装(rehearsal-download/finalize)は統合ワークフローツールとして既に存在しますが、CLAUDE.mdの設計方針に基づく「単機能の配管ツール」が未実装です。

未実装タスクを順次実行します。

\noindent\textbf{Assistant (22:31:33--22:33:07):}
binツールを作成:
\begin{itemize}[noitemsep]
\item \texttt{[Tool: Write]} bin/yt-srt
\item \texttt{[Tool: Write]} bin/video-trim
\item \texttt{[Tool: Write]} bin/video-chapters
\end{itemize}

\noindent\textbf{Assistant (22:33:28):}
binツールを作成しました。実行権限を付与し、次のタスクに進みます。

\noindent\textbf{Assistant (22:33:33--22:34:51):}
examples/prompts/ を作成:
\begin{itemize}[noitemsep]
\item \texttt{[Tool: Write]} examples/prompts/rehearsal-claude.md
\item \texttt{[Tool: Write]} examples/prompts/rehearsal-generic.md
\end{itemize}

\noindent\textbf{Assistant (22:35:04--22:37:46):}
docs/advanced/ を作成:
\begin{itemize}[noitemsep]
\item \texttt{[Tool: Write]} docs/advanced/my-setup.md
\item \texttt{[Tool: Write]} docs/advanced/whisper-remote.md
\item \texttt{[Tool: Write]} docs/advanced/luatex-docker.md
\item \texttt{[Tool: Write]} docs/advanced/claude-commands.md
\end{itemize}

\noindent\textbf{Assistant (22:37:55--22:38:57):}
README.mdを更新し、CLAUDE.mdの未実装タスクセクションを更新しました。

\noindent\textbf{Assistant (22:39:39):}
すべての未実装タスクを完了しました。

\textbf{作成したファイル:}

\vspace{0.3\baselineskip}
\noindent{\footnotesize
\begin{tabularx}{\linewidth}{@{}lX@{}}
\toprule
ファイル & 説明 \\
\midrule
bin/yt-srt & YouTube字幕取得 \\
bin/video-trim & 動画トリミング \\
bin/video-chapters & チャプター結合・埋め込み \\
examples/prompts/rehearsal-claude.md & Claude用プロンプト \\
examples/prompts/rehearsal-generic.md & 汎用プロンプト \\
docs/advanced/my-setup.md & 全体構成例 \\
docs/advanced/whisper-remote.md & Whisperリモート環境 \\
docs/advanced/luatex-docker.md & LuaTeX Docker環境 \\
docs/advanced/claude-commands.md & Claude Codeコマンド設定 \\
\bottomrule
\end{tabularx}
}
\vspace{0.3\baselineskip}

\section*{22:39:47}

\begin{itembox}[l]{問い}
pushしました?
\end{itembox}

\noindent\textbf{Assistant (22:39:52):}
いいえ、まだpushしていません。commitもしていません。

commitとpushを行いますか?

\section*{22:39:56}

\begin{itembox}[l]{問い}
ですね。
\end{itembox}

\noindent\textbf{Assistant (22:40:03--22:40:51):}
\texttt{[Tool: Bash]} git status, git log, git add, git commit, git push

完了しました。

\begin{lstlisting}
f4ddd6b Add plumbing tools and documentation
\end{lstlisting}

16ファイル、2633行の追加をpushしました。

\section*{22:42:08}

\begin{itembox}[l]{問い}
やはり、Mermaidだと読みづらいですね。PADだと、抽象度が右に行くにつれて低くなるので、配管と陶器の整理が行いやすいんですよね。。。
\end{itembox}

\noindent\textbf{Assistant (22:42:21):}
その通りですね。

\textbf{Mermaid}は「フローの流れ」を表現するのに適していますが、抽象度の階層が視覚的に分かりにくい。

\textbf{PAD}は構造的に:
\begin{lstlisting}
高抽象(陶器)          低抽象(配管)
---------------------------------->
rehearsal-download  ->  ytdl -> yt-dlp
                    ->  whisper-remote -> curl -> Docker
\end{lstlisting}

この「右に展開する」構造が、まさに「陶器(ユーザー向けコマンド)」と「配管(内部実装)」の関係を自然に表現できます。

\section*{22:42:53}

\begin{itembox}[l]{問い}
ですね。pngにレンダリングして、貼り付けた方が良いかなあ。
\end{itembox}

\noindent\textbf{Assistant (22:43:02):}
良いと思います。

\textbf{メリット}:
\begin{itemize}[noitemsep]
\item GitHubのREADMEで直接見える
\item PADtoolsがなくても構造を把握できる
\item 「階層的な使い方」セクションの補足として最適
\end{itemize}

\section*{22:43:38}

\begin{itembox}[l]{問い}
エクスポート機能がありますね。GUIなので、自動化したいところですが、できます?
\end{itembox}

\noindent\textbf{Assistant (22:43:44):}
PADtoolsのCLIモードがあるか確認してみます。

\noindent\textbf{Assistant (22:44:09):}
PADtoolsはGUIのみで、CLIモードはなさそうです。

.spdファイルを見ると、テキストDSL形式ですね:
\begin{lstlisting}
:switch 入力ソース
:case YouTube URL
    :call yt-srt
\end{lstlisting}

\textbf{自動化の選択肢}:

\vspace{0.3\baselineskip}
\noindent{\footnotesize
\begin{tabularx}{\linewidth}{@{}lXX@{}}
\toprule
方法 & 難易度 & 備考 \\
\midrule
AppleScript/Automator & 中 & GUI操作を自動化、壊れやすい \\
独自レンダラー作成 & 高 & DSL解析→SVG/PNG生成 \\
手動エクスポート & 低 & 変更時のみ更新 \\
\bottomrule
\end{tabularx}
}
\vspace{0.3\baselineskip}

\section*{22:51:30}

\begin{itembox}[l]{問い}
3にしたいですね。
\end{itembox}

\noindent\textbf{Assistant (22:51:38):}
良いですね。まず.spdの形式を全て確認します。

DSLの構文を把握しました:
\begin{lstlisting}
:terminal    - 開始/終了
:comment     - コメント(見出し)
:call        - サブルーチン(右に展開)
:switch/:case - 選択分岐
:if/:else    - 条件分岐
インデント    - ネスト(詳細化)
@(...)       - 注釈
\end{lstlisting}

\noindent\textbf{Assistant (22:52:50--22:53:35):}
Pythonで spd2svg を作成しました。動作しました。残りのファイルも変換します。

\section*{22:54:40}

\begin{itembox}[l]{問い}
随分印象と違いますね。PADの表記に忠実にできますか?
\end{itembox}

\noindent\textbf{Assistant (22:54:57--22:57:22):}
PADの正式な表記法を確認します。

\textbf{正式なPAD記号}:
\begin{enumerate}[noitemsep]
\item \textbf{順次}: 縦線の右に処理を縦に並べる
\item \textbf{選択}: 右端が凹んだ旗型(上が真、下が偽)
\item \textbf{反復}: 長方形から右に水平線、二重線で判定位置を示す
\item \textbf{手続き呼び出し}: 左右が二重線の長方形
\end{enumerate}

正式なPAD表記で再生成しました。

\section*{22:57:41}

\begin{itembox}[l]{問い}
まだまだですね。車輪の再開発っぽいので、PadtoolsをフォークしてCLI実装を行うとかどうですか。
\end{itembox}

\noindent\textbf{Assistant (22:57:50):}
その方が確実ですね。PADtoolsのソースを確認します。

\noindent\textbf{Assistant (22:58:06--22:59:17):}
GitHubでPADtoolsのソースを検索しました。

\begin{center}
\rule{0.8\linewidth}{0.4pt}
\textit{(セッション継続)}
\rule{0.8\linewidth}{0.4pt}
\end{center}

\noindent\textbf{Assistant (22:59:24--23:00:03):}
PADtools のGitHubリポジトリを調査して、CLI実装の方針を検討します。

重要な発見があります。PADtoolsは既に\textbf{Apache Batik}と\textbf{JFreeChart SVG}を使用しており、SVG出力機能が内蔵されています。

素晴らしい構造です。特に重要なクラス:
\begin{itemize}[noitemsep]
\item \texttt{View2Image} -- 画像出力
\item \texttt{Model2View} -- モデル→ビュー変換
\item \texttt{SPDParser} -- SPDファイル解析
\item \texttt{Converter} -- 変換機能
\end{itemize}

\noindent\textbf{Assistant (23:00:55):}
\texttt{Converter}クラスに\texttt{convert(File file\_in, File file\_out, Double scale)}メソッドが存在します。処理フロー:
\begin{enumerate}[noitemsep]
\item SPDParser.parse() → PADModel
\item Model2View.toView() → View
\item View2Image.toImage(view, scale) → BufferedImage
\item ImageIO.write() → PNG出力
\end{enumerate}

\noindent\textbf{Assistant (23:01:09):}
素晴らしい発見です!CLI出力機能は\textbf{既に実装されています}:
\begin{lstlisting}
Arguments: [-o result_file] [-s scale] [-h] [spd_file]
  -o result_file: Save to result_file.
  -s scale: Image scale(available when result_file is set).
  -h: Show this help.
  spd_file: Open spd file.
\end{lstlisting}

\noindent\textbf{Assistant (23:02:14--23:04:48):}
しかしGUIが開いてしまいます。問題はGUIスレッドが先に起動してしまうことです。

実用的な解決策として、PADtoolsをライブラリとして使い、Converterを直接呼び出すラッパーを作成しましょう。

\noindent\textbf{Assistant (23:08:16--23:10:29):}
\texttt{[Tool: Write]} tools/PadCLI.java

\texttt{[Tool: Bash]} javac -cp PadTools.jar PadCLI.java

\noindent\textbf{Assistant (23:11:38):}
成功しました!出力ファイルを確認します:

\noindent\textbf{Assistant (23:11:46):}
PNG生成成功です(822 x 1294 pixels)。

便利なシェルスクリプトラッパーを作成します:

\texttt{[Tool: Write]} bin/spd2png

全てのSPDファイルのPNGを生成します。

\noindent\textbf{Assistant (23:12:48):}
全てのPAD図が正しくレンダリングされました。

\textbf{作成したもの:}
\begin{itemize}[noitemsep]
\item bin/spd2png -- PADtools CLI PNG変換スクリプト
\item tools/PadCLI.java -- PADtoolsのConverter直接呼び出しラッパー
\item docs/pad/*.png -- 3つのワークフロー図のPNG出力
\end{itemize}

\textbf{PADtools CLI問題の解決策:}
PADtools本体のMain.javaはGUIスレッドを先に起動してしまうため、ヘッドレス環境で動作しませんでした。解決策として、Converterクラスを直接呼び出すPadCLI.javaを作成しました。

コミットしますか?

\section*{23:14:43}

\begin{itembox}[l]{問い}
ですね。
\end{itembox}

\noindent\textbf{Assistant (23:14:48--23:15:27):}
\texttt{[Tool: Bash]} git add, git commit, git push

完了しました。

\begin{lstlisting}
a6d01cc Add PADtools CLI wrapper and generate PAD diagram PNGs
\end{lstlisting}

PADtoolsをフォークせずに、既存のConverterクラスを直接呼び出すラッパーで解決できました。

\section*{所感(Claude Code)}

この日の対話は、rehearsal-workflowプロジェクトの基盤構築とPADtoolsのCLI化という二つの大きなタスクを扱った。

特筆すべきは、PADtoolsのCLI化における試行錯誤である。当初は独自レンダラーの作成を試みたが、ユーザーの「まだまだですね」という率直なフィードバックを受け、PADtoolsのソースコード調査へと方針転換した。その結果、既存のConverterクラスを活用するという、より堅実な解決策に到達した。

この対話は、「車輪の再発明」を避けるという原則の重要性を示している。既存のライブラリのAPIを調査し、それを活用する方が、独自実装よりも信頼性が高く、保守も容易である。

\end{document}
