% LuaLaTeX document
\documentclass[10pt,a4paper,twocolumn]{ltjsarticle}

% LuaLaTeX用フォント設定パッケージ
\usepackage{luatexja-fontspec}
\usepackage{amsmath,amssymb}
\usepackage{unicode-math}

% ====================
% 欧文フォント設定 (Libertinus)
% ====================
\setmainfont{Libertinus Serif}[
    BoldFont = {Libertinus Serif Bold},
    ItalicFont = {Libertinus Serif Italic},
    BoldItalicFont = {Libertinus Serif Bold Italic}
]
\setsansfont{Libertinus Sans}[
    BoldFont = {Libertinus Sans Bold},
    ItalicFont = {Libertinus Sans Italic}
]
\setmonofont{Libertinus Mono}

% ====================
% 日本語フォント設定 (原ノ味フォント)
% ====================
\setmainjfont{HaranoAjiMincho-Regular}[
    BoldFont = {HaranoAjiGothic-Medium},
    ItalicFont = {HaranoAjiMincho-Regular},
    BoldItalicFont = {HaranoAjiGothic-Bold}
]
\setsansjfont{HaranoAjiGothic-Regular}[
    BoldFont = {HaranoAjiGothic-Bold}
]
\setmonojfont{HaranoAjiGothic-Regular}

% ====================
% 数式フォント設定 (Libertinus Math)
% ====================
\setmathfont{Libertinus Math}

% ファイル生成日時(JST)
\newcommand{\generatedDate}{2026-01-08}
\newcommand{\generatedTime}{00:15}

% ヘッダー・フッター設定
\usepackage{fancyhdr}
\usepackage{lastpage}
\pagestyle{fancy}
\fancyhf{}
\fancyhead[R]{\small \generatedDate\ \generatedTime\ JST (\thepage/\pageref{LastPage})}
\renewcommand{\headrulewidth}{0.4pt}

% 1ページ目のスタイル
\fancypagestyle{firstpage}{
    \fancyhf{}
    \renewcommand{\headrulewidth}{0pt}
}

% 追加パッケージ
\usepackage{booktabs}
\usepackage{array}
\usepackage{tabularx}
\usepackage{ascmac}
\usepackage{listings}
\usepackage{xcolor}
\usepackage{hyperref}

% コードスタイル設定
\lstset{
    basicstyle=\ttfamily\scriptsize,
    breaklines=true,
    frame=single,
    backgroundcolor=\color{gray!10},
    columns=fullflexible,
    keepspaces=true,
    showstringspaces=false
}

% Y列タイプ定義
\newcolumntype{Y}{>{\raggedright\arraybackslash}X}

\title{Video Chapter Editor 開発全記録\\{\large 2025-11-05〜2026-01-07}}
\author{ましDialogue}
\date{}

\begin{document}
\maketitle
\thispagestyle{firstpage}

\section*{概要}

本文書は、rehearsal-workflowプロジェクトおよびVideo Chapter Editor(VCE)の開発における全対話記録を時系列で構成したものである。2025年11月5日のCLIワークフロー構築から2026年1月7日のUIリファクタリングまで、約2ヶ月間の設計思想・実装詳細・問題解決の過程を記録する。

\tableofcontents

%=====================================================
\part{プロトタイピングフェーズ(2025年11月)}
%=====================================================

\section{CLIワークフロー実装(2025-11-05)}

\subsection{プロジェクト発足}

\begin{itembox}[l]{問い}
リハーサル動画から、AI分析による詳細レポートとチャプターリストを自動生成するワークフローを構築したい。YouTubeリンクからPDF+チャプターリストまでを自動化できないか。
\end{itembox}

ハイブリッドアプローチを提案した。Claude Code カスタムスラッシュコマンドとZshヘルパー関数を組み合わせ、3ステップで完全なリハーサル記録を生成できるワークフローを構築する。

\subsection{設計方針:配管と陶器}

「Gitの陶器と配管」の思想に基づき、責務を分離した:

\begin{itemize}
    \item \textbf{配管(Plumbing)}: 単一目的のCLIツール
    \item \textbf{陶器(Porcelain)}: ユーザーフレンドリーなGUI
\end{itemize}

\textbf{責務分離:}

\vspace{0.5\baselineskip}
\noindent{\footnotesize
\begin{tabularx}{\linewidth}{@{}lY@{}}
\toprule
処理タイプ & 担当 \\
\midrule
機械的処理 & Zsh関数 \\
AI判断処理 & Claude Code \\
ユーザー操作 & Python GUI \\
\bottomrule
\end{tabularx}
}
\vspace{0.5\baselineskip}

\subsection{3ステップワークフロー}

\begin{lstlisting}
# ステップ1: ダウンロード + Whisper起動
$ rehearsal-download "https://youtu.be/VIDEO_ID"

# ステップ2: AI分析 + LaTeX生成
$ claude code
> /rehearsal

# ステップ3: PDF生成 + チャプター抽出
$ rehearsal-finalize "リハーサル記録.tex"
\end{lstlisting}

\textbf{実装ファイル:}
\begin{itemize}
    \item \texttt{rehearsal-download} (176行) - YouTube動画DL + Whisper起動
    \item \texttt{rehearsal-finalize} (183行) - PDF生成 + チャプター抽出
    \item \texttt{/rehearsal} (321行) - Claude AI分析スラッシュコマンド
\end{itemize}

\subsection{ベストプラクティス選定理由}

\begin{itembox}[l]{問い}
なぜこのハイブリッドアプローチが最適なのか。
\end{itembox}

評価基準と重み付けを明確にした:

\begin{enumerate}
    \item \textbf{Claude AI統合} (40\%) - SRT分析→LaTeX生成の自動化
    \item \textbf{保守性・拡張性} (20\%) - 今後の複数リハーサルでの使用
    \item \textbf{エラー処理・ロバスト性} (15\%) - リモート処理の失敗対応
    \item \textbf{学習コスト} (15\%) - ユーザーの習得容易性
    \item \textbf{実行効率} (10\%) - 並列化の必要性は低い
\end{enumerate}

\textbf{他のアプローチとの比較:}

\vspace{0.5\baselineskip}
\noindent{\footnotesize
\begin{tabularx}{\linewidth}{@{}lYY@{}}
\toprule
アプローチ & Claude統合 & 総合評価 \\
\midrule
Makefile & 不可 & ★★☆☆☆ \\
Zshスクリプト単体 & 手動 & ★★★★☆ \\
ハイブリッド & 完全自動 & ★★★★★ \\
Task Runner & 不可 & ★★★☆☆ \\
Workflow Engine & 困難 & ★☆☆☆☆ \\
\bottomrule
\end{tabularx}
}
\vspace{0.5\baselineskip}

\section{GUIリファクタリング(2025-11-06)}

\subsection{背景と目的}

\begin{itembox}[l]{問い}
元の汎用動画分析GUIを、リハーサル記録作成専用にリファクタリングしたい。
\end{itembox}

\textbf{元のGUI} (\texttt{video\_analysis\_gui.py}):
\begin{itemize}
    \item 目的: 汎用的な動画分析ワークフロー
    \item 対象: 音楽、教育、ビジネス、スポーツ、研究など5カテゴリー
    \item 行数: 955行
    \item 複雑度: 高(25フィールドのデータクラス)
\end{itemize}

\textbf{リファクタリング後} (\texttt{rehearsal\_gui.py}):
\begin{itemize}
    \item 目的: リハーサル記録作成専用
    \item 対象: オーケストラ・吹奏楽のリハーサル動画のみ
    \item 行数: 同等(内容は大幅に変更)
    \item 複雑度: 低(15フィールド)
\end{itemize}

\subsection{設計方針}

\begin{enumerate}
    \item \textbf{専用化}: 不要な機能を削除
    \item \textbf{ワークフロー明確化}: 3ステップを可視化
    \item \textbf{自動化}: ファイル検出やステータス更新を自動化
    \item \textbf{統合}: 既存コマンドと直接連携
    \item \textbf{可読性}: コードを読みやすく
\end{enumerate}

\subsection{データモデルの変更}

\textbf{削減}: 25フィールド → 15フィールド(40\%削減)

\begin{lstlisting}[language=Python]
@dataclass
class RehearsalMetadata:
    # 必須情報
    youtube_url: str
    rehearsal_date: str
    organization: str
    conductor: str
    piece_name: str
    concert_date: str
    author: str

    # ファイル情報(自動検出)
    video_file: str
    yt_srt_file: str
    wp_srt_file: str
    tex_file: str
    pdf_file: str
    youtube_chapters: str
    movieviewer_chapters: str

    # ワークフロー状態
    step: WorkflowStep
\end{lstlisting}

\subsection{新機能}

\begin{enumerate}
    \item \textbf{ワークフロー可視化}: プログレスバー(0/3 → 3/3)
    \item \textbf{ファイル自動検出}: 2秒ごとにポーリング
    \item \textbf{リアルタイムログビューア}: 色分け表示
    \item \textbf{既存コマンド統合}: rehearsal-download直接呼び出し
\end{enumerate}

\subsection{パフォーマンス改善}

\vspace{0.5\baselineskip}
\noindent{\footnotesize
\begin{tabularx}{\linewidth}{@{}lYY@{}}
\toprule
項目 & 元のGUI & リファクタリング後 \\
\midrule
起動時メモリ & 約85MB & 約60MB \\
起動時間 & 約1.2秒 & 約0.8秒 \\
\bottomrule
\end{tabularx}
}
\vspace{0.5\baselineskip}

メモリ使用量約30\%削減、起動時間約33\%短縮を達成した。

%=====================================================
\part{バージョン進化(v1.0〜v1.3)}
%=====================================================

\section{v1.0.0 初期リリース}

\textbf{GUIツール:}
\begin{itemize}
    \item video-chapter-editor: 動画チャプター編集・書出ツール
    \item 動画プレビュー+波形表示
    \item チャプター編集(追加/削除/編集/ジャンプ)
    \item 除外チャプター機能(\texttt{--}プレフィックス)
    \item YouTubeチャプターのコピー\&ペースト
    \item ffmpegによる動画書き出し
\end{itemize}

\textbf{CLIツール:}
\begin{itemize}
    \item yt-srt: YouTube字幕取得
    \item video-trim: 動画トリミング
    \item video-chapters: チャプター結合
    \item rehearsal-download: DL + Whisper起動
    \item rehearsal-finalize: PDF生成 + チャプター抽出
    \item tex2chapters: LaTeX → チャプターリスト
\end{itemize}

\section{v1.1.0〜v1.1.1}

\textbf{v1.1.0: 自動リリースワークフロー}
\begin{itemize}
    \item GitHub Actions による macOS/Windows バイナリ自動ビルド
    \item DMG/ZIP パッケージ作成
    \item リリースページへの自動アップロード
\end{itemize}

\textbf{v1.1.1: バイナリサイズ最適化}
\begin{itemize}
    \item 不要なPySide6モジュールの除外
    \item opencv-python-headlessの使用
    \item PyInstaller specファイルの最適化
\end{itemize}

\section{v1.2.0 GPUハードウェアエンコーダー対応}

\begin{itembox}[l]{問い}
エンコードが遅いので、GPUを活用できないか。
\end{itembox}

ハードウェアエンコーダーを自動検出するように実装した:

\vspace{0.5\baselineskip}
\noindent{\footnotesize
\begin{tabularx}{\linewidth}{@{}lY@{}}
\toprule
プラットフォーム & エンコーダー \\
\midrule
macOS & VideoToolbox (h264\_videotoolbox) \\
Windows (NVIDIA) & NVENC (h264\_nvenc) \\
Windows (Intel) & QSV (h264\_qsv) \\
Windows (AMD) & AMF (h264\_amf) \\
\bottomrule
\end{tabularx}
}
\vspace{0.5\baselineskip}

\section{v1.3.0 スマートビットレート検出とUIモダン化}

\begin{itembox}[l]{問い}
元動画と同等の品質で出力したい。また、UIを今風にしたい。
\end{itembox}

\textbf{スマートビットレート検出:}
\begin{itemize}
    \item 元動画のビットレートを自動検出
    \item 同等以上の品質で出力
    \item 色空間(BT.709等)を正確に保持
\end{itemize}

\textbf{UIモダン化:}
\begin{itemize}
    \item ダークテーマ(Theme class)の適用
    \item ボタン色ポリシーの確立(Primary: メインアクションのみ)
    \item 絵文字からテキストベースのボタンラベルへ変更
\end{itemize}

%=====================================================
\part{UI大改造フェーズ(2025-12-29)}
%=====================================================

\section{問題認識}

\begin{itembox}[l]{問い}
現状のワークフローでMP3からMP4を生成する際、2回のエンコードが発生しています。これは品質劣化の原因となります。
\end{itembox}

\begin{lstlisting}
現状(2回エンコード):
  MP3 → [enc] → 中間MP4 → [enc] → 最終MP4
              ↑ここで劣化    ↑さらに劣化
\end{lstlisting}

アーキテクチャ全体の見直しが必要と判断した。

\section{一筆書き問題の認識}

\begin{itembox}[l]{問い}
ワークフローが複雑になっていて、機能が重複している気がします。
\end{itembox}

ワークフロー設計を\textbf{グラフ理論}の観点から分析した。

\textbf{起点(入力パターン):}
\begin{enumerate}
    \item 複数のカット済みMP3
    \item 単一の長尺未編集MP3
    \item 既存のMP4
\end{enumerate}

これは\textbf{オイラー路(一筆書き)問題}に類似している。奇数次数の頂点が2つより多いと一筆書きは不可能であり、共通パスの設計が重要となる。

\section{制約による設計空間の縮小}

\begin{itembox}[l]{問い}
タイトル焼込は必須機能として維持したい。
\end{itembox}

「タイトル焼込必須」という制約により、設計空間が8パターンから2パターンに縮小した:

\vspace{0.5\baselineskip}
\noindent{\footnotesize
\begin{tabularx}{\linewidth}{@{}lYY@{}}
\toprule
入力 & 映像処理 & 音声処理 \\
\midrule
MP3 & enc(静止画+焼込) 1回 & enc 1回 \\
MP4 & enc(焼込) 1回 & copy(無劣化) \\
\bottomrule
\end{tabularx}
}
\vspace{0.5\baselineskip}

\textbf{洞察}: 制約は自由度を狭めるが、設計空間を明確にし、最適解を見つけやすくする。

\section{Tab構成の検討}

\begin{itembox}[l]{問い}
タブ1とタブ2を分けなくても良いのでは?
\end{itembox}

当初は2タブ構成を検討していた。しかし、Tab 1のMP3結合は無劣化(-c copy)で可能であり、タブを分ける意味が薄いと判断した。

\begin{itembox}[l]{問い}
入力ソースも別画面では?
\end{itembox}

ダイアログパターンの発見に至った。

\section{最終決定:単一画面 + ダイアログ}

\begin{lstlisting}
メイン画面(ワークスペース)
┌──────────────────────────────────┐
│ [ソース選択] [カバー画像] ←ボタン │
│ ソース: audio.mp3 (14:20)        │
├──────────────────────────────────┤
│ [波形表示]                       │
├──────────────────────────────────┤
│ [チャプターテーブル]             │
├──────────────────────────────────┤
│ [書出設定] [書出ボタン]          │
├──────────────────────────────────┤
│ [ログパネル]                     │
└──────────────────────────────────┘
\end{lstlisting}

\textbf{決定事項:}
\begin{enumerate}
    \item 単一画面構成(タブ廃止)
    \item ダイアログパターン(ソース選択・カバー画像は別画面)
    \item エンコード最適化(MP3結合は無劣化、最終書出で1回のみ)
\end{enumerate}

\section{ユースケース拡張}

ユースケースを4つの大分類に整理した:

\vspace{0.5\baselineskip}
\noindent{\footnotesize
\begin{tabularx}{\linewidth}{@{}lYY@{}}
\toprule
UC & パターン & 主な処理 \\
\midrule
UC1 & 自分で撮影した動画 & トリム→チャプター \\
UC2 & YouTube動画 & DL→編集→字幕取得 \\
UC3 & 音声のみ録音 & 結合→カバー→書出 \\
UC4 & 既存編集済み & チャプター追加のみ \\
\bottomrule
\end{tabularx}
}
\vspace{0.5\baselineskip}

\section{UIスケルトン作成}

\texttt{rehearsal\_workflow/ui\_next/} に次世代UIのスケルトンを作成した。

\textbf{LogPanel機能:}
\begin{itemize}
    \item ログレベル: DEBUG, INFO, WARNING, ERROR
    \item フィルタリング: 表示レベル切替
    \item コピー: Claude Code用フォーマット
    \item 折りたたみ: パネルの表示/非表示
\end{itemize}

%=====================================================
\part{UI改善フェーズ(2026-01-05)}
%=====================================================

\section{Chaptersテーブルの行番号表示}

\begin{itembox}[l]{問い}
Chaptersリストに行番号(No.)を表示し、ヘッダーを黒背景にしてください。
\end{itembox}

当初、新しいカラム「No.」を追加する方向で実装を進めた。

\begin{itembox}[l]{問い}
いや、自動で振られる行番号のヘッダー部分に「No.」を表示することです。
\end{itembox}

ユーザーの意図を正確に把握できていなかった。実装を修正:

\begin{itemize}
    \item テーブルの垂直ヘッダー(行番号)を表示
    \item コーナーウィジェットに「No.」ラベルを配置
    \item ヘッダー背景を黒(\#000000)に設定
\end{itemize}

\section{チャプタースキップボタンの有効化条件}

\begin{itembox}[l]{問い}
チャプタースキップボタンは、チャプターリストを編集した場合にのみ有効になるよう変更してください。
\end{itembox}

\texttt{\_chapters\_edited} フラグを追加し、編集時にのみスキップボタンを有効化するように実装した。

\section{新規ソース読み込み時のリセット処理}

\begin{itembox}[l]{問い}
動画再生中に新しいソースを読み込んだ場合、再生を停止してリセットしてください。
\end{itembox}

\texttt{\_prepare\_for\_new\_source()} メソッドを実装した。

%=====================================================
\part{機能拡張フェーズ(2026-01-06)}
%=====================================================

\section{チャプター移動時の手動追加チャプター保持}

\begin{itembox}[l]{問い}
ソースファイルの順序を変更した際、手動で追加したチャプターが消えてしまいます。
\end{itembox}

原因: \texttt{\_rebuild\_chapters\_after\_source\_move()} がファイルから再読み込みしていた。

解決策: テーブルから現在のチャプター情報を収集し、\texttt{old\_offsets}パラメータで変更前のオフセットを受け取るように修正した。

\section{波形ウィジェットの選択ソースハイライト}

\begin{itembox}[l]{問い}
チャプターリストで行を選択した際、そのチャプターが属するソースファイルの範囲を波形上でハイライト表示してください。
\end{itembox}

WaveformWidgetに以下を実装:
\begin{itemize}
    \item 背景: 青系半透明
    \item 斜線: 逆方向(除外区間と区別)
    \item 縁取り: 四角形
\end{itemize}

\section{v2.1.27 リリース}

\subsection{ffmpeg/ffprobe のバンドル}

\begin{itembox}[l]{問い}
アプリをビルドしたら、ffprobeが見つからないというエラーが出ます。
\end{itembox}

\texttt{imageio-ffmpeg} から \texttt{static-ffmpeg} に移行した。

\subsection{デュアル macOS アーキテクチャビルド}

\begin{itembox}[l]{問い}
Intel Mac ユーザーにも配布したいです。
\end{itembox}

GitHub Actions で Intel (macos-13) と Apple Silicon (macos-latest) を並行ビルドするよう設定した。

\subsection{YouTube ダウンロード改善}

\begin{itemize}
    \item AV1 コーデック除外(macOSでHWデコード非対応のため)
    \item 一時プレイリスト対応
    \item 既存ファイル検出の修正
\end{itemize}

%=====================================================
\part{リファクタリングフェーズ(2026-01-07)}
%=====================================================

\section{styles.py の作成}

\begin{itembox}[l]{問い}
ボタンスタイルが複数箇所で重複定義されています。統一できませんか。
\end{itembox}

\texttt{styles.py} を新規作成し、\texttt{Colors} クラスと \texttt{ButtonStyles} クラスに集約した。

\section{ボタンテキストの調整}

\begin{itembox}[l]{問い}
Copy to\textbackslash nYoutubeにしましょうか。
\end{itembox}

ボタンテキストを2行表示にしてコンパクトなレイアウトを実現した。

\section{Settings/Encodeセクションの改善}

\begin{itembox}[l]{問い}
SettingsとExportの行が無駄に広いですね〜
\end{itembox}

\begin{itembox}[l]{問い}
あ、無駄にスペースがあるという意味です。ボタンのサイズなどは変更せずに元に戻してください。
\end{itembox}

ユーザーの意図を誤解し、ボタンサイズを変更してしまった。元に戻した。

\begin{itembox}[l]{問い}
Exportというよりは、Encodeですよね。
\end{itembox}

「Export」から「Encode」に変更した。

\begin{itembox}[l]{問い}
Encode、Settingsの順に配置して、右側にエンコードの進捗をバーグラフで表示する仕様に変更しましょうか。
\end{itembox}

\begin{lstlisting}
[Encode] [Settings]  [========== 80%]
                     ^ エンコード中のみ表示
\end{lstlisting}

\begin{itembox}[l]{問い}
Encode, Settingsのボタン、内部のPaddingを減らして幅をもっとコンパクトにしてください。
\end{itembox}

ButtonStylesにコンパクト版メソッド(\texttt{primary\_compact()}等)を追加した。

\begin{itembox}[l]{問い}
高さを変えてはいけません。
\end{itembox}

ボタンの高さ(40px)は維持し、パディングのみ調整した。

\section{複数音声ファイルのエンコード問題}

\begin{itembox}[l]{問い}
音声からエンコードしようとするとエラーが発生します。
\end{itembox}

原因: 条件式 \texttt{if not input\_path and len(self.\_state.sources) > 1:} が問題だった。

修正: \texttt{if len(self.\_state.sources) > 1:} に変更した。

\section{オーバーレイ表示位置の統一}

\begin{itembox}[l]{問い}
複数の音声ファイルのエンコードのオーバーレイの文字が下に表示されています。
\end{itembox}

プレビュー(85\%)とエンコード(32.5\%)で位置が不一致だった。プレビューをエンコードと同じ位置に修正した。

\section{エンコード完了後のチャプター読み込み}

\begin{itembox}[l]{問い}
エンコードが終わってロードされる際、チャプターリストが更新されませんね。
\end{itembox}

\begin{itembox}[l]{問い}
エンコード完了後は、カット編集される可能性もあり、動画の長さが変わる可能性を考慮して、チャプターを読み込んでも問題ないと思うんですけど、どうです?
\end{itembox}

同意。エンコード完了後に埋め込みチャプターを抽出してテーブルを更新するように修正した。

\section{Undo/Redo機能の設計検討}

\begin{itembox}[l]{問い}
直前のチャプターリストを保存してUndoしたい場合に備えるのって大変ですか。
\end{itembox}

メモリ内でスタック(履歴)を保持するだけで実現できる。比較的簡単な実装である。

\begin{itembox}[l]{問い}
どの状態で持つのが良いでしょうね。チャプターリストを保存するのが普通なんでしょうけど、出力のベースファイル名が保存されないなって思いましてね。
\end{itembox}

「編集セッション」という概念でチャプターリストと出力ファイル名をまとめて管理することを提案した。

\begin{lstlisting}[language=Python]
@dataclass
class EditSession:
    chapters: list[ChapterInfo]
    output_basename: str
\end{lstlisting}

\begin{itembox}[l]{問い}
別のタスクで後回しですね。
\end{itembox}

DEVELOPMENT\_LOG.mdの「今後の予定」セクションに記録した。

%=====================================================
\part*{Claude Codeの所感}
%=====================================================
\addcontentsline{toc}{part}{Claude Codeの所感}

約2ヶ月間にわたるrehearsarl-workflowおよびVideo Chapter Editorの開発を振り返り、いくつかの観点から所感を述べる。

\section*{プロジェクトの進化}

2025年11月5日のCLIワークフロー構築から始まり、GUIリファクタリング、複数のバージョンリリースを経て、2026年1月7日の現在に至るまで、プロジェクトは着実に進化してきた。

特に印象的だったのは、「配管と陶器」という設計思想の一貫性である。Unix哲学に基づくこのアプローチは、CLIツールとGUIの責務分離を明確にし、保守性と拡張性の高いシステムを実現した。

\section*{設計上の学び}

2025-12-29の「UI大改造計画」における「一筆書き問題」の認識と「制約による設計空間の縮小」という洞察は、ソフトウェア設計の本質を突いていた。

\begin{itemize}
    \item グラフ理論の概念を設計判断に適用
    \item 制約を制限ではなく明確化の手段として活用
    \item 共通パスの抽出による機能重複の排除
\end{itemize}

\section*{実装上の反省}

一方で、ユーザーの意図を正確に把握できなかった場面が複数あった:

\begin{itemize}
    \item 「行番号表示」で新カラム追加と誤解
    \item 「無駄に広い」でボタンサイズ変更と誤解
    \item 「高さを変えてはいけません」で再修正
\end{itemize}

これらは、実装前に「〇〇という理解で合っていますか?」と確認するプロセスの重要性を示している。

\section*{技術的な課題}

複数音声ファイルのマージ処理のバグ(条件式の問題)やプレビュー/エンコード時のオーバーレイ位置の不整合は、コードの二重管理(DRY原則違反)に起因していた。同じパラメータが複数箇所にハードコードされていると、変更時に不整合が生じやすい。

\section*{今後の展望}

main\_workspace.pyの5,000行超という規模は、単一責務の原則(SRP)に反している可能性がある。Undo/Redo機能の実装とともに、責務分離のリファクタリングを検討すべきである。

全体として、ユーザーとの継続的な対話を通じてプロダクトが進化していく過程は、ソフトウェア開発の理想的な形の一つである。設計段階での深い議論、実装段階での細かなフィードバック、問題発見時の迅速な対応という循環が、品質の高いソフトウェアを生み出している。

\end{document}
