%% dialog_2025-12-23_full.tex
%% 2025年12月23日の開発対話記録(175メッセージ)

\section{2025年12月23日:配管ツールの実装とPADtoolsのCLI対応}

この日の開発セッションでは、CLAUDE.mdに記載されていた未実装タスクの実行と、PAD図のPNG出力機能の実装が行われた。

\subsection{未実装タスクの実行}

\begin{userbox}
未実装タスクを実行してください。
\end{userbox}

\begin{assistantbox}
CLAUDE.mdに記載されている未実装タスクを実行します。まず現在のプロジェクト構造を確認します。

プロジェクトの全体像を把握しました。現在の実装(rehearsal-download/finalize)は統合ワークフローツールとして既に存在しますが、CLAUDE.mdの設計方針に基づく「単機能の配管ツール」が未実装です。

未実装タスクを順次実行します。
\end{assistantbox}

以下のタスクがTodoWriteで管理された:
\begin{enumerate}
\item bin/yt-srt の整理(現在の bin/ を確認し、yt-srt を作成)
\item bin/video-trim の作成
\item bin/video-chapters の作成
\item examples/prompts/ の作成(Claude用・汎用プロンプト)
\item README.md の更新(新構成に合わせる)
\end{enumerate}

これらのタスクは順次実行され、以下のファイルが作成された:

\vspace{0.5\baselineskip}
\noindent{\footnotesize
\begin{tabularx}{\linewidth}{@{}lX@{}}
\toprule
ファイル & 説明 \\
\midrule
bin/yt-srt & YouTube字幕取得 \\
bin/video-trim & 動画トリミング \\
bin/video-chapters & チャプター結合・埋め込み \\
examples/prompts/rehearsal-claude.md & Claude用プロンプト \\
examples/prompts/rehearsal-generic.md & ChatGPT/Gemini等汎用プロンプト \\
docs/advanced/my-setup.md & 全体構成例 \\
docs/advanced/whisper-remote.md & Whisperリモート環境 \\
docs/advanced/luatex-docker.md & LuaTeX Docker環境 \\
docs/advanced/claude-commands.md & Claude Codeコマンド設定 \\
\bottomrule
\end{tabularx}
}
\vspace{0.5\baselineskip}

\begin{userbox}
pushしました?
\end{userbox}

\begin{assistantbox}
いいえ、まだpushしていません。commitもしていません。

commitとpushを行いますか?
\end{assistantbox}

\begin{userbox}
ですね。
\end{userbox}

コミットとプッシュが実行され、以下の結果となった:

\begin{lstlisting}
f4ddd6b Add plumbing tools and documentation
\end{lstlisting}

16ファイル、2633行の追加がプッシュされた。

\subsection{PAD図に関する議論}

\begin{userbox}
やはり、Mermaidだと読みづらいですね。PADだと、抽象度が右に行くにつれて低くなるので、配管と陶器の整理が行いやすいんですよね。。。
\end{userbox}

\begin{assistantbox}
その通りですね。

\textbf{Mermaid}は「フローの流れ」を表現するのに適していますが、抽象度の階層が視覚的に分かりにくい。

\textbf{PAD}は構造的に:
\begin{lstlisting}
高抽象(陶器)          低抽象(配管)
---------------------------------------->
rehearsal-download  ->  ytdl -> yt-dlp
                    ->  whisper-remote -> curl -> Docker
\end{lstlisting}

この「右に展開する」構造が、まさに「陶器(ユーザー向けコマンド)」と「配管(内部実装)」の関係を自然に表現できます。

docs/pad/ に .spd ファイルがあるので、PADtoolsで視覚化すれば、どのツールがどの配管を呼び出しているか、依存関係が一目で把握できますね。
\end{assistantbox}

\begin{userbox}
ですね。pngにレンダリングして、貼り付けた方が良いかなあ。
\end{userbox}

\begin{assistantbox}
良いと思います。

\textbf{メリット}:
\begin{itemize}
\item GitHubのREADMEで直接見える
\item PADtoolsがなくても構造を把握できる
\item 「階層的な使い方」セクションの補足として最適
\end{itemize}

PADtoolsのエクスポート機能で出力できますか?それとも画面キャプチャになりますか?
\end{assistantbox}

\begin{userbox}
エクスポート機能がありますね。GUIなので、自動化したいところですが、できます?
\end{userbox}

PADtoolsのCLIモードの調査が開始された。インストール先は \texttt{/Users/mashi/local/src/padtools1.4} であることが確認された。

\subsection{PADtools CLI問題の発見}

JARファイルの解析により、CLIオプションが存在することが判明した:

\begin{lstlisting}
Arguments: [-o result_file] [-s scale] [-h] [spd_file]
\end{lstlisting}

しかし、実際にCLIモードで実行しようとすると、GUIが起動してしまう問題が発生した。

\begin{lstlisting}[language=bash]
java -jar PadTools.jar -o output.png -s 2.0 input.spd
# -> GUIが起動してしまう

java -Djava.awt.headless=true -jar PadTools.jar ...
# -> HeadlessException が発生
\end{lstlisting}

\subsection{独自レンダラーの試みと断念}

\begin{userbox}
3にしたいですね。
\end{userbox}

「独自レンダラー作成」オプションが選択され、\texttt{tools/spd2svg.py}が作成された。しかし、出力結果は正式なPAD表記とは異なるものだった。

\begin{userbox}
随分印象と違いますね。PADの表記に忠実にできますか?
\end{userbox}

PADの正式な表記法(二村良彦による設計\footnote{PAD表記法については東京大学の資料を参照:\texttt{lecture.ecc.u-tokyo.ac.jp/hideo-t/references/pad/pad.html}})を調査し、以下の要素を実装しようとした:
\begin{enumerate}
\item 縦線(スパイン)を左端に配置
\item 処理ボックスは縦線の右に接続
\item call(手続き呼び出し)は左右に二重線
\item 選択(if/switch)は旗型(右端が凹み、分岐で尖る)
\item コメントは点線枠
\end{enumerate}

しかし、結果は依然として不十分だった。

\begin{userbox}
まだまだですね。車輪の再開発っぽいので、PadtoolsをフォークしてCLI実装を行うとかどうですか。
\end{userbox}

\subsection{PADtools Converterの直接呼び出し}

PADtoolsのGitHubリポジトリ\footnote{\texttt{https://github.com/knaou/padtools}}を調査した結果、\texttt{Converter}クラスが変換ロジックを持っていることが判明した。

\begin{lstlisting}
Converter.convert(File file_in, File file_out, Double scale)
\end{lstlisting}

処理フロー:
\begin{enumerate}
\item SPDParser.parse() → PADModel
\item Model2View.toView() → View
\item View2Image.toImage(view, scale) → BufferedImage
\item ImageIO.write() → PNG出力
\end{enumerate}

問題は、Main.javaがGUIスレッドを先に起動してしまうことだった。解決策として、Converterクラスを直接呼び出すラッパーを作成した。

\begin{lstlisting}[language=java,caption=tools/PadCLI.java]
import padtools.converter.Converter;
import java.io.File;

public class PadCLI {
    public static void main(String[] args) {
        if (args.length < 2) {
            System.err.println("Usage: PadCLI <input.spd> <output.png> [scale]");
            System.exit(1);
        }
        File input = new File(args[0]);
        File output = new File(args[1]);
        Double scale = args.length >= 3 ? Double.parseDouble(args[2]) : 2.0;

        Converter converter = new Converter();
        converter.convert(input, output, scale);
    }
}
\end{lstlisting}

\begin{lstlisting}[language=bash]
# コンパイルと実行
javac -cp PadTools.jar PadCLI.java
java -Djava.awt.headless=true -cp PadTools.jar:libs/*:. PadCLI \
    docs/pad/workflow-basic.spd /tmp/test-pad.png 2.0
\end{lstlisting}

PNG生成に成功した(822 x 1294 pixels)。

\subsection{spd2pngスクリプトの作成}

便利なシェルスクリプトラッパーが作成された。

\begin{lstlisting}[language=bash,caption=bin/spd2png]
#!/bin/bash
PADTOOLS_HOME="${PADTOOLS_HOME:-/Users/mashi/local/src/padtools1.4}"
SCRIPT_DIR="$(cd "$(dirname "$0")/.." && pwd)"

CLASSPATH="$PADTOOLS_HOME/PadTools.jar"
for jar in "$PADTOOLS_HOME/libs/"*.jar; do
    CLASSPATH="$CLASSPATH:$jar"
done
CLASSPATH="$CLASSPATH:$SCRIPT_DIR/tools"

java -Djava.awt.headless=true -cp "$CLASSPATH" PadCLI "$@"
\end{lstlisting}

全てのSPDファイルがPNGに変換された:
\begin{itemize}
\item workflow-basic.png(111KB, 822x1294px)
\item workflow-main.png(117KB)
\item workflow-advanced.png(210KB)
\end{itemize}

\begin{userbox}
ですね。
\end{userbox}

コミットが実行され、作業が完了した。

\begin{lstlisting}
a6d01cc Add PADtools CLI wrapper and generate PAD diagram PNGs
\end{lstlisting}

\subsection{この日の成果}

\begin{enumerate}
\item CLAUDE.mdの未実装タスクをすべて完了
\item 配管ツール(yt-srt, video-trim, video-chapters)を作成
\item プロンプト例と環境構築ガイドを作成
\item PADtoolsのCLI問題を解決し、spd2pngツールを作成
\item 全てのPAD図をPNGとして出力
\end{enumerate}

PADtoolsをフォークせずに、既存のConverterクラスを直接呼び出すラッパーで問題が解決された。これは、既存のライブラリを活用しながら必要な機能だけを追加するという、配管と陶器の設計思想に沿った解決策であった。

