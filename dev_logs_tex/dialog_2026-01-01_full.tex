%% dialog_2026-01-01_full.tex
%% 2026年1月1日の開発対話記録(6メッセージ)

\section{2026年1月1日:iPad版への展望}

新年最初のセッションは、video-chapter-editorのiPad対応可能性についての技術的検討が行われた短い対話となった。

\subsection{iPad移植の技術的課題}

\begin{userbox}
このアプリ、iPad用にリリースするとなると大変ですかね。
\end{userbox}

\begin{assistantbox}
かなり大変です。現在の構成ではiOSへの直接移植は困難です。
\end{assistantbox}

主な技術的障壁が整理された:

\vspace{0.5\baselineskip}
\noindent{\footnotesize
\begin{tabularx}{\linewidth}{@{}lXX@{}}
\toprule
要素 & 現状 & iPad移植 \\
\midrule
GUI & PySide6 (Qt) & iOS非対応 \\
言語 & Python & iOS公式サポートなし \\
動画処理 & ffmpeg (CLI) & 要ネイティブビルド \\
ファイルアクセス & 自由 & サンドボックス制限 \\
\bottomrule
\end{tabularx}
}
\vspace{0.5\baselineskip}

\subsection{移植アプローチの検討}

4つの選択肢が提示された:

\begin{enumerate}
\item \textbf{ネイティブ書き直し}(最も現実的)
  \begin{itemize}
  \item Swift + SwiftUI
  \item AVFoundation(動画処理)
  \item 工数: 大
  \end{itemize}

\item \textbf{クロスプラットフォーム}
  \begin{itemize}
  \item Flutter / React Native
  \item Kotlin Multiplatform
  \item 工数: 中〜大
  \end{itemize}

\item \textbf{Web + サーバー構成}
  \begin{itemize}
  \item フロントエンド: Web (PWA)
  \item バックエンド: 現在のPython処理をAPI化
  \item iPad: Safari/WebViewで利用
  \item 工数: 中
  \end{itemize}

\item \textbf{Catalyst}
  \begin{itemize}
  \item macOS版を先に作成 → iPad互換
  \item Qt/PySide6では難しい
  \end{itemize}
\end{enumerate}

\subsection{Tauri経由のSwift移行案}

\begin{userbox}
Tauri版を作成して、Swiftに向かうのはいかがでしょう。
\end{userbox}

\begin{assistantbox}
良いアプローチです。Tauri 2.0はiOS/Androidサポートがあります。
\end{assistantbox}

段階的な移行パスが提案された:

\begin{lstlisting}[basicstyle=\ttfamily\small]
現在 (PySide6)
    ↓
Tauri 2.0 (Rust + Web)
    ↓
├── デスクトップ: そのまま利用
├── iPad: Tauri iOS (実験的)
└── 必要なら: Swift書き直し (知見を活用)
\end{lstlisting}

Tauriのメリット:

\vspace{0.5\baselineskip}
\noindent{\footnotesize
\begin{tabularx}{\linewidth}{@{}lX@{}}
\toprule
項目 & 内容 \\
\midrule
コード共有 & デスクトップ/モバイルで同一フロントエンド \\
パフォーマンス & Rustバックエンドは高速 \\
バイナリサイズ & Electronより小さい(約10MB) \\
学習効果 & Rust → Swift移行時に役立つ \\
\bottomrule
\end{tabularx}
}
\vspace{0.5\baselineskip}

\begin{userbox}
ちょっと考えますね。
\end{userbox}

\begin{assistantbox}
了解です。検討事項が多いですからね。
\end{assistantbox}

\subsection{この日の要点}

2026年最初のセッションでは、具体的な開発作業ではなく、アプリの将来的な展開について検討が行われた。PySide6/Pythonベースの現在のアーキテクチャからiPad対応への道筋として、Tauri 2.0を中間ステップとし、必要に応じてSwiftへ移行するという段階的アプローチが有力な選択肢として浮上した。

