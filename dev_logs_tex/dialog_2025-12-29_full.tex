%% dialog_2025-12-29_full.tex
%% 2025年12月29日の開発対話記録(1739メッセージ)

\section{2025年12月29日:UI大改造とui\_nextパッケージの実装}

この日は、1739メッセージに及ぶ過去最大規模のセッションとなった。タブベースUIから単一ワークスペース+ダイアログパターンへの移行、開発ログの整理、配管と陶器の分界点マトリクス作成、そしてui\_nextパッケージの完全実装が行われた。

\subsection{タブ構成の再検討}

\begin{userbox}
そうなるとタブ1とタブ2を分けなくても良いような気がしますが、いかがでしょう。まあ、陶器が巨大になるんですけど。笑
\end{userbox}

\begin{assistantbox}
その通りですね。現在の機能を整理すると:

\textbf{タブ1(MP3結合)の実体:}
\begin{itemize}
\item 複数MP3 → 結合MP3(\texttt{-c copy}で無劣化)
\item チャプターファイル自動生成
\end{itemize}

\textbf{タブ2(編集・書出)の実体:}
\begin{itemize}
\item MP3/MP4読込
\item カバー画像設定
\item 波形表示・チャプター編集
\item 書出
\end{itemize}

タブ1の出力がそのままタブ2の入力になるため、分ける必然性は薄いですね。
\end{assistantbox}

\begin{userbox}
1の入力ソースも別画面ですかね。その際に、チャプターを自動挿入、2のカバー画像もユーザーが明示的に保存しなくても自動挿入という感じで、統一感が出ますね。
\end{userbox}

\begin{assistantbox}
良いですね。以下の構成を提案します:

\textbf{メインワークスペース(1画面)}
\begin{itemize}
\item 動画/波形プレビュー
\item チャプターリスト
\item 書出ボタン
\end{itemize}

\textbf{モーダルダイアログ}
\begin{enumerate}
\item \textbf{SourceSelectionDialog}: MP3複数選択 or 単一MP3/MP4選択 → 自動でチャプター挿入
\item \textbf{CoverImageDialog}: 画像選択・クロップ → 自動で状態に反映
\end{enumerate}

ユーザーは「開く」「カバー設定」ボタンを押すだけで、バックグラウンドで処理が完了します。
\end{assistantbox}

\subsection{開発ログの再編成}

\begin{userbox}
DEVELOPMENT\_LOG\_as\_of\_2025-12-29.mdとして、一連のログを集約して書き出してください。今回の大改造以降のログをDEVELOPMENT\_LOG.mdとして新たに作成してください。
\end{userbox}

過去のログがアーカイブされ、新しい開発ログが作成された。

\begin{userbox}
昨日のグラフ表現に関するディスカッションを追加して、トレーサビリティを確保してください。
\end{userbox}

一筆書き問題(オイラー路)としてワークフローを分析した議論が追加された:

\begin{lstlisting}[basicstyle=\ttfamily\tiny]
起点1 --> 結合 --> カバー --+
起点2 ---------> カバー ----+--> チャプター編集 -> 書出
起点3 ---------------------+
\end{lstlisting}

\begin{userbox}
ダイアグラムの可視性向上のために、Mermaid記法にて記述してください。
\end{userbox}

Mermaid記法でフロー図が追加された:

\begin{lstlisting}[basicstyle=\ttfamily\tiny]
graph LR
    A[複数MP3] --> B[MP3結合]
    B --> C[カバー画像設定]
    D[単一MP3] --> C
    E[MP4] --> F[チャプター編集]
    C --> F
    F --> G[書出]
\end{lstlisting}

\begin{userbox}
この一連のディスカッションをPADで可視化する利点はありますか。
\end{userbox}

\begin{assistantbox}
PADとMermaidの使い分けを整理すると:

\vspace{0.5\baselineskip}
\noindent{\footnotesize
\begin{tabularx}{\linewidth}{@{}lXX@{}}
\toprule
 & PAD & Mermaid \\
\midrule
用途 & 実装詳細・制御フロー & 設計概要・データフロー \\
強み & 条件分岐・ループの表現 & 状態遷移・依存関係 \\
ツール & 専用エディタ必要 & テキストのみ \\
\bottomrule
\end{tabularx}
}
\vspace{0.5\baselineskip}

現段階では設計議論が中心なので、Mermaidが適切です。実装に入ったらPADで詳細化するのが良いでしょう。
\end{assistantbox}

\subsection{ユースケースの詳細化}

\begin{userbox}
UC3に、編集済みと要編集の派生パターンがありますね。
\end{userbox}

ユースケースが詳細化された:

\vspace{0.5\baselineskip}
\noindent{\footnotesize
\begin{tabularx}{\linewidth}{@{}lX@{}}
\toprule
ユースケース & 説明 \\
\midrule
UC1 & 複数MP3(カット済み)→ 結合 → カバー → 書出 \\
UC2 & 単一MP3(長尺)→ カバー → チャプター編集 → 書出 \\
UC3a & MP4(編集済み)→ そのまま書出 \\
UC3b & MP4(要編集)→ チャプター編集 → 書出 \\
UC4 & 複数MP3 → 結合 → カバー → チャプター編集 → 書出 \\
\bottomrule
\end{tabularx}
}
\vspace{0.5\baselineskip}

\begin{userbox}
LOGを時系列逆にしましょうか。
\end{userbox}

開発ログが逆時系列順(最新が上)に変更された。

\subsection{配管と陶器の分界点マトリクス}

\begin{userbox}
「ユースケース詳細マトリクス」が、配管と陶器の分界点を可視化する重要な表になりそうです。
\end{userbox}

\begin{userbox}
ツールのサブコマンドも入れてしまいましょう。
\end{userbox}

\begin{userbox}
配管部分の操作をUI操作とバックヤードのサブコマンドに分けましょうか。
\end{userbox}

詳細なマトリクスが作成された:

\vspace{0.5\baselineskip}
\noindent{\footnotesize
\begin{tabularx}{\linewidth}{@{}lXXX@{}}
\toprule
ステップ & ユーザー操作 & プラットフォーム & 要実装 \\
\midrule
ソース選択 & ファイルD\&D & QFileDialog & --- \\
MP3結合 & 結合ボタン & --- & ffmpeg -c copy \\
カバー設定 & ダイアログ & QDialog, QImage & ImageCropWidget \\
波形表示 & --- & --- & WaveformWorker \\
書出 & 書出ボタン & --- & ExportWorker \\
\bottomrule
\end{tabularx}
}
\vspace{0.5\baselineskip}

\begin{userbox}
バックエンドをOS、PySide6などの機能と、開発者側で実装しないといけないことを弁別してください。
\end{userbox}

\begin{userbox}
yt-dlpやwhisper-remoteは、開発者サイドで実装したので、要実装に移動しましょうか。
\end{userbox}

\begin{userbox}
luatex-pdfも自作ツールなんですよね。
\end{userbox}

\begin{userbox}
我ながら、色々と作ってますね。
\end{userbox}

自作ツール群が明確化された:

\vspace{0.5\baselineskip}
\noindent{\footnotesize
\begin{tabularx}{\linewidth}{@{}lX@{}}
\toprule
ツール & 機能 \\
\midrule
ytdl & Claude CLIでファイル名自動生成+yt-dlp \\
whisper-remote & リモートサーバーでのWhisper文字起こし \\
luatex-pdf & リモートLuaLaTeXコンパイル \\
yt-srt & YouTube字幕取得 \\
\bottomrule
\end{tabularx}
}
\vspace{0.5\baselineskip}

\subsection{UI設計の課題と対応}

\begin{userbox}
現状を踏まえて懸念点を
\end{userbox}

\begin{assistantbox}
主な懸念点:

\begin{enumerate}
\item \textbf{UIの複雑化}: 単一ワークスペースが巨大になる可能性
\item \textbf{状態管理}: プロジェクト状態の一貫性維持
\item \textbf{ログ管理}: デバッグ情報の可視化
\item \textbf{エラーハンドリング}: ユーザーへの適切なフィードバック
\end{enumerate}

優先度として、2(UI再設計)から着手し、その後1(ロギング)を実装することを提案します。
\end{assistantbox}

\begin{userbox}
2からですかね。1のUIは、スケルトンだけ作成してください。エラー通知はUIに、コマンドラインのログ表示機能を作成して...
\end{userbox}

\subsection{ui\_nextパッケージの作成}

新しいUIパッケージ構造が作成された:

\begin{lstlisting}[basicstyle=\ttfamily\tiny]
rehearsal_workflow/ui_next/
+-- __init__.py
+-- __main__.py
+-- app.py              # VideoChapterEditorNext
+-- main_workspace.py   # MainWorkspace
+-- dialogs.py          # SourceSelectionDialog, CoverImageDialog
+-- log_panel.py        # LogPanel with LogLevel
+-- models.py           # ChapterInfo, ColorspaceInfo
+-- workers.py          # MergeWorker, ExportWorker, WaveformWorker
+-- image_crop.py       # ImageCropWidget
\end{lstlisting}

\begin{userbox}
どうやって起動するんでしたっけ
\end{userbox}

\begin{assistantbox}
モジュールとして実行します:

\begin{lstlisting}[language=bash,basicstyle=\ttfamily\tiny]
python -m rehearsal_workflow.ui_next
\end{lstlisting}

\texttt{\_\_main\_\_.py}があるため、パッケージを直接実行できます。
\end{assistantbox}

\subsection{動画プレビュー機能の実装}

\begin{userbox}
python -m rehearsal\_workflow.ui\_nextを確認しました。この状態で右側に、動画読み込みUIを追加してみてください。
\end{userbox}

QMediaPlayer、QVideoWidget、QAudioOutputを使用した動画プレビュー機能が実装された:

\begin{lstlisting}[language=python,basicstyle=\ttfamily\tiny]
from PySide6.QtMultimedia import QMediaPlayer, QAudioOutput
from PySide6.QtMultimediaWidgets import QVideoWidget

self.media_player = QMediaPlayer()
self.audio_output = QAudioOutput()
self.video_widget = QVideoWidget()
self.media_player.setAudioOutput(self.audio_output)
self.media_player.setVideoOutput(self.video_widget)
\end{lstlisting}

\subsection{ウィンドウサイズとクロスプラットフォーム対応}

\begin{userbox}
GUIのサイズを1440×900固定で、macおよびWindowsで問題なく表示を行うことができるようにデザインしましょう。
\end{userbox}

クロスプラットフォームのフォント検出機能が実装された:

\begin{lstlisting}[language=python,basicstyle=\ttfamily\tiny]
def get_system_font() -> str:
    system = platform.system()
    if system == "Darwin":  # macOS
        return "Helvetica Neue"
    elif system == "Windows":
        return "Segoe UI"
    else:  # Linux
        return "Noto Sans"

def get_monospace_font() -> str:
    system = platform.system()
    if system == "Darwin":
        return "Menlo"
    elif system == "Windows":
        return "Consolas"
    else:
        return "Noto Sans Mono"
\end{lstlisting}

当初「SF Pro Text」を使用していたが、Qtで警告が出たためHelvetica Neue/Menloに変更された。

\subsection{レイアウトの再構成}

\begin{userbox}
select sourceの下にあるwaveform displayはVideo Previewの下におきたいですね。Chaptersはこの位置にしましょう。Video Previewは、いっぱいに広げてください。
\end{userbox}

\begin{userbox}
操作系は、全て左に寄せましょうか。操作の順に左側に移動してください。
\end{userbox}

\begin{userbox}
Cover imageからExportも左に移動してください。
\end{userbox}

\begin{userbox}
ログも左に寄せましょうかね。
\end{userbox}

最終的なレイアウト構成:

\vspace{0.5\baselineskip}
\noindent{\footnotesize
\begin{tabularx}{\linewidth}{@{}lX@{}}
\toprule
位置 & コンポーネント \\
\midrule
左パネル(480px固定) & ソース選択、カバー画像、チャプター、書出、ログ \\
右パネル(残り) & 動画プレビュー、波形表示、再生コントロール \\
\bottomrule
\end{tabularx}
}
\vspace{0.5\baselineskip}

\begin{userbox}
左側の幅を1:3くらいで調整してみてください。
\end{userbox}

\begin{userbox}
半々くらいになってますね。
\end{userbox}

QSplitterの\texttt{setSizes()}が意図通りに動作しなかったため、\texttt{setFixedWidth(480)}で固定幅を設定する方式に変更された。

\begin{userbox}
全体のサイズを1440×1080にしましょう
\end{userbox}

\begin{userbox}
元の方が良いですね。
\end{userbox}

1440×1080を試したが、ユーザーの好みにより1440×900に戻された。

\subsection{作業ディレクトリのサポート}

\begin{userbox}
起動時にフォルダパスを引数として作業ディレクトリとするようにしてください。
\end{userbox}

\begin{lstlisting}[language=python,basicstyle=\ttfamily\tiny]
def main():
    parser = argparse.ArgumentParser()
    parser.add_argument("work_dir", nargs="?", default=None)
    args = parser.parse_args()

    app = QApplication(sys.argv)
    window = VideoChapterEditorNext(work_dir=args.work_dir)
    window.show()
    sys.exit(app.exec())
\end{lstlisting}

ダイアログも作業ディレクトリを初期ディレクトリとして使用するよう更新された。

\subsection{波形表示のバックグラウンド処理}

\begin{userbox}
波形表示、再生コントロールなどを同様にUIロックを行わないように別スレッドで実装してください。
\end{userbox}

WaveformWorkerクラスがQThreadベースで実装された:

\begin{lstlisting}[language=python,basicstyle=\ttfamily\tiny]
class WaveformWorker(QObject):
    finished = Signal(np.ndarray, int)
    error = Signal(str)

    def __init__(self, file_path: str):
        super().__init__()
        self.file_path = file_path

    def run(self):
        try:
            # ffmpegで音声をPCMに変換
            cmd = [
                "ffmpeg", "-i", self.file_path,
                "-vn", "-ac", "1", "-ar", "8000",
                "-f", "s16le", "-"
            ]
            result = subprocess.run(cmd, capture_output=True)
            samples = np.frombuffer(result.stdout, dtype=np.int16)
            self.finished.emit(samples.astype(np.float32), 8000)
        except Exception as e:
            self.error.emit(str(e))
\end{lstlisting}

\begin{userbox}
\textasciitilde/Dropbox/01\_Projects/00\_Horn\_Works/20260125\_レオケ/2025-12-21を引数にして起動してください。
\end{userbox}

実際の作業ディレクトリでテストが行われた。

\subsection{波形表示の改善}

\begin{userbox}
波形表示のノーマライズを行いましょうか。
\end{userbox}

波形データの正規化が実装された:

\begin{lstlisting}[language=python,basicstyle=\ttfamily\tiny]
max_val = np.max(np.abs(samples))
if max_val > 0:
    samples = samples / max_val
\end{lstlisting}

\begin{userbox}
あと、波形表示の画面解像度に応じた間引き処理も入れましょう。min-max法で良いかと思いますが、いかがでしょうか。
\end{userbox}

\begin{assistantbox}
min-max decimationは最適な選択ですね。ピークを保持しながらデータ量を削減できます。

\begin{lstlisting}[language=python,basicstyle=\ttfamily\tiny]
def decimate_minmax(samples, target_points):
    chunk_size = len(samples) // target_points
    decimated = []
    for i in range(target_points):
        chunk = samples[i*chunk_size:(i+1)*chunk_size]
        decimated.append(np.min(chunk))
        decimated.append(np.max(chunk))
    return np.array(decimated)
\end{lstlisting}
\end{assistantbox}

サンプル数は4000に設定され、min-max法により解像度に応じた表示が実現された。

\subsection{video\_chapter\_editor.pyからの機能移植}

\begin{userbox}
video\_chapter\_editor.pyに実装したものを全て実装してみてください。
\end{userbox}

以下の機能がui\_nextパッケージに移植された:

\textbf{models.py}
\begin{itemize}
\item ChapterInfo(チャプター情報データクラス)
\item ColorspaceInfo(色空間情報データクラス)
\item detect\_gpu\_encoder()(GPU検出)
\item detect\_video\_colorspace()(色空間検出)
\end{itemize}

\textbf{workers.py}
\begin{itemize}
\item MergeWorker(MP3結合)
\item ExportWorker(書出、チャプター埋込、タイトル焼込)
\item WaveformWorker(波形生成)
\end{itemize}

\textbf{image\_crop.py}
\begin{itemize}
\item ImageCropWidget(16:9カバー画像クロップ)
\end{itemize}

\textbf{main\_workspace.py}
\begin{itemize}
\item チャプターテーブルの可視化
\item YouTube用チャプターコピー
\item 書出オプション(GPU/CPU、ビットレート)
\item 除外チャプター(--プレフィックス)の波形ハッチング
\end{itemize}

\subsection{発生したエラーと解決}

\vspace{0.5\baselineskip}
\noindent{\footnotesize
\begin{tabularx}{\linewidth}{@{}lX@{}}
\toprule
エラー & 解決策 \\
\midrule
ImportError(相対インポート) & \_\_main\_\_.pyを作成し\texttt{python -m}で実行 \\
QFontDatabase.isFixedPitch() TypeError & 文字列パラメータを使用 \\
SF Pro Text警告 & Helvetica Neue/Menloに変更 \\
Splitter比率が効かない & setFixedWidth(480)を使用 \\
\bottomrule
\end{tabularx}
}
\vspace{0.5\baselineskip}

\subsection{この日の成果}

\begin{enumerate}
\item \textbf{UI設計の大転換} - タブベースから単一ワークスペース+ダイアログへ
\item \textbf{開発ログの再編成} - アーカイブ化と逆時系列順
\item \textbf{配管と陶器の分界点マトリクス} - 責務の明確化
\item \textbf{ui\_nextパッケージ} - 完全なモジュール構成
\item \textbf{クロスプラットフォーム対応} - フォント検出、1440×900固定
\item \textbf{バックグラウンド処理} - WaveformWorkerによるUIロック回避
\item \textbf{min-max decimation} - 解像度適応型波形表示
\item \textbf{video\_chapter\_editor.py機能の完全移植}
\end{enumerate}

\subsection{技術的なポイント}

\begin{enumerate}
\item \textbf{QSplitterの制限}: \texttt{setSizes()}は比率保証がないため、\texttt{setFixedWidth()}で明示的に幅を指定
\item \textbf{\_\_main\_\_.pyの役割}: パッケージを\texttt{python -m}で実行可能にする
\item \textbf{QObject.moveToThread()}: QThreadと組み合わせてバックグラウンド処理を実現
\item \textbf{min-max decimation}: ピークを保持しつつデータ量を削減する間引き手法
\item \textbf{オイラー路}: ワークフローをグラフ理論で分析し、共通パスを特定
\end{enumerate}

この日は1739メッセージという過去最大規模のセッションとなり、UI設計の根本的な見直しから実装の詳細に至るまで、包括的な作業が行われた。

