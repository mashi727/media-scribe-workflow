%% dialog_2025-12-30_full.tex
%% 2025年12月30日の開発対話記録(1063メッセージ)

\section{2025年12月30日:ui\_nextパッケージの大規模UI改善}

この日は、ui\_nextパッケージに対する大規模な改善作業が行われた。1063メッセージに及ぶ長時間のセッションで、スペクトログラム表示の改善、エンコード設定の統一、再生コントロールのリデザイン、ウィンドウリサイズ機能など、多岐にわたる機能が実装された。

\subsection{スペクトログラム表示の改善}

\begin{userbox}
スペクトログラムの色をSOXのデフォルトと同じ色にしてみてください。また、スペクトログラムは、波形の処理の後に自動的に開始するようにしてください。
\end{userbox}

\begin{assistantbox}
SOXのデフォルトカラーマップ(黒→青→シアン→緑→黄→赤→白)を実装し、波形完了後に自動生成を開始します。
\end{assistantbox}

\begin{userbox}
スペクトログラムの色を、青系にしてください。再生ヘッダの位置がみづらいので。
\end{userbox}

再生ヘッドの視認性を考慮し、青系のカラーマップ(黒→濃紺→青→シアン→白)に変更された。

\subsection{メルスペクトログラムの実装}

\begin{userbox}
演奏と指揮者のトークが区別しやすいようなスペクトログラム処理を行なってみてください。
\end{userbox}

\begin{assistantbox}
演奏とトークを区別しやすくするため、以下の処理を実装します:

\begin{enumerate}
\item \textbf{メルスケール変換} --- 人間の聴覚特性に合わせた周波数スケール(低域を拡大、高域を圧縮)
\item \textbf{スペクトルコントラスト強調} --- 音声と音楽の違いを強調
\end{enumerate}
\end{assistantbox}

SpectrogramWorkerにメルスケール対応が実装された:

\begin{lstlisting}[language=python,basicstyle=\ttfamily\tiny]
def _hz_to_mel(self, hz, np):
    return 2595 * np.log10(1 + hz / 700)

def _mel_to_hz(self, mel, np):
    return 700 * (10 ** (mel / 2595) - 1)

def _create_mel_filterbank(self, n_fft: int, sample_rate: int, n_mels: int, np):
    # 三角メルフィルターバンクを作成
    ...
\end{lstlisting}

視覚的な違いとして、トークは低周波に集中したエネルギー分布と水平方向の縞模様(フォルマント)が現れ、演奏は広い周波数帯域に分布する複雑なパターンが現れる。

\subsection{再生ヘッドと波形表示の改善}

\begin{userbox}
波形上の再生ヘッド位置を黄色でもう少し太くできますかね。
\end{userbox}

再生ヘッドが黄色(\texttt{\#FFEB3B})で3ピクセル幅に変更された。

\begin{userbox}
振幅計算後一瞬だけ波形が表示されて、一旦消えます。消えないようにスペクトログラム計算中も振幅波形の表示を維持するようにしてください。
\end{userbox}

\texttt{paintEvent}が修正され、スペクトログラム計算中も波形を表示したまま、下部に進捗オーバーレイを表示するように変更された。

\subsection{MP3→MP4エンコード設定の統一}

ユーザーは既存の\texttt{mp3tomp4} zsh関数を共有し、現在の実装との比較を要求した:

\begin{userbox}
元々は、[mp3tomp4 zsh関数]でエンコードしていました。違いはありますか
\end{userbox}

\begin{assistantbox}
オリジナルの \texttt{mp3tomp4} zsh関数と現在の ui\_next 実装の違いを比較します:

\vspace{0.5\baselineskip}
\noindent{\footnotesize
\begin{tabularx}{\linewidth}{@{}lXX@{}}
\toprule
項目 & オリジナル & 現在の実装 \\
\midrule
黒画像サイズ & 1920x1080 & 1280x720 \\
フォントサイズ & 56(固定) & h*0.054(相対) \\
CRF & 32 & 23(デフォルト) \\
-threads & 0(全コア使用) & 指定なし \\
\bottomrule
\end{tabularx}
}
\vspace{0.5\baselineskip}
\end{assistantbox}

\begin{userbox}
静止画なので、CRF32で十分ですね。黒画面の解像度は1280x720で、-threads 0にしましょうか。
\end{userbox}

以下の変更が実装された:

\begin{lstlisting}[language=python,basicstyle=\ttfamily\tiny]
# models.py - libx264エンコーダに-threads 0を追加
return [
    '-c:v', 'libx264',
    '-preset', 'ultrafast',
    '-crf', str(crf),
    '-threads', '0',  # 全CPUコアを使用
    '-pix_fmt', 'yuv420p',
]

# workers.py - 音声のみエクスポートでCRF 32を使用
encoder_args = get_encoder_args(self.encoder_id, self.bitrate_kbps, crf=32)
\end{lstlisting}

\subsection{品質ドロップダウンの変更}

\begin{userbox}
Qualityで、元と同じは変なので静止画用と表示してmp3からのエンコードにおいては変更できないようにしましょう。
\end{userbox}

品質オプションが動画用と音声用に分離された:

\begin{lstlisting}[language=python,basicstyle=\ttfamily\tiny]
self._video_quality_options = [
    ("元と同じ (自動)", 0, 23),
    ("高画質 (6Mbps)", 6000, 20),
    ("標準 (4Mbps)", 4000, 23),
    ("軽量 (2Mbps)", 2000, 28),
    ("最小 (1Mbps)", 1000, 32),
]
self._audio_quality_options = [
    ("静止画用 (CRF 32)", 500, 32),
]
\end{lstlisting}

音声ファイル読み込み時は品質コンボボックスが無効化され、「静止画用 (CRF 32)」のみが表示される。

\subsection{エクスポート後の自動読み込み}

\begin{userbox}
export後に、出力した動画をメイン画面に表示するようにしましょうか。
\end{userbox}

エクスポート完了後に出力動画を自動的に読み込む機能が実装された。これにより、すぐに出力結果を確認できる。

\subsection{プログレスバーのステータスバー統合}

\begin{userbox}
Readyのところに進捗が表示されます。プログレスバーもステータスバーに移動してマージして良い感じの進捗表示にしてください。
\end{userbox}

プログレスバーがステータスバーに統合され、以下のような表示となった:

\vspace{0.5\baselineskip}
\noindent{\footnotesize
\begin{tabularx}{\linewidth}{@{}lXXX@{}}
\toprule
状態 & プログレスバー & テキスト & 色 \\
\midrule
待機中 & 非表示 & Ready & 緑 \\
エクスポート中 & 表示・更新 & 00:01:23 45\% & 青 \\
完了 & 100\%表示 & Completed: file.mp4 & 緑 \\
エラー & 非表示 & Failed: error message & 赤 \\
\bottomrule
\end{tabularx}
}
\vspace{0.5\baselineskip}

\subsection{FFmpegエラーの修正}

\begin{userbox}
mp3からエンコードしたmp4を再エンコードするとエラーになりますね。

Output same as Input - exiting
\end{userbox}

エクスポート後に出力動画を読み込むため、再エクスポート時に入出力ファイルが同じになる問題が発生した。出力ファイル名に自動で連番サフィックスを付ける対策が実装された:

\begin{lstlisting}[language=python,basicstyle=\ttfamily\tiny]
if input_path.resolve() == output_path.resolve():
    counter = 2
    while True:
        output_path = self._state.work_dir / f"{Path(output_base).name}_chaptered_{counter}.mp4"
        if not output_path.exists():
            break
        counter += 1
\end{lstlisting}

\subsection{編集エラーの修正}

\begin{userbox}
edit: editing failed

チャプター名に--を付してエンターした時にでるエラーでした。
\end{userbox}

編集中にEnterを押すと、\texttt{edit()}が再度呼ばれて失敗する問題が発見された。編集状態をチェックして分岐する修正が実装された:

\begin{lstlisting}[language=python,basicstyle=\ttfamily\tiny]
def eventFilter(self, obj, event):
    if obj == self._table:
        if event.type() == QEvent.Type.KeyPress:
            if event.key() in (Qt.Key.Key_Return, Qt.Key.Key_Enter):
                # 編集中の場合はデフォルト処理(編集確定)に任せる
                if self._table.state() == QAbstractItemView.State.EditingState:
                    return False  # デフォルト処理に委譲
                # 編集中でなければ編集開始
                ...
\end{lstlisting}

\subsection{焼き込みテキストの二重化問題}

\begin{userbox}
2回目のエンコードの際に、チャプターを追加すると、チャプター名が2重になるなと。良い方法はありませんよね。
\end{userbox}

焼き込み済みMP4を再エクスポートすると、元の焼き込みに新しい焼き込みが追加されて二重表示になる問題が議論された。これは焼き込み(drawtext)の本質的な制限であり、ユーザーの操作の自由度を優先して自動読み込み機能は維持されることとなった。

\subsection{ウィンドウとダイアログのリサイズ対応}

\begin{userbox}
アプリケーションの画面サイズを拡大縮小できるようにしましょうか。比率は維持したままでやってみましょう。
\end{userbox}

\begin{userbox}
ダイアログも同様に同じ割合で拡大縮小できるように実装してください。
\end{userbox}

アスペクト比を維持したリサイズ機能が実装された:

\begin{lstlisting}[language=python,basicstyle=\ttfamily\tiny]
def resizeEvent(self, event):
    if self._resizing:
        return super().resizeEvent(event)
    self._resizing = True
    # アスペクト比計算
    self._resizing = False
    super().resizeEvent(event)
\end{lstlisting}

\vspace{0.5\baselineskip}
\noindent{\footnotesize
\begin{tabularx}{\linewidth}{@{}lXXX@{}}
\toprule
対象 & デフォルトサイズ & 最小サイズ & アスペクト比 \\
\midrule
メインウィンドウ & 1680 x 1050 & 1120 x 700 & 8:5 \\
SourceSelectionDialog & 800 x 600 & 600 x 450 & 4:3 \\
CoverImageDialog & 1344 x 840 & 896 x 560 & 8:5 \\
\bottomrule
\end{tabularx}
}
\vspace{0.5\baselineskip}

\subsection{movie-viewerスタイルの再生コントロール}

\begin{userbox}
右側の再生などのコントロールをvideo-chapter-editorと同様にしてください。また、コントロールのウィジェットの位置は波形の中心に揃えてください。
\end{userbox}

movie-viewerのUIを参考に、以下のような配置に変更された:

\begin{verbatim}
-10s  -1s  -.3s  -1f  [▶]  +1f  +.3s  +1s  +10s
\end{verbatim}

相対シーク用のメソッドが追加された:

\begin{lstlisting}[language=python,basicstyle=\ttfamily\tiny]
def _seek_relative(self, delta_ms: int):
    """現在位置から相対的にシーク"""
    if not self._media_player:
        return
    current = self._media_player.position()
    duration = self._media_player.duration()
    new_pos = max(0, min(duration, current + delta_ms))
    self._media_player.setPosition(new_pos)
\end{lstlisting}

\subsection{スペクトログラムカラーマップの探求}

この日の後半は、スペクトログラムのカラーマップについて多くの試行が行われた:

\begin{enumerate}
\item \textbf{SOXデフォルト}: 黒→紫→青→シアン→緑→黄→赤→白
\item \textbf{JET}: 濃紺→青→シアン→黄→赤→濃い赤
\item \textbf{Inferno}: 黒→紫→マゼンタ→オレンジ→黄→白
\item \textbf{青系}: 黒→青(シンプル)
\item \textbf{最終版}: 黒→青→シアン(黄緑寄り)
\end{enumerate}

\begin{userbox}
くらいのを青、明るいのをシアンという感じにするとどうなりますか
\end{userbox}

\begin{userbox}
良いですね。これでシアンを黄緑に少し寄せてみましょうか。
\end{userbox}

最終的に、以下のカラーマップが採用された:

\begin{lstlisting}[language=python,basicstyle=\ttfamily\tiny]
# カラーマップ: 黒→青→シアン(黄緑寄り)
# 0~50%: 黒→青 (RGB 0,0,0 → 0,0,255)
# 50~100%: 青→黄緑寄りシアン (RGB 0,0,255 → 80,255,120)
\end{lstlisting}

\subsection{UI微調整}

\begin{userbox}
Chapters(埋め込み)の、埋め込みを緑色にしましょうか。
\end{userbox}

QGroupBoxのタイトルはプレーンテキストのみ対応のため、カスタムQLabelに置き換えてリッチテキストで「埋め込み」部分を緑色で表示するように変更された。

\begin{userbox}
Spectrogramではなく、メルスペクトログラムでしたっけ。
\end{userbox}

ラベルが「Spectrogram」から「Mel Spectrogram」に変更され、コンボボックスの幅も140pxから160pxに拡張された。

\subsection{この日の成果}

\begin{enumerate}
\item \textbf{メルスペクトログラム}: 人間の聴覚特性に合わせた周波数表示
\item \textbf{エンコード設定統一}: CRF 32、-threads 0の追加
\item \textbf{品質ドロップダウン分離}: 動画用/音声用の選択肢分離
\item \textbf{自動読み込み}: エクスポート後の即時確認機能
\item \textbf{ステータスバー統合}: プログレスバーとメッセージの一体化
\item \textbf{FFmpegエラー対策}: 同名ファイルの自動リネーム
\item \textbf{編集エラー修正}: EditingStateチェックの追加
\item \textbf{リサイズ対応}: アスペクト比維持のウィンドウ・ダイアログリサイズ
\item \textbf{movie-viewerスタイル}: 中央配置の再生コントロール
\item \textbf{カラーマップ最適化}: 黒→青→シアン(黄緑寄り)の採用
\end{enumerate}

\subsection{技術的なポイント}

\begin{enumerate}
\item \textbf{メルスケール変換}: STFT出力にメルフィルターバンクを適用して人間の聴覚特性を反映
\item \textbf{QGroupBoxの制限}: タイトルはプレーンテキストのみ対応、リッチテキストにはQLabelを使用
\item \textbf{resizeEventの再帰防止}: フラグ(\texttt{\_resizing})を使用して無限再帰を回避
\item \textbf{EditingState}: QTableWidgetの編集状態を確認して適切なイベント処理を行う
\item \textbf{ガンマ補正}: スペクトログラムのコントラスト調整に\texttt{np.power(data, gamma)}を使用
\end{enumerate}

この日は非常に多くの改善が行われ、ui\_nextパッケージの完成度が大幅に向上した。特に、スペクトログラムカラーマップについては多数の試行が行われ、ユーザーの好みに合った表示が実現された。

