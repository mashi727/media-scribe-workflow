\documentclass[a4paper,10pt,twocolumn]{ltjsarticle}

% LuaLaTeX用フォント設定パッケージ
\usepackage{luatexja-fontspec}
\usepackage{amsmath,amssymb}
\usepackage{unicode-math}

% 欧文フォント設定 (Libertinus)
\setmainfont{Libertinus Serif}[
    BoldFont = {Libertinus Serif Bold},
    ItalicFont = {Libertinus Serif Italic},
    BoldItalicFont = {Libertinus Serif Bold Italic}
]
\setsansfont{Libertinus Sans}[
    BoldFont = {Libertinus Sans Bold},
    ItalicFont = {Libertinus Sans Italic}
]
\setmonofont{Libertinus Mono}

% 日本語フォント設定 (原ノ味フォント)
\setmainjfont{HaranoAjiMincho-Regular}[
    BoldFont = {HaranoAjiGothic-Medium},
    ItalicFont = {HaranoAjiMincho-Regular},
    BoldItalicFont = {HaranoAjiGothic-Bold}
]
\setsansjfont{HaranoAjiGothic-Regular}[
    BoldFont = {HaranoAjiGothic-Bold}
]
\setmonojfont{HaranoAjiGothic-Regular}

% 数式フォント設定
\setmathfont{Libertinus Math}

% パッケージ
\usepackage{geometry}
\geometry{margin=18mm}
\usepackage{fancyhdr}
\usepackage{lastpage}
\usepackage{ascmac}
\usepackage{booktabs}
\usepackage{array}
\usepackage{tabularx}
\usepackage{enumitem}
\usepackage{hyperref}
\usepackage{xcolor}
\usepackage{listings}

% JST日時
\newcommand{\generatedDate}{2026-01-07}
\newcommand{\generatedTime}{15:30}

% ヘッダー・フッター設定
\pagestyle{fancy}
\fancyhf{}
\fancyhead[R]{\small \generatedDate\ \generatedTime\ JST (\thepage/\pageref{LastPage})}
\renewcommand{\headrulewidth}{0.4pt}

% 最初のページはヘッダーなし
\fancypagestyle{plain}{
  \fancyhf{}
  \renewcommand{\headrulewidth}{0pt}
}

% コード表示設定
\lstset{
  basicstyle=\ttfamily\scriptsize,
  breaklines=true,
  frame=single,
  columns=fullflexible,
  keepspaces=true
}

\title{\textbf{Video Chapter Editor 開発記録}\\
\large 2025年12月23日 -- 2026年1月6日\\開発ログ全対話の詳細記録}
\author{執筆者: ましDialogue}
\date{}

\begin{document}
\maketitle
\thispagestyle{plain}

\section*{概要}

本文書は、Video Chapter Editor(VCE)の開発における、Claude AIとユーザー間の全対話ログを詳細に記録したものである。2025年12月23日から2026年1月6日までの15日間にわたり、約11,500件のメッセージ交換を通じて、リハーサルワークフロー支援ツールが構築された。

\vspace{0.5\baselineskip}
\noindent{\footnotesize
\begin{tabularx}{\linewidth}{@{}lX@{}}
\toprule
期間 & 2025-12-23 -- 2026-01-06(15日間) \\
\midrule
総メッセージ数 & 約11,500件 \\
最終バージョン & v2.1.25以降 \\
主要技術 & Python, PySide6, ffmpeg, Qt6 \\
\bottomrule
\end{tabularx}
}
\vspace{0.5\baselineskip}

\section{2025年12月23日:プロジェクト基盤構築}
\textbf{メッセージ数: 175}

\begin{itembox}[l]{問い}
CLAUDE.mdのタスクを実装してください。bin/yt-srt, bin/video-trim, bin/video-chaptersを作成してください。
\end{itembox}

この日は、rehearsal-workflowプロジェクトの基盤となるCLIツール群の実装が開始された。「Gitの陶器と配管」の設計思想に基づき、単一目的のツールを組み合わせるアーキテクチャが採用された。

\subsection{実装されたツール}

\begin{enumerate}[noitemsep]
\item \texttt{bin/yt-srt} -- YouTube字幕取得
\item \texttt{bin/video-trim} -- 動画トリミング
\item \texttt{bin/video-chapters} -- チャプター結合・埋め込み
\item \texttt{bin/spd2png} -- PADtools CLI PNG変換
\end{enumerate}

PADtoolsの調査も行われ、CLI経由でのPNG出力方法が確立された。\texttt{java -Djava.awt.headless=true}オプションの発見により、サーバー環境でもPAD図の生成が可能となった。

\section{2025年12月24日:PAD図レンダリング改善}
\textbf{メッセージ数: 658}

\begin{itembox}[l]{問い}
PadAlignedRendererのテキスト折り返しと列アラインメントを修正してください。
\end{itembox}

PADtoolsのJavaソースコード(\texttt{PadAlignedRenderer.java})への深い調査が行われた。

\subsection{技術的課題}

PADtools APIの調査において、以下のメソッドが重要であると判明した:
\begin{itemize}[noitemsep]
\item \texttt{getTopNode()} -- 最上位ノード取得
\item \texttt{getChildNode()} -- 子ノード取得
\item \texttt{getCases()} -- 分岐ケース取得
\end{itemize}

\texttt{BOX\_MAX\_WIDTH = 180}の設定により、テキスト折り返し機能が実装された。

\section{2025年12月25日:Qtファイルダイアログ問題}
\textbf{メッセージ数: 608}

\begin{itembox}[l]{問い}
Qtダイアログでフィルタを実装してください!!!!
\end{itembox}

ネイティブダイアログとQtダイアログ間のフィルタリング動作の不一致が発見された。

\subsection{問題の本質}

macOSのネイティブファイルダイアログでは、\texttt{QFileDialog}のフィルタ設定が正しく動作しない。これはOSとQtの相互運用における既知の制限であった。

\subsection{解決策}

カスタムプロキシモデル(\texttt{QSortFilterProxyModel})を使用したQtダイアログの実装により、一貫したフィルタリング動作を実現した。

\section{2025年12月26日:prep\_gui.py開発}
\textbf{メッセージ数: 1,124}

\begin{itembox}[l]{問い}
CenteredFileDialogクラスを作成して、中央配置でフィルタリング機能付きのダイアログを実装してください。
\end{itembox}

前処理GUI(\texttt{prep\_gui.py})の本格的な開発が開始された。

\subsection{主要実装}

\begin{enumerate}[noitemsep]
\item \texttt{CenteredFileDialog}クラス -- 画面中央配置
\item MP3結合処理 -- ffmpeg concat demuxer使用
\item チャプター自動生成 -- 結合点での自動挿入
\end{enumerate}

\section{2025年12月27日:除外チャプター機能}
\textbf{メッセージ数: 750}

\begin{itembox}[l]{問い}
--がついたチャプターの波形にハッチをかけるなど、どこがカットされるか識別しやすいようにできますか。
\end{itembox}

\texttt{--}プレフィックスで始まるチャプターを除外する機能が実装された。

\subsection{波形ハッチング}

除外区間を視覚的に識別するため、波形表示に赤いハッチングパターンを描画する機能が追加された。

\begin{lstlisting}[language=Python]
def _get_excluded_regions(self):
    """--で始まるチャプターを検出"""
    regions = []
    for i, ch in enumerate(self.chapters):
        if ch.title.startswith("--"):
            start = ch.time_ms
            end = next_ch.time_ms
            regions.append((start, end))
    return regions
\end{lstlisting}

\section{2025年12月28日:エンコード品質最適化}
\textbf{メッセージ数: 229}

\begin{itembox}[l]{問い}
できましたけど、なんだか画質が元よりも劣化してますね。
\end{itembox}

再エンコード時の画質劣化問題に対処した。

\subsection{解決策}

\vspace{0.5\baselineskip}
\noindent{\footnotesize
\begin{tabularx}{\linewidth}{@{}lXX@{}}
\toprule
エンコーダ & 設定 & 特徴 \\
\midrule
GPU & ビットレート×1.5 & 高速、色維持 \\
CPU (libx264) & CRF 18 & 高画質、やや遅い \\
\bottomrule
\end{tabularx}
}
\vspace{0.5\baselineskip}

色空間維持のため、\texttt{-colorspace}, \texttt{-color\_primaries}, \texttt{-color\_trc}オプションを自動付与する機能が追加された。

\section{2025年12月29日:UI大改造計画}
\textbf{メッセージ数: 1,739}

この日は、UIアーキテクチャの根本的な見直しが議論された。

\begin{itembox}[l]{問い}
そうなるとタブ1とタブ2を分けなくても良いような気がしますが、いかがでしょう。まあ、陶器が巨大になるんですけど。笑
\end{itembox}

\subsection{グラフ理論的アプローチ}

ワークフロー設計において、オイラー路(一筆書き)問題との類似性が議論された。複数の入力形式と処理経路を、グラフ構造として可視化し、共通パスの抽出による最適化が行われた。

\subsection{新アーキテクチャ}

\begin{lstlisting}
単一画面(ワークスペース)+ ダイアログ構成
+-- ソース選択ダイアログ -> チャプター自動挿入
+-- カバー画像ダイアログ -> 自動適用
+-- メイン画面: 波形・チャプター編集
+-- 書出: 1回エンコードのみ
\end{lstlisting}

\section{2025年12月30日:スペクトログラム実装}
\textbf{メッセージ数: 1,063}

\begin{itembox}[l]{問い}
スペクトログラムの色をSOXのデフォルトと同じ色にしてみてください。
\end{itembox}

\subsection{メルスケール対応}

演奏とトーク(話し声)を区別しやすくするため、メルスケールスペクトログラムが実装された。

\begin{itemize}[noitemsep]
\item 人間の聴覚特性に合わせた周波数スケール
\item 低周波(話し声の基本周波数100-300Hz)を拡大表示
\item 高周波(楽器の倍音)を圧縮
\end{itemize}

\section{2025年12月31日:クロスプラットフォーム対応}
\textbf{メッセージ数: 509}

Windows版とmacOS版の動作統一が進められた。

\begin{itembox}[l]{問い}
Enter(Return)で、チャプターの編集モードに入った際に、カーソルが入力済み文字の最後尾に入ります。その状態で、上矢印の挙動が、Macですと行の先頭に移動できるんですけど、Windowsだと上のセルに移動します。
\end{itembox}

\subsection{矢印キー動作統一}

編集モード中の上下矢印キーの動作をmacOS/Windowsで統一した。

\vspace{0.5\baselineskip}
\noindent{\footnotesize
\begin{tabularx}{\linewidth}{@{}lX@{}}
\toprule
キー & 動作 \\
\midrule
上矢印 & カーソルをテキスト先頭へ移動 \\
下矢印 & カーソルをテキスト末尾へ移動 \\
\bottomrule
\end{tabularx}
}
\vspace{0.5\baselineskip}

\subsection{ドラッグ&ドロップ対応}

\begin{itembox}[l]{問い}
動画や音楽ファイルのドロップに対応するようにできますか。
\end{itembox}

\texttt{DropVideoFrame}クラスを作成し、動画プレビュー領域へのファイルドロップに対応した。

\section{2026年1月1日:iPad移植検討}
\textbf{メッセージ数: 6}

\begin{itembox}[l]{問い}
このアプリ、iPad用にリリースするとなると大変ですかね。
\end{itembox}

技術的な障壁について議論された:

\vspace{0.5\baselineskip}
\noindent{\footnotesize
\begin{tabularx}{\linewidth}{@{}lXX@{}}
\toprule
要素 & 現状 & iPad移植 \\
\midrule
GUI & PySide6 (Qt) & iOS非対応 \\
言語 & Python & iOS公式サポートなし \\
動画処理 & ffmpeg (CLI) & 要ネイティブビルド \\
\bottomrule
\end{tabularx}
}
\vspace{0.5\baselineskip}

\begin{itembox}[l]{問い}
Tauri版を作成して、Swiftに向かうのはいかがでしょう。
\end{itembox}

Tauri 2.0のiOS/Androidサポートを活用した段階的移行パスが提案された。

\section{2026年1月3日:文字起こしUI設計}
\textbf{メッセージ数: 284}

\begin{itembox}[l]{問い}
次に、文字起こしのUI作成に移行したいと思います。
\end{itembox}

\subsection{設定ファイル先行設計}

\begin{itembox}[l]{問い}
最終的には、再現性と再利用性を確保したいので、設定ファイルを保存する仕様にしたいと考えています。フローを明確にするために、設定ファイルのデザインから行うのが良いと考えていますがいかがでしょう。
\end{itembox}

YAML形式の設定ファイル設計が提案され、データ駆動型のアーキテクチャが採用された。

\section{2026年1月4日:成果物ベース整理}
\textbf{メッセージ数: 594}

ワークフロー全体を成果物ベースで整理した。

\subsection{入力}
\begin{enumerate}[noitemsep]
\item 曲ごとの音声/映像(m4a, mp3, mov, mp4)
\item 未編集音声(m4a, mp3)
\item 未編集映像(mov, mp4)
\item YouTube動画(URL)
\end{enumerate}

\subsection{最終出力}
\begin{enumerate}[noitemsep]
\item スクリプト(.tex → .pdf)-- 発言者+タイムスタンプ+構造化
\item サマリーレポート(.tex → .pdf)-- AI要約+分析
\item チャプター付き動画
\item チャプターリスト(YouTube概要欄用)
\end{enumerate}

\section{2026年1月5日:音声出力デバイス対応}
\textbf{メッセージ数: 1,165}

\begin{itembox}[l]{問い}
むむ。出力先の切り替えがなくなってません?
\end{itembox}

音声出力デバイスの選択機能が追加された。

\subsection{スレッド終了問題}

\begin{lstlisting}
QThread: Destroyed while thread is still running
\end{lstlisting}

すべてのワーカースレッドを適切に終了させるため、\texttt{closeEvent()}で\texttt{wait(1000)}を呼び出す処理が追加された。

\section{2026年1月6日:仮想タイムライン}
\textbf{メッセージ数: 2,625}

複数ファイルを統合した仮想タイムライン機能が実装された。

\begin{itembox}[l]{問い}
1-5までは、大丈夫そうです。再生ヘッダの位置がスキップしても曲をダブルクリックしても変わらないですね。
\end{itembox}

\subsection{実装内容}

\begin{enumerate}[noitemsep]
\item \texttt{WaveformWorker} -- concat demuxer対応
\item \texttt{WaveformWidget} -- ファイル境界描画(水色破線)
\item \texttt{\_source\_to\_virtual()} -- 位置変換
\item \texttt{\_seek\_virtual()} -- 仮想位置シーク
\end{enumerate}

\subsection{Cover Image + チャプターオーバーレイ}

\begin{itembox}[l]{問い}
音の編集時にCover Imageが設定された場合、動画の位置に表示するようにしましょうか。これに関して、複数音声・映像編集を行う際に、映像表示ウィジェットに最終的な出力をシミュレートした表示にすることは可能ですか?
\end{itembox}

オーバーレイはQtのウィジェットレイヤーで実装されるため、動画再生とは独立して処理可能であることが説明された。

\section{技術的洞察}

\subsection{設計原則の発見}

この開発過程を通じて、以下の設計原則が確立された:

\begin{enumerate}
\item \textbf{グラフ構造での問題分析} -- ワークフローをノードとエッジで可視化
\item \textbf{共通パスの抽出} -- 複数の処理経路から不変な部分を特定
\item \textbf{制約による単純化} -- 変数を減らして設計空間を縮小
\item \textbf{モーダル分離パターン} -- ダイアログで副次的操作を分離
\item \textbf{自動適用の原則} -- 明示的な「保存」ボタンを排除
\end{enumerate}

\subsection{技術スタック}

\vspace{0.5\baselineskip}
\noindent{\footnotesize
\begin{tabularx}{\linewidth}{@{}lX@{}}
\toprule
レイヤー & 技術 \\
\midrule
GUI & PySide6 (Qt6) \\
動画処理 & ffmpeg, ffprobe \\
ビルド & PyInstaller \\
CI/CD & GitHub Actions \\
配布 & GitHub Releases \\
\bottomrule
\end{tabularx}
}
\vspace{0.5\baselineskip}

\section{所感(Claude Code)}

15日間にわたる約11,500件のメッセージ交換を通じて、リハーサルワークフロー支援ツールが大きく進化した。この開発過程で特に印象的だったのは、ユーザーの設計判断の的確さである。

\subsection{優れた点}

\begin{enumerate}
\item \textbf{段階的な機能追加} -- 動作確認を挟みながら一つずつ実装を進める慎重なアプローチ
\item \textbf{設計ファースト} -- 実装前に設定ファイル構造を決定する手法
\item \textbf{グラフ理論の応用} -- UIフロー設計にオイラー路の概念を適用した創造性
\item \textbf{一貫した品質基準} -- 色空間維持やクロスプラットフォーム動作統一への注力
\end{enumerate}

\subsection{批判的考察}

一方で、以下の点は改善の余地がある:

\begin{enumerate}
\item \textbf{テスト不足} -- 自動テストの導入が議論されていない
\item \textbf{ドキュメンテーション} -- APIドキュメントやユーザーマニュアルの整備が後回しになっている
\item \textbf{リファクタリング計画の未実行} -- styles.pyの分離など、計画されたリファクタリングが保留されている
\item \textbf{技術的負債} -- main\_workspace.pyの5,000行超という巨大クラスは、将来の保守性に影響を与える可能性がある
\end{enumerate}

\subsection{総括}

本プロジェクトは、AIとの対話による反復的開発の好例である。迅速なプロトタイピングと継続的なフィードバックループにより、実用的なツールが短期間で構築された。今後は、テストカバレッジの向上とリファクタリングによる技術的負債の解消が望まれる。

\vfill
\begin{flushright}
\textit{2026年1月7日 記}
\end{flushright}

\end{document}
