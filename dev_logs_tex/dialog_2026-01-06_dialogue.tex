% LuaLaTeX document
\documentclass[10pt,a4paper,twocolumn]{ltjsarticle}

% LuaLaTeX用フォント設定パッケージ
\usepackage{luatexja-fontspec}
\usepackage{amsmath,amssymb}
\usepackage{unicode-math}

% ====================
% 欧文フォント設定 (Libertinus)
% ====================
\setmainfont{Libertinus Serif}[
    BoldFont = {Libertinus Serif Bold},
    ItalicFont = {Libertinus Serif Italic},
    BoldItalicFont = {Libertinus Serif Bold Italic}
]
\setsansfont{Libertinus Sans}[
    BoldFont = {Libertinus Sans Bold},
    ItalicFont = {Libertinus Sans Italic}
]
\setmonofont{DejaVu Sans Mono}[Scale=0.9]

% ====================
% 日本語フォント設定 (原ノ味フォント)
% ====================
\setmainjfont{HaranoAjiMincho-Regular}[
    BoldFont = {HaranoAjiGothic-Medium},
    ItalicFont = {HaranoAjiMincho-Regular},
    BoldItalicFont = {HaranoAjiGothic-Bold}
]
\setsansjfont{HaranoAjiGothic-Regular}[
    BoldFont = {HaranoAjiGothic-Bold}
]
\setmonojfont{HaranoAjiGothic-Regular}

% ====================
% 数式フォント設定 (Libertinus Math)
% ====================
\setmathfont{Libertinus Math}

% ファイル生成日時(JST)
\newcommand{\generatedDate}{2026-01-08}
\newcommand{\generatedTime}{13:35}

% ヘッダー・フッター設定
\usepackage{fancyhdr}
\usepackage{lastpage}
\pagestyle{fancy}
\fancyhf{}
\fancyhead[R]{\small \generatedDate\ \generatedTime\ JST (\thepage/\pageref{LastPage})}
\renewcommand{\headrulewidth}{0.4pt}

% 1ページ目のスタイル
\fancypagestyle{firstpage}{
    \fancyhf{}
    \renewcommand{\headrulewidth}{0pt}
}

% 追加パッケージ
\usepackage{booktabs}
\usepackage{array}
\usepackage{tabularx}
\usepackage{ascmac}
\usepackage{listings}
\usepackage{xcolor}
\usepackage{hyperref}
\usepackage[margin=20mm]{geometry}

% ハイパーリンクの色設定
\hypersetup{
    colorlinks=true,
    linkcolor=blue,
    urlcolor=blue,
    citecolor=blue
}

% コードスタイル設定
\lstset{
    basicstyle=\ttfamily\tiny,
    breaklines=true,
    frame=single,
    backgroundcolor=\color{gray!10},
    columns=fullflexible,
    keepspaces=true,
    showstringspaces=false
}

% Y列タイプ定義
\newcolumntype{Y}{>{\raggedright\arraybackslash}X}

\title{Video Chapter Editor 開発対話記録\\
\large 2026年1月6日:仮想タイムライン完成と大規模UI改善}
\author{ましDialogue}
\date{}

\begin{document}
\maketitle
\thispagestyle{firstpage}

\section*{概要}

本文書は、2026年1月6日に行われたVideo Chapter Editor開発セッションの対話記録である。この日は2625メッセージという史上最大規模のセッションとなり、仮想タイムラインの波形表示・シーク機能の完成、音声出力デバイス選択の改善、波形マーカー表示の最適化、SourceSelectionDialogの簡素化、GitHub ActionsによるCatalina対応ビルドの設定など、多岐にわたる機能改善が行われた。

\tableofcontents

\section{仮想タイムラインの完成}

\subsection{再生位置インジケータの修正}

\begin{itembox}[l]{問い}
1-5までは、大丈夫そうです。再生ヘッダの位置がスキップしても曲をダブルクリックしても変わらないですね。曲はダブルクリックしたものが再生されます。
\end{itembox}

再生位置インジケータの更新が仮想タイムラインに対応していない問題である。\texttt{\_on\_position\_changed}を確認し、仮想タイムラインモード時に\texttt{\_source\_to\_virtual}で全体の中での位置を計算して波形インジケータと時間表示を更新するよう修正した。

\subsection{チャプター対応}

\begin{itembox}[l]{問い}
ですね。次に、--のチャプターの対応(波形表示)と、チャプターをaddする機能がちゃんと動くようにしましょう。
\end{itembox}

問題を発見した。\texttt{\_update\_waveform\_chapters}で\texttt{\_media\_player.duration()}を使っているが、仮想タイムラインでは全体の長さを使う必要がある。\texttt{\_get\_total\_duration()}を使用するよう修正した。

\section{Cover Imageとオーバーレイ表示}

\subsection{Cover Image表示の要望}

\begin{itembox}[l]{問い}
大丈夫そうです。ありがとうございます。あとは、音の編集時にCover Imageが設定された場合、動画の位置に表示するようにしましょうか。これに関して、複数音声・映像編集を行う際に、映像表示ウィジェットに最終的な出力をシミュレートした表示にすることは可能ですか?
\end{itembox}

音声編集時にCover Imageを動画表示領域に表示するのは比較的簡単である。QVideoWidgetの上にQLabelをオーバーレイとして配置し、音声ファイルの場合はCover Imageを表示、映像の場合は非表示とする。

\subsection{チャプター名オーバーレイのシミュレーション}

\begin{itembox}[l]{問い}
ですね。その際に、エンコード時にチャプター名をオーバーレイで焼き込んでいる処理をシミュレートして表示するのは大変でしょうね。
\end{itembox}

実はそれほど大変ではない。QVideoWidget / Cover Imageの上にQLabelをオーバーレイ配置し、再生位置が変わるたびに現在のチャプターを特定してテキストを更新するだけである。完全一致は難しいが、「どこにどのテキストが出るか」のプレビューとしては十分である。

\subsection{Cover Image表示の問題}

\begin{itembox}[l]{問い}
カバー画像を指定しても、オーバーレイ表示されませんね。
\end{itembox}

\begin{itembox}[l]{問い}
09:36:01 INFO  [UI] Chapter overlay enabled\\
09:36:19 INFO  [UI] Cover image updated, is\_audio\_only=True\\
09:36:19 DEBUG [UI] Cover image label size: 1159x608\\
09:36:19 INFO  [UI] Cover image displayed: 1080x608\\
となりますが、表示はされませんね。
\end{itembox}

ログでは正しく処理されているが、QStackedLayoutの問題のようである。別のアプローチで、絶対位置配置に変更した。

\section{音声出力デバイス選択の改善}

\subsection{ホットプラグ対応}

\begin{itembox}[l]{問い}
あとオーディオデバイスリストを開いた時にアップデートするようにしましょうか。アプリ起動後に、APPを接続して表示・選択できるように。
\end{itembox}

\texttt{AudioDeviceComboBox}クラスを作成し、ポップアップ時にデバイスリストを更新する機能を実装した。

\subsection{デバイス切り替え時の音声出力}

\begin{itembox}[l]{問い}
オーディオデバイスの選択はできましたけど、音がでませんね。
\end{itembox}

Qt6では\texttt{setDevice()}ではなく、新しい\texttt{QAudioOutput}オブジェクトを作成する必要があることが判明した。

\subsection{OSボリュームへの委譲}

\begin{itembox}[l]{問い}
まあそうですね。アプリで複雑に実装する必要はないかもしれませんね。Windowsでの実装を考慮すると、どちらもOSに任せるほうがシンプルで良いと思います。1で。
\end{itembox}

アプリ内のVolスライダーを削除し、ボリュームを常に1.0(100\%)に設定してOSのボリュームコントロールに委譲する設計に変更した。

\section{波形マーカー表示の最適化}

\subsection{ファイル境界とチャプターマーカーの区別}

\begin{itembox}[l]{問い}
複数ファイルを読み込んだ際のファイル間をブルーで表示しています。黄色のチャプター区切りと重なると見えなくなるので、ブルーの線幅を太くしてチャプターと重なる場合にも視認できるようにしましょう。
\end{itembox}

\begin{itembox}[l]{問い}
割と見づらいですね。良い方法はありますか?
\end{itembox}

いくつかの改善案を提示した:
\begin{enumerate}
\item ファイル間で背景色を交互に変える(推奨)
\item ファイル境界を上部/下部のみに描画
\item 三角形マーカー
\end{enumerate}

\begin{itembox}[l]{問い}
色は、カットの有無で使っているので避けたいんですけど。仮で3にしてもらえますか?
\end{itembox}

\begin{itembox}[l]{問い}
ちょっとビジーですね。2はどうですか
\end{itembox}

最終的に、ファイル境界は上下12pxの短い水色の線、チャプターマーカーは中央部分の黄色縦線という分離方式が採用された。

\subsection{スペクトログラム後のチャプター消失問題}

\begin{itembox}[l]{問い}
Generating Spectrogramの描画後にチャプターマーカーが消えるんですよね。
\end{itembox}

原因は\texttt{\_on\_spectrogram\_finished}が複数ファイル時に単一ファイルのデュレーションを使用していたこと。\texttt{\_get\_total\_duration()}を使用するよう修正した。

\section{GitHub ActionsによるCatalina対応ビルド}

\begin{itembox}[l]{問い}
このアプリ、Catalina 10.15.7 でIntel Core i7 でも動くようなバイナリを作成できますよね。
\end{itembox}

\begin{itembox}[l]{問い}
GithubのActionsでビルドしてもらう予定です。
\end{itembox}

\texttt{.github/workflows/release.yml}を修正した:
\begin{itemize}
\item \texttt{macos-13}: GitHubの最後のIntel macOSランナー
\item \texttt{MACOSX\_DEPLOYMENT\_TARGET: '10.15'}: Catalina互換を保証
\end{itemize}

\section{SourcesセクションのUI再設計}

\subsection{常時表示への変更}

\begin{itembox}[l]{問い}
複数音声、動画の編集時に表示されるSourcesを常時表示、その右隣に低い高さでSelect Sourceボタンを「Open」に移動しましょう。単一ファイルの時は、1行で、そのほかの仕様は現状のままで構いません。
\end{itembox}

SourceListWidgetを常時表示に変更し、単一ファイルは1行、複数ファイルは3行(prev/current/next)表示とした。

\begin{itembox}[l]{問い}
Youtubeダウンロードの表示をOpenボタンの上にしましょう。
\end{itembox}

\begin{itembox}[l]{問い}
OpenとDownloadボタンの高さをOpenに合わせてください。横幅も同じにしましょう。
\end{itembox}

\begin{itembox}[l]{問い}
DownloadをDLにしましょうか。
\end{itembox}

最終的なレイアウト:
\begin{lstlisting}
+---------------------------------------------+
| YouTube [url input...           ] [DL]      |
| Sources                              [Open] |
| > filename.mp4  (12:34)                     |
+---------------------------------------------+
\end{lstlisting}

\subsection{YouTubeダウンロード進捗表示}

\begin{itembox}[l]{問い}
YoutubeのDLの下に、プログレスバーをコンパクトに表示することは可能ですか
\end{itembox}

\begin{itembox}[l]{問い}
大丈夫そうです。黄緑色にしましょうか。また、複数動画の編集を実装したので、Youtubeのリストダウンロードにも対応したくなりました。
\end{itembox}

QProgressBarを追加し、ライム色(4px高さ)で進捗を表示するようにした。

\section{AV1コーデック問題への対応}

\begin{itembox}[l]{問い}
[AV1 hardware decoding error logs...]\\
既存の動画ですね。
\end{itembox}

yt-dlpのフォーマット設定を修正し、H.264/VP9を優先してAV1を回避するようにした。既存のAV1動画については再ダウンロードまたはffmpegでの変換が必要である。

\begin{lstlisting}[language=Python]
'format': 'bv[vcodec^=avc1]+ba/bv[vcodec^=vp9]+ba/bv*[vcodec!=av01]+ba/b',
\end{lstlisting}

\section{YouTubeファイル名制限の拡張}

\begin{itembox}[l]{問い}
OKです。あとYoutubeでダウンロードされる動画ファイル名が途中でキレてるんですけど、何かやってます?
\end{itembox}

原因を見つけた。yt-dlpのファイル名テンプレートで60バイト制限をかけていた。日本語は1文字3バイト(UTF-8)なので20文字程度で切れてしまう。200バイト(日本語約66文字)に拡張した。

\section{YouTubeプレイリスト対応の計画}

\begin{itembox}[l]{問い}
リストの話に戻りましょう。
\end{itembox}

\begin{itembox}[l]{問い}
2で複数を選択できるのが望ましいですね。
\end{itembox}

プレイリスト内の動画を複数選択できるようにする実装計画を策定した:
\begin{enumerate}
\item PlaylistInfoWorker - プレイリスト情報を非同期取得
\item PlaylistVideoSelectionDialog - チェックボックス付き動画選択UI
\item PlaylistDownloadWorker - 選択動画を順次ダウンロード
\item main\_workspace.py修正 - プレイリストURL検出・フロー追加
\end{enumerate}

\section*{Claude Codeの所感}

本日の開発セッションは2625メッセージという史上最大規模となり、非常に多くの機能改善が行われた。特に印象的だったのは以下の点である。

\textbf{良かった点:}

仮想タイムラインの完成は大きなマイルストーンであった。複数ファイルを一つの連続した波形として表示し、シームレスにシークできる機能は、リハーサル記録の編集において非常に有用である。Qt6のQMediaPlayerの制約(1ファイルずつの再生)を理解した上で、適切な抽象化を行い、ユーザーに一貫した体験を提供できている。

UIの細部への配慮も印象的であった。「ファイル境界とチャプターマーカーの視覚的区別」という問題に対して、三角形マーカー、背景色交互、上下短線という複数の選択肢を提示し、ユーザーの好みと既存の色使い(カットの有無)との整合性を考慮して最適解を見つけていく過程は、良好なコラボレーションの例であった。

\textbf{技術的な学び:}

Qt6の\texttt{QAudioOutput}がOSボリュームと独立して動作すること、\texttt{setDevice()}ではなく新しいオブジェクト生成が必要なこと、\texttt{QStackedLayout}の挙動など、フレームワーク固有の知識が多く得られた。これらは公式ドキュメントだけでは分かりにくい実践的な知見である。

\textbf{反省点:}

Cover Image表示の問題で、最初にQStackedLayoutを使用したが、絶対位置配置に変更する必要があった。ウィジェットの重なり表示については、Qt6の仕様をより深く理解しておくべきであった。

また、AV1コーデック問題への対応は、ダウンロード時の回避のみで、既存動画の再生問題は解決できていない。Qt/QMediaPlayerの制限であり、アプリ側での対処は難しいが、ユーザーへの適切なガイダンスを提供すべきであった。

\textbf{今後の課題:}

YouTubeプレイリスト対応の実装計画は策定されたが、まだ完成していない。また、2625メッセージという長大なセッションは、コンテキストの管理という点でも課題があった。適切なタイミングでのコミット、進捗のまとめ、次回への引き継ぎを意識することが重要である。

全体として、ユーザーの要望を丁寧に聞き取り、段階的に改善を重ねていく姿勢は維持できた。特に「完璧を求めすぎず、実用的な解決策を選ぶ」という判断(例:OSボリュームへの委譲)は、ソフトウェア開発において重要な視点である。

\end{document}
