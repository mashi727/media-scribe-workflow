%% dialog_2025-12-28_full.tex
%% 2025年12月28日の開発対話記録(229メッセージ)

\section{2025年12月28日:UI改善とワークフロー最適化の設計}

この日は、エンコード品質の改善(GPU/CPU選択、色空間保持)、v1.3.0のリリース、モダンダークUIへの刷新、そしてMP3→MP4変換における2回エンコード問題の解決に向けた設計議論が行われた。

\subsection{エンコード品質の改善}

\begin{userbox}
できましたけど、なんだか画質が元よりも劣化してますね。
\end{userbox}

\begin{assistantbox}
再エンコードによる世代劣化ですね。同じビットレートでも再エンコードすると品質は落ちます。

対策として:
\begin{enumerate}
\item 「元と同じ」のビットレートを1.2〜1.5倍にする --- 元の品質を維持
\item CPUエンコーダ(libx264)でCRFモードを使う --- 品質優先で必要なビットレートを自動決定
\end{enumerate}
\end{assistantbox}

\begin{userbox}
GPUは1で、CPUは2でということにしましょうか。
\end{userbox}

以下の方針で実装された:
\begin{itemize}
\item \textbf{GPU}: 「元と同じ」選択時はビットレート×1.5倍
\item \textbf{CPU}: CRFモード(CRF 18)で高画質
\end{itemize}

\begin{userbox}
なんかCPUの方が早いんですけど。
\end{userbox}

\begin{assistantbox}
静止画動画だからですね。CRFモードは変化のないフレームは少ないビットで済むので高速。ultrafastプリセットを使用しており、静止画はフレーム間差分がほぼゼロなので処理が軽い。GPUはデータ転送オーバーヘッドがあるので、単純な動画では逆に遅くなることがあります。
\end{assistantbox}

\subsection{色空間の保持}

\begin{userbox}
CPUは色合いが少し変わってしまいますね。
\end{userbox}

\begin{userbox}
-colorspaceを追加しましょうか。
\end{userbox}

\texttt{detect\_video\_colorspace()}関数が追加され、ffprobeで元動画の色空間を検出し、エンコード時に維持する機能が実装された:

\begin{lstlisting}[language=python,basicstyle=\ttfamily\tiny]
@dataclass
class ColorspaceInfo:
    """色空間情報"""
    color_space: str = ""
    color_primaries: str = ""
    color_transfer: str = ""

    def get_ffmpeg_args(self) -> List[str]:
        args = []
        if self.color_space and self.color_space != "unknown":
            args.extend(['-colorspace', self.color_space])
        if self.color_primaries and self.color_primaries != "unknown":
            args.extend(['-color_primaries', self.color_primaries])
        if self.color_transfer and self.color_transfer != "unknown":
            args.extend(['-color_trc', self.color_transfer])
        return args
\end{lstlisting}

\begin{userbox}
まあまあ保持されますね。
\end{userbox}

\subsection{v1.3.0のリリース}

\begin{userbox}
今の状態までのアップデートをコミットして、Pushリリースしてください。
\end{userbox}

バージョンを1.3.0に更新し、以下のファイルが変更された:
\begin{itemize}
\item \texttt{rehearsal\_workflow/\_\_init\_\_.py}
\item \texttt{pyproject.toml}
\item \texttt{video\_chapter\_editor.spec}
\item \texttt{README.md}(ダウンロードリンク更新)
\end{itemize}

リリースノートには以下の新機能が記載された:
\begin{itemize}
\item ビットレート自動検出(「元と同じ(自動)」オプション)
\item GPU: ビットレート×1.5、CPU: CRF 18
\item 色空間の自動検出と保持
\end{itemize}

\subsection{モダンダークUIへの刷新}

\begin{userbox}
video-chapter-editorのUIをシンプルにかっこよくしてください。
\end{userbox}

Themeクラスが導入され、統一されたカラーパレットとスタイル定義が実装された:

\begin{lstlisting}[language=python,basicstyle=\ttfamily\tiny]
class Theme:
    """統一されたUIテーマ定義"""
    # 背景色(暗い順)
    BG_DARKEST = "#0f0f0f"
    BG_DARK = "#1a1a1a"
    BG_BASE = "#242424"
    BG_ELEVATED = "#2d2d2d"
    BG_HOVER = "#363636"
    BG_ACTIVE = "#404040"

    # テキスト色
    TEXT_PRIMARY = "#f0f0f0"
    TEXT_SECONDARY = "#a0a0a0"
    TEXT_MUTED = "#666666"

    # アクセントカラー
    ACCENT = "#3b82f6"
    ACCENT_HOVER = "#2563eb"
    ACCENT_ACTIVE = "#1d4ed8"

    # ステータスカラー
    SUCCESS = "#22c55e"
    WARNING = "#f59e0b"
    DANGER = "#ef4444"

    RADIUS = "8px"
    RADIUS_SM = "4px"

    @classmethod
    def button_primary(cls) -> str: ...
    @classmethod
    def button_secondary(cls) -> str: ...
    @classmethod
    def button_danger(cls) -> str: ...
\end{lstlisting}

\subsection{ボタンスタイルのポリシー統一}

\begin{userbox}
タブ1のボタンに色がついているものとついていないものがありますが、明確なポリシーはありますか
\end{userbox}

\begin{assistantbox}
提案:どちらかのポリシーに統一

\begin{enumerate}
\item \textbf{機能ベース}: 主要アクションのみPrimary、他はすべてSecondary
\item \textbf{推奨度ベース}: よく使う操作をPrimaryで目立たせる
\end{enumerate}
\end{assistantbox}

\begin{userbox}
1
\end{userbox}

機能ベースのポリシーが採用され、以下のように統一された:

\vspace{0.5\baselineskip}
\noindent{\footnotesize
\begin{tabularx}{\linewidth}{@{}lX@{}}
\toprule
スタイル & 用途 \\
\midrule
Primary(青) & 各タブの主要アクション(結合実行、書出)のみ \\
Secondary(グレー) & その他すべて(追加、削除、選択、貼り付け、開く、ジャンプなど) \\
Danger(赤) & 中止ボタン \\
\bottomrule
\end{tabularx}
}
\vspace{0.5\baselineskip}

\begin{userbox}
タブ2のチャプター表示において、時間が見切れるのを防止してください。
\end{userbox}

時間列の幅を90px→110pxに拡張し、「H:MM:SS.mmm」形式が見切れないように修正された。

\subsection{ワークフロー最適化の設計議論}

\begin{userbox}
mp3の動画エンコードで、2回エンコードを行うのが美しくないと考えています。mp3の場合は、曲ごとのカットが終わっている場合と、長時間の未編集の場合があると考えていますが、エンコードを一度だけ行うようなおすすめのワークフローはどんなものがありますか
\end{userbox}

\begin{assistantbox}
現状の問題点を整理します:

\textbf{現在のワークフロー(2回エンコード)}
\begin{lstlisting}
タブ1: MP3結合 -> MP4作成(1回目エンコード)
タブ2: MP4読込 -> チャプター編集 -> 書出(2回目エンコード)
\end{lstlisting}

\textbf{提案: 1回エンコードのワークフロー}

\vspace{0.5\baselineskip}
\noindent{\footnotesize
\begin{tabularx}{\linewidth}{@{}lX@{}}
\toprule
ケース & 推奨ワークフロー \\
\midrule
曲別MP3(カット済み) & タブ1で結合+チャプター自動生成 → 直接最終MP4出力 \\
長時間MP3(未編集) & タブ2でMP3を直接読込 → チャプター編集 → 書出 \\
\bottomrule
\end{tabularx}
}
\vspace{0.5\baselineskip}
\end{assistantbox}

\subsection{一筆書き問題としての分析}

\begin{userbox}
なるほど。一筆書き問題に見えてきました。機能重複は、起点が3つ以上あるということですね。
\end{userbox}

\begin{assistantbox}
その通りです。グラフ理論的に整理すると:

\textbf{起点(入力パターン):}
\begin{enumerate}
\item 複数MP3(カット済み)
\item 単一MP3(長尺未編集)
\item 既存MP4
\end{enumerate}

\textbf{終点:} チャプター付きMP4

一筆書きできない構造:
\begin{lstlisting}
起点1 --> 結合 --> カバー --+
起点2 ---------> カバー ----+--> チャプター編集 -> 書出
起点3 ---------------------+
\end{lstlisting}

\textbf{解決策:共通パスを1つにして、入口を分岐}
\begin{lstlisting}
+- 起点1: 複数MP3 -> 結合 -+
+- 起点2: 単一MP3 ---------+--> [カバー画像?] -> チャプター編集 -> 書出
+- 起点3: MP4 ------------+
                           ^
                      ここを1つのTabに
\end{lstlisting}
\end{assistantbox}

\subsection{タイトル焼き込みと劣化の関係}

\begin{userbox}
焼き込みありを前提にして、劣化と手間を極限したいんですよね。
\end{userbox}

タイトル焼き込みが必須の場合の最適化が議論された:

\vspace{0.5\baselineskip}
\noindent{\footnotesize
\begin{tabularx}{\linewidth}{@{}lXX@{}}
\toprule
入力 & 映像処理 & 音声処理 \\
\midrule
MP3+静止画 & enc(生成+焼込)1回 & enc 1回 \\
MP4 & enc(焼込)1回 & copy(無劣化) \\
\bottomrule
\end{tabularx}
}
\vspace{0.5\baselineskip}

重要な発見として、MP4入力時は音声を\texttt{-c:a copy}で無劣化コピーできることが確認された。

\subsection{最終的なタブ構成の提案}

\begin{userbox}
タブ1は、複数のMP3の読み込み、一つのMP3の出力とチャプターファイルの出力のみにし、カバー画像は、タブ2からボタンで呼び出し別画面が起動する流れもアリでしょうか。
\end{userbox}

\begin{assistantbox}
その方がスッキリしますね。

\textbf{提案の構造:}

\textbf{Tab 1: MP3結合(シンプル化)}
\begin{itemize}
\item 入力: 複数MP3
\item 出力: 結合済みMP3 + chapters.txt
\item カバー画像機能なし
\end{itemize}

\textbf{Tab 2: 編集・書出}
\begin{itemize}
\item ファイル読込: MP3(Tab 1経由 or 直接)/ MP4
\item カバー画像設定 → 別画面ダイアログ
\item 波形・チャプター編集
\item 書出
\end{itemize}

\textbf{各ケースのフロー:}

\vspace{0.5\baselineskip}
\noindent{\footnotesize
\begin{tabularx}{\linewidth}{@{}lX@{}}
\toprule
ケース & フロー \\
\midrule
A. 複数MP3 & Tab 1で結合 → Tab 2読込 → カバー設定 → 書出 \\
B. 単体MP3 & Tab 2で直接読込 → カバー設定 → 書出 \\
C. MP4 & Tab 2で読込 → 書出(カバー不要) \\
\bottomrule
\end{tabularx}
}
\vspace{0.5\baselineskip}
\end{assistantbox}

\begin{userbox}
この場合、タブ1操作における劣化はほぼないという認識で良いですね。
\end{userbox}

Tab 1では結合のみでエンコードを行わないため、劣化は発生しない設計となった。

\subsection{report-workflowとの連携}

\begin{userbox}
以降のフローも考慮しておきたいと考えています。動画をYoutubeに登録してsrtをダウンロード、WhisperでSRTをダウンロード。これからClaude Codeのプロンプトにて所望の形式で概要を作成、保存という流れです。これは別UIのアプリで作成する方針で進めていますが、この方針に対して意見はありますか。
\end{userbox}

\begin{assistantbox}
設計方針として適切だと考えます。

\textbf{分離が適切な根拠:}
\begin{enumerate}
\item \textbf{責務の明確な境界}
  \begin{itemize}
  \item video-chapter-editor: 映像加工(決定論的処理)
  \item report-workflow: テキスト生成(AI/LLM処理)
  \end{itemize}
\item \textbf{入力の性質が異なる}
  \begin{itemize}
  \item 前者:ファイル操作、波形、プレビュー
  \item 後者:SRT解析、プロンプト設計、テキスト編集
  \end{itemize}
\item \textbf{「配管と陶器」の思想に合致}
\end{enumerate}

\textbf{提案:共通の作業ディレクトリ規約}

\begin{lstlisting}
work/
+-- input/           # 元素材
+-- output.mp4       # video-chapter-editor出力
+-- chapters.txt     # チャプターリスト
+-- transcript.srt   # 字幕(YouTube or Whisper)
+-- report.tex       # report-workflow出力
\end{lstlisting}

この規約があれば、両アプリは疎結合のまま連携できます。
\end{assistantbox}

\subsection{この日の成果}

\begin{enumerate}
\item \textbf{エンコード品質改善} - GPU×1.5倍ビットレート、CPU CRF 18
\item \textbf{色空間保持機能} - ffprobeで検出、ffmpegで維持
\item \textbf{v1.3.0リリース} - GitHub Actionsで自動ビルド
\item \textbf{モダンダークUI} - Themeクラスによる統一スタイル
\item \textbf{ボタンポリシー統一} - 機能ベース(Primary=主要アクションのみ)
\item \textbf{ワークフロー最適化設計} - 1回エンコードへの移行計画
\item \textbf{タブ構成再設計} - Tab 1は結合のみ、Tab 2で共通編集
\end{enumerate}

\subsection{技術的なポイント}

\begin{enumerate}
\item \textbf{静止画動画のエンコード特性}: CPU(CRF)の方がGPUより高速かつ高品質
\item \textbf{音声の無劣化コピー}: タイトル焼き込み時もMP4入力なら\texttt{-c:a copy}で音声劣化ゼロ
\item \textbf{グラフ理論的設計分析}: 一筆書き問題として入力パターンの分岐を整理
\item \textbf{責務分離}: video-chapter-editorとreport-workflowの明確な境界設計
\end{enumerate}

