% LuaLaTeX Document - Video Chapter Editor 開発対話記録
% 2025年12月23日〜2026年1月6日
\documentclass[a4paper,10pt,twocolumn]{ltjsarticle}

% ===== パッケージ =====
\usepackage{luatexja-fontspec}
\usepackage{amsmath,amssymb}
\usepackage{unicode-math}
\usepackage{geometry}
\usepackage{fancyhdr}
\usepackage{lastpage}
\usepackage{xcolor}
\usepackage{tcolorbox}
\usepackage{enumitem}
\usepackage{booktabs}
\usepackage{array}
\usepackage{tabularx}
\usepackage{listings}
\usepackage{ascmac}
\usepackage{hyperref}
\usepackage{titlesec}

% ===== ページ設定 =====
\geometry{left=15mm,right=15mm,top=25mm,bottom=20mm}

% ===== フォント設定 =====
\setmainfont{Libertinus Serif}[
    BoldFont = {Libertinus Serif Bold},
    ItalicFont = {Libertinus Serif Italic},
    BoldItalicFont = {Libertinus Serif Bold Italic}
]
\setsansfont{Libertinus Sans}[
    BoldFont = {Libertinus Sans Bold},
    ItalicFont = {Libertinus Sans Italic}
]
\setmonofont{Libertinus Mono}

\setmainjfont{HaranoAjiMincho-Regular}[
    BoldFont = {HaranoAjiGothic-Medium},
    ItalicFont = {HaranoAjiMincho-Regular},
    BoldItalicFont = {HaranoAjiGothic-Bold}
]
\setsansjfont{HaranoAjiGothic-Regular}[
    BoldFont = {HaranoAjiGothic-Bold}
]
\setmonojfont{HaranoAjiGothic-Regular}

\setmathfont{Libertinus Math}

% ===== 日時設定(JST) =====
\newcommand{\generatedDate}{2026-01-07}
\newcommand{\generatedTime}{22:30}

% ===== ヘッダー・フッター =====
\pagestyle{fancy}
\fancyhf{}
\fancyhead[R]{\small \generatedDate\ \generatedTime\ JST (\thepage/\pageref{LastPage})}
\renewcommand{\headrulewidth}{0.4pt}
\fancypagestyle{plain}{
    \fancyhf{}
    \renewcommand{\headrulewidth}{0pt}
}

% ===== タイトル =====
\title{\textbf{Video Chapter Editor 開発対話記録}\\
{\large 2025年12月23日〜2026年1月6日}}
\author{執筆者:ましDialogue}
\date{}

% ===== コードスタイル =====
\lstset{
    basicstyle=\ttfamily\scriptsize,
    breaklines=true,
    frame=single,
    backgroundcolor=\color{gray!10},
    columns=fullflexible,
    keepspaces=true,
    escapeinside={(*@}{@*)},
    showstringspaces=false
}

% ===== カスタム環境 =====
\newtcolorbox{userbox}{
    colback=blue!5,
    colframe=blue!50!black,
    title={\textbf{問い}},
    fonttitle=\bfseries,
    boxrule=0.5pt
}

\newtcolorbox{aibox}{
    colback=green!5,
    colframe=green!50!black,
    boxrule=0.3pt
}

% ===== セクションスタイル =====
\titleformat{\section}
    {\normalfont\large\bfseries}{\thesection}{1em}{}
\titleformat{\subsection}
    {\normalfont\normalsize\bfseries}{\thesubsection}{1em}{}

% ===== 本文開始 =====
\begin{document}
\maketitle
\thispagestyle{plain}

\section*{概要}

本文書は、Video Chapter Editor(VCE)の開発過程における対話記録を、トピック別に構造化してまとめたものである。2025年12月23日から2026年1月6日までの約2週間にわたる開発セッションを、日付順ではなく機能・テーマ別に整理し、一次資料としての価値を保ちながら、開発の論理的な流れを示すことを目的とする。

対話の内容は、エラーメッセージの詳細など非本質的な部分を適切に省略しつつも、思考過程と設計判断の記録として重要な部分は忠実に保持している。

\tableofcontents

%==============================================================================
\section{設計思想:配管と陶器}
%==============================================================================

\subsection{Gitの陶器と配管の適用}

本プロジェクトの設計思想は、Gitにおける「陶器(porcelain)と配管(plumbing)」の概念に基づいている。

\begin{userbox}
未実装タスクを実行してください。
\end{userbox}

この指示を受け、CLAUDE.mdに記載された未実装タスクの実行を開始した。プロジェクトの全体像を把握した結果、既存の統合ワークフローツール(rehearsal-download/finalize)とは別に、CLAUDE.mdの設計方針に基づく「単機能の配管ツール」が必要であることが判明した。

\begin{userbox}
やはり、Mermaidだと読みづらいですね。PADだと、抽象度が右に行くにつれて低くなるので、配管と陶器の整理が行いやすいんですよね。
\end{userbox}

Mermaidは「フローの流れ」を表現するのに適しているが、抽象度の階層が視覚的に分かりにくい。一方、PAD(Problem Analysis Diagram)は構造的に「右に展開する」形式であり、これが「陶器(ユーザー向けコマンド)」と「配管(内部実装)」の関係を自然に表現できる。

\begin{lstlisting}
高抽象(陶器)          低抽象(配管)
────────────────────────────────────>
rehearsal-download  → ytdl → yt-dlp
                    → whisper-remote → curl → Docker
\end{lstlisting}

\subsection{階層的なツール構成}

作成した配管ツールは以下の階層で整理された:

\vspace{0.5\baselineskip}
\noindent{\footnotesize
\begin{tabularx}{\linewidth}{@{}lX@{}}
\toprule
階層 & 内容 \\
\midrule
誰でも使える & yt-srt(YouTube字幕取得) \\
興味があれば & プロンプト例 \\
本気でやりたい人向け & 環境構築ガイド \\
\bottomrule
\end{tabularx}
}
\vspace{0.5\baselineskip}

%==============================================================================
\section{基盤ツールの構築}
%==============================================================================

\subsection{配管ツールの実装}

2025年12月23日のセッションで、以下の配管ツールを作成した:

\begin{enumerate}[itemsep=0pt]
    \item \texttt{bin/yt-srt} -- YouTube字幕取得
    \item \texttt{bin/video-trim} -- 動画トリミング
    \item \texttt{bin/video-chapters} -- チャプター結合・埋め込み
\end{enumerate}

これらはffmpegやyt-dlpを内部で呼び出す薄いラッパーとして設計され、単一の責務に特化している。

\subsection{PADtools CLI(spd2png)の作成}

\begin{userbox}
ですね。pngにレンダリングして、貼り付けた方が良いかなあ。
\end{userbox}

PAD図をPNG形式で出力するため、PADtoolsのCLI機能を調査した。PADtools本体のMain.javaはCLI引数の解析ロジックを持っているものの、実際にはGUIを起動してしまう実装であることが判明した。

\begin{userbox}
3にしたいですね。
\end{userbox}

独自レンダラーの作成を選択したが、最初のフローチャート風実装はPADの正式な表記法と異なっていた。

\begin{userbox}
まだまだですね。車輪の再開発っぽいので、PadtoolsをフォークしてCLI実装を行うとかどうですか。
\end{userbox}

最終的に、PADtoolsのConverterクラスを直接呼び出すラッパー(PadCLI.java)を作成することで解決した:

\begin{lstlisting}[language=Java]
// PadCLI.java - PADtools Converter直接呼び出し
import padtools.converter.Converter;
public class PadCLI {
    public static void main(String[] args) {
        Converter.convert(
            new File(args[0]),
            new File(args[1]),
            Double.parseDouble(args[2])
        );
    }
}
\end{lstlisting}

これにより、ヘッドレス環境でPAD図をPNG出力できるようになった。

%==============================================================================
\section{GUIインフラストラクチャの進化}
%==============================================================================

\subsection{ファイルダイアログとフィルタリング}

2025年12月25日のセッションで、Qtのファイルダイアログにカスタムフィルタリング機能を実装した。

\begin{userbox}
Sourceの選択ダイアログのデフォルトをmp4にして、mp3のボタンと入れ替えましょう。
\end{userbox}

QSortFilterProxyModelを継承したFileFilterProxyModelクラスを作成し、拡張子に基づくファイルフィルタリングを実現した。

\subsection{CenteredFileDialogの実装}

2025年12月26日のセッションで、ダイアログを親ウィンドウの中央に配置し、ダークテーマを適用したCenteredFileDialogクラスを作成した。

\begin{lstlisting}[language=Python]
class CenteredFileDialog(QDialog):
    """センタリング&ダークテーマ適用ダイアログ"""
    def __init__(self, parent=None):
        super().__init__(parent)
        # 親ウィンドウの中央に配置
        if parent:
            geo = parent.geometry()
            self.move(
                geo.center() - self.rect().center()
            )
\end{lstlisting}

\subsection{ソース選択ダイアログの刷新}

2026年1月5日のセッションで、ソース選択ダイアログを大幅に改良した。

\begin{userbox}
Select Sourceから、Browseを押した時のディレクトリ選択のUIがお好みなんですけど。Select Sourceのダイアログを、Browseを押した時のSelect Directoryを同じようにできますか?
\end{userbox}

\begin{userbox}
フィルダによって、関係ないファイルを表示しないように。尚且つダークで表示してください。
\end{userbox}

QFileDialogの標準フィルタは非マッチファイルを非選択にするだけで非表示にしないため、カスタムプロキシモデルを作成してフィルタに合わないファイルを完全に非表示にした。

\begin{userbox}
フォルダーツリーにフォルダのツリーを表示することは可能ですか
\end{userbox}

左側の「場所」リストを階層的なフォルダツリーに置き換え、QFileSystemModelを使用してディレクトリのみを表示するようにした。

\begin{userbox}
Local filesとYoutubeをタブにして、先ほど作成したファイル選択の画面を実装できますか
\end{userbox}

最終的に、QTabWidgetによる「Local Files」と「YouTube」のタブ切り替えインターフェースを実装した。Local Filesタブには以下の機能が統合された:

\begin{enumerate}[itemsep=0pt]
    \item 左側:フォルダツリー(QTreeView + QFileSystemModel)
    \item 右側:ファイルリスト(QTreeView + MediaFilterProxyModel)
    \item Video/Audioトグルボタンによるフィルタ切り替え
    \item \texttt{..}による親ディレクトリナビゲーション
\end{enumerate}

%==============================================================================
\section{動画処理機能}
%==============================================================================

\subsection{エンコーディング品質とカラースペース}

2025年12月28日のセッションで、GPU(VideoToolbox)とCPU(libx264)のエンコード品質を比較した。

CPU(CRFモード)の方が品質が高いことが確認され、カラースペースの検出と保持機能も実装した:

\begin{lstlisting}[language=bash]
# カラースペース検出
ffprobe -v error -select_streams v:0 \
  -show_entries stream=color_space,\
  color_primaries,color_trc \
  -of default=noprint_wrappers=1 input.mp4

# 保持オプション
-colorspace bt709 -color_primaries bt709 \
-color_trc bt709
\end{lstlisting}

\subsection{除外チャプター機能}

2025年12月27日のセッションで、チャプター名に\texttt{--}プレフィックスを付けることでエクスポート時に除外する機能を実装した。

\begin{userbox}
除外チャプターのハッチングについてですが、波形の色を薄くするだけでわかりやすいと思います。
\end{userbox}

波形表示では、除外チャプターの領域にハッチング(斜線パターン)を描画し、視覚的に区別できるようにした。

%==============================================================================
\section{可視化機能}
%==============================================================================

\subsection{波形表示とハッチング}

波形表示では、除外チャプターの領域を視覚的に区別するため、ハッチングパターンを実装した。QPainterのsetBrushStyleを使用し、Qt.BDiagPatternによる斜線パターンを描画している。

\subsection{スペクトログラム実装}

2025年12月30日のセッションで、SOXスタイルのカラーマップを使用したスペクトログラム表示を実装した。

\begin{userbox}
スペクトログラムのカラーをSOX風の、青→緑→黄色→赤風に変えることはできますか?
\end{userbox}

メルスケール変換を実装し、人間の聴覚特性に合わせた周波数スケールで表示するようにした。また、再生ヘッドの視認性を向上させるため、黄色・3ピクセル幅のスタイルを適用した。

%==============================================================================
\section{クロスプラットフォーム対応}
%==============================================================================

2025年12月31日のセッションで、macOSとWindowsの両プラットフォームでの動作統一を行った。

\begin{enumerate}[itemsep=0pt]
    \item 矢印キー動作の統一(チャプター間移動)
    \item メニューバーフォントサイズの標準化(16px)
    \item ドラッグ&ドロップサポートの追加
\end{enumerate}

GitHub Actionsによる自動ビルドワークフローを整備し、macOS(DMG)とWindows(ZIP)の両形式でリリースを自動化した。

%==============================================================================
\section{YouTubeダウンロード統合}
%==============================================================================

2026年1月5日のセッションで、yt-dlpを使用したYouTubeダウンロード機能を実装した。

\begin{userbox}
さて、youtubeのダウンロード、どう実装しましょうかね。良いアイデアはありますか
\end{userbox}

技術的選択肢を検討した結果、yt-dlp Pythonパッケージを採用した。

\begin{userbox}
Select SourceタブにYoutubeタブをつけるとどうなりますか。質問の回答です。1フォーマット動画優先ですが、音質も良いものをお願いします。最高品質で良いですね。字幕はあれば取得してください。
\end{userbox}

以下の機能を実装した:

\begin{enumerate}[itemsep=0pt]
    \item URL入力欄とダウンロードボタン
    \item 進捗バーによるダウンロード状況表示
    \item 字幕(SRT)の自動取得オプション
    \item ダウンロード後の自動ソース設定
\end{enumerate}

\begin{userbox}
OSを含めて、UIをロックしないように。また、Fetch Infoは不要です。そのままダウンロードを初めてください。
\end{userbox}

Fetch Infoボタンを削除し、URLを入力してDownloadを押すだけで非同期ダウンロードを開始するシンプルなインターフェースに変更した。

%==============================================================================
\section{UI再設計とグラフ理論的アプローチ}
%==============================================================================

2025年12月29日のセッションで、UIの大規模な再設計について議論した。

\begin{userbox}
今のUIは3系統の入力インターフェースがあるのですが、Select Sourceで全てまかなうようにしたいです。
\end{userbox}

タブベースのインターフェースから、単一ワークスペース+ダイアログ形式への移行を計画した。

議論の中で、グラフ理論的なアプローチが提案された:

\begin{quote}
オイラー路を辿るように、各ノード(機能)を一度だけ通過する直線的なワークフローを設計する。分岐点では「行かなかった道」を視覚的に示すことで、ユーザーが「何を選択したか」を常に把握できるようにする。
\end{quote}

この考え方は、後のソース選択ダイアログの設計に反映された。

%==============================================================================
\section{アーティファクトベースのワークフロー}
%==============================================================================

2026年1月4日のセッションで、ワークフローをアーティファクト(成果物)中心に再整理した。

\begin{userbox}
入力ファイル、中間ファイル、出力ファイルの観点でまとめてください。
\end{userbox}

\vspace{0.5\baselineskip}
\noindent{\footnotesize
\newcolumntype{Y}{>{\raggedright\arraybackslash}X}
\begin{tabularx}{\linewidth}{@{}lYY@{}}
\toprule
分類 & ファイル & 説明 \\
\midrule
入力 & 生動画/音声 & リハーサル録画 \\
中間 & SRT & 自動文字起こし \\
最終 & チャプター付きMP4 & YouTube投稿用 \\
\bottomrule
\end{tabularx}
}
\vspace{0.5\baselineskip}

この整理により、各処理ステップの入出力が明確になり、ワークフロー全体の見通しが良くなった。

%==============================================================================
\section{将来の展望}
%==============================================================================

\subsection{iPad移植の検討}

2026年1月1日のセッションで、iPad版の可能性について議論した。

\begin{userbox}
このアプリ、iPad用にリリースするとなると大変ですかね。
\end{userbox}

主な技術的障壁として、PySide6がiOS非対応であること、Pythonの公式iOSサポートがないこと、ffmpegのネイティブビルドが必要なことが挙げられた。

\begin{userbox}
Tauri版を作成して、Swiftに向かうのはいかがでしょう。
\end{userbox}

Tauri 2.0はiOS/Androidサポートがあるため、以下の移行パスが提案された:

\begin{lstlisting}
現在 (PySide6)
    |
Tauri 2.0 (Rust + Web)
    |
    +-- デスクトップ: そのまま利用
    +-- iPad: Tauri iOS (実験的)
    +-- 必要なら: Swift書き直し
\end{lstlisting}

\subsection{リファクタリング計画}

2026年1月6日のセッションで、コードの保守性向上のためのリファクタリング計画を策定した。

\vspace{0.5\baselineskip}
\noindent{\footnotesize
\begin{tabularx}{\linewidth}{@{}lXl@{}}
\toprule
ファイル & 主な問題 & 優先度 \\
\midrule
main\_workspace.py & God Class(133メソッド) & HIGH \\
workers.py & drawtext重複4箇所 & HIGH \\
dialogs.py & \_button\_style()重複6箇所 & MEDIUM \\
\bottomrule
\end{tabularx}
}
\vspace{0.5\baselineskip}

Phase 1では重複コードの抽出(styles.py新規作成)、Phase 2ではユーティリティMixinの作成、Phase 3では責務分離(後日検討)という段階的なアプローチを採用する。

%==============================================================================
\section*{Claude Codeの所感}
%==============================================================================

本開発プロジェクトに携わって、いくつかの重要な観点を得た。

\textbf{肯定的な側面:}

ユーザーの設計思想である「配管と陶器」のアプローチは、ツールの再利用性と保守性を高める上で極めて有効であった。単一目的のツールを組み合わせることで、複雑なワークフローも柔軟に構築できる。

UI設計においては、ユーザーからの継続的なフィードバックが重要な役割を果たした。特に「ネイティブダイアログが元に戻った」といった即座の指摘は、実装の品質向上に直結した。

\textbf{批判的な側面:}

開発過程で、main\_workspace.pyが133メソッドを持つGod Classに成長してしまった点は反省材料である。機能追加を優先するあまり、適切なタイミングでのリファクタリングを怠った結果と言える。

また、PADtoolsのCLI化にあたり、最初に独自レンダラーを作成しようとしたアプローチは、「車輪の再発明」であった。既存のConverterクラスを直接利用するというユーザーの提案の方が、はるかに効率的かつ正確な解決策であった。

\textbf{今後への示唆:}

本プロジェクトは、対話的な開発プロセスの有効性を示している。ユーザーとAIの継続的なやり取りにより、要件の明確化、設計の洗練、実装の改善が進み、最終的に実用的なツールが完成した。

一方で、長期的な保守性を考慮すると、定期的なリファクタリングの機会を設けることが重要である。Phase 1〜3の計画が策定されたことは、この点での改善の第一歩と言える。

\vfill
\begin{flushright}
{\small 2026年1月7日 Claude Code}
\end{flushright}

\end{document}
