%% dialog_2025-12-25_full.tex
%% 2025年12月25日の開発対話記録(608メッセージ)

\section{2025年12月25日:Qtダイアログとファイルフィルタリング}

この日は、Qtファイルダイアログでのフィルタリング機能の実装と、設定保存/読み込み機能、ytdlスクリプトのGUI統合が行われた。

\subsection{ファイルフィルタ機能の要望}

\begin{userbox}
ファイルオープンのダイアログを開いた際に、関係のないファイルをフィルタする機能を実装してください。
\end{userbox}

最初の試みでは、QFileDialogの標準的なフィルタ機能(\texttt{setNameFilter}、\texttt{setNameFilters})を使用したが、Qtダイアログでは正しく機能しなかった。

\begin{userbox}
フィルターが機能していないようですよ
\end{userbox}

\begin{userbox}
フィルタを選択しても、対象のファイルに切り替わらないですね。
\end{userbox}

ネイティブOSダイアログでは正常に機能するが、Qtダイアログでは動作しないという状況が判明した。

\begin{userbox}
OS配下のダイアログは、うまくいきますね。
\end{userbox}

ユーザーはあくまでQtダイアログでの実装を希望した。

\begin{userbox}
いえいえ、Qtダイアログでフィルタを実装してください!!!!
\end{userbox}

\subsection{FileFilterProxyModelの実装}

最終的に、\texttt{QSortFilterProxyModel}を継承したカスタムプロキシモデルを作成することで解決した。

\begin{lstlisting}[language=python,caption=FileFilterProxyModel,basicstyle=\ttfamily\tiny]
class FileFilterProxyModel(QSortFilterProxyModel):
    """ファイル拡張子でフィルタリングするプロキシモデル"""
    def __init__(self, extensions=None, parent=None):
        super().__init__(parent)
        self.extensions = extensions or []

    def set_extensions(self, extensions):
        self.extensions = extensions
        self.invalidateFilter()

    def filterAcceptsRow(self, source_row, source_parent):
        model = self.sourceModel()
        index = model.index(source_row, 0, source_parent)
        file_path = model.filePath(index)

        # ディレクトリは常に表示
        if model.isDir(index):
            return True

        if not self.extensions:
            return True

        suffix = Path(file_path).suffix.lower()
        return suffix in self.extensions
\end{lstlisting}

\begin{userbox}
機能しますが、フォルダをグレーアウトするのはいただけませんね。
\end{userbox}

\begin{assistantbox}
フォルダを常に表示するようにフィルタ設定を修正します。
\end{assistantbox}

\texttt{filterAcceptsRow}メソッドで、ディレクトリの場合は常に\texttt{True}を返すよう修正することで解決した。

\subsection{yt-srtスクリプトのエラーハンドリング修正}

\begin{userbox}
字幕ファイルもダウンロードされていませんけど。
\end{userbox}

\begin{assistantbox}
問題を発見しました。\texttt{yt-dlp}は字幕がなくても終了コード0を返します。字幕ファイルが実際に生成されたかを確認するように修正します。
\end{assistantbox}

\texttt{bin/yt-srt}スクリプトが、字幕ファイルが存在しない場合でも成功を返していた問題が修正された。実行前後のSRTファイル数を比較し、新しいファイルが生成されなかった場合はエラーを返すよう変更された。

\subsection{設定の保存/読み込み機能}

\begin{userbox}
この一連の処理を保存する機能を実装しましょう。
\end{userbox}

\begin{assistantbox}
どのような保存機能をお考えですか?

\begin{enumerate}
\item \textbf{設定の保存/読み込み} - ゴール、メタデータ、ファイルパス等をJSONで保存し、後で復元
\item \textbf{セッション自動保存} - アプリ終了時に自動保存、起動時に復元
\item \textbf{プロジェクトファイル} - 文字起こし結果やプロンプトも含めて一式保存
\end{enumerate}
\end{assistantbox}

\begin{userbox}
まずは、1ですかね。
\end{userbox}

JSON形式での設定保存/読み込み機能が実装された。保存される項目:
\begin{itemize}
\item ゴールタイプ
\item タイトル、日時、キーパーソン、組織、利用者
\item YouTube URL / 動画ファイルパス
\item チャプターファイル
\end{itemize}

\begin{userbox}
設定読込のダイアログも、ファイルを開くのと同じようにQtで実装してセンターになるようにしてください。
\end{userbox}

設定保存/読み込みダイアログにも\texttt{FileFilterProxyModel}を適用し、JSONファイルのみが表示されるよう修正された。

\subsection{ytdlスクリプトのGUI統合}

ユーザーは既存の\texttt{ytdl}関数(Claude CLIでファイル名を自動生成する機能付き)を共有した。

\begin{userbox}
元々は、こんなスクリプトで動画とSRTをダウンロードしてましたよ。
\end{userbox}

\begin{assistantbox}
高機能なスクリプトですね。主な特徴:

\begin{enumerate}
\item \textbf{Claudeでタイトル短縮} - ファイル名を自動生成
\item \textbf{リモート/ローカル自動選択} - zeusサーバーのClaude CLIを使用
\item \textbf{字幕リネーム} - \texttt{\{name\}.ja.srt} → \texttt{\{name\}\_yt.srt}
\item \textbf{Safariクッキー使用} - 認証済み動画もダウンロード可能
\end{enumerate}
\end{assistantbox}

\begin{userbox}
ダウンロードするビデオのファイル名を、GUIにて指定するように変更したいと思います。
\end{userbox}

\begin{userbox}
ダウンロードしたsrtは、ファイル名\_yt.srtにしてください。
\end{userbox}

\begin{userbox}
加えて、作業フォルダに同名の動画がある場合は、動画のダウンロードをスキップしてください。
\end{userbox}

これらの仕様変更に基づき、\texttt{bin/ytdl}スクリプトが作成され、GUIにファイル名入力フィールドが追加された。

\subsection{この日の成果}

\begin{enumerate}
\item \textbf{FileFilterProxyModel}の実装 - Qtダイアログでのファイルフィルタリング
\item \textbf{設定保存/読み込み}機能の追加 - JSON形式
\item \textbf{yt-srt}スクリプトの修正 - 字幕未取得時のエラー処理
\item \textbf{ytdl}スクリプトのGUI統合 - ファイル名指定、同名ファイルスキップ
\item ファイル名入力フィールドのUI追加
\end{enumerate}

\subsection{技術的なポイント}

この日の開発で重要だった技術的ポイント:

\begin{enumerate}
\item \textbf{Qtネイティブダイアログの制限}: 標準の\texttt{setNameFilter}はネイティブダイアログでは機能するが、Qtダイアログ(\texttt{DontUseNativeDialog})では機能しない
\item \textbf{QSortFilterProxyModelの活用}: カスタムプロキシモデルを作成することで、ファイルシステムモデルのフィルタリングを実現
\item \textbf{yt-dlpの終了コード}: 字幕が存在しなくても終了コード0を返すため、実際のファイル生成を確認する必要がある
\end{enumerate}

