% LuaLaTeX document
\documentclass[10pt,a4paper,twocolumn]{ltjsarticle}

% LuaLaTeX用フォント設定パッケージ
\usepackage{luatexja-fontspec}
\usepackage{amsmath,amssymb}
\usepackage{unicode-math}

% ====================
% 欧文フォント設定 (Libertinus)
% ====================
\setmainfont{Libertinus Serif}[
    BoldFont = {Libertinus Serif Bold},
    ItalicFont = {Libertinus Serif Italic},
    BoldItalicFont = {Libertinus Serif Bold Italic}
]
\setsansfont{Libertinus Sans}[
    BoldFont = {Libertinus Sans Bold},
    ItalicFont = {Libertinus Sans Italic}
]
\setmonofont{Libertinus Mono}

% ====================
% 日本語フォント設定 (原ノ味フォント)
% ====================
\setmainjfont{HaranoAjiMincho-Regular}[
    BoldFont = {HaranoAjiGothic-Medium},
    ItalicFont = {HaranoAjiMincho-Regular},
    BoldItalicFont = {HaranoAjiGothic-Bold}
]
\setsansjfont{HaranoAjiGothic-Regular}[
    BoldFont = {HaranoAjiGothic-Bold}
]
\setmonojfont{HaranoAjiGothic-Regular}

% ====================
% 数式フォント設定 (Libertinus Math)
% ====================
\setmathfont{Libertinus Math}

% ファイル生成日時(JST)
\newcommand{\generatedDate}{2026-01-08}
\newcommand{\generatedTime}{00:30}

% ヘッダー・フッター設定
\usepackage{fancyhdr}
\usepackage{lastpage}
\pagestyle{fancy}
\fancyhf{}
\fancyhead[R]{\small \generatedDate\ \generatedTime\ JST (\thepage/\pageref{LastPage})}
\renewcommand{\headrulewidth}{0.4pt}

% 1ページ目のスタイル
\fancypagestyle{firstpage}{
    \fancyhf{}
    \renewcommand{\headrulewidth}{0pt}
}

% 追加パッケージ
\usepackage{booktabs}
\usepackage{array}
\usepackage{tabularx}
\usepackage{ascmac}
\usepackage{listings}
\usepackage{xcolor}
\usepackage{hyperref}
\usepackage[margin=18mm]{geometry}

% コードスタイル設定
\lstset{
    basicstyle=\ttfamily\scriptsize,
    breaklines=true,
    frame=single,
    backgroundcolor=\color{gray!10},
    columns=fullflexible,
    keepspaces=true,
    showstringspaces=false
}

% Y列タイプ定義
\newcolumntype{Y}{>{\raggedright\arraybackslash}X}

\title{Video Chapter Editor 開発対話記録\\
\large 2025-11-05 〜 2026-01-07}
\author{ましDialogue}
\date{}

\begin{document}
\maketitle
\thispagestyle{firstpage}

\section*{概要}

本文書は、Video Chapter Editor(VCE)の開発過程における対話を詳細に記録したものである。CLIワークフローのプロトタイピングから、GUIリファクタリング、UI大改造、そして継続的な機能改善まで、約2ヶ月間の開発過程を網羅する。

\tableofcontents

%======================================
\part{プロトタイピング期}
%======================================

\section{CLIワークフロー設計 (2025-11-05)}

\subsection{ハイブリッドアプローチの選定}

\begin{itembox}[l]{問い}
YouTubeリンクからPDF + チャプターリストを自動生成するワークフローを構築したい。どのようなアプローチが最適か。
\end{itembox}

5つのアプローチを比較検討した結果、\textbf{Claude Code カスタムスラッシュコマンド + Zshヘルパー関数}のハイブリッドアプローチを採用した。

\vspace{0.5\baselineskip}
\noindent{\footnotesize
\begin{tabularx}{\linewidth}{@{}lYY@{}}
\toprule
評価項目 & 重み & ハイブリッド \\
\midrule
Claude統合 & 40\% & 完全自動 \\
保守性 & 20\% & 優秀 \\
エラー処理 & 15\% & 自動 \\
学習コスト & 15\% & 低 \\
実行効率 & 10\% & 対応可 \\
\bottomrule
\end{tabularx}
}
\vspace{0.5\baselineskip}

\begin{itembox}[l]{問い}
なぜMakefileやWorkflow Engineではなく、このハイブリッドアプローチなのか。
\end{itembox}

最大の理由は\textbf{Claude AI統合}である。SRTファイル(6000行以上)の分析と、音楽的に意味のあるLaTeXレポート生成は、AIの独壇場である。

\begin{lstlisting}[language=bash]
# 最適な責任分離
機械的処理 → Zsh関数
  - ファイルダウンロード (ytdl-claude)
  - Whisper起動 (whisper-remote)
  - PDFコンパイル (luatex-pdf)

AI判断処理 → Claude Code
  - SRT分析 (6000行以上の字幕)
  - 文脈理解 (音楽的意味)
  - 誤認識修正 (「ホルモン」→「ホルン」)
\end{lstlisting}

\subsection{3ステップワークフロー}

\begin{itembox}[l]{問い}
具体的なワークフローの流れは?
\end{itembox}

以下の3ステップで完結する:

\begin{lstlisting}[language=bash]
# Step 1: ダウンロード + Whisper起動
$ rehearsal-download "https://youtu.be/ID"
# → 動画.mp4, 動画_yt.srt
# → whisper-remote 起動

# Step 2: AI分析 + LaTeX生成
$ claude code
> /rehearsal
# → 両SRTを統合分析
# → LuaTeX形式レポート生成

# Step 3: PDF生成 + チャプター抽出
$ rehearsal-finalize "記録.tex"
# → PDF, YouTubeチャプター, Movie Viewerチャプター
\end{lstlisting}

\subsection{/rehearsalコマンドの設計}

\begin{itembox}[l]{問い}
Claude AIはSRTファイルをどのように分析するのか。
\end{itembox}

YouTube字幕とWhisper字幕の\textbf{相補的統合}を行う:

\begin{itemize}
    \item \textbf{YouTube字幕}: 時系列構造の把握(演奏中も継続記録)
    \item \textbf{Whisper字幕}: 指揮者の指示の高精度記録(発話内容の精度)
\end{itemize}

出力構造は8セクションで構成:
\begin{enumerate}
    \item リハーサル概要(基本情報、時系列、データソース)
    \item 時系列展開(曲/楽章ごとのセクション)
    \item パート別指示まとめ(著者パート向け抽出)
    \item 音楽用語集(イタリア語音楽用語)
    \item 擬音語パターン(指揮者の口唱記録)
    \item リハーサルの特徴(指揮者の教育スタイル)
    \item 重要な瞬間(キーシーン記録)
    \item Summary(総括、謝辞)
\end{enumerate}

\textbf{タイムスタンプ形式}: すべてのセクションに\texttt{[HH:MM:SS.mmm]}を付与し、後続のチャプター抽出に対応。

\section{GUIリファクタリング (2025-11-06)}

\subsection{元のGUIの分析}

\begin{itembox}[l]{問い}
元の\texttt{video\_analysis\_gui.py}の問題点は何か。
\end{itembox}

\textbf{強み}:
\begin{itemize}
    \item 包括的なプリセットシステム(5カテゴリー)
    \item 25フィールドの柔軟なカスタマイズ
    \item YAML出力によるバッチ処理対応
\end{itemize}

\textbf{弱み}:
\begin{itemize}
    \item 過剰な汎用性(リハーサル記録には不要な機能多数)
    \item ワークフロー不明確(実行順序が不明瞭)
    \item 既存コマンドとの分離(rehearsal-download等との統合なし)
    \item ファイル管理が手動(パスの手動入力)
\end{itemize}

\subsection{リファクタリング方針}

\begin{itembox}[l]{問い}
どのような設計方針でリファクタリングするのか。
\end{itembox}

5つの設計方針を定義:

\begin{enumerate}
    \item \textbf{専用化}: リハーサル記録作成のみに特化
    \item \textbf{ワークフロー明確化}: 3ステップを可視化
    \item \textbf{自動化}: ファイル検出、ステータス更新を自動化
    \item \textbf{統合}: 既存コマンドを直接呼び出し
    \item \textbf{可読性}: 詳細なコメント、明確な命名
\end{enumerate}

\subsection{データモデルの簡素化}

\begin{lstlisting}[language=Python]
# 元: VideoMetadata (25フィールド)
@dataclass
class VideoMetadata:
    url: str
    category: str  # music, education, ...
    content_type: str
    # ... 合計25フィールド

# 新: RehearsalMetadata (15フィールド)
@dataclass
class RehearsalMetadata:
    youtube_url: str
    rehearsal_date: str
    organization: str
    conductor: str
    piece_name: str
    # ... 合計15フィールド
    step: WorkflowStep  # 状態管理
\end{lstlisting}

\textbf{削減効果}: 25フィールド → 15フィールド(40\%削減)

\subsection{パフォーマンス改善}

\vspace{0.5\baselineskip}
\noindent{\footnotesize
\begin{tabularx}{\linewidth}{@{}lYY@{}}
\toprule
項目 & 元 & リファクタリング後 \\
\midrule
起動時メモリ & 約85MB & 約60MB(30\%削減) \\
起動時間 & 約1.2秒 & 約0.8秒(33\%短縮) \\
データクラス & 25フィールド & 15フィールド \\
プリセット & 35種類 & 0種類 \\
\bottomrule
\end{tabularx}
}
\vspace{0.5\baselineskip}

%======================================
\part{機能実装期}
%======================================

\section{エクスポート機能 (2025-12-26)}

\subsection{基本機能の実装}

\begin{itembox}[l]{問い}
動画書き出し機能の要件は何か。
\end{itembox}

\begin{itemize}
    \item ffmpegによる動画書き出し(QThread非同期処理)
    \item チャプターメタデータ埋め込み(FFMETADATA1形式)
    \item チャプター名の映像焼き込み(drawtext filter)
    \item 進捗バー表示(ffmpeg stderrパース)
\end{itemize}

\subsection{3タブ構成}

初期実装は3タブ構成:
\begin{enumerate}
    \item MP3結合タブ
    \item 編集タブ(波形表示、チャプターテーブル)
    \item 書出タブ
\end{enumerate}

\section{除外チャプター機能 (2025-12-27)}

\subsection{--プレフィックス仕様}

\begin{itembox}[l]{問い}
休憩や準備時間など、エクスポートから除外したいチャプターをどう扱うか。
\end{itembox}

チャプター名の先頭に\texttt{--}を付けることで、エクスポート時に自動除外する仕様を実装。

\begin{lstlisting}
# チャプターリスト例
00:00:00.000 オープニング
00:05:23.000 --休憩
00:15:00.000 第1曲 木星
00:20:30.000 --MC
00:22:00.000 第2曲 威風堂々
\end{lstlisting}

\subsection{波形ハッチング表示}

\begin{itembox}[l]{問い}
除外区間を視覚的に分かりやすく表示したい。
\end{itembox}

波形ウィジェットに除外区間の可視化を追加:
\begin{itemize}
    \item \texttt{\_get\_excluded\_regions()}: 除外区間の特定
    \item \texttt{paintEvent()}: 半透明赤背景 + 斜線ハッチング
    \item \texttt{itemChanged}シグナル: リアルタイム更新
\end{itemize}

\subsection{YouTubeチャプター連携}

\begin{itembox}[l]{問い}
YouTubeのチャプター形式との連携は?
\end{itembox}

\begin{itemize}
    \item \textbf{コピー機能}(📋ボタン): ミリ秒なし形式でクリップボードへ
    \item \textbf{貼り付け機能}(Cmd+V): YouTube形式をパースして読み込み
\end{itemize}

YouTube形式のパース仕様:
\begin{lstlisting}
# 対応形式
M:SS    タイトル   → 0:MM:SS.000
MM:SS   タイトル   → 0:MM:SS.000
H:MM:SS タイトル   → H:MM:SS.000
\end{lstlisting}

\subsection{0:00:00.000からの開始保証}

\begin{itembox}[l]{問い}
先頭が0秒でない場合の処理は?
\end{itembox}

以下の場合に\texttt{0:00:00.000 --開始}を自動追加:
\begin{itemize}
    \item 動画のみ読込時
    \item YouTube貼付け時(先頭が0でない場合)
    \item チャプター読込時(先頭が0でない場合)
\end{itemize}

%======================================
\part{UI大改造期}
%======================================

\section{一筆書き問題の認識 (2025-12-29)}

\subsection{グラフ理論的アプローチ}

\begin{itembox}[l]{問い}
現状のワークフローでMP3からMP4を生成する際、2回のエンコードが発生する問題がある。どう解決するか。
\end{itembox}

ワークフロー設計を\textbf{グラフ理論}の観点から分析した。問題の本質は\textbf{オイラー路(一筆書き)問題}に類似している。

\textbf{起点(入力パターン)}が3つ存在:
\begin{enumerate}
    \item 複数のカット済みMP3
    \item 単一の長尺未編集MP3
    \item 既存のMP4
\end{enumerate}

\textbf{処理ノード}の分析:

\vspace{0.5\baselineskip}
\noindent{\footnotesize
\begin{tabularx}{\linewidth}{@{}lY@{}}
\toprule
ノード & 必要な起点 \\
\midrule
結合 & 起点1のみ \\
カバー画像 & MP3入力時のみ \\
チャプター編集 & 全起点 \\
書出 & 全起点 \\
\bottomrule
\end{tabularx}
}
\vspace{0.5\baselineskip}

\textbf{洞察}: 機能重複は「起点が3つ以上ある」ことに起因。共通パスを明確にし、入口のみを分岐させることで重複を排除できる。

\subsection{制約による設計空間の縮小}

\begin{itembox}[l]{問い}
タイトル焼込を必須とした場合、設計はどう変わるか。
\end{itembox}

\textbf{制約なし}: 8パターンの設計空間

\textbf{制約追加後(タイトル焼込必須)}: 2パターンに縮小

\begin{lstlisting}
最適化されたエンコード戦略:

MP3入力:
  映像: enc(静止画+焼込) 1回
  音声: enc 1回

MP4入力:
  映像: enc(焼込) 1回
  音声: copy(無劣化)
\end{lstlisting}

\textbf{洞察}: 制約は自由度を狭めるが、設計空間を明確にし、最適解を見つけやすくする。

\subsection{Tab構成の検討}

\begin{itembox}[l]{問い}
2タブ構成(入力準備 + 編集・書出)でどうか。
\end{itembox}

問題点を指摘された:Tab 1のMP3結合は無劣化(-c copy)で可能。中間ファイルが不要ならタブを分ける意味が薄い。

\begin{itembox}[l]{問い}
タブ1とタブ2を分けなくても良いのでは?
\end{itembox}

陶器(UI)が巨大化する懸念があったが、「入力ソースも別画面では?」という提案からダイアログパターンの発見に至った。

\section{単一画面 + ダイアログ設計 (2025-12-29)}

\subsection{モーダル分離パターン}

\begin{itembox}[l]{問い}
最終的なUI構成はどうなったか。
\end{itembox}

\textbf{単一画面 + ダイアログ}のアーキテクチャを採用:

\begin{lstlisting}
メイン画面(MainWorkspace)
+-----------------------------------+
| [ソース選択] [カバー画像] ボタン  |
| ソース: audio.mp3 (14:20)         |
| カバー: cover.jpg (1920x1080)     |
|-----------------------------------|
| [波形表示]                        |
|-----------------------------------|
| [チャプターテーブル]              |
| | 00:00 | 第1曲 ホルスト 木星    |
| | 05:23 | 第2曲 エルガー 威風堂々|
|-----------------------------------|
| [書出設定] [書出]                 |
+-----------------------------------+

ダイアログ(モーダル)
- SourceSelectionDialog
- CoverImageDialog
\end{lstlisting}

\subsection{決定事項}

\begin{enumerate}
    \item \textbf{単一画面構成}: タブを廃止、メイン画面1つに統合
    \item \textbf{ダイアログパターン}: ソース選択・カバー画像は別画面
    \item \textbf{自動適用}: ダイアログを閉じると自動で反映
    \item \textbf{エンコード最適化}: 最終書出で1回のみ
\end{enumerate}

\subsection{設計原則の抽出}

本議論から得られた知見:

\begin{enumerate}
    \item \textbf{グラフ構造での問題分析}: ワークフローを有向グラフとして捉える
    \item \textbf{共通パスの抽出}: 複数起点の合流点を中心に設計
    \item \textbf{制約による単純化}: 制約は選択肢を狭めるが設計を明確化
    \item \textbf{モーダル分離}: 複雑な入力はダイアログに分離
    \item \textbf{自動適用の原則}: 明示的な「保存」ボタン不要
\end{enumerate}

%======================================
\part{機能拡張期}
%======================================

\section{文字起こしワークフロー設計 (2026-01-03)}

\subsection{TeX/LaTeXアナロジー}

\begin{itembox}[l]{問い}
既存のカスタムコマンド群(rehearsal.md: 302行、srt-meeting-report.md: 549行)はモノリシックで再利用性がない。どう改善するか。
\end{itembox}

TeX/LaTeXの構造に倣った設計を提案:

\vspace{0.5\baselineskip}
\noindent{\footnotesize
\begin{tabularx}{\linewidth}{@{}lYY@{}}
\toprule
TeX/LaTeX & 本システム & 役割 \\
\midrule
.tex & workflow.yaml & 具体的な値 \\
.cls/.sty & profiles/*.yaml & 構造定義 \\
\textbackslash section & fields.title & 内容記述 \\
マクロ展開 & プロンプト生成 & 変数展開 \\
\bottomrule
\end{tabularx}
}
\vspace{0.5\baselineskip}

\subsection{分離の原則}

\begin{lstlisting}
設定ファイル(YAML)     → 「何を」処理するか
プロファイル(YAML)     → 「どのような構造で」
プロンプト(Markdown)   → 「AIに何を指示するか」
テンプレート(TeX)      → 「どのような形式で」出力
\end{lstlisting}

\subsection{入力状態の場合分け}

\begin{itembox}[l]{問い}
YouTube経由とローカル動画で処理フローが異なる。どう整理するか。
\end{itembox}

7つの入力状態を列挙:

\vspace{0.5\baselineskip}
\noindent{\footnotesize
\begin{tabularx}{\linewidth}{@{}cYYYYY@{}}
\toprule
\# & 動画 & YT字幕 & Whisper & 処理 \\
\midrule
1 & YouTube URL & - & - & ytdl+任意Whisper \\
2 & ローカルのみ & - & - & Whisper/手動 \\
3 & ローカル & \checkmark & - & 任意Whisper追加 \\
4 & ローカル & - & \checkmark & 処理可能 \\
5 & ローカル & \checkmark & \checkmark & 処理可能(統合) \\
6 & ローカル & - & - & 手動SRTで処理 \\
7 & DL済み & - & - & yt-srt/Whisper \\
\bottomrule
\end{tabularx}
}
\vspace{0.5\baselineskip}

\subsection{YAMLライフサイクル}

YAMLファイルは\textbf{プロジェクトマニフェスト}として、全ワークフローの開始前に作成する。

\begin{lstlisting}[language=bash]
1. YAML作成(ユーザー)
   ├── プロファイル選択
   ├── ソース情報入力
   └── フィールド入力
       ↓
2. 前処理(Video Chapter Editor)
   ├── YAML読込 → source.path
   ├── トリミング・チャプター作成
   └── YAML更新 → source.state
       ↓
3. 文字起こし(Transcription Workflow)
   ├── YAML読込 → state確認
   ├── SRT取得(必要に応じて)
   └── AI処理 → 出力生成
\end{lstlisting}

\section{UI改善 (2026-01-05)}

\subsection{行番号表示}

\begin{itembox}[l]{問い}
Chaptersリストに行番号(No.)を表示し、ヘッダーを黒背景にしたい。
\end{itembox}

当初、新しいカラム「No.」を追加する方向で実装を進めたが、ユーザーの意図は自動で振られる行番号のヘッダー部分(コーナーウィジェット)に「No.」を表示することだった。

実装内容:
\begin{itemize}
    \item テーブルの垂直ヘッダー(行番号)を表示
    \item コーナーウィジェット(左上隅)に「No.」ラベルを配置
    \item ヘッダー背景を黒(\#000000)に設定
\end{itemize}

\subsection{チャプタースキップボタンの有効化条件}

\begin{itembox}[l]{問い}
チャプタースキップボタンはいつ有効になるべきか。
\end{itembox}

チャプターリストを編集した場合にのみ有効になるよう変更:

\begin{itemize}
    \item \texttt{\_chapters\_edited}フラグを追加(初期値: False)
    \item チャプター追加/削除/編集/ペースト時にTrueに設定
    \item 新しいメディア読み込み時にリセット
\end{itemize}

\section{v2.1.27リリース (2026-01-06)}

\subsection{デュアルアーキテクチャビルド}

\begin{itembox}[l]{問い}
Intel Macユーザーにも配布したい。
\end{itembox}

GitHub Actionsで両アーキテクチャを並行ビルド:

\vspace{0.5\baselineskip}
\noindent{\footnotesize
\begin{tabularx}{\linewidth}{@{}lYY@{}}
\toprule
ランナー & アーキテクチャ & 出力 \\
\midrule
macos-13 & Intel x86\_64 & *-Intel.dmg \\
macos-latest & Apple Silicon & *-AppleSilicon.dmg \\
\bottomrule
\end{tabularx}
}
\vspace{0.5\baselineskip}

\subsection{ffmpeg/ffprobeバンドル}

\begin{itembox}[l]{問い}
\texttt{imageio-ffmpeg}はffprobeを含まない。動画の長さ取得に問題がある。
\end{itembox}

\texttt{static-ffmpeg}パッケージに移行:

\vspace{0.5\baselineskip}
\noindent{\footnotesize
\begin{tabularx}{\linewidth}{@{}lYY@{}}
\toprule
パッケージ & ffmpeg & ffprobe \\
\midrule
imageio-ffmpeg & \checkmark & × \\
static-ffmpeg & \checkmark & \checkmark \\
\bottomrule
\end{tabularx}
}
\vspace{0.5\baselineskip}

\subsection{AV1コーデック除外}

\begin{itembox}[l]{問い}
macOSでAV1のハードウェアデコードが非対応の問題がある。
\end{itembox}

format文字列を変更し、H.264優先・AV1除外に:

\begin{lstlisting}
# 変更後のformat文字列
bv[vcodec^=avc1]+ba/
bv[vcodec!^=av01]+ba/b
\end{lstlisting}

%======================================
\part{リファクタリング期}
%======================================

\section{styles.pyの作成 (2026-01-07)}

\subsection{ボタンスタイルの統一}

\begin{itembox}[l]{問い}
\texttt{\_button\_style()}が5箇所で重複定義されている。統一できないか。
\end{itembox}

\texttt{rehearsal\_workflow/ui/styles.py}を新規作成:

\begin{lstlisting}[language=Python]
class Colors:
    PRIMARY = "#3b82f6"
    PRIMARY_HOVER = "#2563eb"
    DANGER = "#dc2626"
    BACKGROUND_DARK = "#1a1a1a"
    # ...

class ButtonStyles:
    @staticmethod
    def primary() -> str:
        return f"""
            QPushButton {{
                background: {Colors.PRIMARY};
                padding: 8px 16px;
                border-radius: 6px;
            }}
        """

    @staticmethod
    def primary_compact() -> str:
        # padding 4px 12px
\end{lstlisting}

\subsection{Export → Encode名称変更}

\begin{itembox}[l]{問い}
Exportというよりは、Encodeですよね。
\end{itembox}

確かに動画のエンコード処理なので「Encode」が正確である。ボタンテキストを変更した。

\subsection{レイアウト改善}

\begin{itembox}[l]{問い}
Encode、Settingsの順に配置して、右側にエンコードの進捗をバーグラフで表示する仕様に変更しましょうか。
\end{itembox}

以下のレイアウトに変更:

\begin{lstlisting}
[Encode] [Settings]  [========== 80%]
                     ^ エンコード中のみ表示
\end{lstlisting}

\section{複数音声ファイルのマージ問題}

\begin{itembox}[l]{問い}
音声からエンコードしようとするとエラーが発生する。
\end{itembox}

ログを確認すると、8つのMP3ファイルを読み込んでいるにもかかわらず、ffmpegには1つのファイルしか渡されていなかった。

\textbf{原因}:

\begin{lstlisting}[language=Python]
# 修正前
if not input_path and len(self._state.sources) > 1:
    # マージ処理

# 修正後
if len(self._state.sources) > 1:
    # マージ処理
\end{lstlisting}

複数ソースがある場合でも\texttt{video\_path}が設定されていたため、マージ処理がスキップされていた。

\section{オーバーレイ表示位置の統一}

\begin{itembox}[l]{問い}
複数音声ファイルのエンコードのオーバーレイの文字が下に表示されている。プレビューとエンコードで位置が異なる。
\end{itembox}

プレビュー時とエンコード時でオーバーレイ位置が不一致だった:

\vspace{0.5\baselineskip}
\noindent{\footnotesize
\begin{tabularx}{\linewidth}{@{}lYY@{}}
\toprule
 & プレビュー & エンコード \\
\midrule
位置 & 85\%(下部) & 32.5\%(上部寄り中央) \\
フォントサイズ & 3.5\% & 4\% \\
\bottomrule
\end{tabularx}
}
\vspace{0.5\baselineskip}

プレビューをエンコードと同じ位置(32.5\%、フォントサイズ4\%)に修正した。

\section{Undo/Redo機能の設計検討}

\begin{itembox}[l]{問い}
直前のチャプターリストを保存してUndoしたい場合に備えるのって大変ですか。外部ファイルに依存しない形で実装したい。
\end{itembox}

メモリ内でスタック(履歴)を保持するだけで実現できる。比較的簡単な実装である。

\begin{itembox}[l]{問い}
どの状態で持つのが良いか。チャプターリストだけでなく、出力のベースファイル名も保存したい。
\end{itembox}

「編集セッション」としてまとめることを提案:

\begin{lstlisting}[language=Python]
@dataclass
class EditSession:
    chapters: list[ChapterInfo]
    output_basename: str
    # 必要に応じて追加
\end{lstlisting}

\begin{itembox}[l]{問い}
編集セッションの表示は別に行いますか?
\end{itembox}

履歴パネルでの表示を検討。操作名(「チャプター追加」「エンコード完了後」など)を記録し、クリックで任意の状態に復元できるUIを想定。

\begin{itembox}[l]{問い}
別のタスクで後回しですね。
\end{itembox}

DEVELOPMENT\_LOG.mdの「今後の予定」セクションに追記:

\begin{itemize}
    \item \textbf{Undo/Redo + 履歴パネル機能}
    \item 編集セッションを単位として保存
    \item 履歴パネルで操作一覧表示、クリックで復元
    \item メモリ内で管理(外部ファイル非依存)
\end{itemize}

%======================================
\section*{Claude Codeの所感}
%======================================

約2ヶ月間のVCE開発過程を振り返って、いくつかの観察と反省を述べる。

\subsection*{良かった点}

\textbf{グラフ理論的アプローチの発見}は、本プロジェクトにおける最大の知見であった。ワークフロー設計を「一筆書き問題」として捉え直すことで、機能重複の本質的原因が明らかになった。制約を追加することで設計空間が縮小し、最適解が見つけやすくなるという逆説的な効果も興味深い。

\textbf{TeX/LaTeXアナロジー}による文字起こしワークフロー設計も、既存の概念モデルを活用した良い例である。.texファイルと.clsファイルの関係を、設定ファイルとプロファイルの関係に対応させることで、分離の原則が自然に導かれた。

\textbf{ユーザーとの対話}を通じて、意図を正確に把握する重要性を再認識した。「行番号を表示したい」という要望に対して、新しいカラムを追加するのではなく、コーナーウィジェットに「No.」を表示するという意図を汲み取れたのは良かった。

\subsection*{反省点}

\textbf{Settings/Exportセクションの改善}では、最初にボタンサイズを変更してしまった。「無駄に広い」という表現から、スペースの問題ではなくサイズの問題と早合点してしまった。ユーザーの意図を正確に把握することの重要性を示す反面教師となった。

\textbf{複数音声ファイルのマージ処理のバグ}は、条件式\texttt{if not input\_path and ...}の問題であった。単体テストでは見つけにくい類の不具合であり、実際の使用シナリオでのテストの重要性を再認識した。

\subsection*{今後の課題}

Undo/Redo機能は後回しとなったが、設計の方向性は明確になった。履歴パネルのUI設計(配置場所、表示形式、操作性)については、実装時に詳細な検討が必要である。

また、main\_workspace.pyが5,162行、133メソッドという巨大なGod Classになっている問題がある。Phase 3として責務分離(ChapterManager、MediaPlaybackController等)を検討しているが、機能影響を最小限に抑えながらのリファクタリングは慎重に進める必要がある。

全体として、ユーザーとの対話を通じてプロダクトが着実に改善されていく過程は、ソフトウェア開発の理想的な形の一つだと感じた。特に「配管と陶器」の思想に基づく設計は、Unix哲学の現代的な適用例として興味深い。

\end{document}
