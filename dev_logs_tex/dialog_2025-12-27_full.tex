%% dialog_2025-12-27_full.tex
%% 2025年12月27日の開発対話記録(750メッセージ)

\section{2025年12月27日:パッケージ化とリリース準備}

この日は、前日に実装されたチャプター除外機能のUI改善(波形ハッチング)から始まり、パッケージのリネーム、pip install対応、GitHub Actionsによるリリース自動化まで、配布に向けた大規模な整備作業が行われた。

\subsection{波形へのハッチング表示}

\begin{userbox}
ありがとうございます。確認した範囲では大丈夫そうです。加えて、--がついたチャプターの波形にハッチをかけるなど、どこがカットされるか識別しやすいようにできますか。
\end{userbox}

WaveformWidgetに除外区間のハッチング機能が実装された:

\begin{lstlisting}[language=python,basicstyle=\ttfamily\tiny]
def _get_excluded_regions(self) -> List[Tuple[int, int]]:
    """除外チャプター(--で始まる)の区間を取得"""
    if not self.chapters or self.duration_ms <= 0:
        return []
    excluded_regions = []
    sorted_chapters = sorted(self.chapters, key=lambda c: c.time_ms)
    for i, ch in enumerate(sorted_chapters):
        if ch.title.startswith("--"):
            start_ms = ch.time_ms
            if i + 1 < len(sorted_chapters):
                end_ms = sorted_chapters[i + 1].time_ms
            else:
                end_ms = self.duration_ms
            excluded_regions.append((start_ms, end_ms))
    return excluded_regions
\end{lstlisting}

\begin{userbox}
どのタイミングでハッチがかかりますか?チャプターリストのバッファに変化があった場合に、ハッチがかかるようにしてください。
\end{userbox}

チャプターテーブルの\texttt{itemChanged}シグナルに接続し、リアルタイムでハッチングが更新されるよう改善された。

\subsection{YouTubeチャプター貼り付け機能}

\begin{userbox}
Youtubeのチャプターを貼り付ける機能です。Cmd(ctrl)-Vで、クリップボードにyoutubeのチャプター形式がある場合、このテーブルに貼り付ける機能を実装してください。
\end{userbox}

\begin{assistantbox}
YouTubeチャプター形式の貼り付け機能を実装します。Cmd+V(Mac)/ Ctrl+V(Windows/Linux)を検出し、YouTube形式なら貼り付け、そうでなければ通常の動作を行います。
\end{assistantbox}

サポートする形式:
\begin{itemize}
\item \texttt{0:00 タイトル} - M:SS形式
\item \texttt{00:00 タイトル} - MM:SS形式
\item \texttt{1:23:45 タイトル} - H:MM:SS形式
\item \texttt{01:23:45 タイトル} - HH:MM:SS形式
\end{itemize}

\subsection{0:00:00.000開始の明示的処理}

\begin{userbox}
0:00:00.000から始まらない問題に対応しましょう。動画のみを読み込んだ場合、0:00:00.000 開始を必ず入れるしましょう。また、Youtubeのチャプターを貼り付ける場合およびチャプターファイルを読み込む場合、0:00:00.000から始まらない場合は「0:00:00.000 --開始」を自動的に入れる。0:00:00.000からある場合はそのまま。という使用はいかがでしょうか。
\end{userbox}

\begin{assistantbox}
良い仕様ですね。明示的で予測しやすい動作になります。
\end{assistantbox}

\begin{userbox}
UNIX的ではないですけどね。
\end{userbox}

\begin{assistantbox}
おっしゃる通りです。暗黙の動作は予測しにくい。

\textbf{UNIX的なアプローチ:明示的に指定されたものだけを除外する}

冒頭を除外したければ、ユーザーが \texttt{0:00:00.000 --準備} を追加する。現状の実装のままで問題ありません。
\end{assistantbox}

最終的に、ユーザーが明示的にチャプターを追加する必要があるという方針が採用された。ただし、利便性のため:
\begin{itemize}
\item \textbf{動画のみ読込}: \texttt{0:00:00.000 開始}を自動追加
\item \textbf{YouTube貼付け/チャプター読込}: 0から始まらない場合は\texttt{0:00:00.000 --開始}を追加
\end{itemize}

\subsection{未使用ファイルの削除とパッケージ整理}

\begin{userbox}
./concat\_mp4\_with\_chapters.pyって使ってませんよね。
\end{userbox}

\begin{assistantbox}
使っていません。prep\_gui.pyで同等機能が実装されています。削除しますか?
\end{assistantbox}

\begin{userbox}
削除しましょう。
\end{userbox}

続いて、大量のテストメディアファイル(約1.6GB)も削除された。\texttt{.gitignore}が更新され、以下が追加された:

\begin{lstlisting}[basicstyle=\ttfamily\tiny]
# メディアファイル
*.mp3
*.m4a
*.wav

# 画像ファイル(ドキュメント用は除外)
*.jpg
*.jpeg
*.png
!docs/**/*.jpg
!docs/**/*.png
\end{lstlisting}

\subsection{pip install対応}

\begin{userbox}
pipでインストールできるようにしてください。
\end{userbox}

\texttt{pyproject.toml}が作成され、パッケージ構造が整備された:

\begin{lstlisting}[basicstyle=\ttfamily\tiny]
[build-system]
requires = ["hatchling"]
build-backend = "hatchling.build"

[project]
name = "rehearsal-workflow"
version = "1.1.0"
dependencies = [
    "PySide6>=6.6.0",
    "PyYAML>=6.0",
    "numpy>=1.24.0",
    "opencv-python>=4.8.0",
]

[project.scripts]
video-chapter-editor = "rehearsal_workflow.video_chapter_editor:main"
report-workflow = "rehearsal_workflow.report_workflow:main"
\end{lstlisting}

\subsection{コマンド名のリネーム}

\begin{userbox}
pre\_guiとworkflow\_gui、guiであることはわかっているので、機能が想像できる適切な名称に変えたいですね。
\end{userbox}

\begin{assistantbox}
現在の機能から考えると:

\vspace{0.5\baselineskip}
\noindent{\footnotesize
\begin{tabularx}{\linewidth}{@{}lXX@{}}
\toprule
現在 & 機能 & 提案 \\
\midrule
prep\_gui & 動画編集+チャプター+書出 & video-chapter / chapter-editor \\
workflow\_gui & 汎用ワークフロー全体 & rehearsal-flow / report-flow \\
\bottomrule
\end{tabularx}
}
\vspace{0.5\baselineskip}
\end{assistantbox}

\begin{userbox}
video-chapter-editorとreport-workflowにしましょうか。長いですかね。
\end{userbox}

\begin{assistantbox}
CLIコマンドとしては少し長いですが、明確で良いと思います。
\end{assistantbox}

\begin{userbox}
そうですね。気に入らなければ、自分でシンボリックリンクを作れば良いですね。
\end{userbox}

以下のリネームが実行された:
\begin{itemize}
\item \texttt{gui/} → \texttt{rehearsal\_workflow/}
\item \texttt{prep\_gui.py} → \texttt{video\_chapter\_editor.py}
\item \texttt{workflow\_gui.py} → \texttt{report\_workflow.py}
\end{itemize}

\subsection{GitHub Actions リリース自動化}

\begin{userbox}
ですね。あとactionsに登録して、vceだけリリースしてください。その際、バイナリはワンファイルとしフォルダをドロップするとそこを作業用ディレクトリとして起動する仕様にしてください。
\end{userbox}

GitHub Actionsワークフロー(\texttt{.github/workflows/release.yml})が作成された:

\begin{lstlisting}[language=yaml,basicstyle=\ttfamily\tiny]
name: Release
on:
  push:
    tags:
      - 'v*'

jobs:
  build-macos:
    runs-on: macos-latest
    steps:
      - name: Build with PyInstaller
        run: pyinstaller video_chapter_editor.spec
      - name: Create DMG
        run: |
          hdiutil create -volname "Video Chapter Editor" \
            -srcfolder dmg_contents \
            -ov -format UDZO \
            "Video-Chapter-Editor-${{ github.ref_name }}-macOS.dmg"

  build-windows:
    runs-on: windows-latest
    steps:
      - name: Build with PyInstaller
        run: |
          pyinstaller --onefile --windowed --name "VideoChapterEditor" ...
\end{lstlisting}

\begin{userbox}
Windows版も作ってください。
\end{userbox}

Windows版のビルドジョブも追加された。

\subsection{フォルダドロップ対応}

フォルダをドロップすると作業ディレクトリとして起動する機能が実装された:

\begin{lstlisting}[language=python,basicstyle=\ttfamily\tiny]
def main():
    import argparse
    parser = argparse.ArgumentParser(
        description="動画チャプター編集・書出ツール",
        prog="video-chapter-editor"
    )
    parser.add_argument(
        "working_dir",
        nargs="?",
        default=None,
        help="作業ディレクトリ(フォルダをドロップして起動可能)"
    )
    args = parser.parse_args()
    # ... validates and uses working_dir
\end{lstlisting}

PyInstallerの設定(\texttt{video\_chapter\_editor.spec})には、macOSでのフォルダドラッグ\&ドロップ対応のため\texttt{argv\_emulation=True}が設定された。

\subsection{この日の成果}

\begin{enumerate}
\item \textbf{波形ハッチング}の実装 - 除外区間の視覚的表示
\item \textbf{YouTubeチャプター貼り付け} - Cmd+Vでの貼り付け機能
\item \textbf{0:00:00.000開始処理} - 明示的なチャプター追加
\item \textbf{未使用ファイル削除} - 1.6GB削減
\item \textbf{pip install対応} - pyproject.tomlによるパッケージ化
\item \textbf{コマンド名リネーム} - video-chapter-editor, report-workflow
\item \textbf{GitHub Actions} - macOS/Windows自動リリース
\item \textbf{フォルダドロップ対応} - 引数による作業ディレクトリ指定
\end{enumerate}

\subsection{技術的なポイント}

\begin{enumerate}
\item \textbf{UNIX哲学の議論}: 暗黙の動作より明示的な指定を優先する設計方針が確認された
\item \textbf{hatchlingビルドシステム}: setuptoolsではなくhatchlingを採用
\item \textbf{PyInstaller + argv\_emulation}: macOSでのフォルダドロップを実現
\item \textbf{クロスプラットフォームビルド}: GitHub Actionsでmacosおよびwindows-latestランナーを使用
\end{enumerate}

