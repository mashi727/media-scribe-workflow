% LuaLaTeX document
\documentclass[10pt,a4paper,twocolumn]{ltjarticle}

% ファイル生成日時(JST)
\newcommand{\generatedDate}{2026-01-11}
\newcommand{\generatedTime}{06:30}

% LuaLaTeX用フォント設定パッケージ
\usepackage{luatexja-fontspec}
\usepackage{amsmath,amssymb}
\usepackage{unicode-math}

% 欧文フォント設定 (Libertinus)
\setmainfont{Libertinus Serif}[
    BoldFont = {Libertinus Serif Bold},
    ItalicFont = {Libertinus Serif Italic},
    BoldItalicFont = {Libertinus Serif Bold Italic}
]
\setsansfont{Libertinus Sans}[
    BoldFont = {Libertinus Sans Bold},
    ItalicFont = {Libertinus Sans Italic}
]
\setmonofont{DejaVu Sans Mono}[Scale=0.8]

% 日本語フォント設定 (原ノ味フォント)
\setmainjfont{HaranoAjiMincho-Regular}[
    BoldFont = {HaranoAjiGothic-Medium},
    ItalicFont = {HaranoAjiMincho-Regular},
    BoldItalicFont = {HaranoAjiGothic-Bold}
]
\setsansjfont{HaranoAjiGothic-Regular}[
    BoldFont = {HaranoAjiGothic-Bold}
]
\setmonojfont{HaranoAjiGothic-Regular}

% 数式フォント設定
\setmathfont{Libertinus Math}

% パッケージ
\usepackage{geometry}
\geometry{left=18mm,right=18mm,top=25mm,bottom=20mm}
\usepackage{fancyhdr}
\usepackage{lastpage}
\usepackage{ascmac}
\usepackage{booktabs}
\usepackage{array}
\usepackage{tabularx}
\usepackage{listings}
\usepackage{xcolor}
\usepackage{hyperref}

% listings設定
\lstset{
    basicstyle=\ttfamily\scriptsize,
    breaklines=true,
    frame=single,
    columns=fullflexible,
    keepspaces=true,
    backgroundcolor=\color{gray!10}
}

% ヘッダー・フッター設定
\pagestyle{fancy}
\fancyhf{}
\fancyhead[R]{\small \generatedDate\ \generatedTime\ JST (\thepage/\pageref{LastPage})}
\renewcommand{\headrulewidth}{0.4pt}

% 1ページ目のスタイル
\fancypagestyle{firstpage}{
    \fancyhf{}
    \renewcommand{\headrulewidth}{0pt}
}

% タイトル
\title{\textbf{VCE機能図の設計対話}\\
\large ---処理・コンテンツ・クラスの関係性を整理する---}
\author{massy--Claude Dialogue}
\date{}

\begin{document}
\maketitle
\thispagestyle{firstpage}

\section{はじめに}

本文書は、VCE(Video Chapter Editor)の機能図(vce\_functional\_diagram.mmd)を設計する過程での対話を記録したものである。機能図の目的、構成要素の分類、および2軸モデルの設計に至るまでの議論を収録している。

\section{リポジトリの目的の明確化}

\begin{itembox}[l]{問い}
VCEのアプリについて、一つずつ上流から整理しましょう。
\end{itembox}

VCEアプリを上流から整理していく方針で進めることにした。

\begin{itembox}[l]{問い}
vce\_functional\_diagram.mmdに関して、認識を揃えてこの図を完成させたいと思います。
\end{itembox}

現在の図の構成を確認した結果、以下の構造であった。

\begin{itemize}
    \item Activity層: 動画入手、素材準備、編集作業
    \item External層: YouTube、ローカルファイル、前処理ツール
    \item VCE層: UI操作、表示機能、編集機能、内部Manager
    \item Output層: プロジェクト、動画出力、テキスト出力
    \item CLI層: vce-encode、vce-split
\end{itemize}

\begin{itembox}[l]{問い}
Activity層に編集作業はいらないですね。VCEでも編集しますので、未編集を前提としています。複数音声は微妙ですけど。
\end{itembox}

Activity層から「編集作業」を削除。VCEへの入力は未編集素材が前提であり、編集はVCE内で行う。

\begin{itembox}[l]{問い}
曲だけを切り出した音声の結合も行えるから。
\end{itembox}

VCEは音声ファイル(MP3)も入力として受け付け、仮想タイムラインで結合してチャプターを付与できる。複数MP4/MP3には以下の2つの意味がある。

\begin{itemize}
    \item 曲切り出し済み: 各ファイルが1曲に対応
    \item レコーダー都合: 連続素材が分割されている
\end{itemize}

VCEはこれらを区別せず、ユーザーが解釈して使用する。

\section{VCEの目的の再定義}

\begin{itembox}[l]{問い}
用途は、編集して配布用のMP4を作成する。がVCEが行うこと。VCEで行う編集とは、音声のノーマライズ、要不要部分の指定、チャプター位置の指定、カバー画像の設定(音声のみ、必要に応じて)のこと。という感じですかね。
\end{itembox}

VCEの目的と編集機能を以下のように整理した。

\vspace{0.5\baselineskip}
\noindent{\footnotesize
\begin{tabularx}{\linewidth}{@{}lX@{}}
\toprule
編集項目 & 説明 \\
\midrule
音声ノーマライズ & 音量調整 \\
要不要部分の指定 & 除外区間設定 \\
チャプター位置指定 & チャプターマーカー設定 \\
カバー画像設定 & 音声のみの場合、必要に応じて \\
\bottomrule
\end{tabularx}
}
\vspace{0.5\baselineskip}

\begin{itembox}[l]{問い}
あ、VCEの目的ではなくて、このリポジトリのワークフローが目指すところということですね。念の為。
\end{itembox}

重要な補足があった。「メディアの再利用性・相互運用性向上」はVCE単体ではなく、リポジトリ全体(media-scribe-workflow)のワークフローが目指すところである。

\vspace{0.5\baselineskip}
\noindent{\footnotesize
\begin{tabularx}{\linewidth}{@{}lX@{}}
\toprule
概念 & 説明 \\
\midrule
標準化 & 再利用性・相互運用性を高める形式への変換 \\
配布 & 再利用可能な形で提供(他者・自分) \\
Digitalization & 散在するメディアを整理・構造化 \\
\bottomrule
\end{tabularx}
}
\vspace{0.5\baselineskip}

\section{改行表示の問題}

\begin{itembox}[l]{問い}
えっと、改行の\textbackslash nが見えるんですけど、これってちゃんと改行できるようになりませんかね。
\end{itembox}

MermaidのPNG出力で\texttt{\textbackslash n}がそのまま表示されていた問題。Mermaidでは\texttt{\textbackslash n}ではなく\texttt{<br>}を使用することで改行が正しくレンダリングされる。

\section{グラフ構造の議論}

\begin{itembox}[l]{問い}
あと、この図の趣旨に反するかもしれませんが、最終的にグラフとしてみたいという思いがあるので、処理(状態を変化させる機能)、コンテンツ(対象)と、処理をまとめたクラスという感じで記述してもらいたいですね。
\end{itembox}

グラフ理論的な観点からの整理が求められた。以下の3種類を区別する。

\vspace{0.5\baselineskip}
\noindent{\footnotesize
\begin{tabularx}{\linewidth}{@{}lXl@{}}
\toprule
種別 & 説明 & 形状案 \\
\midrule
コンテンツ & 処理対象のデータ & 角丸四角 \\
処理 & 状態を変化させる機能 & 六角形 \\
クラス & 処理をまとめた実装 & 四角 \\
\bottomrule
\end{tabularx}
}
\vspace{0.5\baselineskip}

\begin{itembox}[l]{問い}
なるほど、こうきましたか。これ、レイヤで表現できないですか?
\end{itembox}

subgraphを使用したレイヤー構造を試みた。Manager層、処理層、コンテンツ層の3層構造。

\begin{itembox}[l]{問い}
もしかしてPAD図の方が良い?
\end{itembox}

PAD図との比較検討を行った。

\vspace{0.5\baselineskip}
\noindent{\footnotesize
\newcolumntype{Y}{>{\raggedright\arraybackslash}X}
\begin{tabularx}{\linewidth}{@{}lYY@{}}
\toprule
観点 & Mermaid & PAD \\
\midrule
得意 & グラフ構造、データフロー & 手続き的フロー、階層構造 \\
表現 & ノード間の多対多関係 & 処理の順序・分岐・繰返し \\
\bottomrule
\end{tabularx}
}
\vspace{0.5\baselineskip}

結論として、全体図(アーキテクチャ)にはMermaid、詳細図(処理フロー)にはPADという使い分けが適切。

\section{Manager層と処理層の関係}

\begin{itembox}[l]{問い}
ちょっと考えましょう。コンテンツ層の縦の流れは、良いと思います。マネージャ層と処理層の関係が難しいですね。マネージャ層は、実装するクラスになりますか?
\end{itembox}

3層モデルにおけるManager層と処理層の関係について議論した。

\begin{itemize}
    \item Manager層: 実装クラス(SourceFileManager等)
    \item 処理層: 状態変化の機能(ファイル分類等)
\end{itemize}

問題点として、Managerは処理を「実装する」クラスであり、処理はManagerの「メソッド」として存在するため、両方を別レイヤーにすると冗長になる可能性がある。

\section{案Bの採用}

\begin{itembox}[l]{問い}
最終的に実装を行うクラスとするならば、処理をブラッシュアップして、同じ処理を適用する良否が見えやすい案Bが良いかと、案Aだとクラスあり気になってしまいますね。どう考えますか
\end{itembox}

案Bの優位性について合意した。

\vspace{0.5\baselineskip}
\noindent{\footnotesize
\newcolumntype{Y}{>{\raggedright\arraybackslash}X}
\begin{tabularx}{\linewidth}{@{}lY@{}}
\toprule
案 & 特徴 \\
\midrule
案A(クラスありき) & クラス設計→処理を割り当て。既存のクラス構造に引きずられる \\
案B(処理ありき) & 処理を洗い出し→類似性を発見→クラスに集約。処理の本質から設計を見直せる \\
\bottomrule
\end{tabularx}
}
\vspace{0.5\baselineskip}

案Bで見えること:
\begin{enumerate}
    \item 処理の重複がないか
    \item 責務の偏りがないか
    \item グルーピングの妥当性
\end{enumerate}

\section{2軸モデルの設計}

\begin{itembox}[l]{問い}
コンテンツを左に、縦軸に時間、状態の変化、および目的の意味を持たせましょう。また、横軸には、目的と手段の意味を持たせましょうか。
\end{itembox}

2軸モデルの設計に至った。

\begin{itemize}
    \item 縦軸(Y): 時間・状態変化・目的
    \item 横軸(X): 目的→手段(What→How→Who)
\end{itemize}

\begin{lstlisting}
     目的 <-----------> 手段
      |    What    How    Who
      |  コンテンツ  処理   実装
時間  |
状態  |  MP4/MP3 -> 分類 -> SrcMgr
変化  |     |
      |  SourceFile -> 追加 -> SrcMgr
目的  |     |
      |  Chapter -> 章編集 -> ChapMgr
      |     |
      |  .vce.json -> 保存 -> ChapMgr
      |     |
      |  出力MP4 -> 変換 -> ExpOrc
      v
\end{lstlisting}

\begin{itembox}[l]{問い}
どうです?いけそうですか
\end{itembox}

Mermaidのsubgraph機能を使用して実現可能。各列をsubgraphで作成し、列内はdirection TBで縦方向(時間軸)、全体はflowchart LRで横方向(目的→手段)とする。

\section{Claude Code氏の所感}

本対話を通じて、ソフトウェア設計における概念モデリングの重要性を改めて認識した。

\subsection{設計思想について}

「処理ありき」(案B)の採用は、ドメイン駆動設計(DDD)における「ユビキタス言語」の考え方と通底する。クラス構造から出発するのではなく、まず「何をするか」(処理)を明確にし、その後で「誰がするか」(実装)を決定するアプローチは、責務の適切な配置を促進する。

\subsection{2軸モデルの意義}

縦軸に「時間・状態変化・目的」、横軸に「目的→手段」を配置する2軸モデルは、以下の観点で有用である。

\begin{enumerate}
    \item \textbf{因果関係の可視化}: コンテンツの状態遷移と、それを引き起こす処理の関係が明確になる
    \item \textbf{責務の検証}: 処理がどのクラスに属するべきかを、処理の性質から判断できる
    \item \textbf{リファクタリングの指針}: 類似処理の発見や、責務の偏りの検出が容易になる
\end{enumerate}

\subsection{批判的観点}

一方で、以下の課題も認識している。

\begin{itemize}
    \item \textbf{複雑性}: 2軸モデルは理解に一定の学習コストを要する
    \item \textbf{動的側面の欠如}: 実行時のイベント駆動的な振る舞いは、この静的な図では表現しきれない
    \item \textbf{ツールの制約}: Mermaidのレイアウト制御には限界があり、意図した配置にならない場合がある
\end{itemize}

\subsection{学際的観点}

本対話は、情報アーキテクチャ、認知科学、グラフ理論の交点に位置する。「目的→手段」の軸は手段目的連鎖分析(Means-End Chain)を想起させ、マーケティングや意思決定理論との接点がある。また、コンテンツの状態遷移を縦軸とする発想は、オートマトン理論やペトリネットとの親和性も感じられる。

設計図の作成は、単なるドキュメンテーションではなく、設計思考そのものを深化させるプロセスであることを、本対話は示している。

\end{document}
