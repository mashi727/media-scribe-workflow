% LuaLaTeX document
\documentclass[10pt,a4paper,twocolumn]{ltjsarticle}

% LuaLaTeX用フォント設定パッケージ
\usepackage{luatexja-fontspec}
\usepackage{amsmath,amssymb}
\usepackage{unicode-math}

% ====================
% 欧文フォント設定 (Libertinus)
% ====================
\setmainfont{Libertinus Serif}[
    BoldFont = {Libertinus Serif Bold},
    ItalicFont = {Libertinus Serif Italic},
    BoldItalicFont = {Libertinus Serif Bold Italic}
]
\setsansfont{Libertinus Sans}[
    BoldFont = {Libertinus Sans Bold},
    ItalicFont = {Libertinus Sans Italic}
]
\setmonofont{Libertinus Mono}

% ====================
% 日本語フォント設定 (原ノ味フォント)
% ====================
\setmainjfont{HaranoAjiMincho-Regular}[
    BoldFont = {HaranoAjiGothic-Medium},
    ItalicFont = {HaranoAjiMincho-Regular},
    BoldItalicFont = {HaranoAjiGothic-Bold}
]
\setsansjfont{HaranoAjiGothic-Regular}[
    BoldFont = {HaranoAjiGothic-Bold}
]
\setmonojfont{HaranoAjiGothic-Regular}

% ====================
% 数式フォント設定 (Libertinus Math)
% ====================
\setmathfont{Libertinus Math}

% ファイル生成日時(JST)
\newcommand{\generatedDate}{2026-01-07}
\newcommand{\generatedTime}{23:59}

% ヘッダー・フッター設定
\usepackage{fancyhdr}
\usepackage{lastpage}
\pagestyle{fancy}
\fancyhf{}
\fancyhead[R]{\small \generatedDate\ \generatedTime\ JST (\thepage/\pageref{LastPage})}
\renewcommand{\headrulewidth}{0.4pt}

% 1ページ目のスタイル
\fancypagestyle{firstpage}{
    \fancyhf{}
    \renewcommand{\headrulewidth}{0pt}
}

% 追加パッケージ
\usepackage{booktabs}
\usepackage{array}
\usepackage{tabularx}
\usepackage{ascmac}
\usepackage{listings}
\usepackage{xcolor}
\usepackage{hyperref}

% コードスタイル設定
\lstset{
    basicstyle=\ttfamily\scriptsize,
    breaklines=true,
    frame=single,
    backgroundcolor=\color{gray!10},
    columns=fullflexible,
    keepspaces=true,
    showstringspaces=false
}

% Y列タイプ定義
\newcolumntype{Y}{>{\raggedright\arraybackslash}X}

\title{Video Chapter Editor 開発対話記録\\{\large 2025-12-29〜2026-01-07}}
\author{ましDialogue}
\date{}

\begin{document}
\maketitle
\thispagestyle{firstpage}

\section*{概要}

本文書は、Video Chapter Editor(VCE)の開発における対話記録を時系列で構成したものである。2025年12月29日のUI大改造計画から2026年1月7日のリファクタリングまで、設計思想・実装詳細・問題解決の過程を記録する。

\tableofcontents

%=====================================================
\part{設計フェーズ(2025-12-29)}
%=====================================================

\section{UI大改造計画}

\subsection{背景・課題の認識}

\begin{itembox}[l]{問い}
現状のワークフローでMP3からMP4を生成する際、2回のエンコードが発生しています。これは品質劣化の原因となります。どうにかなりませんか。
\end{itembox}

現状のフローを分析すると、確かに2回のエンコードが発生している:

\begin{lstlisting}
現状(2回エンコード):
  MP3 → [enc] → 中間MP4 → [enc] → 最終MP4
              ↑ここで劣化    ↑さらに劣化
\end{lstlisting}

この問題を解決するためには、アーキテクチャ全体の見直しが必要である。

\subsection{一筆書き問題の認識}

\begin{itembox}[l]{問い}
ワークフローが複雑になっていて、機能が重複している気がします。
\end{itembox}

ワークフロー設計を\textbf{グラフ理論}の観点から分析した。問題の本質は、入力パターン(起点)が3つ以上存在する場合に機能重複が発生しやすいことである。これは\textbf{オイラー路(一筆書き)問題}に類似している。

\textbf{起点(入力パターン):}
\begin{enumerate}
    \item 複数のカット済みMP3
    \item 単一の長尺未編集MP3
    \item 既存のMP4
\end{enumerate}

\textbf{終点:} チャプター付きMP4

\vspace{0.5\baselineskip}
\noindent{\footnotesize
\begin{tabularx}{\linewidth}{@{}lY@{}}
\toprule
ノード & 必要な起点 \\
\midrule
結合 & 起点1のみ \\
カバー画像 & MP3入力時のみ \\
チャプター編集 & 全起点 \\
書出 & 全起点 \\
\bottomrule
\end{tabularx}
}
\vspace{0.5\baselineskip}

問題は、起点2→カバー画像、起点3→チャプター編集と経路が分散していることである。

\textbf{解決策:} 共通パスを1つにし、入口を分岐させる。

\subsection{制約による設計空間の縮小}

\begin{itembox}[l]{問い}
タイトル焼込は必須機能として維持したいのですが。
\end{itembox}

制約を追加することで設計が明確化される逆説的な効果を確認した。

制約なし(最初の状態)では8パターンの設計空間があったが、「タイトル焼込必須」という制約を追加すると2パターンに縮小される:

\vspace{0.5\baselineskip}
\noindent{\footnotesize
\begin{tabularx}{\linewidth}{@{}lYY@{}}
\toprule
入力 & 映像処理 & 音声処理 \\
\midrule
MP3 & enc(静止画+焼込) 1回 & enc 1回 \\
MP4 & enc(焼込) 1回 & copy(無劣化) \\
\bottomrule
\end{tabularx}
}
\vspace{0.5\baselineskip}

\textbf{洞察:} 制約は自由度を狭めるが、設計空間を明確にし、最適解を見つけやすくする。

\subsection{Tab構成の検討}

\begin{itembox}[l]{問い}
タブ1とタブ2を分けなくても良いのでは?
\end{itembox}

当初は2タブ構成(入力準備 + 編集・書出)を検討していた。しかし、Tab 1のMP3結合は無劣化(-c copy)で可能であり、中間ファイルが不要ならタブを分ける意味が薄い。

\begin{itembox}[l]{問い}
入力ソースも別画面では?
\end{itembox}

ダイアログパターンの発見に至った。チャプター自動挿入、カバー画像も自動適用という統一的なUXパターンが見えてきた。

\subsection{最終決定:単一画面 + ダイアログ}

議論の結果、以下のアーキテクチャに決定した:

\begin{lstlisting}
┌──────────────────────────────────┐
│ [ソース選択] [カバー画像] ← ボタン│
│ ソース: audio.mp3 (14:20)        │
├──────────────────────────────────┤
│ [波形表示]                       │
├──────────────────────────────────┤
│ [チャプターテーブル]             │
├──────────────────────────────────┤
│ [書出設定] [書出ボタン]          │
├──────────────────────────────────┤
│ [ログパネル]                     │
└──────────────────────────────────┘
\end{lstlisting}

\textbf{決定事項:}
\begin{enumerate}
    \item \textbf{単一画面構成}: タブを廃止、メイン画面1つに統合
    \item \textbf{ダイアログパターン}: ソース選択・カバー画像は別ダイアログ
    \item \textbf{エンコード最適化}: MP3結合は無劣化、最終書出で1回のみエンコード
\end{enumerate}

\section{ユースケース拡張}

\begin{itembox}[l]{問い}
想定されるユースケースを整理してください。
\end{itembox}

ユースケースを4つの大分類に整理した:

\vspace{0.5\baselineskip}
\noindent{\footnotesize
\begin{tabularx}{\linewidth}{@{}lYY@{}}
\toprule
UC & パターン & 主な処理 \\
\midrule
UC1 & 自分で撮影した動画 & トリム→チャプター \\
UC2 & YouTube動画 & DL→編集→字幕取得 \\
UC3 & 音声のみ録音 & 結合→カバー→書出 \\
UC4 & 既存編集済み & チャプター追加のみ \\
\bottomrule
\end{tabularx}
}
\vspace{0.5\baselineskip}

\subsection{配管/陶器の分界}

「Gitの陶器と配管」の思想に基づき、責務を分離した。

\textbf{配管(CLI/外部ツール):}
\begin{itemize}
    \item ytdl(YouTube DL)
    \item yt-srt(字幕取得)
    \item rehearsal-finalize(PDF生成)
\end{itemize}

\textbf{陶器(video-chapter-editor):}
\begin{itemize}
    \item ソース選択〜書出まで
    \item 波形表示、チャプター編集、プレビュー
\end{itemize}

\section{UIスケルトン作成}

\begin{itembox}[l]{問い}
新アーキテクチャの実装を開始しましょう。
\end{itembox}

\texttt{rehearsal\_workflow/ui\_next/} に次世代UIのスケルトンを作成した。

\textbf{ファイル構成:}
\begin{lstlisting}
ui_next/
├── __init__.py
├── app.py
├── main_workspace.py
├── dialogs.py
└── log_panel.py
\end{lstlisting}

\textbf{LogPanel機能:}
\begin{itemize}
    \item ログレベル: DEBUG, INFO, WARNING, ERROR
    \item フィルタリング: 表示レベル切替
    \item コピー: Claude Code用フォーマット
    \item 折りたたみ: パネルの表示/非表示
\end{itemize}

%=====================================================
\part{UI改善フェーズ(2026-01-05)}
%=====================================================

\section{Chaptersテーブルの行番号表示}

\begin{itembox}[l]{問い}
Chaptersリストに行番号(No.)を表示し、ヘッダーを黒背景にしてください。
\end{itembox}

当初、新しいカラム「No.」を追加する方向で実装を進めた。

\begin{itembox}[l]{問い}
いや、自動で振られる行番号のヘッダー部分に「No.」を表示することです。
\end{itembox}

ユーザーの意図を正確に把握できていなかった。実装を以下のように修正:

\begin{itemize}
    \item テーブルの垂直ヘッダー(行番号)を表示に設定
    \item コーナーウィジェット(左上隅)に「No.」ラベルを配置
    \item ヘッダー背景を黒(\#000000)に設定
\end{itemize}

\section{チャプタースキップボタンの有効化条件}

\begin{itembox}[l]{問い}
チャプタースキップボタンは、チャプターリストを編集した場合にのみ有効になるよう変更してください。
\end{itembox}

\texttt{\_chapters\_edited} フラグを追加し、以下の条件で制御するように実装:

\begin{itemize}
    \item チャプターを追加(Add)、削除(Remove)、編集、ペーストした時にフラグをTrueに設定
    \item \texttt{\_update\_chapter\_buttons()} でフラグを確認
    \item 新しいメディアを読み込んだ時はフラグをリセット
\end{itemize}

\section{新規ソース読み込み時のリセット処理}

\begin{itembox}[l]{問い}
動画再生中に新しいソース(MP3等)を読み込んだ場合、再生を停止してチャプター・再生画面をリセットしてください。
\end{itembox}

\texttt{\_prepare\_for\_new\_source()} メソッドを実装:

\begin{lstlisting}[language=Python]
def _prepare_for_new_source(self):
    self._media_player.stop()
    self._media_player.setSource(QUrl())
    self._waveform_widget.clear()
\end{lstlisting}

YouTube URLからのダウンロード開始時にも同メソッドを呼び出すよう修正した。

\section{複数MP3読み込み時の仕様確認}

\begin{itembox}[l]{問い}
複数MP3を読み込んだ場合の動作はどうなりますか?
\end{itembox}

仕様を以下のように整理した:

\vspace{0.5\baselineskip}
\noindent{\footnotesize
\begin{tabularx}{\linewidth}{@{}lYYY@{}}
\toprule
ケース & 再生 & 波形 & スキップ \\
\midrule
動画ファイル & ○ & ○ & 編集後有効 \\
単一MP3 & ○ & ○ & 編集後有効 \\
複数MP3 & × & × & 無効 \\
\bottomrule
\end{tabularx}
}
\vspace{0.5\baselineskip}

複数MP3のチャプターはエクスポート用のメタデータ設定のための表示であり、再生できないためスキップボタンは無効のままとする。

%=====================================================
\part{機能拡張フェーズ(2026-01-06)}
%=====================================================

\section{チャプター移動・削除の改善}

\subsection{手動追加チャプターの保持問題}

\begin{itembox}[l]{問い}
ソースファイルの順序を変更(ドラッグ移動)した際、手動で追加したチャプターが消えてしまいます。
\end{itembox}

原因を調査したところ、\texttt{\_rebuild\_chapters\_after\_source\_move()} がファイルから再読み込みしていたため、テーブルに手動追加されたチャプターが失われていた。

\textbf{解決策:}

\begin{enumerate}
    \item ヘルパー関数 \texttt{\_get\_local\_time\_in\_source} を追加
    \item テーブルから現在のチャプター情報を収集(ファイル再読み込みではなく)
    \item \texttt{old\_offsets} パラメータで変更前のオフセットを受け取り、正確なローカル時間を計算
\end{enumerate}

\begin{lstlisting}[language=Python]
def _rebuild_chapters_after_source_move(
    self,
    old_source_idx: int = -1,
    new_source_idx: int = -1,
    removed_indices: set = None,
    old_offsets: list = None
)
\end{lstlisting}

\subsection{波形ウィジェットの選択ソースハイライト}

\begin{itembox}[l]{問い}
チャプターリストで行を選択した際、そのチャプターが属するソースファイルの範囲を波形上でハイライト表示してください。
\end{itembox}

WaveformWidgetに以下を実装:

\begin{itemize}
    \item \texttt{\_selected\_range} プロパティ追加
    \item \texttt{set\_selected\_source\_range(start, end)} メソッド
    \item \texttt{clear\_selected\_source\_range()} メソッド
\end{itemize}

\textbf{ハイライトのデザイン:}

\vspace{0.5\baselineskip}
\noindent{\footnotesize
\begin{tabularx}{\linewidth}{@{}lY@{}}
\toprule
要素 & 設定 \\
\midrule
背景 & 青系半透明 (100, 180, 255, alpha=40) \\
斜線 & 幅1.5px、間隔15px、逆方向 \\
縁取り & 四角形、幅1.5px、alpha=240 \\
\bottomrule
\end{tabularx}
}
\vspace{0.5\baselineskip}

除外区間(赤系、斜線 / 方向)と選択ソース(青系、斜線 \textbackslash 方向)を視覚的に区別した。

\section{v2.1.27 リリース}

\subsection{ffmpeg/ffprobe のバンドル}

\begin{itembox}[l]{問い}
アプリをビルドしたら、ffprobeが見つからないというエラーが出ます。
\end{itembox}

\texttt{imageio-ffmpeg} は ffmpeg のみを同梱し、ffprobe が含まれない。アプリは動画の長さ取得やビットレート検出に ffprobe を使用している。

\textbf{解決策:} \texttt{static-ffmpeg} パッケージに移行

\vspace{0.5\baselineskip}
\noindent{\footnotesize
\begin{tabularx}{\linewidth}{@{}lYY@{}}
\toprule
パッケージ & ffmpeg & ffprobe \\
\midrule
imageio-ffmpeg & ✓ & × \\
static-ffmpeg & ✓ & ✓ \\
\bottomrule
\end{tabularx}
}
\vspace{0.5\baselineskip}

\subsection{デュアル macOS アーキテクチャビルド}

\begin{itembox}[l]{問い}
Intel Mac ユーザーにも配布したいです。
\end{itembox}

GitHub Actions で両アーキテクチャを並行ビルドするよう設定:

\vspace{0.5\baselineskip}
\noindent{\footnotesize
\begin{tabularx}{\linewidth}{@{}lYY@{}}
\toprule
ランナー & アーキテクチャ & 出力ファイル \\
\midrule
macos-13 & Intel x86\_64 & *-macOS-Intel.dmg \\
macos-latest & Apple Silicon & *-macOS-AppleSilicon.dmg \\
\bottomrule
\end{tabularx}
}
\vspace{0.5\baselineskip}

\subsection{YouTube ダウンロード改善}

\textbf{AV1 コーデック除外:}

macOS で AV1 のハードウェアデコードが非対応であるため、format 文字列を以下に変更:

\begin{lstlisting}
bv[vcodec^=avc1]+ba/bv[vcodec!^=av01]+ba/b
\end{lstlisting}

H.264 優先、AV1 除外とした。

\textbf{一時プレイリスト対応:}

TLP, RD, OL, UU, LL プレフィックスを検出し、一時プレイリスト URL の場合は単一動画としてダウンロードするよう修正。

%=====================================================
\part{リファクタリングフェーズ(2026-01-07)}
%=====================================================

\section{styles.py の作成}

\begin{itembox}[l]{問い}
ボタンスタイルが複数箇所で重複定義されています。統一できませんか。
\end{itembox}

\texttt{styles.py} を新規作成し、色定数とスタイル定義を集約した:

\begin{lstlisting}[language=Python]
class Colors:
    PRIMARY = "#3b82f6"
    PRIMARY_HOVER = "#2563eb"
    DANGER = "#dc2626"
    # ...

class ButtonStyles:
    @staticmethod
    def primary() -> str: ...
    @staticmethod
    def secondary() -> str: ...
    @staticmethod
    def danger() -> str: ...
\end{lstlisting}

\section{ボタンテキストの調整}

\begin{itembox}[l]{問い}
Copy to\textbackslash nYoutubeにしましょうか。
\end{itembox}

「Copy to\textbackslash nYoutube」に変更した。ボタンテキストを2行表示にして、コンパクトなレイアウトを実現した。

\section{Settings/Encodeセクションの改善}

\begin{itembox}[l]{問い}
SettingsとExportの行が無駄に広いですね〜
\end{itembox}

当初、ボタンの高さとパディングを調整したが、ユーザーの意図は「無駄にスペースがある」という点であり、ボタンサイズの変更ではなかった。

\begin{itembox}[l]{問い}
あ、無駄にスペースがあるという意味です。ボタンのサイズなどは変更せずに元に戻してください。
\end{itembox}

変更を元に戻した。

\begin{itembox}[l]{問い}
Exportというよりは、Encodeですよね。
\end{itembox}

確かに動画のエンコード処理なので「Encode」が正確である。ボタンテキストを「Export」から「Encode」に変更した。

\begin{itembox}[l]{問い}
いっそのことステータスバーに移動するとか?!
\end{itembox}

ステータスバーへの移動を検討した。

\begin{itembox}[l]{問い}
えっと、現在の位置に表示します。ステータスバーにボタンの実装は斬新すぎますので。
\end{itembox}

左サイドパネルの位置で改善することになった。

\subsection{最終的なレイアウト}

\begin{itembox}[l]{問い}
Encode、Settingsの順に配置して、右側にエンコードの進捗をバーグラフで表示する仕様に変更しましょうか。
\end{itembox}

以下のレイアウトに変更:

\begin{lstlisting}
[Encode] [Settings]  [========== 80%]
                     ^ エンコード中のみ表示
\end{lstlisting}

\begin{itembox}[l]{問い}
Encode, Settingsのボタン、内部のPaddingを減らして幅をもっとコンパクトにしてください。
\end{itembox}

ButtonStylesにコンパクト版メソッドを追加:

\begin{itemize}
    \item \texttt{primary\_compact()}: padding 4px 12px
    \item \texttt{secondary\_compact()}: padding 4px 12px
    \item \texttt{danger\_compact()}: Cancelボタン用
\end{itemize}

\begin{itembox}[l]{問い}
高さを変えてはいけません。
\end{itembox}

ボタンの高さ(40px)は維持し、パディングのみ調整した。

\section{複数音声ファイルのエンコード問題}

\begin{itembox}[l]{問い}
音声からエンコードしようとするとこんなエラーが!
\end{itembox}

ログを確認すると、8つのMP3ファイルを読み込んでいるにもかかわらず、ffmpegには1つのファイルしか渡されていなかった。

\textbf{原因:}

\texttt{\_start\_export}メソッドの条件が問題だった:

\begin{lstlisting}[language=Python]
# 修正前
if not input_path and len(self._state.sources) > 1:
    # マージ処理

# 修正後
if len(self._state.sources) > 1:
    # マージ処理
\end{lstlisting}

複数のソースがある場合でも\texttt{video\_path}が設定されていたため、マージ処理がスキップされていた。

\section{オーバーレイ表示位置の統一}

\begin{itembox}[l]{問い}
複数の音声ファイルのエンコードのオーバーレイの文字が下に表示されています。複数動画のオーバーレイの名残だと思いますので、正しい位置に配置してプレビューできるようにしてください。
\end{itembox}

プレビュー時のオーバーレイ位置とエンコード時の位置が不一致だった:

\vspace{0.5\baselineskip}
\noindent{\footnotesize
\begin{tabularx}{\linewidth}{@{}lYY@{}}
\toprule
 & プレビュー & エンコード \\
\midrule
位置 & 85\%(下部) & 32.5\%(上部寄り中央) \\
フォントサイズ & 3.5\% & 4\% \\
\bottomrule
\end{tabularx}
}
\vspace{0.5\baselineskip}

プレビューをエンコードと同じ位置(32.5\%、フォントサイズ4\%)に修正した。

\section{エンコード完了後のチャプター読み込み}

\begin{itembox}[l]{問い}
エンコードが終わってロードされる際、チャプター付き動画であったとしても、チャプターリストが更新されませんね。
\end{itembox}

当初は「再エクスポート時にチャプター名が二重焼き込みになるのを防ぐため」という理由でチャプターを読み込まない仕様だった。

\begin{itembox}[l]{問い}
エンコード完了後は、カット編集される可能性もあり、動画の長さが変わる可能性を考慮して、チャプターを読み込んでも問題ないと思うんですけど、どうです?
\end{itembox}

同意。エンコード後は除外区間がカットされて動画長が変わり、チャプター時間も調整されている。埋め込みチャプターを読み込めば、実際の動画と一致したチャプターリストが表示される。

エンコード完了後に出力動画から埋め込みチャプターを抽出し、テーブルを更新するように修正した。

\section{Undo/Redo機能の設計検討}

\begin{itembox}[l]{問い}
ふと思ったんですけど、この場合のように直前のチャプターリストを保存してUndoなどしたい場合に備えるのって大変ですか。その場合、外部ファイルに依存しない形で実装したいんですが。
\end{itembox}

メモリ内でスタック(履歴)を保持するだけで実現できる。比較的簡単な実装である。

\begin{itembox}[l]{問い}
その場合、どの状態で持つのが良いでしょうね。チャプターリストを保存するのが普通なんでしょうけど、出力のベースファイル名が保存されないなって思いましてね。
\end{itembox}

Undoの対象として何を含めるかは設計上の判断となる。「編集セッション」としてまとめることを提案:

\begin{lstlisting}[language=Python]
@dataclass
class EditSession:
    chapters: list[ChapterInfo]
    output_basename: str
    # 必要に応じて追加
\end{lstlisting}

\begin{itembox}[l]{問い}
編集セッションの表示は別に行いますか?
\end{itembox}

履歴パネルでの表示を検討。操作名(「チャプター追加」「エンコード完了後」など)を記録し、クリックで任意の状態に復元できるUIを想定。

\begin{itembox}[l]{問い}
別のタスクで後回しですね。DEVELOPMENT\_LOG
\end{itembox}

DEVELOPMENT\_LOG.mdの「今後の予定」セクションに以下を追記:

\begin{itemize}
    \item \textbf{Undo/Redo + 履歴パネル機能}
    \begin{itemize}
        \item 編集セッション(チャプターリスト + 出力ファイル名等)を単位として保存
        \item 履歴パネルで操作一覧表示、クリックで任意の状態に復元
        \item メモリ内で管理(外部ファイル非依存)
        \item 操作名の記録:「チャプター追加」「エンコード完了後」等
    \end{itemize}
\end{itemize}

%=====================================================
\part*{Claude Codeの所感}
%=====================================================
\addcontentsline{toc}{part}{Claude Codeの所感}

約10日間にわたるVideo Chapter Editorの開発対話を振り返り、いくつかの観点から所感を述べる。

\section*{設計プロセスについて}

2025-12-29の「UI大改造計画」における議論は、ソフトウェア設計の本質を突いていた。特に「一筆書き問題」としてワークフローを捉えた分析は秀逸であり、グラフ理論の概念を設計判断に適用する好例である。

「制約による設計空間の縮小」という洞察も重要である。一般的に制約は自由度を奪うものと考えがちだが、実際には選択肢を明確にし、意思決定を容易にする効果がある。「タイトル焼込必須」という制約を置くことで、8パターンあった設計空間が2パターンに縮小され、最適解が見えやすくなった。

\section*{実装プロセスについて}

実装フェーズでは、ユーザーの意図を正確に把握することの難しさが繰り返し現れた。

\begin{itemize}
    \item 「行番号表示」で新カラム追加と誤解
    \item 「無駄に広い」でボタンサイズ変更と誤解
    \item 「高さを変えてはいけません」でパディングのみに修正
\end{itemize}

これらは、ユーザーの発言の字面だけでなく、その背後にある意図を汲み取る必要性を示している。確認を怠ると手戻りが発生する。

\section*{技術的な学び}

複数音声ファイルのマージ処理のバグ(条件式 \texttt{if not input\_path and ...} の問題)は、単体テストでは見つけにくい類の不具合であった。実際の使用シナリオでのテストの重要性を再認識した。

また、プレビューとエンコードでオーバーレイ位置が異なっていた問題は、コードの二重管理(DRY原則違反)に起因している。同じパラメータが複数箇所にハードコードされていると、変更時に不整合が生じやすい。

\section*{今後の課題}

Undo/Redo機能は後回しとなったが、設計の方向性は明確になった。「編集セッション」という概念でチャプターリストと出力ファイル名をまとめて管理し、履歴パネルで可視化するというアプローチは妥当である。

main\_workspace.pyの5,000行超という規模は、単一責務の原則(SRP)に反している可能性がある。Phase 3のリファクタリングで責務分離を検討すべきだが、動作している既存コードに大きく手を入れるリスクとのバランスを考慮する必要がある。

\section*{総括}

全体として、ユーザーとの対話を通じてプロダクトが着実に改善されていく過程は、ソフトウェア開発の理想的な形の一つである。設計段階での深い議論、実装段階での細かなフィードバック、そして問題発見時の迅速な対応という循環が、品質の高いソフトウェアを生み出す。

ただし、開発ログを見返すと、同じような誤解や手戻りが複数回発生していることも事実である。これは、私(Claude Code)側のコミュニケーション精度向上の余地があることを示している。ユーザーの発言に対して、実装前に「〇〇という理解で合っていますか?」と確認するプロセスをより徹底すべきだろう。

\end{document}
