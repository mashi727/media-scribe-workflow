\documentclass[10pt,a4paper,twocolumn]{ltjsarticle}

% LuaLaTeX用フォント設定パッケージ
\usepackage{luatexja-fontspec}
\usepackage{amsmath,amssymb}
\usepackage{unicode-math}

% ====================
% 欧文フォント設定 (Libertinus)
% ====================
\setmainfont{Libertinus Serif}[
    BoldFont = {Libertinus Serif Bold},
    ItalicFont = {Libertinus Serif Italic},
    BoldItalicFont = {Libertinus Serif Bold Italic}
]
\setsansfont{Libertinus Sans}[
    BoldFont = {Libertinus Sans Bold},
    ItalicFont = {Libertinus Sans Italic}
]
\setmonofont{Libertinus Mono}

% lstlisting用等幅フォント (DejaVu Sans Mono)
\newfontfamily\dejavumono{DejaVu Sans Mono}[Scale=0.85]

% ====================
% 日本語フォント設定 (原ノ味フォント)
% ====================
\setmainjfont{HaranoAjiMincho-Regular}[
    BoldFont = {HaranoAjiGothic-Medium},
    ItalicFont = {HaranoAjiMincho-Regular},
    BoldItalicFont = {HaranoAjiGothic-Bold}
]
\setsansjfont{HaranoAjiGothic-Regular}[
    BoldFont = {HaranoAjiGothic-Bold}
]
\setmonojfont{HaranoAjiGothic-Regular}

% ====================
% 数式フォント設定 (Libertinus Math)
% ====================
\setmathfont{Libertinus Math}

% パッケージ
\usepackage{booktabs}
\usepackage{array}
\usepackage{tabularx}
\usepackage{longtable}
\usepackage{listings}
\usepackage{xcolor}
\usepackage{hyperref}
\usepackage{ascmac}
\usepackage{fancyvrb}
\usepackage{tcolorbox}
\tcbuselibrary{breakable,skins}
\usepackage{geometry}
\geometry{left=15mm,right=15mm,top=20mm,bottom=20mm}

% ファイル生成日時(JST)
\newcommand{\generatedDate}{2026-01-08}
\newcommand{\generatedTime}{01:15}

% ヘッダー・フッター設定
\usepackage{fancyhdr}
\usepackage{lastpage}
\pagestyle{fancy}
\fancyhf{}
\fancyhead[R]{\small \generatedDate\ \generatedTime\ JST (\thepage/\pageref{LastPage})}
\renewcommand{\headrulewidth}{0.4pt}

% 1ページ目のスタイル
\fancypagestyle{plain}{
  \fancyhf{}
  \renewcommand{\headrulewidth}{0pt}
}

% コードブロック設定(DejaVu Sans Mono使用、アスキーアート対応)
\lstset{
  basicstyle=\dejavumono\tiny,
  breaklines=true,
  breakatwhitespace=false,
  columns=fixed,
  keepspaces=true,
  frame=single,
  framerule=0.4pt,
  rulecolor=\color{gray!60},
  backgroundcolor=\color{gray!5},
  xleftmargin=1.5mm,
  xrightmargin=1.5mm,
  aboveskip=1mm,
  belowskip=1mm,
  lineskip=-0.5pt,
}

% 対話ボックスの定義(シンプルなモノクロスタイル)
% userbox: ascmac itembox風の控えめな枠線
\newtcolorbox{userbox}[1][]{
  colback=white,
  colframe=gray!60,
  boxrule=0.5pt,
  arc=0pt,
  fonttitle=\bfseries\small,
  title={問い},
  coltitle=black,
  colbacktitle=gray!15,
  breakable,
  left=2mm,
  right=2mm,
  top=1mm,
  bottom=1mm,
  #1
}

% assistantbox: 枠線なし、控えめな左ボーダーのみ
\newtcolorbox{assistantbox}[1][]{
  colback=white,
  colframe=white,
  borderline west={1.5pt}{0pt}{gray!40},
  boxrule=0pt,
  arc=0pt,
  fonttitle=\bfseries\small,
  title={},
  breakable,
  left=3mm,
  right=2mm,
  top=1mm,
  bottom=1mm,
  #1
}

% Y列タイプ定義
\newcolumntype{Y}{>{\raggedright\arraybackslash}X}

% タイトル
\title{\textbf{Video Chapter Editor 開発記録 完全版}\\
\large 対話記録・開発ログ・設計文書\\
\normalsize 2025年12月23日〜2026年1月7日}
\author{執筆者:まし\\対話相手:Claude Code (Anthropic)}
\date{}

\begin{document}

\maketitle
\thispagestyle{plain}


\section*{はじめに}

本文書は、Video Chapter Editor(VCE)の開発に関する一次資料を可能な限り網羅的に収録したものである。VCEは、リハーサル動画のチャプター編集を効率化するためのデスクトップアプリケーションであり、PySide6(Qt6)を用いて開発された。

本文書は以下の3部で構成される:

\begin{enumerate}
\item \textbf{第I部:対話記録} --- 2025年12月23日から2026年1月6日まで、ユーザーとClaude Codeの間で交わされた9,000以上のメッセージを時系列で収録
\item \textbf{第II部:開発ログ} --- バージョン履歴、機能実装の経緯、変更履歴を記録
\item \textbf{第III部:設計・実装文書} --- アーキテクチャ設計、ワークフロー設計、環境構築ガイドを収録
\end{enumerate}

各資料は、技術的な議論、設計上の決定、実装の詳細、試行錯誤の過程を省略することなく記録しており、開発プロセスの一次資料としての価値を保つことを目指している。

\vspace{0.5\baselineskip}

\tableofcontents

\newpage

%% =====================================
%% 第I部:対話記録
%% =====================================

\part{対話記録}

この部では、開発期間中に行われた全ての対話を日付順に収録する。各日の対話は、その日に行われた開発作業の流れに沿って構成されている。



%% === 統合された対話記録 ===


%% ====== dialog_2025-12-23_full.tex ======
%% dialog_2025-12-23_full.tex
%% 2025年12月23日の開発対話記録(175メッセージ)

\section{2025年12月23日:配管ツールの実装とPADtoolsのCLI対応}

この日の開発セッションでは、CLAUDE.mdに記載されていた未実装タスクの実行と、PAD図のPNG出力機能の実装が行われた。

\subsection{未実装タスクの実行}

\begin{userbox}
未実装タスクを実行してください。
\end{userbox}

\begin{assistantbox}
CLAUDE.mdに記載されている未実装タスクを実行します。まず現在のプロジェクト構造を確認します。

プロジェクトの全体像を把握しました。現在の実装(rehearsal-download/finalize)は統合ワークフローツールとして既に存在しますが、CLAUDE.mdの設計方針に基づく「単機能の配管ツール」が未実装です。

未実装タスクを順次実行します。
\end{assistantbox}

以下のタスクがTodoWriteで管理された:
\begin{enumerate}
\item bin/yt-srt の整理(現在の bin/ を確認し、yt-srt を作成)
\item bin/video-trim の作成
\item bin/video-chapters の作成
\item examples/prompts/ の作成(Claude用・汎用プロンプト)
\item README.md の更新(新構成に合わせる)
\end{enumerate}

これらのタスクは順次実行され、以下のファイルが作成された:

\vspace{0.5\baselineskip}
\noindent{\footnotesize
\begin{tabularx}{\linewidth}{@{}lX@{}}
\toprule
ファイル & 説明 \\
\midrule
bin/yt-srt & YouTube字幕取得 \\
bin/video-trim & 動画トリミング \\
bin/video-chapters & チャプター結合・埋め込み \\
examples/prompts/rehearsal-claude.md & Claude用プロンプト \\
examples/prompts/rehearsal-generic.md & ChatGPT/Gemini等汎用プロンプト \\
docs/advanced/my-setup.md & 全体構成例 \\
docs/advanced/whisper-remote.md & Whisperリモート環境 \\
docs/advanced/luatex-docker.md & LuaTeX Docker環境 \\
docs/advanced/claude-commands.md & Claude Codeコマンド設定 \\
\bottomrule
\end{tabularx}
}
\vspace{0.5\baselineskip}

\begin{userbox}
pushしました?
\end{userbox}

\begin{assistantbox}
いいえ、まだpushしていません。commitもしていません。

commitとpushを行いますか?
\end{assistantbox}

\begin{userbox}
ですね。
\end{userbox}

コミットとプッシュが実行され、以下の結果となった:

\begin{lstlisting}
f4ddd6b Add plumbing tools and documentation
\end{lstlisting}

16ファイル、2633行の追加がプッシュされた。

\subsection{PAD図に関する議論}

\begin{userbox}
やはり、Mermaidだと読みづらいですね。PADだと、抽象度が右に行くにつれて低くなるので、配管と陶器の整理が行いやすいんですよね。。。
\end{userbox}

\begin{assistantbox}
その通りですね。

\textbf{Mermaid}は「フローの流れ」を表現するのに適していますが、抽象度の階層が視覚的に分かりにくい。

\textbf{PAD}は構造的に:
\begin{lstlisting}
高抽象(陶器)          低抽象(配管)
---------------------------------------->
rehearsal-download  ->  ytdl -> yt-dlp
                    ->  whisper-remote -> curl -> Docker
\end{lstlisting}

この「右に展開する」構造が、まさに「陶器(ユーザー向けコマンド)」と「配管(内部実装)」の関係を自然に表現できます。

docs/pad/ に .spd ファイルがあるので、PADtoolsで視覚化すれば、どのツールがどの配管を呼び出しているか、依存関係が一目で把握できますね。
\end{assistantbox}

\begin{userbox}
ですね。pngにレンダリングして、貼り付けた方が良いかなあ。
\end{userbox}

\begin{assistantbox}
良いと思います。

\textbf{メリット}:
\begin{itemize}
\item GitHubのREADMEで直接見える
\item PADtoolsがなくても構造を把握できる
\item 「階層的な使い方」セクションの補足として最適
\end{itemize}

PADtoolsのエクスポート機能で出力できますか?それとも画面キャプチャになりますか?
\end{assistantbox}

\begin{userbox}
エクスポート機能がありますね。GUIなので、自動化したいところですが、できます?
\end{userbox}

PADtoolsのCLIモードの調査が開始された。インストール先は \texttt{/Users/mashi/local/src/padtools1.4} であることが確認された。

\subsection{PADtools CLI問題の発見}

JARファイルの解析により、CLIオプションが存在することが判明した:

\begin{lstlisting}
Arguments: [-o result_file] [-s scale] [-h] [spd_file]
\end{lstlisting}

しかし、実際にCLIモードで実行しようとすると、GUIが起動してしまう問題が発生した。

\begin{lstlisting}[language=bash]
java -jar PadTools.jar -o output.png -s 2.0 input.spd
# -> GUIが起動してしまう

java -Djava.awt.headless=true -jar PadTools.jar ...
# -> HeadlessException が発生
\end{lstlisting}

\subsection{独自レンダラーの試みと断念}

\begin{userbox}
3にしたいですね。
\end{userbox}

「独自レンダラー作成」オプションが選択され、\texttt{tools/spd2svg.py}が作成された。しかし、出力結果は正式なPAD表記とは異なるものだった。

\begin{userbox}
随分印象と違いますね。PADの表記に忠実にできますか?
\end{userbox}

PADの正式な表記法(二村良彦による設計\footnote{PAD表記法については東京大学の資料を参照:\texttt{lecture.ecc.u-tokyo.ac.jp/hideo-t/references/pad/pad.html}})を調査し、以下の要素を実装しようとした:
\begin{enumerate}
\item 縦線(スパイン)を左端に配置
\item 処理ボックスは縦線の右に接続
\item call(手続き呼び出し)は左右に二重線
\item 選択(if/switch)は旗型(右端が凹み、分岐で尖る)
\item コメントは点線枠
\end{enumerate}

しかし、結果は依然として不十分だった。

\begin{userbox}
まだまだですね。車輪の再開発っぽいので、PadtoolsをフォークしてCLI実装を行うとかどうですか。
\end{userbox}

\subsection{PADtools Converterの直接呼び出し}

PADtoolsのGitHubリポジトリ\footnote{\texttt{https://github.com/knaou/padtools}}を調査した結果、\texttt{Converter}クラスが変換ロジックを持っていることが判明した。

\begin{lstlisting}
Converter.convert(File file_in, File file_out, Double scale)
\end{lstlisting}

処理フロー:
\begin{enumerate}
\item SPDParser.parse() → PADModel
\item Model2View.toView() → View
\item View2Image.toImage(view, scale) → BufferedImage
\item ImageIO.write() → PNG出力
\end{enumerate}

問題は、Main.javaがGUIスレッドを先に起動してしまうことだった。解決策として、Converterクラスを直接呼び出すラッパーを作成した。

\begin{lstlisting}[language=java,caption=tools/PadCLI.java]
import padtools.converter.Converter;
import java.io.File;

public class PadCLI {
    public static void main(String[] args) {
        if (args.length < 2) {
            System.err.println("Usage: PadCLI <input.spd> <output.png> [scale]");
            System.exit(1);
        }
        File input = new File(args[0]);
        File output = new File(args[1]);
        Double scale = args.length >= 3 ? Double.parseDouble(args[2]) : 2.0;

        Converter converter = new Converter();
        converter.convert(input, output, scale);
    }
}
\end{lstlisting}

\begin{lstlisting}[language=bash]
# コンパイルと実行
javac -cp PadTools.jar PadCLI.java
java -Djava.awt.headless=true -cp PadTools.jar:libs/*:. PadCLI \
    docs/pad/workflow-basic.spd /tmp/test-pad.png 2.0
\end{lstlisting}

PNG生成に成功した(822 x 1294 pixels)。

\subsection{spd2pngスクリプトの作成}

便利なシェルスクリプトラッパーが作成された。

\begin{lstlisting}[language=bash,caption=bin/spd2png]
#!/bin/bash
PADTOOLS_HOME="${PADTOOLS_HOME:-/Users/mashi/local/src/padtools1.4}"
SCRIPT_DIR="$(cd "$(dirname "$0")/.." && pwd)"

CLASSPATH="$PADTOOLS_HOME/PadTools.jar"
for jar in "$PADTOOLS_HOME/libs/"*.jar; do
    CLASSPATH="$CLASSPATH:$jar"
done
CLASSPATH="$CLASSPATH:$SCRIPT_DIR/tools"

java -Djava.awt.headless=true -cp "$CLASSPATH" PadCLI "$@"
\end{lstlisting}

全てのSPDファイルがPNGに変換された:
\begin{itemize}
\item workflow-basic.png(111KB, 822x1294px)
\item workflow-main.png(117KB)
\item workflow-advanced.png(210KB)
\end{itemize}

\begin{userbox}
ですね。
\end{userbox}

コミットが実行され、作業が完了した。

\begin{lstlisting}
a6d01cc Add PADtools CLI wrapper and generate PAD diagram PNGs
\end{lstlisting}

\subsection{この日の成果}

\begin{enumerate}
\item CLAUDE.mdの未実装タスクをすべて完了
\item 配管ツール(yt-srt, video-trim, video-chapters)を作成
\item プロンプト例と環境構築ガイドを作成
\item PADtoolsのCLI問題を解決し、spd2pngツールを作成
\item 全てのPAD図をPNGとして出力
\end{enumerate}

PADtoolsをフォークせずに、既存のConverterクラスを直接呼び出すラッパーで問題が解決された。これは、既存のライブラリを活用しながら必要な機能だけを追加するという、配管と陶器の設計思想に沿った解決策であった。



%% ====== dialog_2025-12-24_full.tex ======
%% dialog_2025-12-24_full.tex
%% 2025年12月24日の開発対話記録(658メッセージ)

\section{2025年12月24日:PADレンダラーのスタイル改善}

この日は、PadAlignedRendererのカラム整列機能とテキスト折り返し機能の実装、およびオリジナルPADtoolsスタイルへの復元作業が行われた。658メッセージに及ぶ長時間の反復的な改善作業となった。

\subsection{カラム整列機能の要望}

\begin{userbox}
一つづつの箱が、深さに応じて縦が揃うように改良できないものですかね。
\end{userbox}

\begin{assistantbox}
PADtoolsにはカラム揃え機能がありません。現在の実装では各ボックスが子を相対位置で配置するため、深さごとの縦揃えができていません。

解決策として、2パスレンダリングを実装する必要があります:
\begin{enumerate}
\item \textbf{パス1}: ツリーを走査して各深さの最大幅を計算
\item \textbf{パス2}: 固定カラム幅で配置
\end{enumerate}

PADtoolsをフォークせずに、カスタムレンダラーで実装してみます。
\end{assistantbox}

\texttt{PadAlignedRenderer.java}が作成され、2パスレンダリングアルゴリズムが実装された。コンパイル時にPADtoolsのAPIとの不整合(43個のエラー)が発生したが、JARファイルを逆コンパイルして正しいメソッド名を特定することで解決した。

主なAPI修正点:
\begin{itemize}
\item \texttt{model.getChild()} → \texttt{model.getTopNode()}
\item \texttt{node.getCall()} → \texttt{node.getText()}
\item \texttt{ifNode.getElseChild()} → \texttt{ifNode.getFalseNode()}
\item \texttt{wc.getChild()} → \texttt{wc.getChildNode()}
\end{itemize}

\subsection{テキスト折り返し機能の追加}

\begin{userbox}
ですね。その際に、内容の記述を適切に改行してあまり右に伸びないようにしたいですね。
\end{userbox}

\begin{assistantbox}
テキストの折り返し機能を追加します。長いテキストが右に伸びすぎないよう、最大幅を設定して自動改行させます。
\end{assistantbox}

以下の機能が実装された:
\begin{itemize}
\item \texttt{BOX\_MAX\_WIDTH = 180}の定数を追加
\item 論理的な区切り点(→、、・、空白)での折り返し
\item 動的なボックス高さの計算
\end{itemize}

\begin{lstlisting}[language=java,basicstyle=\ttfamily\tiny]
private List<String> wrapText(String text, double maxWidth,
                               FontMetrics metrics) {
    List<String> lines = new ArrayList<>();
    if (metrics.stringWidth(text) <= maxWidth) {
        lines.add(text);
        return lines;
    }
    // Break at logical points
    String[] segments = text.split("(?<=[->|、・ ])|(?=[->|、・ ])");
    // ... wrapping logic
}
\end{lstlisting}

\subsection{オリジナルスタイルへの復元}

\begin{userbox}
あら、随分記法が変わりましたね。元の雰囲気を生かしつつ改良できませんか
\end{userbox}

オリジナルPADtoolsの出力と比較した結果、以下の違いが特定された:

\vspace{0.5\baselineskip}
\noindent{\footnotesize
\begin{tabularx}{\linewidth}{@{}lYY@{}}
\toprule
要素 & 元のスタイル & 現在の実装 \\
\midrule
端子 & 楕円 & 角丸四角 \\
呼出 & 両端に二重線 & 背景色付き四角 \\
コメント & 括弧のみ & 点線枠 \\
選択 & 矢印型 & 四角+文字 \\
色 & 白黒のみ & 色付き \\
\bottomrule
\end{tabularx}
}
\vspace{0.5\baselineskip}

これらの要素を順次修正し、オリジナルスタイルを復元した。

\subsection{線の接続パターンの修正}

\begin{userbox}
線が繋がっていたりいなかったり。忠実に再現できませんかね。
\end{userbox}

オリジナルPADtoolsの線描画パターンを分析した結果:
\begin{enumerate}
\item 左側の縦線(スパイン)が全要素を接続
\item 呼出ボックスの右側から子要素への縦線
\item 選択分岐の縦線がペナントの左端として共有
\end{enumerate}

\subsection{条件分岐(Switch/If)の改善}

\begin{userbox}
少しずつ直しましょう。条件分岐のボックスを同様に作成してください。
\end{userbox}

条件分岐の形状がシェブロン(両端が尖る)からペナント(左が平坦、右が尖る)に修正された。

\begin{userbox}
条件分岐をもっと修正しましょう。まず、左端から旗のようになってペナント形状が縦に繋がっています。
\end{userbox}

\begin{assistantbox}
オリジナルとの違いがわかりました:

\textbf{オリジナルPADtools:}
\begin{itemize}
\item 縦線がペナントの左端として共有されている
\item 条件テキストは縦線の左側に配置
\item ペナントは縦線から始まって右に伸びる
\end{itemize}

\textbf{現在の実装:}
\begin{itemize}
\item 縦線がxの位置にある
\item ペナントはx + conditionWidthから始まる(縦線と離れている)
\end{itemize}

修正します。
\end{assistantbox}

\begin{userbox}
音声のみ、動画のみの間に線はありませんね。
\end{userbox}

ケース間の水平線が削除された。

\begin{userbox}
おお線画なくなりましたけど、最も上と下は残してください。
\end{userbox}

ブロック最上部と最下部の水平線のみ残す形に修正された。

\begin{userbox}
両方あり、音声のみの上下に線を引いてくださいな
\end{userbox}

各ケースの上下に水平線が追加された。

\begin{userbox}
水平線が、入りませんね。
\end{userbox}

水平線が\texttt{arrowX}までしか描画されていなかったため、\texttt{arrowTipX}(矢印の先端位置)まで延長するよう修正された。

\subsection{シンプルな方針への転換}

長時間の反復的な修正作業の後、ユーザーはシンプルな方針への転換を提案した。

\begin{userbox}
うーん。難しいですね。改行なしにして、水平方向のみの移動で深さを揃えるようにしましょうか。
\end{userbox}

\begin{assistantbox}
そうですね。シンプルな方針に変更しましょう。

\textbf{新しい方針:}
\begin{enumerate}
\item \textbf{テキスト折り返しなし} - 1行で表示
\item \textbf{水平方向のみで深さを表現} - 同じ深さのノードは同じX座標
\item \textbf{縦方向は単純に上から下へ流れる}
\end{enumerate}

これにより:
\begin{itemize}
\item ペナント形状の複雑な水平線問題が解消
\item レイアウトが予測しやすくなる
\item オリジナルPADtoolsに近いシンプルな構造
\end{itemize}
\end{assistantbox}

\begin{userbox}
ですねー
\end{userbox}

テキスト折り返し機能が削除され、水平方向のみで深さを表現するシンプルなレイアウトに変更された。

\subsection{この日のコミット履歴}

この日は多くのコミットが行われた:

\begin{enumerate}
\item \texttt{e0c0383} - Add column-aligned PAD renderer with text wrapping
\item \texttt{5dc1e95} - Restore original PADtools visual style with column alignment
\item \texttt{8fc8051} - Fix vertical connecting lines to match original PADtools style
\item \texttt{4636c96} - Change condition branch shape to pennant
\item \texttt{8c60a11} - Fix condition branch layout: vertical line connects pennants
\item \texttt{fe10cad} - Improve Switch/If rendering: vertically connected pennants
\item \texttt{586462a} - Remove horizontal separator lines between Switch/If cases
\item \texttt{73e8a58} - Add top and bottom border lines for Switch/If blocks
\item \texttt{596038c} - Add horizontal separator lines between each case in Switch/If
\end{enumerate}

\subsection{この日の教訓}

この日の開発で得られた教訓は以下の通りである:

\begin{enumerate}
\item \textbf{既存ライブラリのAPI調査の重要性}: JARファイルの逆コンパイルにより正しいメソッド名を特定できた
\item \textbf{反復的な改善の限界}: 細かな視覚的調整を繰り返すよりも、シンプルな方針に転換する方が効果的な場合がある
\item \textbf{オリジナルとの比較の重要性}: 常にオリジナル出力と比較しながら作業を進めることで、差異を明確に把握できた
\end{enumerate}



%% ====== dialog_2025-12-25_full.tex ======
%% dialog_2025-12-25_full.tex
%% 2025年12月25日の開発対話記録(608メッセージ)

\section{2025年12月25日:Qtダイアログとファイルフィルタリング}

この日は、Qtファイルダイアログでのフィルタリング機能の実装と、設定保存/読み込み機能、ytdlスクリプトのGUI統合が行われた。

\subsection{ファイルフィルタ機能の要望}

\begin{userbox}
ファイルオープンのダイアログを開いた際に、関係のないファイルをフィルタする機能を実装してください。
\end{userbox}

最初の試みでは、QFileDialogの標準的なフィルタ機能(\texttt{setNameFilter}、\texttt{setNameFilters})を使用したが、Qtダイアログでは正しく機能しなかった。

\begin{userbox}
フィルターが機能していないようですよ
\end{userbox}

\begin{userbox}
フィルタを選択しても、対象のファイルに切り替わらないですね。
\end{userbox}

ネイティブOSダイアログでは正常に機能するが、Qtダイアログでは動作しないという状況が判明した。

\begin{userbox}
OS配下のダイアログは、うまくいきますね。
\end{userbox}

ユーザーはあくまでQtダイアログでの実装を希望した。

\begin{userbox}
いえいえ、Qtダイアログでフィルタを実装してください!!!!
\end{userbox}

\subsection{FileFilterProxyModelの実装}

最終的に、\texttt{QSortFilterProxyModel}を継承したカスタムプロキシモデルを作成することで解決した。

\begin{lstlisting}[language=python,caption=FileFilterProxyModel,basicstyle=\ttfamily\tiny]
class FileFilterProxyModel(QSortFilterProxyModel):
    """ファイル拡張子でフィルタリングするプロキシモデル"""
    def __init__(self, extensions=None, parent=None):
        super().__init__(parent)
        self.extensions = extensions or []

    def set_extensions(self, extensions):
        self.extensions = extensions
        self.invalidateFilter()

    def filterAcceptsRow(self, source_row, source_parent):
        model = self.sourceModel()
        index = model.index(source_row, 0, source_parent)
        file_path = model.filePath(index)

        # ディレクトリは常に表示
        if model.isDir(index):
            return True

        if not self.extensions:
            return True

        suffix = Path(file_path).suffix.lower()
        return suffix in self.extensions
\end{lstlisting}

\begin{userbox}
機能しますが、フォルダをグレーアウトするのはいただけませんね。
\end{userbox}

\begin{assistantbox}
フォルダを常に表示するようにフィルタ設定を修正します。
\end{assistantbox}

\texttt{filterAcceptsRow}メソッドで、ディレクトリの場合は常に\texttt{True}を返すよう修正することで解決した。

\subsection{yt-srtスクリプトのエラーハンドリング修正}

\begin{userbox}
字幕ファイルもダウンロードされていませんけど。
\end{userbox}

\begin{assistantbox}
問題を発見しました。\texttt{yt-dlp}は字幕がなくても終了コード0を返します。字幕ファイルが実際に生成されたかを確認するように修正します。
\end{assistantbox}

\texttt{bin/yt-srt}スクリプトが、字幕ファイルが存在しない場合でも成功を返していた問題が修正された。実行前後のSRTファイル数を比較し、新しいファイルが生成されなかった場合はエラーを返すよう変更された。

\subsection{設定の保存/読み込み機能}

\begin{userbox}
この一連の処理を保存する機能を実装しましょう。
\end{userbox}

\begin{assistantbox}
どのような保存機能をお考えですか?

\begin{enumerate}
\item \textbf{設定の保存/読み込み} - ゴール、メタデータ、ファイルパス等をJSONで保存し、後で復元
\item \textbf{セッション自動保存} - アプリ終了時に自動保存、起動時に復元
\item \textbf{プロジェクトファイル} - 文字起こし結果やプロンプトも含めて一式保存
\end{enumerate}
\end{assistantbox}

\begin{userbox}
まずは、1ですかね。
\end{userbox}

JSON形式での設定保存/読み込み機能が実装された。保存される項目:
\begin{itemize}
\item ゴールタイプ
\item タイトル、日時、キーパーソン、組織、利用者
\item YouTube URL / 動画ファイルパス
\item チャプターファイル
\end{itemize}

\begin{userbox}
設定読込のダイアログも、ファイルを開くのと同じようにQtで実装してセンターになるようにしてください。
\end{userbox}

設定保存/読み込みダイアログにも\texttt{FileFilterProxyModel}を適用し、JSONファイルのみが表示されるよう修正された。

\subsection{ytdlスクリプトのGUI統合}

ユーザーは既存の\texttt{ytdl}関数(Claude CLIでファイル名を自動生成する機能付き)を共有した。

\begin{userbox}
元々は、こんなスクリプトで動画とSRTをダウンロードしてましたよ。
\end{userbox}

\begin{assistantbox}
高機能なスクリプトですね。主な特徴:

\begin{enumerate}
\item \textbf{Claudeでタイトル短縮} - ファイル名を自動生成
\item \textbf{リモート/ローカル自動選択} - zeusサーバーのClaude CLIを使用
\item \textbf{字幕リネーム} - \texttt{\{name\}.ja.srt} → \texttt{\{name\}\_yt.srt}
\item \textbf{Safariクッキー使用} - 認証済み動画もダウンロード可能
\end{enumerate}
\end{assistantbox}

\begin{userbox}
ダウンロードするビデオのファイル名を、GUIにて指定するように変更したいと思います。
\end{userbox}

\begin{userbox}
ダウンロードしたsrtは、ファイル名\_yt.srtにしてください。
\end{userbox}

\begin{userbox}
加えて、作業フォルダに同名の動画がある場合は、動画のダウンロードをスキップしてください。
\end{userbox}

これらの仕様変更に基づき、\texttt{bin/ytdl}スクリプトが作成され、GUIにファイル名入力フィールドが追加された。

\subsection{この日の成果}

\begin{enumerate}
\item \textbf{FileFilterProxyModel}の実装 - Qtダイアログでのファイルフィルタリング
\item \textbf{設定保存/読み込み}機能の追加 - JSON形式
\item \textbf{yt-srt}スクリプトの修正 - 字幕未取得時のエラー処理
\item \textbf{ytdl}スクリプトのGUI統合 - ファイル名指定、同名ファイルスキップ
\item ファイル名入力フィールドのUI追加
\end{enumerate}

\subsection{技術的なポイント}

この日の開発で重要だった技術的ポイント:

\begin{enumerate}
\item \textbf{Qtネイティブダイアログの制限}: 標準の\texttt{setNameFilter}はネイティブダイアログでは機能するが、Qtダイアログ(\texttt{DontUseNativeDialog})では機能しない
\item \textbf{QSortFilterProxyModelの活用}: カスタムプロキシモデルを作成することで、ファイルシステムモデルのフィルタリングを実現
\item \textbf{yt-dlpの終了コード}: 字幕が存在しなくても終了コード0を返すため、実際のファイル生成を確認する必要がある
\end{enumerate}



%% ====== dialog_2025-12-26_full.tex ======
%% dialog_2025-12-26_full.tex
%% 2025年12月26日の開発対話記録(1124メッセージ)

\section{2025年12月26日:prep\_gui.pyの本格的なUI/UX改善とエクスポート機能の実装}

この日は、prep\_gui.pyの大規模な改善作業が行われた。1124メッセージに及ぶ長時間のセッションで、UI/UXの改善、エクスポート機能の実装、チャプター除外機能の追加が行われた。

\subsection{CenteredFileDialogの実装}

\begin{userbox}
ダイアログは、必ずWindowの中心に配置して欲しいのでQtのダイアログをカスタムしてフィルタ機能も追加してください。
\end{userbox}

\begin{assistantbox}
CenteredFileDialogクラスを作成します。showEventで親ウィンドウの中心に配置し、拡張子フィルタも適用します。
\end{assistantbox}

以下の機能を持つカスタムファイルダイアログが実装された:

\begin{itemize}
\item 親ウィンドウの中心に配置
\item QFileSystemModel.setNameFilters()による拡張子フィルタ
\item setNameFilterDisables(False)で対象外ファイルを非表示
\end{itemize}

\subsection{MP3結合とMP4変換}

\begin{userbox}
mp3の結合は、どのように行われますか
\end{userbox}

\begin{assistantbox}
ffmpegのconcatデムクサを使用します。ファイルリストを作成し、\texttt{-f concat -safe 0 -i list.txt}で結合します。
\end{assistantbox}

ユーザーは既存のシェル関数を共有した:

\begin{lstlisting}[language=bash,basicstyle=\ttfamily\tiny]
mp3tomp4 () {
    local mp3_file="$1"
    local image_file="$2"
    local output_file="${mp3_file:r}.mp4"
    local font="/System/Library/Fonts/ヒラギノ角ゴシック W6.ttc"
    local drawtext="drawtext=fontfile='${font}':textfile='${tmpfile}'"
    drawtext="${drawtext}:fontsize=56:fontcolor=white:borderw=2"
    drawtext="${drawtext}:box=1:boxcolor=black@0.6:boxborderw=15"
    drawtext="${drawtext}:x=(w-text_w)/2:y=h*0.325-th/2"
    ffmpeg -loop 1 -i "$image_file" -i "$mp3_file" \
        -c:v libx264 -preset ultrafast -tune stillimage -crf 32 \
        -c:a aac -b:a 192k -vf "${drawtext},${vf_pad}" \
        -pix_fmt yuv420p -shortest -movflags +faststart "$output_file"
}
\end{lstlisting}

このスタイル(ヒラギノフォント、半透明黒背景ボックス、画面上部1/3の位置)が後のチャプター名焼き込み機能に踏襲された。

\subsection{エンコード進捗表示の実装}

\begin{userbox}
エンコードの進捗状況も表示するようにできますか。
\end{userbox}

\begin{assistantbox}
ffmpegの出力から\texttt{time=}行を解析して進捗を表示します。QProcessを使用してリアルタイムで読み取ります。
\end{assistantbox}

\begin{userbox}
加えて、エンコード中にUIをロックしないようにしてください。
\end{userbox}

MergeWorker(QThread)を使用したバックグラウンド処理が実装された。

\subsection{タブ構成の変更}

\begin{userbox}
タブ構成で、最終出力のタブはあまり設定することもないので、タブ2にマージして出力する仕様に変更しようと考えていますが、いかがでしょう。
\end{userbox}

\begin{assistantbox}
良いと思います。ExportTabをEditTabに統合し、書出オプション(チャプター埋込、タイトル埋込)と書出ボタンを追加します。
\end{assistantbox}

3タブ構成から2タブ構成に変更された:
\begin{itemize}
\item タブ1: MergeTab(MP3結合・MP4変換)
\item タブ2: EditTab(チャプター編集・書出)
\end{itemize}

\subsection{再生コントロールのレイアウト改善}

\begin{userbox}
メインの再生早送りなどのボタンを中心に持ってきてください。また、再生ボタンの色を明るめの緑色で統一してください。
\end{userbox}

\begin{userbox}
画像と波形のグラフの幅を厳密に揃えてください。また、ボタンの配置は、波形の中心に配置してください。
\end{userbox}

AspectRatioWidgetにinnerWidthChangedシグナルを追加し、映像の幅変更時に波形・コントロールエリアの幅を同期する機能が実装された。

\begin{userbox}
ボタンを映像に重ねてほしいわけではありません。波形表示の下にあるコントロールボタンを中央に配置してください。というお願いでした。
\end{userbox}

この対話では、一時的にVideoOverlayWidget(映像上に再生ボタンを重ねる)が実装されたが、ユーザーの意図とは異なることが判明し、削除された。

\subsection{エクスポート機能の実装}

\begin{userbox}
大丈夫です!いよいよ書出の実装を行いましょう。
\end{userbox}

ExportWorkerクラスが作成され、以下の機能が実装された:

\begin{itemize}
\item FFMETADATA1形式のメタデータファイル生成
\item チャプター情報の埋め込み(\texttt{-map\_chapters 1})
\item タイトルの埋め込み
\item ストリームコピー(再エンコードなし)による高速書出
\end{itemize}

\subsection{チャプター名の映像焼き込み}

\begin{userbox}
チャプターごとに、チャプター名を動画に表示させたいと考えています。
\end{userbox}

\begin{assistantbox}
ffmpegのdrawtextフィルターを使用します。既存のmp3tomp4スクリプトのスタイルを踏襲して実装します。
\end{assistantbox}

\begin{userbox}
映像に5秒間のみのやきこみではなく、再生時に、常に再生しているチャプター名を表示する仕様にしてください。
\end{userbox}

当初は各チャプター開始時に5秒間のみ表示する仕様だったが、チャプター全体で常時表示する仕様に変更された。

\begin{lstlisting}[language=python,basicstyle=\ttfamily\tiny]
def _create_drawtext_filter(self) -> str:
    for i, ch in enumerate(self.chapters):
        start_sec = ch.time_ms / 1000.0
        # 次のチャプターの開始時間まで、または動画終了まで表示
        if i + 1 < len(self.chapters):
            end_sec = self.chapters[i + 1].time_ms / 1000.0
        else:
            end_sec = self.total_duration_ms / 1000.0

        drawtext = (
            f"drawtext=fontfile='{self.FONT_PATH}'"
            f":textfile='{textfiles[i]}'"
            f":fontsize={self.FONT_SIZE}"
            f":fontcolor=white:borderw=2:bordercolor=black"
            f":box=1:boxcolor=black@0.6:boxborderw=15"
            f":x=(w-text_w)/2:y=h*0.325-th/2"
            f":enable='between(t,{start_sec:.3f},{end_sec:.3f})'"
        )
\end{lstlisting}

\subsection{YouTube用チャプターコピー機能}

\begin{userbox}
チャプターファイルの書き出しの際に、Youtube用のms表記のないものを書き出すチェックボックスを追加してください。
\end{userbox}

\begin{userbox}
ちょっと仕様を変更しましょう。Youtube用のチャプターはコピペするだけなので、ファイルに書き出しを行う必要がないので、チェックボックスも削除してコピーボタンを押すとYoutubeのチャプター用にmsが削除されたチャプターリストがコピーされるという仕様に変更してください。
\end{userbox}

ファイル書き出しからクリップボードコピーに仕様変更された。📋ボタンでYouTube用フォーマット(HH:MM:SS タイトル)がコピーされる。

\subsection{チャプター除外機能(--プレフィックス)}

\begin{userbox}
動画のチャプターで、先頭に--をつけたチャプターを除外する実装です。--をチャプター名に付したチャプターを保存する際には、--のチャプターを除いてチャプター名の焼き込みを行うとともに、除外したチャプターの時間を適切に調整してチャプターファイルに保存する仕様に変更したいのです。
\end{userbox}

\begin{assistantbox}
理解しました。例えば:

\textbf{元:}
\begin{enumerate}
\item 0:00:00 オープニング
\item 0:15:27 曲A
\item 0:30:00 --休憩(15分間カット)
\item 0:45:00 曲B
\item 1:00:00 --準備中(10分間カット)
\item 1:10:00 曲C
\end{enumerate}

\textbf{書出後:}
\begin{enumerate}
\item 0:00:00 オープニング
\item 0:15:27 曲A
\item 0:30:00 曲B(15分短縮)
\item 0:45:00 曲C(25分短縮)
\end{enumerate}
\end{assistantbox}

以下のメソッドが実装された:

\begin{itemize}
\item \texttt{\_process\_excluded\_chapters()}: 除外区間・保持区間・調整後チャプターを計算
\item \texttt{\_create\_trim\_concat\_filter()}: ffmpegのtrim→concatフィルター生成
\item 保持区間を結合し、調整後の時間でチャプター埋込・焼込
\end{itemize}

\subsection{進捗バーの実装}

\begin{userbox}
エンコードの進捗状況をどこかに表示したいですね。
\end{userbox}

QProgressBarが追加され、ffmpegの\texttt{time=HH:MM:SS.xx}出力を解析してパーセンテージを表示する機能が実装された。

\begin{lstlisting}[language=python,basicstyle=\ttfamily\tiny]
self.export_progress = QProgressBar()
self.export_progress.setStyleSheet("""
    QProgressBar {
        border: 1px solid #555;
        border-radius: 4px;
        background: #2a2a2a;
        height: 20px;
        text-align: center;
        color: #fff;
    }
    QProgressBar::chunk {
        background: qlineargradient(x1:0, y1:0, x2:1, y2:0,
            stop:0 #4CAF50, stop:1 #66BB6A);
        border-radius: 3px;
    }
""")
\end{lstlisting}

\subsection{この日の成果}

\begin{enumerate}
\item \textbf{CenteredFileDialog}の実装 - 親ウィンドウ中央配置、拡張子フィルタ
\item \textbf{MergeWorker}によるバックグラウンドエンコード
\item \textbf{タブ統合} - 3タブから2タブ構成に変更
\item \textbf{ExportWorker}の実装 - チャプター埋込、タイトル焼込
\item \textbf{チャプター名常時表示} - 各チャプター区間中の連続表示
\item \textbf{YouTube用チャプターコピー} - ワンクリックでクリップボードコピー
\item \textbf{チャプター除外機能} - \texttt{--}プレフィックスで区間カット・時間調整
\item \textbf{進捗バー} - ffmpeg出力解析によるリアルタイム表示
\end{enumerate}

\subsection{技術的なポイント}

\begin{enumerate}
\item \textbf{QSortFilterProxyModelの限界}: 当初カスタムプロキシモデルでフィルタを試みたが、QFileSystemModel.setNameFilters()の方が適切だった
\item \textbf{VideoOverlayWidgetの不要な実装}: ユーザーの意図を誤解し一時的に実装したが、不要と判明し削除
\item \textbf{drawtext + textfile}: 日本語文字のエスケープ問題を回避するため、textfileパラメータを使用
\item \textbf{ffmpeg filter\_complex}: 除外区間のカット処理には、trim→concat→drawtextの複合フィルターが必要
\end{enumerate}

この日は非常に多くの機能が実装され、prep\_gui.pyの行数が大幅に増加した。特にエクスポート機能とチャプター除外機能は複雑なffmpegフィルターチェーンを必要とし、実装に時間を要した。



%% ====== dialog_2025-12-27_full.tex ======
%% dialog_2025-12-27_full.tex
%% 2025年12月27日の開発対話記録(750メッセージ)

\section{2025年12月27日:パッケージ化とリリース準備}

この日は、前日に実装されたチャプター除外機能のUI改善(波形ハッチング)から始まり、パッケージのリネーム、pip install対応、GitHub Actionsによるリリース自動化まで、配布に向けた大規模な整備作業が行われた。

\subsection{波形へのハッチング表示}

\begin{userbox}
ありがとうございます。確認した範囲では大丈夫そうです。加えて、--がついたチャプターの波形にハッチをかけるなど、どこがカットされるか識別しやすいようにできますか。
\end{userbox}

WaveformWidgetに除外区間のハッチング機能が実装された:

\begin{lstlisting}[language=python,basicstyle=\ttfamily\tiny]
def _get_excluded_regions(self) -> List[Tuple[int, int]]:
    """除外チャプター(--で始まる)の区間を取得"""
    if not self.chapters or self.duration_ms <= 0:
        return []
    excluded_regions = []
    sorted_chapters = sorted(self.chapters, key=lambda c: c.time_ms)
    for i, ch in enumerate(sorted_chapters):
        if ch.title.startswith("--"):
            start_ms = ch.time_ms
            if i + 1 < len(sorted_chapters):
                end_ms = sorted_chapters[i + 1].time_ms
            else:
                end_ms = self.duration_ms
            excluded_regions.append((start_ms, end_ms))
    return excluded_regions
\end{lstlisting}

\begin{userbox}
どのタイミングでハッチがかかりますか?チャプターリストのバッファに変化があった場合に、ハッチがかかるようにしてください。
\end{userbox}

チャプターテーブルの\texttt{itemChanged}シグナルに接続し、リアルタイムでハッチングが更新されるよう改善された。

\subsection{YouTubeチャプター貼り付け機能}

\begin{userbox}
Youtubeのチャプターを貼り付ける機能です。Cmd(ctrl)-Vで、クリップボードにyoutubeのチャプター形式がある場合、このテーブルに貼り付ける機能を実装してください。
\end{userbox}

\begin{assistantbox}
YouTubeチャプター形式の貼り付け機能を実装します。Cmd+V(Mac)/ Ctrl+V(Windows/Linux)を検出し、YouTube形式なら貼り付け、そうでなければ通常の動作を行います。
\end{assistantbox}

サポートする形式:
\begin{itemize}
\item \texttt{0:00 タイトル} - M:SS形式
\item \texttt{00:00 タイトル} - MM:SS形式
\item \texttt{1:23:45 タイトル} - H:MM:SS形式
\item \texttt{01:23:45 タイトル} - HH:MM:SS形式
\end{itemize}

\subsection{0:00:00.000開始の明示的処理}

\begin{userbox}
0:00:00.000から始まらない問題に対応しましょう。動画のみを読み込んだ場合、0:00:00.000 開始を必ず入れるしましょう。また、Youtubeのチャプターを貼り付ける場合およびチャプターファイルを読み込む場合、0:00:00.000から始まらない場合は「0:00:00.000 --開始」を自動的に入れる。0:00:00.000からある場合はそのまま。という使用はいかがでしょうか。
\end{userbox}

\begin{assistantbox}
良い仕様ですね。明示的で予測しやすい動作になります。
\end{assistantbox}

\begin{userbox}
UNIX的ではないですけどね。
\end{userbox}

\begin{assistantbox}
おっしゃる通りです。暗黙の動作は予測しにくい。

\textbf{UNIX的なアプローチ:明示的に指定されたものだけを除外する}

冒頭を除外したければ、ユーザーが \texttt{0:00:00.000 --準備} を追加する。現状の実装のままで問題ありません。
\end{assistantbox}

最終的に、ユーザーが明示的にチャプターを追加する必要があるという方針が採用された。ただし、利便性のため:
\begin{itemize}
\item \textbf{動画のみ読込}: \texttt{0:00:00.000 開始}を自動追加
\item \textbf{YouTube貼付け/チャプター読込}: 0から始まらない場合は\texttt{0:00:00.000 --開始}を追加
\end{itemize}

\subsection{未使用ファイルの削除とパッケージ整理}

\begin{userbox}
./concat\_mp4\_with\_chapters.pyって使ってませんよね。
\end{userbox}

\begin{assistantbox}
使っていません。prep\_gui.pyで同等機能が実装されています。削除しますか?
\end{assistantbox}

\begin{userbox}
削除しましょう。
\end{userbox}

続いて、大量のテストメディアファイル(約1.6GB)も削除された。\texttt{.gitignore}が更新され、以下が追加された:

\begin{lstlisting}[basicstyle=\ttfamily\tiny]
# メディアファイル
*.mp3
*.m4a
*.wav

# 画像ファイル(ドキュメント用は除外)
*.jpg
*.jpeg
*.png
!docs/**/*.jpg
!docs/**/*.png
\end{lstlisting}

\subsection{pip install対応}

\begin{userbox}
pipでインストールできるようにしてください。
\end{userbox}

\texttt{pyproject.toml}が作成され、パッケージ構造が整備された:

\begin{lstlisting}[basicstyle=\ttfamily\tiny]
[build-system]
requires = ["hatchling"]
build-backend = "hatchling.build"

[project]
name = "rehearsal-workflow"
version = "1.1.0"
dependencies = [
    "PySide6>=6.6.0",
    "PyYAML>=6.0",
    "numpy>=1.24.0",
    "opencv-python>=4.8.0",
]

[project.scripts]
video-chapter-editor = "rehearsal_workflow.video_chapter_editor:main"
report-workflow = "rehearsal_workflow.report_workflow:main"
\end{lstlisting}

\subsection{コマンド名のリネーム}

\begin{userbox}
pre\_guiとworkflow\_gui、guiであることはわかっているので、機能が想像できる適切な名称に変えたいですね。
\end{userbox}

\begin{assistantbox}
現在の機能から考えると:

\vspace{0.5\baselineskip}
\noindent{\footnotesize
\begin{tabularx}{\linewidth}{@{}lXX@{}}
\toprule
現在 & 機能 & 提案 \\
\midrule
prep\_gui & 動画編集+チャプター+書出 & video-chapter / chapter-editor \\
workflow\_gui & 汎用ワークフロー全体 & rehearsal-flow / report-flow \\
\bottomrule
\end{tabularx}
}
\vspace{0.5\baselineskip}
\end{assistantbox}

\begin{userbox}
video-chapter-editorとreport-workflowにしましょうか。長いですかね。
\end{userbox}

\begin{assistantbox}
CLIコマンドとしては少し長いですが、明確で良いと思います。
\end{assistantbox}

\begin{userbox}
そうですね。気に入らなければ、自分でシンボリックリンクを作れば良いですね。
\end{userbox}

以下のリネームが実行された:
\begin{itemize}
\item \texttt{gui/} → \texttt{rehearsal\_workflow/}
\item \texttt{prep\_gui.py} → \texttt{video\_chapter\_editor.py}
\item \texttt{workflow\_gui.py} → \texttt{report\_workflow.py}
\end{itemize}

\subsection{GitHub Actions リリース自動化}

\begin{userbox}
ですね。あとactionsに登録して、vceだけリリースしてください。その際、バイナリはワンファイルとしフォルダをドロップするとそこを作業用ディレクトリとして起動する仕様にしてください。
\end{userbox}

GitHub Actionsワークフロー(\texttt{.github/workflows/release.yml})が作成された:

\begin{lstlisting}[language=yaml,basicstyle=\ttfamily\tiny]
name: Release
on:
  push:
    tags:
      - 'v*'

jobs:
  build-macos:
    runs-on: macos-latest
    steps:
      - name: Build with PyInstaller
        run: pyinstaller video_chapter_editor.spec
      - name: Create DMG
        run: |
          hdiutil create -volname "Video Chapter Editor" \
            -srcfolder dmg_contents \
            -ov -format UDZO \
            "Video-Chapter-Editor-${{ github.ref_name }}-macOS.dmg"

  build-windows:
    runs-on: windows-latest
    steps:
      - name: Build with PyInstaller
        run: |
          pyinstaller --onefile --windowed --name "VideoChapterEditor" ...
\end{lstlisting}

\begin{userbox}
Windows版も作ってください。
\end{userbox}

Windows版のビルドジョブも追加された。

\subsection{フォルダドロップ対応}

フォルダをドロップすると作業ディレクトリとして起動する機能が実装された:

\begin{lstlisting}[language=python,basicstyle=\ttfamily\tiny]
def main():
    import argparse
    parser = argparse.ArgumentParser(
        description="動画チャプター編集・書出ツール",
        prog="video-chapter-editor"
    )
    parser.add_argument(
        "working_dir",
        nargs="?",
        default=None,
        help="作業ディレクトリ(フォルダをドロップして起動可能)"
    )
    args = parser.parse_args()
    # ... validates and uses working_dir
\end{lstlisting}

PyInstallerの設定(\texttt{video\_chapter\_editor.spec})には、macOSでのフォルダドラッグ\&ドロップ対応のため\texttt{argv\_emulation=True}が設定された。

\subsection{この日の成果}

\begin{enumerate}
\item \textbf{波形ハッチング}の実装 - 除外区間の視覚的表示
\item \textbf{YouTubeチャプター貼り付け} - Cmd+Vでの貼り付け機能
\item \textbf{0:00:00.000開始処理} - 明示的なチャプター追加
\item \textbf{未使用ファイル削除} - 1.6GB削減
\item \textbf{pip install対応} - pyproject.tomlによるパッケージ化
\item \textbf{コマンド名リネーム} - video-chapter-editor, report-workflow
\item \textbf{GitHub Actions} - macOS/Windows自動リリース
\item \textbf{フォルダドロップ対応} - 引数による作業ディレクトリ指定
\end{enumerate}

\subsection{技術的なポイント}

\begin{enumerate}
\item \textbf{UNIX哲学の議論}: 暗黙の動作より明示的な指定を優先する設計方針が確認された
\item \textbf{hatchlingビルドシステム}: setuptoolsではなくhatchlingを採用
\item \textbf{PyInstaller + argv\_emulation}: macOSでのフォルダドロップを実現
\item \textbf{クロスプラットフォームビルド}: GitHub Actionsでmacosおよびwindows-latestランナーを使用
\end{enumerate}



%% ====== dialog_2025-12-28_full.tex ======
%% dialog_2025-12-28_full.tex
%% 2025年12月28日の開発対話記録(229メッセージ)

\section{2025年12月28日:UI改善とワークフロー最適化の設計}

この日は、エンコード品質の改善(GPU/CPU選択、色空間保持)、v1.3.0のリリース、モダンダークUIへの刷新、そしてMP3→MP4変換における2回エンコード問題の解決に向けた設計議論が行われた。

\subsection{エンコード品質の改善}

\begin{userbox}
できましたけど、なんだか画質が元よりも劣化してますね。
\end{userbox}

\begin{assistantbox}
再エンコードによる世代劣化ですね。同じビットレートでも再エンコードすると品質は落ちます。

対策として:
\begin{enumerate}
\item 「元と同じ」のビットレートを1.2〜1.5倍にする --- 元の品質を維持
\item CPUエンコーダ(libx264)でCRFモードを使う --- 品質優先で必要なビットレートを自動決定
\end{enumerate}
\end{assistantbox}

\begin{userbox}
GPUは1で、CPUは2でということにしましょうか。
\end{userbox}

以下の方針で実装された:
\begin{itemize}
\item \textbf{GPU}: 「元と同じ」選択時はビットレート×1.5倍
\item \textbf{CPU}: CRFモード(CRF 18)で高画質
\end{itemize}

\begin{userbox}
なんかCPUの方が早いんですけど。
\end{userbox}

\begin{assistantbox}
静止画動画だからですね。CRFモードは変化のないフレームは少ないビットで済むので高速。ultrafastプリセットを使用しており、静止画はフレーム間差分がほぼゼロなので処理が軽い。GPUはデータ転送オーバーヘッドがあるので、単純な動画では逆に遅くなることがあります。
\end{assistantbox}

\subsection{色空間の保持}

\begin{userbox}
CPUは色合いが少し変わってしまいますね。
\end{userbox}

\begin{userbox}
-colorspaceを追加しましょうか。
\end{userbox}

\texttt{detect\_video\_colorspace()}関数が追加され、ffprobeで元動画の色空間を検出し、エンコード時に維持する機能が実装された:

\begin{lstlisting}[language=python,basicstyle=\ttfamily\tiny]
@dataclass
class ColorspaceInfo:
    """色空間情報"""
    color_space: str = ""
    color_primaries: str = ""
    color_transfer: str = ""

    def get_ffmpeg_args(self) -> List[str]:
        args = []
        if self.color_space and self.color_space != "unknown":
            args.extend(['-colorspace', self.color_space])
        if self.color_primaries and self.color_primaries != "unknown":
            args.extend(['-color_primaries', self.color_primaries])
        if self.color_transfer and self.color_transfer != "unknown":
            args.extend(['-color_trc', self.color_transfer])
        return args
\end{lstlisting}

\begin{userbox}
まあまあ保持されますね。
\end{userbox}

\subsection{v1.3.0のリリース}

\begin{userbox}
今の状態までのアップデートをコミットして、Pushリリースしてください。
\end{userbox}

バージョンを1.3.0に更新し、以下のファイルが変更された:
\begin{itemize}
\item \texttt{rehearsal\_workflow/\_\_init\_\_.py}
\item \texttt{pyproject.toml}
\item \texttt{video\_chapter\_editor.spec}
\item \texttt{README.md}(ダウンロードリンク更新)
\end{itemize}

リリースノートには以下の新機能が記載された:
\begin{itemize}
\item ビットレート自動検出(「元と同じ(自動)」オプション)
\item GPU: ビットレート×1.5、CPU: CRF 18
\item 色空間の自動検出と保持
\end{itemize}

\subsection{モダンダークUIへの刷新}

\begin{userbox}
video-chapter-editorのUIをシンプルにかっこよくしてください。
\end{userbox}

Themeクラスが導入され、統一されたカラーパレットとスタイル定義が実装された:

\begin{lstlisting}[language=python,basicstyle=\ttfamily\tiny]
class Theme:
    """統一されたUIテーマ定義"""
    # 背景色(暗い順)
    BG_DARKEST = "#0f0f0f"
    BG_DARK = "#1a1a1a"
    BG_BASE = "#242424"
    BG_ELEVATED = "#2d2d2d"
    BG_HOVER = "#363636"
    BG_ACTIVE = "#404040"

    # テキスト色
    TEXT_PRIMARY = "#f0f0f0"
    TEXT_SECONDARY = "#a0a0a0"
    TEXT_MUTED = "#666666"

    # アクセントカラー
    ACCENT = "#3b82f6"
    ACCENT_HOVER = "#2563eb"
    ACCENT_ACTIVE = "#1d4ed8"

    # ステータスカラー
    SUCCESS = "#22c55e"
    WARNING = "#f59e0b"
    DANGER = "#ef4444"

    RADIUS = "8px"
    RADIUS_SM = "4px"

    @classmethod
    def button_primary(cls) -> str: ...
    @classmethod
    def button_secondary(cls) -> str: ...
    @classmethod
    def button_danger(cls) -> str: ...
\end{lstlisting}

\subsection{ボタンスタイルのポリシー統一}

\begin{userbox}
タブ1のボタンに色がついているものとついていないものがありますが、明確なポリシーはありますか
\end{userbox}

\begin{assistantbox}
提案:どちらかのポリシーに統一

\begin{enumerate}
\item \textbf{機能ベース}: 主要アクションのみPrimary、他はすべてSecondary
\item \textbf{推奨度ベース}: よく使う操作をPrimaryで目立たせる
\end{enumerate}
\end{assistantbox}

\begin{userbox}
1
\end{userbox}

機能ベースのポリシーが採用され、以下のように統一された:

\vspace{0.5\baselineskip}
\noindent{\footnotesize
\begin{tabularx}{\linewidth}{@{}lX@{}}
\toprule
スタイル & 用途 \\
\midrule
Primary(青) & 各タブの主要アクション(結合実行、書出)のみ \\
Secondary(グレー) & その他すべて(追加、削除、選択、貼り付け、開く、ジャンプなど) \\
Danger(赤) & 中止ボタン \\
\bottomrule
\end{tabularx}
}
\vspace{0.5\baselineskip}

\begin{userbox}
タブ2のチャプター表示において、時間が見切れるのを防止してください。
\end{userbox}

時間列の幅を90px→110pxに拡張し、「H:MM:SS.mmm」形式が見切れないように修正された。

\subsection{ワークフロー最適化の設計議論}

\begin{userbox}
mp3の動画エンコードで、2回エンコードを行うのが美しくないと考えています。mp3の場合は、曲ごとのカットが終わっている場合と、長時間の未編集の場合があると考えていますが、エンコードを一度だけ行うようなおすすめのワークフローはどんなものがありますか
\end{userbox}

\begin{assistantbox}
現状の問題点を整理します:

\textbf{現在のワークフロー(2回エンコード)}
\begin{lstlisting}
タブ1: MP3結合 -> MP4作成(1回目エンコード)
タブ2: MP4読込 -> チャプター編集 -> 書出(2回目エンコード)
\end{lstlisting}

\textbf{提案: 1回エンコードのワークフロー}

\vspace{0.5\baselineskip}
\noindent{\footnotesize
\begin{tabularx}{\linewidth}{@{}lX@{}}
\toprule
ケース & 推奨ワークフロー \\
\midrule
曲別MP3(カット済み) & タブ1で結合+チャプター自動生成 → 直接最終MP4出力 \\
長時間MP3(未編集) & タブ2でMP3を直接読込 → チャプター編集 → 書出 \\
\bottomrule
\end{tabularx}
}
\vspace{0.5\baselineskip}
\end{assistantbox}

\subsection{一筆書き問題としての分析}

\begin{userbox}
なるほど。一筆書き問題に見えてきました。機能重複は、起点が3つ以上あるということですね。
\end{userbox}

\begin{assistantbox}
その通りです。グラフ理論的に整理すると:

\textbf{起点(入力パターン):}
\begin{enumerate}
\item 複数MP3(カット済み)
\item 単一MP3(長尺未編集)
\item 既存MP4
\end{enumerate}

\textbf{終点:} チャプター付きMP4

一筆書きできない構造:
\begin{lstlisting}
起点1 --> 結合 --> カバー --+
起点2 ---------> カバー ----+--> チャプター編集 -> 書出
起点3 ---------------------+
\end{lstlisting}

\textbf{解決策:共通パスを1つにして、入口を分岐}
\begin{lstlisting}
+- 起点1: 複数MP3 -> 結合 -+
+- 起点2: 単一MP3 ---------+--> [カバー画像?] -> チャプター編集 -> 書出
+- 起点3: MP4 ------------+
                           ^
                      ここを1つのTabに
\end{lstlisting}
\end{assistantbox}

\subsection{タイトル焼き込みと劣化の関係}

\begin{userbox}
焼き込みありを前提にして、劣化と手間を極限したいんですよね。
\end{userbox}

タイトル焼き込みが必須の場合の最適化が議論された:

\vspace{0.5\baselineskip}
\noindent{\footnotesize
\begin{tabularx}{\linewidth}{@{}lXX@{}}
\toprule
入力 & 映像処理 & 音声処理 \\
\midrule
MP3+静止画 & enc(生成+焼込)1回 & enc 1回 \\
MP4 & enc(焼込)1回 & copy(無劣化) \\
\bottomrule
\end{tabularx}
}
\vspace{0.5\baselineskip}

重要な発見として、MP4入力時は音声を\texttt{-c:a copy}で無劣化コピーできることが確認された。

\subsection{最終的なタブ構成の提案}

\begin{userbox}
タブ1は、複数のMP3の読み込み、一つのMP3の出力とチャプターファイルの出力のみにし、カバー画像は、タブ2からボタンで呼び出し別画面が起動する流れもアリでしょうか。
\end{userbox}

\begin{assistantbox}
その方がスッキリしますね。

\textbf{提案の構造:}

\textbf{Tab 1: MP3結合(シンプル化)}
\begin{itemize}
\item 入力: 複数MP3
\item 出力: 結合済みMP3 + chapters.txt
\item カバー画像機能なし
\end{itemize}

\textbf{Tab 2: 編集・書出}
\begin{itemize}
\item ファイル読込: MP3(Tab 1経由 or 直接)/ MP4
\item カバー画像設定 → 別画面ダイアログ
\item 波形・チャプター編集
\item 書出
\end{itemize}

\textbf{各ケースのフロー:}

\vspace{0.5\baselineskip}
\noindent{\footnotesize
\begin{tabularx}{\linewidth}{@{}lX@{}}
\toprule
ケース & フロー \\
\midrule
A. 複数MP3 & Tab 1で結合 → Tab 2読込 → カバー設定 → 書出 \\
B. 単体MP3 & Tab 2で直接読込 → カバー設定 → 書出 \\
C. MP4 & Tab 2で読込 → 書出(カバー不要) \\
\bottomrule
\end{tabularx}
}
\vspace{0.5\baselineskip}
\end{assistantbox}

\begin{userbox}
この場合、タブ1操作における劣化はほぼないという認識で良いですね。
\end{userbox}

Tab 1では結合のみでエンコードを行わないため、劣化は発生しない設計となった。

\subsection{report-workflowとの連携}

\begin{userbox}
以降のフローも考慮しておきたいと考えています。動画をYoutubeに登録してsrtをダウンロード、WhisperでSRTをダウンロード。これからClaude Codeのプロンプトにて所望の形式で概要を作成、保存という流れです。これは別UIのアプリで作成する方針で進めていますが、この方針に対して意見はありますか。
\end{userbox}

\begin{assistantbox}
設計方針として適切だと考えます。

\textbf{分離が適切な根拠:}
\begin{enumerate}
\item \textbf{責務の明確な境界}
  \begin{itemize}
  \item video-chapter-editor: 映像加工(決定論的処理)
  \item report-workflow: テキスト生成(AI/LLM処理)
  \end{itemize}
\item \textbf{入力の性質が異なる}
  \begin{itemize}
  \item 前者:ファイル操作、波形、プレビュー
  \item 後者:SRT解析、プロンプト設計、テキスト編集
  \end{itemize}
\item \textbf{「配管と陶器」の思想に合致}
\end{enumerate}

\textbf{提案:共通の作業ディレクトリ規約}

\begin{lstlisting}
work/
+-- input/           # 元素材
+-- output.mp4       # video-chapter-editor出力
+-- chapters.txt     # チャプターリスト
+-- transcript.srt   # 字幕(YouTube or Whisper)
+-- report.tex       # report-workflow出力
\end{lstlisting}

この規約があれば、両アプリは疎結合のまま連携できます。
\end{assistantbox}

\subsection{この日の成果}

\begin{enumerate}
\item \textbf{エンコード品質改善} - GPU×1.5倍ビットレート、CPU CRF 18
\item \textbf{色空間保持機能} - ffprobeで検出、ffmpegで維持
\item \textbf{v1.3.0リリース} - GitHub Actionsで自動ビルド
\item \textbf{モダンダークUI} - Themeクラスによる統一スタイル
\item \textbf{ボタンポリシー統一} - 機能ベース(Primary=主要アクションのみ)
\item \textbf{ワークフロー最適化設計} - 1回エンコードへの移行計画
\item \textbf{タブ構成再設計} - Tab 1は結合のみ、Tab 2で共通編集
\end{enumerate}

\subsection{技術的なポイント}

\begin{enumerate}
\item \textbf{静止画動画のエンコード特性}: CPU(CRF)の方がGPUより高速かつ高品質
\item \textbf{音声の無劣化コピー}: タイトル焼き込み時もMP4入力なら\texttt{-c:a copy}で音声劣化ゼロ
\item \textbf{グラフ理論的設計分析}: 一筆書き問題として入力パターンの分岐を整理
\item \textbf{責務分離}: video-chapter-editorとreport-workflowの明確な境界設計
\end{enumerate}



%% ====== dialog_2025-12-29_full.tex ======
%% dialog_2025-12-29_full.tex
%% 2025年12月29日の開発対話記録(1739メッセージ)

\section{2025年12月29日:UI大改造とui\_nextパッケージの実装}

この日は、1739メッセージに及ぶ過去最大規模のセッションとなった。タブベースUIから単一ワークスペース+ダイアログパターンへの移行、開発ログの整理、配管と陶器の分界点マトリクス作成、そしてui\_nextパッケージの完全実装が行われた。

\subsection{タブ構成の再検討}

\begin{userbox}
そうなるとタブ1とタブ2を分けなくても良いような気がしますが、いかがでしょう。まあ、陶器が巨大になるんですけど。笑
\end{userbox}

\begin{assistantbox}
その通りですね。現在の機能を整理すると:

\textbf{タブ1(MP3結合)の実体:}
\begin{itemize}
\item 複数MP3 → 結合MP3(\texttt{-c copy}で無劣化)
\item チャプターファイル自動生成
\end{itemize}

\textbf{タブ2(編集・書出)の実体:}
\begin{itemize}
\item MP3/MP4読込
\item カバー画像設定
\item 波形表示・チャプター編集
\item 書出
\end{itemize}

タブ1の出力がそのままタブ2の入力になるため、分ける必然性は薄いですね。
\end{assistantbox}

\begin{userbox}
1の入力ソースも別画面ですかね。その際に、チャプターを自動挿入、2のカバー画像もユーザーが明示的に保存しなくても自動挿入という感じで、統一感が出ますね。
\end{userbox}

\begin{assistantbox}
良いですね。以下の構成を提案します:

\textbf{メインワークスペース(1画面)}
\begin{itemize}
\item 動画/波形プレビュー
\item チャプターリスト
\item 書出ボタン
\end{itemize}

\textbf{モーダルダイアログ}
\begin{enumerate}
\item \textbf{SourceSelectionDialog}: MP3複数選択 or 単一MP3/MP4選択 → 自動でチャプター挿入
\item \textbf{CoverImageDialog}: 画像選択・クロップ → 自動で状態に反映
\end{enumerate}

ユーザーは「開く」「カバー設定」ボタンを押すだけで、バックグラウンドで処理が完了します。
\end{assistantbox}

\subsection{開発ログの再編成}

\begin{userbox}
DEVELOPMENT\_LOG\_as\_of\_2025-12-29.mdとして、一連のログを集約して書き出してください。今回の大改造以降のログをDEVELOPMENT\_LOG.mdとして新たに作成してください。
\end{userbox}

過去のログがアーカイブされ、新しい開発ログが作成された。

\begin{userbox}
昨日のグラフ表現に関するディスカッションを追加して、トレーサビリティを確保してください。
\end{userbox}

一筆書き問題(オイラー路)としてワークフローを分析した議論が追加された:

\begin{lstlisting}[basicstyle=\ttfamily\tiny]
起点1 --> 結合 --> カバー --+
起点2 ---------> カバー ----+--> チャプター編集 -> 書出
起点3 ---------------------+
\end{lstlisting}

\begin{userbox}
ダイアグラムの可視性向上のために、Mermaid記法にて記述してください。
\end{userbox}

Mermaid記法でフロー図が追加された:

\begin{lstlisting}[basicstyle=\ttfamily\tiny]
graph LR
    A[複数MP3] --> B[MP3結合]
    B --> C[カバー画像設定]
    D[単一MP3] --> C
    E[MP4] --> F[チャプター編集]
    C --> F
    F --> G[書出]
\end{lstlisting}

\begin{userbox}
この一連のディスカッションをPADで可視化する利点はありますか。
\end{userbox}

\begin{assistantbox}
PADとMermaidの使い分けを整理すると:

\vspace{0.5\baselineskip}
\noindent{\footnotesize
\begin{tabularx}{\linewidth}{@{}lXX@{}}
\toprule
 & PAD & Mermaid \\
\midrule
用途 & 実装詳細・制御フロー & 設計概要・データフロー \\
強み & 条件分岐・ループの表現 & 状態遷移・依存関係 \\
ツール & 専用エディタ必要 & テキストのみ \\
\bottomrule
\end{tabularx}
}
\vspace{0.5\baselineskip}

現段階では設計議論が中心なので、Mermaidが適切です。実装に入ったらPADで詳細化するのが良いでしょう。
\end{assistantbox}

\subsection{ユースケースの詳細化}

\begin{userbox}
UC3に、編集済みと要編集の派生パターンがありますね。
\end{userbox}

ユースケースが詳細化された:

\vspace{0.5\baselineskip}
\noindent{\footnotesize
\begin{tabularx}{\linewidth}{@{}lX@{}}
\toprule
ユースケース & 説明 \\
\midrule
UC1 & 複数MP3(カット済み)→ 結合 → カバー → 書出 \\
UC2 & 単一MP3(長尺)→ カバー → チャプター編集 → 書出 \\
UC3a & MP4(編集済み)→ そのまま書出 \\
UC3b & MP4(要編集)→ チャプター編集 → 書出 \\
UC4 & 複数MP3 → 結合 → カバー → チャプター編集 → 書出 \\
\bottomrule
\end{tabularx}
}
\vspace{0.5\baselineskip}

\begin{userbox}
LOGを時系列逆にしましょうか。
\end{userbox}

開発ログが逆時系列順(最新が上)に変更された。

\subsection{配管と陶器の分界点マトリクス}

\begin{userbox}
「ユースケース詳細マトリクス」が、配管と陶器の分界点を可視化する重要な表になりそうです。
\end{userbox}

\begin{userbox}
ツールのサブコマンドも入れてしまいましょう。
\end{userbox}

\begin{userbox}
配管部分の操作をUI操作とバックヤードのサブコマンドに分けましょうか。
\end{userbox}

詳細なマトリクスが作成された:

\vspace{0.5\baselineskip}
\noindent{\footnotesize
\begin{tabularx}{\linewidth}{@{}lXXX@{}}
\toprule
ステップ & ユーザー操作 & プラットフォーム & 要実装 \\
\midrule
ソース選択 & ファイルD\&D & QFileDialog & --- \\
MP3結合 & 結合ボタン & --- & ffmpeg -c copy \\
カバー設定 & ダイアログ & QDialog, QImage & ImageCropWidget \\
波形表示 & --- & --- & WaveformWorker \\
書出 & 書出ボタン & --- & ExportWorker \\
\bottomrule
\end{tabularx}
}
\vspace{0.5\baselineskip}

\begin{userbox}
バックエンドをOS、PySide6などの機能と、開発者側で実装しないといけないことを弁別してください。
\end{userbox}

\begin{userbox}
yt-dlpやwhisper-remoteは、開発者サイドで実装したので、要実装に移動しましょうか。
\end{userbox}

\begin{userbox}
luatex-pdfも自作ツールなんですよね。
\end{userbox}

\begin{userbox}
我ながら、色々と作ってますね。
\end{userbox}

自作ツール群が明確化された:

\vspace{0.5\baselineskip}
\noindent{\footnotesize
\begin{tabularx}{\linewidth}{@{}lX@{}}
\toprule
ツール & 機能 \\
\midrule
ytdl & Claude CLIでファイル名自動生成+yt-dlp \\
whisper-remote & リモートサーバーでのWhisper文字起こし \\
luatex-pdf & リモートLuaLaTeXコンパイル \\
yt-srt & YouTube字幕取得 \\
\bottomrule
\end{tabularx}
}
\vspace{0.5\baselineskip}

\subsection{UI設計の課題と対応}

\begin{userbox}
現状を踏まえて懸念点を
\end{userbox}

\begin{assistantbox}
主な懸念点:

\begin{enumerate}
\item \textbf{UIの複雑化}: 単一ワークスペースが巨大になる可能性
\item \textbf{状態管理}: プロジェクト状態の一貫性維持
\item \textbf{ログ管理}: デバッグ情報の可視化
\item \textbf{エラーハンドリング}: ユーザーへの適切なフィードバック
\end{enumerate}

優先度として、2(UI再設計)から着手し、その後1(ロギング)を実装することを提案します。
\end{assistantbox}

\begin{userbox}
2からですかね。1のUIは、スケルトンだけ作成してください。エラー通知はUIに、コマンドラインのログ表示機能を作成して...
\end{userbox}

\subsection{ui\_nextパッケージの作成}

新しいUIパッケージ構造が作成された:

\begin{lstlisting}[basicstyle=\ttfamily\tiny]
rehearsal_workflow/ui_next/
+-- __init__.py
+-- __main__.py
+-- app.py              # VideoChapterEditorNext
+-- main_workspace.py   # MainWorkspace
+-- dialogs.py          # SourceSelectionDialog, CoverImageDialog
+-- log_panel.py        # LogPanel with LogLevel
+-- models.py           # ChapterInfo, ColorspaceInfo
+-- workers.py          # MergeWorker, ExportWorker, WaveformWorker
+-- image_crop.py       # ImageCropWidget
\end{lstlisting}

\begin{userbox}
どうやって起動するんでしたっけ
\end{userbox}

\begin{assistantbox}
モジュールとして実行します:

\begin{lstlisting}[language=bash,basicstyle=\ttfamily\tiny]
python -m rehearsal_workflow.ui_next
\end{lstlisting}

\texttt{\_\_main\_\_.py}があるため、パッケージを直接実行できます。
\end{assistantbox}

\subsection{動画プレビュー機能の実装}

\begin{userbox}
python -m rehearsal\_workflow.ui\_nextを確認しました。この状態で右側に、動画読み込みUIを追加してみてください。
\end{userbox}

QMediaPlayer、QVideoWidget、QAudioOutputを使用した動画プレビュー機能が実装された:

\begin{lstlisting}[language=python,basicstyle=\ttfamily\tiny]
from PySide6.QtMultimedia import QMediaPlayer, QAudioOutput
from PySide6.QtMultimediaWidgets import QVideoWidget

self.media_player = QMediaPlayer()
self.audio_output = QAudioOutput()
self.video_widget = QVideoWidget()
self.media_player.setAudioOutput(self.audio_output)
self.media_player.setVideoOutput(self.video_widget)
\end{lstlisting}

\subsection{ウィンドウサイズとクロスプラットフォーム対応}

\begin{userbox}
GUIのサイズを1440×900固定で、macおよびWindowsで問題なく表示を行うことができるようにデザインしましょう。
\end{userbox}

クロスプラットフォームのフォント検出機能が実装された:

\begin{lstlisting}[language=python,basicstyle=\ttfamily\tiny]
def get_system_font() -> str:
    system = platform.system()
    if system == "Darwin":  # macOS
        return "Helvetica Neue"
    elif system == "Windows":
        return "Segoe UI"
    else:  # Linux
        return "Noto Sans"

def get_monospace_font() -> str:
    system = platform.system()
    if system == "Darwin":
        return "Menlo"
    elif system == "Windows":
        return "Consolas"
    else:
        return "Noto Sans Mono"
\end{lstlisting}

当初「SF Pro Text」を使用していたが、Qtで警告が出たためHelvetica Neue/Menloに変更された。

\subsection{レイアウトの再構成}

\begin{userbox}
select sourceの下にあるwaveform displayはVideo Previewの下におきたいですね。Chaptersはこの位置にしましょう。Video Previewは、いっぱいに広げてください。
\end{userbox}

\begin{userbox}
操作系は、全て左に寄せましょうか。操作の順に左側に移動してください。
\end{userbox}

\begin{userbox}
Cover imageからExportも左に移動してください。
\end{userbox}

\begin{userbox}
ログも左に寄せましょうかね。
\end{userbox}

最終的なレイアウト構成:

\vspace{0.5\baselineskip}
\noindent{\footnotesize
\begin{tabularx}{\linewidth}{@{}lX@{}}
\toprule
位置 & コンポーネント \\
\midrule
左パネル(480px固定) & ソース選択、カバー画像、チャプター、書出、ログ \\
右パネル(残り) & 動画プレビュー、波形表示、再生コントロール \\
\bottomrule
\end{tabularx}
}
\vspace{0.5\baselineskip}

\begin{userbox}
左側の幅を1:3くらいで調整してみてください。
\end{userbox}

\begin{userbox}
半々くらいになってますね。
\end{userbox}

QSplitterの\texttt{setSizes()}が意図通りに動作しなかったため、\texttt{setFixedWidth(480)}で固定幅を設定する方式に変更された。

\begin{userbox}
全体のサイズを1440×1080にしましょう
\end{userbox}

\begin{userbox}
元の方が良いですね。
\end{userbox}

1440×1080を試したが、ユーザーの好みにより1440×900に戻された。

\subsection{作業ディレクトリのサポート}

\begin{userbox}
起動時にフォルダパスを引数として作業ディレクトリとするようにしてください。
\end{userbox}

\begin{lstlisting}[language=python,basicstyle=\ttfamily\tiny]
def main():
    parser = argparse.ArgumentParser()
    parser.add_argument("work_dir", nargs="?", default=None)
    args = parser.parse_args()

    app = QApplication(sys.argv)
    window = VideoChapterEditorNext(work_dir=args.work_dir)
    window.show()
    sys.exit(app.exec())
\end{lstlisting}

ダイアログも作業ディレクトリを初期ディレクトリとして使用するよう更新された。

\subsection{波形表示のバックグラウンド処理}

\begin{userbox}
波形表示、再生コントロールなどを同様にUIロックを行わないように別スレッドで実装してください。
\end{userbox}

WaveformWorkerクラスがQThreadベースで実装された:

\begin{lstlisting}[language=python,basicstyle=\ttfamily\tiny]
class WaveformWorker(QObject):
    finished = Signal(np.ndarray, int)
    error = Signal(str)

    def __init__(self, file_path: str):
        super().__init__()
        self.file_path = file_path

    def run(self):
        try:
            # ffmpegで音声をPCMに変換
            cmd = [
                "ffmpeg", "-i", self.file_path,
                "-vn", "-ac", "1", "-ar", "8000",
                "-f", "s16le", "-"
            ]
            result = subprocess.run(cmd, capture_output=True)
            samples = np.frombuffer(result.stdout, dtype=np.int16)
            self.finished.emit(samples.astype(np.float32), 8000)
        except Exception as e:
            self.error.emit(str(e))
\end{lstlisting}

\begin{userbox}
\textasciitilde/Dropbox/01\_Projects/00\_Horn\_Works/20260125\_レオケ/2025-12-21を引数にして起動してください。
\end{userbox}

実際の作業ディレクトリでテストが行われた。

\subsection{波形表示の改善}

\begin{userbox}
波形表示のノーマライズを行いましょうか。
\end{userbox}

波形データの正規化が実装された:

\begin{lstlisting}[language=python,basicstyle=\ttfamily\tiny]
max_val = np.max(np.abs(samples))
if max_val > 0:
    samples = samples / max_val
\end{lstlisting}

\begin{userbox}
あと、波形表示の画面解像度に応じた間引き処理も入れましょう。min-max法で良いかと思いますが、いかがでしょうか。
\end{userbox}

\begin{assistantbox}
min-max decimationは最適な選択ですね。ピークを保持しながらデータ量を削減できます。

\begin{lstlisting}[language=python,basicstyle=\ttfamily\tiny]
def decimate_minmax(samples, target_points):
    chunk_size = len(samples) // target_points
    decimated = []
    for i in range(target_points):
        chunk = samples[i*chunk_size:(i+1)*chunk_size]
        decimated.append(np.min(chunk))
        decimated.append(np.max(chunk))
    return np.array(decimated)
\end{lstlisting}
\end{assistantbox}

サンプル数は4000に設定され、min-max法により解像度に応じた表示が実現された。

\subsection{video\_chapter\_editor.pyからの機能移植}

\begin{userbox}
video\_chapter\_editor.pyに実装したものを全て実装してみてください。
\end{userbox}

以下の機能がui\_nextパッケージに移植された:

\textbf{models.py}
\begin{itemize}
\item ChapterInfo(チャプター情報データクラス)
\item ColorspaceInfo(色空間情報データクラス)
\item detect\_gpu\_encoder()(GPU検出)
\item detect\_video\_colorspace()(色空間検出)
\end{itemize}

\textbf{workers.py}
\begin{itemize}
\item MergeWorker(MP3結合)
\item ExportWorker(書出、チャプター埋込、タイトル焼込)
\item WaveformWorker(波形生成)
\end{itemize}

\textbf{image\_crop.py}
\begin{itemize}
\item ImageCropWidget(16:9カバー画像クロップ)
\end{itemize}

\textbf{main\_workspace.py}
\begin{itemize}
\item チャプターテーブルの可視化
\item YouTube用チャプターコピー
\item 書出オプション(GPU/CPU、ビットレート)
\item 除外チャプター(--プレフィックス)の波形ハッチング
\end{itemize}

\subsection{発生したエラーと解決}

\vspace{0.5\baselineskip}
\noindent{\footnotesize
\begin{tabularx}{\linewidth}{@{}lX@{}}
\toprule
エラー & 解決策 \\
\midrule
ImportError(相対インポート) & \_\_main\_\_.pyを作成し\texttt{python -m}で実行 \\
QFontDatabase.isFixedPitch() TypeError & 文字列パラメータを使用 \\
SF Pro Text警告 & Helvetica Neue/Menloに変更 \\
Splitter比率が効かない & setFixedWidth(480)を使用 \\
\bottomrule
\end{tabularx}
}
\vspace{0.5\baselineskip}

\subsection{この日の成果}

\begin{enumerate}
\item \textbf{UI設計の大転換} - タブベースから単一ワークスペース+ダイアログへ
\item \textbf{開発ログの再編成} - アーカイブ化と逆時系列順
\item \textbf{配管と陶器の分界点マトリクス} - 責務の明確化
\item \textbf{ui\_nextパッケージ} - 完全なモジュール構成
\item \textbf{クロスプラットフォーム対応} - フォント検出、1440×900固定
\item \textbf{バックグラウンド処理} - WaveformWorkerによるUIロック回避
\item \textbf{min-max decimation} - 解像度適応型波形表示
\item \textbf{video\_chapter\_editor.py機能の完全移植}
\end{enumerate}

\subsection{技術的なポイント}

\begin{enumerate}
\item \textbf{QSplitterの制限}: \texttt{setSizes()}は比率保証がないため、\texttt{setFixedWidth()}で明示的に幅を指定
\item \textbf{\_\_main\_\_.pyの役割}: パッケージを\texttt{python -m}で実行可能にする
\item \textbf{QObject.moveToThread()}: QThreadと組み合わせてバックグラウンド処理を実現
\item \textbf{min-max decimation}: ピークを保持しつつデータ量を削減する間引き手法
\item \textbf{オイラー路}: ワークフローをグラフ理論で分析し、共通パスを特定
\end{enumerate}

この日は1739メッセージという過去最大規模のセッションとなり、UI設計の根本的な見直しから実装の詳細に至るまで、包括的な作業が行われた。



%% ====== dialog_2025-12-30_full.tex ======
%% dialog_2025-12-30_full.tex
%% 2025年12月30日の開発対話記録(1063メッセージ)

\section{2025年12月30日:ui\_nextパッケージの大規模UI改善}

この日は、ui\_nextパッケージに対する大規模な改善作業が行われた。1063メッセージに及ぶ長時間のセッションで、スペクトログラム表示の改善、エンコード設定の統一、再生コントロールのリデザイン、ウィンドウリサイズ機能など、多岐にわたる機能が実装された。

\subsection{スペクトログラム表示の改善}

\begin{userbox}
スペクトログラムの色をSOXのデフォルトと同じ色にしてみてください。また、スペクトログラムは、波形の処理の後に自動的に開始するようにしてください。
\end{userbox}

\begin{assistantbox}
SOXのデフォルトカラーマップ(黒→青→シアン→緑→黄→赤→白)を実装し、波形完了後に自動生成を開始します。
\end{assistantbox}

\begin{userbox}
スペクトログラムの色を、青系にしてください。再生ヘッダの位置がみづらいので。
\end{userbox}

再生ヘッドの視認性を考慮し、青系のカラーマップ(黒→濃紺→青→シアン→白)に変更された。

\subsection{メルスペクトログラムの実装}

\begin{userbox}
演奏と指揮者のトークが区別しやすいようなスペクトログラム処理を行なってみてください。
\end{userbox}

\begin{assistantbox}
演奏とトークを区別しやすくするため、以下の処理を実装します:

\begin{enumerate}
\item \textbf{メルスケール変換} --- 人間の聴覚特性に合わせた周波数スケール(低域を拡大、高域を圧縮)
\item \textbf{スペクトルコントラスト強調} --- 音声と音楽の違いを強調
\end{enumerate}
\end{assistantbox}

SpectrogramWorkerにメルスケール対応が実装された:

\begin{lstlisting}[language=python,basicstyle=\ttfamily\tiny]
def _hz_to_mel(self, hz, np):
    return 2595 * np.log10(1 + hz / 700)

def _mel_to_hz(self, mel, np):
    return 700 * (10 ** (mel / 2595) - 1)

def _create_mel_filterbank(self, n_fft: int, sample_rate: int, n_mels: int, np):
    # 三角メルフィルターバンクを作成
    ...
\end{lstlisting}

視覚的な違いとして、トークは低周波に集中したエネルギー分布と水平方向の縞模様(フォルマント)が現れ、演奏は広い周波数帯域に分布する複雑なパターンが現れる。

\subsection{再生ヘッドと波形表示の改善}

\begin{userbox}
波形上の再生ヘッド位置を黄色でもう少し太くできますかね。
\end{userbox}

再生ヘッドが黄色(\texttt{\#FFEB3B})で3ピクセル幅に変更された。

\begin{userbox}
振幅計算後一瞬だけ波形が表示されて、一旦消えます。消えないようにスペクトログラム計算中も振幅波形の表示を維持するようにしてください。
\end{userbox}

\texttt{paintEvent}が修正され、スペクトログラム計算中も波形を表示したまま、下部に進捗オーバーレイを表示するように変更された。

\subsection{MP3→MP4エンコード設定の統一}

ユーザーは既存の\texttt{mp3tomp4} zsh関数を共有し、現在の実装との比較を要求した:

\begin{userbox}
元々は、[mp3tomp4 zsh関数]でエンコードしていました。違いはありますか
\end{userbox}

\begin{assistantbox}
オリジナルの \texttt{mp3tomp4} zsh関数と現在の ui\_next 実装の違いを比較します:

\vspace{0.5\baselineskip}
\noindent{\footnotesize
\begin{tabularx}{\linewidth}{@{}lXX@{}}
\toprule
項目 & オリジナル & 現在の実装 \\
\midrule
黒画像サイズ & 1920x1080 & 1280x720 \\
フォントサイズ & 56(固定) & h*0.054(相対) \\
CRF & 32 & 23(デフォルト) \\
-threads & 0(全コア使用) & 指定なし \\
\bottomrule
\end{tabularx}
}
\vspace{0.5\baselineskip}
\end{assistantbox}

\begin{userbox}
静止画なので、CRF32で十分ですね。黒画面の解像度は1280x720で、-threads 0にしましょうか。
\end{userbox}

以下の変更が実装された:

\begin{lstlisting}[language=python,basicstyle=\ttfamily\tiny]
# models.py - libx264エンコーダに-threads 0を追加
return [
    '-c:v', 'libx264',
    '-preset', 'ultrafast',
    '-crf', str(crf),
    '-threads', '0',  # 全CPUコアを使用
    '-pix_fmt', 'yuv420p',
]

# workers.py - 音声のみエクスポートでCRF 32を使用
encoder_args = get_encoder_args(self.encoder_id, self.bitrate_kbps, crf=32)
\end{lstlisting}

\subsection{品質ドロップダウンの変更}

\begin{userbox}
Qualityで、元と同じは変なので静止画用と表示してmp3からのエンコードにおいては変更できないようにしましょう。
\end{userbox}

品質オプションが動画用と音声用に分離された:

\begin{lstlisting}[language=python,basicstyle=\ttfamily\tiny]
self._video_quality_options = [
    ("元と同じ (自動)", 0, 23),
    ("高画質 (6Mbps)", 6000, 20),
    ("標準 (4Mbps)", 4000, 23),
    ("軽量 (2Mbps)", 2000, 28),
    ("最小 (1Mbps)", 1000, 32),
]
self._audio_quality_options = [
    ("静止画用 (CRF 32)", 500, 32),
]
\end{lstlisting}

音声ファイル読み込み時は品質コンボボックスが無効化され、「静止画用 (CRF 32)」のみが表示される。

\subsection{エクスポート後の自動読み込み}

\begin{userbox}
export後に、出力した動画をメイン画面に表示するようにしましょうか。
\end{userbox}

エクスポート完了後に出力動画を自動的に読み込む機能が実装された。これにより、すぐに出力結果を確認できる。

\subsection{プログレスバーのステータスバー統合}

\begin{userbox}
Readyのところに進捗が表示されます。プログレスバーもステータスバーに移動してマージして良い感じの進捗表示にしてください。
\end{userbox}

プログレスバーがステータスバーに統合され、以下のような表示となった:

\vspace{0.5\baselineskip}
\noindent{\footnotesize
\begin{tabularx}{\linewidth}{@{}lXXX@{}}
\toprule
状態 & プログレスバー & テキスト & 色 \\
\midrule
待機中 & 非表示 & Ready & 緑 \\
エクスポート中 & 表示・更新 & 00:01:23 45\% & 青 \\
完了 & 100\%表示 & Completed: file.mp4 & 緑 \\
エラー & 非表示 & Failed: error message & 赤 \\
\bottomrule
\end{tabularx}
}
\vspace{0.5\baselineskip}

\subsection{FFmpegエラーの修正}

\begin{userbox}
mp3からエンコードしたmp4を再エンコードするとエラーになりますね。

Output same as Input - exiting
\end{userbox}

エクスポート後に出力動画を読み込むため、再エクスポート時に入出力ファイルが同じになる問題が発生した。出力ファイル名に自動で連番サフィックスを付ける対策が実装された:

\begin{lstlisting}[language=python,basicstyle=\ttfamily\tiny]
if input_path.resolve() == output_path.resolve():
    counter = 2
    while True:
        output_path = self._state.work_dir / f"{Path(output_base).name}_chaptered_{counter}.mp4"
        if not output_path.exists():
            break
        counter += 1
\end{lstlisting}

\subsection{編集エラーの修正}

\begin{userbox}
edit: editing failed

チャプター名に--を付してエンターした時にでるエラーでした。
\end{userbox}

編集中にEnterを押すと、\texttt{edit()}が再度呼ばれて失敗する問題が発見された。編集状態をチェックして分岐する修正が実装された:

\begin{lstlisting}[language=python,basicstyle=\ttfamily\tiny]
def eventFilter(self, obj, event):
    if obj == self._table:
        if event.type() == QEvent.Type.KeyPress:
            if event.key() in (Qt.Key.Key_Return, Qt.Key.Key_Enter):
                # 編集中の場合はデフォルト処理(編集確定)に任せる
                if self._table.state() == QAbstractItemView.State.EditingState:
                    return False  # デフォルト処理に委譲
                # 編集中でなければ編集開始
                ...
\end{lstlisting}

\subsection{焼き込みテキストの二重化問題}

\begin{userbox}
2回目のエンコードの際に、チャプターを追加すると、チャプター名が2重になるなと。良い方法はありませんよね。
\end{userbox}

焼き込み済みMP4を再エクスポートすると、元の焼き込みに新しい焼き込みが追加されて二重表示になる問題が議論された。これは焼き込み(drawtext)の本質的な制限であり、ユーザーの操作の自由度を優先して自動読み込み機能は維持されることとなった。

\subsection{ウィンドウとダイアログのリサイズ対応}

\begin{userbox}
アプリケーションの画面サイズを拡大縮小できるようにしましょうか。比率は維持したままでやってみましょう。
\end{userbox}

\begin{userbox}
ダイアログも同様に同じ割合で拡大縮小できるように実装してください。
\end{userbox}

アスペクト比を維持したリサイズ機能が実装された:

\begin{lstlisting}[language=python,basicstyle=\ttfamily\tiny]
def resizeEvent(self, event):
    if self._resizing:
        return super().resizeEvent(event)
    self._resizing = True
    # アスペクト比計算
    self._resizing = False
    super().resizeEvent(event)
\end{lstlisting}

\vspace{0.5\baselineskip}
\noindent{\footnotesize
\begin{tabularx}{\linewidth}{@{}lXXX@{}}
\toprule
対象 & デフォルトサイズ & 最小サイズ & アスペクト比 \\
\midrule
メインウィンドウ & 1680 x 1050 & 1120 x 700 & 8:5 \\
SourceSelectionDialog & 800 x 600 & 600 x 450 & 4:3 \\
CoverImageDialog & 1344 x 840 & 896 x 560 & 8:5 \\
\bottomrule
\end{tabularx}
}
\vspace{0.5\baselineskip}

\subsection{movie-viewerスタイルの再生コントロール}

\begin{userbox}
右側の再生などのコントロールをvideo-chapter-editorと同様にしてください。また、コントロールのウィジェットの位置は波形の中心に揃えてください。
\end{userbox}

movie-viewerのUIを参考に、以下のような配置に変更された:

\begin{verbatim}
-10s  -1s  -.3s  -1f  [▶]  +1f  +.3s  +1s  +10s
\end{verbatim}

相対シーク用のメソッドが追加された:

\begin{lstlisting}[language=python,basicstyle=\ttfamily\tiny]
def _seek_relative(self, delta_ms: int):
    """現在位置から相対的にシーク"""
    if not self._media_player:
        return
    current = self._media_player.position()
    duration = self._media_player.duration()
    new_pos = max(0, min(duration, current + delta_ms))
    self._media_player.setPosition(new_pos)
\end{lstlisting}

\subsection{スペクトログラムカラーマップの探求}

この日の後半は、スペクトログラムのカラーマップについて多くの試行が行われた:

\begin{enumerate}
\item \textbf{SOXデフォルト}: 黒→紫→青→シアン→緑→黄→赤→白
\item \textbf{JET}: 濃紺→青→シアン→黄→赤→濃い赤
\item \textbf{Inferno}: 黒→紫→マゼンタ→オレンジ→黄→白
\item \textbf{青系}: 黒→青(シンプル)
\item \textbf{最終版}: 黒→青→シアン(黄緑寄り)
\end{enumerate}

\begin{userbox}
くらいのを青、明るいのをシアンという感じにするとどうなりますか
\end{userbox}

\begin{userbox}
良いですね。これでシアンを黄緑に少し寄せてみましょうか。
\end{userbox}

最終的に、以下のカラーマップが採用された:

\begin{lstlisting}[language=python,basicstyle=\ttfamily\tiny]
# カラーマップ: 黒→青→シアン(黄緑寄り)
# 0~50%: 黒→青 (RGB 0,0,0 → 0,0,255)
# 50~100%: 青→黄緑寄りシアン (RGB 0,0,255 → 80,255,120)
\end{lstlisting}

\subsection{UI微調整}

\begin{userbox}
Chapters(埋め込み)の、埋め込みを緑色にしましょうか。
\end{userbox}

QGroupBoxのタイトルはプレーンテキストのみ対応のため、カスタムQLabelに置き換えてリッチテキストで「埋め込み」部分を緑色で表示するように変更された。

\begin{userbox}
Spectrogramではなく、メルスペクトログラムでしたっけ。
\end{userbox}

ラベルが「Spectrogram」から「Mel Spectrogram」に変更され、コンボボックスの幅も140pxから160pxに拡張された。

\subsection{この日の成果}

\begin{enumerate}
\item \textbf{メルスペクトログラム}: 人間の聴覚特性に合わせた周波数表示
\item \textbf{エンコード設定統一}: CRF 32、-threads 0の追加
\item \textbf{品質ドロップダウン分離}: 動画用/音声用の選択肢分離
\item \textbf{自動読み込み}: エクスポート後の即時確認機能
\item \textbf{ステータスバー統合}: プログレスバーとメッセージの一体化
\item \textbf{FFmpegエラー対策}: 同名ファイルの自動リネーム
\item \textbf{編集エラー修正}: EditingStateチェックの追加
\item \textbf{リサイズ対応}: アスペクト比維持のウィンドウ・ダイアログリサイズ
\item \textbf{movie-viewerスタイル}: 中央配置の再生コントロール
\item \textbf{カラーマップ最適化}: 黒→青→シアン(黄緑寄り)の採用
\end{enumerate}

\subsection{技術的なポイント}

\begin{enumerate}
\item \textbf{メルスケール変換}: STFT出力にメルフィルターバンクを適用して人間の聴覚特性を反映
\item \textbf{QGroupBoxの制限}: タイトルはプレーンテキストのみ対応、リッチテキストにはQLabelを使用
\item \textbf{resizeEventの再帰防止}: フラグ(\texttt{\_resizing})を使用して無限再帰を回避
\item \textbf{EditingState}: QTableWidgetの編集状態を確認して適切なイベント処理を行う
\item \textbf{ガンマ補正}: スペクトログラムのコントラスト調整に\texttt{np.power(data, gamma)}を使用
\end{enumerate}

この日は非常に多くの改善が行われ、ui\_nextパッケージの完成度が大幅に向上した。特に、スペクトログラムカラーマップについては多数の試行が行われ、ユーザーの好みに合った表示が実現された。



%% ====== dialog_2025-12-31_full.tex ======
%% dialog_2025-12-31_full.tex
%% 2025年12月31日の開発対話記録(509メッセージ)

\section{2025年12月31日:ドラッグ&ドロップ実装とセミオートアップデーター}

この日は、UI操作の統一化から始まり、ドラッグ&ドロップ機能の本格実装、セミオートアップデーターの追加、Split Chapters機能の実装まで、v2.1.9からv2.1.25までの17回のリリースが行われた密度の高い開発セッションとなった。

\subsection{矢印キー挙動の統一(v2.1.9)}

\begin{userbox}
Enter(Return)で、チャプターの編集モードに入った際に、カーソルが入力済み文字の最後尾に入ります。その状態で、上矢印の挙動が、Macですと行の先頭に移動できるんですけど、Windowsだと上のセルに移動します。感覚的には、Macの方がイメージ通りですので、Windowsでも同じ挙動になるようにしてください。
\end{userbox}

\begin{assistantbox}
チャプター編集時の矢印キーの挙動を統一します。セル編集中の上下矢印キーを処理するコードを確認・修正します。
\end{assistantbox}

編集モード中のキーイベント処理が実装された:

\begin{lstlisting}[language=python,basicstyle=\ttfamily\tiny]
if self._table.state() == QAbstractItemView.State.EditingState:
    if key == Qt.Key.Key_Up:
        editor = self._table.findChild(QLineEdit)
        if editor:
            editor.setCursorPosition(0)
            return True
    elif key == Qt.Key.Key_Down:
        editor = self._table.findChild(QLineEdit)
        if editor:
            editor.setCursorPosition(len(editor.text()))
            return True
\end{lstlisting}

\subsection{メニューバーフォントサイズ統一(v2.1.10)}

\begin{userbox}
メニューバーのフォントサイズを、Window内のフォントサイズと同じにしてください。
\end{userbox}

QMenuBarとQMenuに\texttt{font-size: 16px}を追加し、ウィンドウ内UIと統一された。

\subsection{ドラッグ&ドロップ機能の実装}

\subsubsection{仕様策定}

\begin{userbox}
動画や音楽ファイルのドロップに対応するようにできますか。
\end{userbox}

\begin{userbox}
ドロップの仕様を決めたいと思います。決める必要がある事項を質問してください。
\end{userbox}

以下の仕様が決定された:

\vspace{0.5\baselineskip}
\noindent{\footnotesize
\begin{tabularx}{\linewidth}{@{}lX@{}}
\toprule
項目 & 仕様 \\
\midrule
ドロップ領域 & 動画プレビュー領域のみ \\
複数ファイル & 音声は結合リストに追加、動画は最初の1つのみ \\
上書き動作 & 確認なしで置き換え \\
フォルダ & 作業ディレクトリとして設定 \\
\bottomrule
\end{tabularx}
}
\vspace{0.5\baselineskip}

\subsubsection{DropVideoFrameクラス(v2.1.11)}

最初の実装として、QFrameを拡張したDropVideoFrameクラスが作成された:

\begin{lstlisting}[language=python,basicstyle=\ttfamily\tiny]
class DropVideoFrame(QFrame):
    files_dropped = Signal(list)
    folder_dropped = Signal(str)

    def __init__(self, parent=None):
        super().__init__(parent)
        self.setAcceptDrops(True)
        self._drag_active = False

    def dragEnterEvent(self, event):
        # Accept video/audio files and folders
        ...

    def dropEvent(self, event):
        # Emit files_dropped or folder_dropped signal
        ...
\end{lstlisting}

\subsubsection{チャプターペースト機能(v2.1.12-13)}

\begin{userbox}
続いて、チャプターのリストにペーストができる仕様にしてください。Youtubeのように.mmmがない場合は.000でパディングする仕様にしてください。
\end{userbox}

\begin{userbox}
video-chapter-editorと同じように、必ずチャプタの先頭に0:00:00.000を入れるようにしてください。
\end{userbox}

\texttt{paste\_chapters()}メソッドが追加され、YouTube形式(HH:MM:SS)とフル形式(HH:MM:SS.mmm)の両方に対応した。また、先頭に0:00:00.000がない場合は自動追加する機能も実装された。

\subsubsection{ドロップ時のバグ修正シリーズ(v2.1.14-17)}

\begin{userbox}
複数のmp3をドロップした時の挙動はどうなりましたっけ
\end{userbox}

確認の結果、\texttt{\_on\_source\_changed()}が未定義であることが判明し、\texttt{\_load\_source\_media()}に修正された(v2.1.15)。

\begin{userbox}
動画などをドロップした際の作業ディレクトリは、元の動画などのファイルが保存されているフォルダにしてください。
\end{userbox}

ファイルドロップ時に親フォルダを作業ディレクトリとして設定する機能が追加された(v2.1.16)。

\begin{userbox}
動画を一つドロップすると、Select Sourceでmp4を読み込んだ時と同じ挙動になりますよね。今は、ならないんですけど。
\end{userbox}

\texttt{\_prepare\_for\_new\_source()}ヘルパーメソッドを抽出し、Select Sourceダイアログとドロップ時の挙動を統一した(v2.1.17)。

\subsubsection{About/Helpダイアログ(v2.1.18)}

\begin{userbox}
バージョンの確認ができるようにしてください。あと、ヘルプに何も入ってませんね。
\end{userbox}

Help > About ダイアログにバージョン番号、機能一覧、GitHubリンクを追加。Help > Keyboard Shortcutsに操作説明を追加した。

\subsubsection{QVideoWidgetによるドロップイベントブロック問題}

\begin{userbox}
動画ファイルをドロップしても何も起きないですね。どうしたもんでしょう。
\end{userbox}

\begin{assistantbox}
QVideoWidgetがDropVideoFrameの上を覆っていたため、ドロップイベントが親フレームに届いていませんでした。
\end{assistantbox}

最初にイベントフィルタを使用した解決を試みたが機能せず(v2.1.19)、最終的に透明なDropOverlayウィジェットをQStackedLayoutのStackAllモードで配置する方式で解決した(v2.1.20):

\begin{lstlisting}[language=python,basicstyle=\ttfamily\tiny]
class DropOverlay(QWidget):
    files_dropped = Signal(list)
    folder_dropped = Signal(str)

    def __init__(self, parent=None):
        super().__init__(parent)
        self.setAcceptDrops(True)
        self.setAttribute(Qt.WidgetAttribute.WA_TranslucentBackground)
\end{lstlisting}

\subsubsection{ステータスバー更新問題(v2.1.21)}

\begin{userbox}
ドロップした際に、作業ディレクトリが切り替わらないですね。
\end{userbox}

デバッグ出力により、内部では\texttt{work\_dir}が正しく更新されているが、ステータスバーの表示が更新されていないことが判明:

\begin{userbox}
ステータスバーの表示が変わりません。
\end{userbox}

\texttt{work\_dir\_changed = Signal(object)}を追加し、app.pyで受け取ってステータスバーを更新する機能が実装された。

\subsection{セミオートアップデーター(v2.1.22)}

\begin{userbox}
Actionsで、アプリのアップデーターを準備することは可能ですか。
\end{userbox}

\begin{userbox}
オンラインでバージョンチェックして、アップデーターのダウンロードからインストールまでを行うことは可能ですか
\end{userbox}

3つのアプローチが提案された:

\vspace{0.5\baselineskip}
\noindent{\footnotesize
\begin{tabularx}{\linewidth}{@{}lX@{}}
\toprule
方式 & 概要 \\
\midrule
完全自動(Sparkle) & 署名必須、開発工数大 \\
半自動 & ダウンロード→Finder表示、すぐ実装可能 \\
通知のみ & リンク表示のみ、最もシンプル \\
\bottomrule
\end{tabularx}
}
\vspace{0.5\baselineskip}

\begin{userbox}
2ですかね。
\end{userbox}

\texttt{updater.py}モジュールが新規作成された:

\begin{lstlisting}[language=python,basicstyle=\ttfamily\tiny]
class UpdateChecker(QObject):
    update_available = Signal(str, str, str)  # version, url, notes
    check_finished = Signal()
    error = Signal(str)

    def run(self):
        # Checks GitHub API, compares versions, emits signal if update available

class UpdateDownloader(QObject):
    progress = Signal(int)  # 0-100
    finished = Signal(str)  # downloaded file path
    error = Signal(str)

def mount_and_open_dmg(dmg_path: str) -> bool:
    # macOS: hdiutil attach, open in Finder
\end{lstlisting}

アップデート手順:
\begin{enumerate}
\item 起動3秒後にGitHub Releases APIをバックグラウンドチェック
\item 新バージョンがあればステータスバーに通知表示
\item クリック→確認ダイアログ→ダウンロード開始
\item プログレスバーでダウンロード進捗を表示
\item macOS: DMGをマウントしてFinderで開く / Windows: ZIPをExplorerで開く
\end{enumerate}

\begin{userbox}
差分ではないんですね。
\end{userbox}

\begin{assistantbox}
その通りです。現在の実装はフルダウンロード方式です。毎回約80MBのDMG/ZIPをダウンロード。差分アップデートにはbsdiff等のパッチ生成・適用ロジックが必要で、現状のサイズでは許容範囲と判断しました。
\end{assistantbox}

\subsection{Split Chapters機能(v2.1.23)}

\begin{userbox}
エンコード時に、チャプターごと個別のファイルとして出力するオプションを追加し、Exportの左にトグルを追加してください。
\end{userbox}

\texttt{SplitExportWorker}クラスがworkers.pyに追加された:

\begin{lstlisting}[language=python,basicstyle=\ttfamily\tiny]
def sanitize_filename(name: str) -> str:
    """ファイル名に使えない文字を置換"""
    invalid_chars = '<>:"/\\|?*'
    for char in invalid_chars:
        name = name.replace(char, '_')
    return name.strip()

class SplitExportWorker(QThread):
    progress = Signal(int)          # 全体進捗 (0-100)
    chapter_completed = Signal(str) # 各チャプター完了時
    finished = Signal(list)         # 出力ファイルリスト
    error = Signal(str)
\end{lstlisting}

動作仕様:
\begin{itemize}
\item 「Split Chapters」チェックボックスをONでExport
\item 各チャプターが個別ファイルとして出力
\item 出力ファイル名: \texttt{\{base\}\_01\_\{chapter\_title\}.mp4}
\item \texttt{--}で始まる除外チャプターは出力されない
\item Embed Chap有効時は各ファイルにタイトル焼き込み
\end{itemize}

\subsection{PyInstaller問題の修正(v2.1.24-25)}

\begin{userbox}
mac版を実行すると開けないんですけど。なんでしょう。
\end{userbox}

ターミナルから実行した結果、\texttt{No module named 'email'}エラーが発生:

\begin{lstlisting}[basicstyle=\ttfamily\tiny]
ModuleNotFoundError: No module named 'email'
[PYI-275:ERROR] Failed to execute script 'run_video_chapter_editor'
\end{lstlisting}

\begin{assistantbox}
原因が判明しました。PyInstallerの設定で\texttt{email}モジュールを除外していますが、\texttt{urllib.request}(アップデーターで使用)がそれに依存しています。
\end{assistantbox}

PyInstallerのexcludesから\texttt{email}, \texttt{html}, \texttt{http}を削除し、v2.1.24でリリースされた。

\begin{userbox}
Aboutのバージョン、間違ってません?
\end{userbox}

\texttt{VERSION}定数が更新されていなかったため、v2.1.25で修正された。

\subsection{この日の成果}

\begin{enumerate}
\item \textbf{UI操作統一} - 矢印キー挙動、メニューフォントサイズ
\item \textbf{ドラッグ&ドロップ} - DropOverlay方式で完全実装
\item \textbf{チャプターペースト} - YouTube形式対応、先頭自動追加
\item \textbf{セミオートアップデーター} - GitHub Releases連携
\item \textbf{Split Chapters} - チャプター個別ファイル出力
\item \textbf{17バージョンリリース} - v2.1.9〜v2.1.25
\end{enumerate}

\subsection{技術的なポイント}

\begin{enumerate}
\item \textbf{QVideoWidgetのイベント処理}: 内部ウィジェットがイベントをブロックするため、イベントフィルタではなくQStackedLayout.StackAllモードのオーバーレイで解決
\item \textbf{Signal/Slotパターン}: \texttt{work\_dir\_changed}シグナルでコンポーネント間通信を実現
\item \textbf{PyInstaller依存関係}: urllib.requestはemail, html, httpモジュールに依存しており、除外リストから削除が必要
\item \textbf{GitHub Releases API}: バージョン比較とダウンロードURLの取得に使用
\end{enumerate}



%% ====== dialog_2026-01-01_full.tex ======
%% dialog_2026-01-01_full.tex
%% 2026年1月1日の開発対話記録(6メッセージ)

\section{2026年1月1日:iPad版への展望}

新年最初のセッションは、video-chapter-editorのiPad対応可能性についての技術的検討が行われた短い対話となった。

\subsection{iPad移植の技術的課題}

\begin{userbox}
このアプリ、iPad用にリリースするとなると大変ですかね。
\end{userbox}

\begin{assistantbox}
かなり大変です。現在の構成ではiOSへの直接移植は困難です。
\end{assistantbox}

主な技術的障壁が整理された:

\vspace{0.5\baselineskip}
\noindent{\footnotesize
\begin{tabularx}{\linewidth}{@{}lXX@{}}
\toprule
要素 & 現状 & iPad移植 \\
\midrule
GUI & PySide6 (Qt) & iOS非対応 \\
言語 & Python & iOS公式サポートなし \\
動画処理 & ffmpeg (CLI) & 要ネイティブビルド \\
ファイルアクセス & 自由 & サンドボックス制限 \\
\bottomrule
\end{tabularx}
}
\vspace{0.5\baselineskip}

\subsection{移植アプローチの検討}

4つの選択肢が提示された:

\begin{enumerate}
\item \textbf{ネイティブ書き直し}(最も現実的)
  \begin{itemize}
  \item Swift + SwiftUI
  \item AVFoundation(動画処理)
  \item 工数: 大
  \end{itemize}

\item \textbf{クロスプラットフォーム}
  \begin{itemize}
  \item Flutter / React Native
  \item Kotlin Multiplatform
  \item 工数: 中〜大
  \end{itemize}

\item \textbf{Web + サーバー構成}
  \begin{itemize}
  \item フロントエンド: Web (PWA)
  \item バックエンド: 現在のPython処理をAPI化
  \item iPad: Safari/WebViewで利用
  \item 工数: 中
  \end{itemize}

\item \textbf{Catalyst}
  \begin{itemize}
  \item macOS版を先に作成 → iPad互換
  \item Qt/PySide6では難しい
  \end{itemize}
\end{enumerate}

\subsection{Tauri経由のSwift移行案}

\begin{userbox}
Tauri版を作成して、Swiftに向かうのはいかがでしょう。
\end{userbox}

\begin{assistantbox}
良いアプローチです。Tauri 2.0はiOS/Androidサポートがあります。
\end{assistantbox}

段階的な移行パスが提案された:

\begin{lstlisting}[basicstyle=\ttfamily\small]
現在 (PySide6)
    ↓
Tauri 2.0 (Rust + Web)
    ↓
├── デスクトップ: そのまま利用
├── iPad: Tauri iOS (実験的)
└── 必要なら: Swift書き直し (知見を活用)
\end{lstlisting}

Tauriのメリット:

\vspace{0.5\baselineskip}
\noindent{\footnotesize
\begin{tabularx}{\linewidth}{@{}lX@{}}
\toprule
項目 & 内容 \\
\midrule
コード共有 & デスクトップ/モバイルで同一フロントエンド \\
パフォーマンス & Rustバックエンドは高速 \\
バイナリサイズ & Electronより小さい(約10MB) \\
学習効果 & Rust → Swift移行時に役立つ \\
\bottomrule
\end{tabularx}
}
\vspace{0.5\baselineskip}

\begin{userbox}
ちょっと考えますね。
\end{userbox}

\begin{assistantbox}
了解です。検討事項が多いですからね。
\end{assistantbox}

\subsection{この日の要点}

2026年最初のセッションでは、具体的な開発作業ではなく、アプリの将来的な展開について検討が行われた。PySide6/Pythonベースの現在のアーキテクチャからiPad対応への道筋として、Tauri 2.0を中間ステップとし、必要に応じてSwiftへ移行するという段階的アプローチが有力な選択肢として浮上した。



%% ====== dialog_2026-01-03_full.tex ======
%% dialog_2026-01-03_full.tex
%% 2026年1月3日の開発対話記録(284メッセージ)

\section{2026年1月3日:文字起こしワークフローUI設計とYAMLスキーマ}

この日は、文字起こしワークフローのUI設計に着手し、YAMLベースの設定ファイル設計、プロファイルベースのアーキテクチャ、入力状態の列挙、ワークフロー境界の明確化、そしてPADおよびスイムレーン図による可視化まで、284メッセージに及ぶ大規模な設計議論が行われた。

\subsection{文字起こしワークフローUI設計の開始}

\begin{userbox}
次に、文字起こしのUI作成に移行したいと思います。
\end{userbox}

\begin{assistantbox}
report\_workflow.pyをベースに、transcription\_workflow.pyを作成します。
\end{assistantbox}

\begin{userbox}
前処理のワークフローと同様な設計思想にて実装したいと思います。まず、機能を整理しましょう。
\end{userbox}

video-chapter-editorと同様の設計思想として、以下の機能が整理された:

\vspace{0.5\baselineskip}
\noindent{\footnotesize
\begin{tabularx}{\linewidth}{@{}lX@{}}
\toprule
機能カテゴリ & 内容 \\
\midrule
入力 & 動画ファイル、SRTファイル、YouTube URL \\
SRT取得 & YouTube字幕、Whisper文字起こし \\
プロファイル & オーケストラリハーサル、ホルンレッスン、会議 \\
出力 & LaTeX文書、Markdown \\
\bottomrule
\end{tabularx}
}
\vspace{0.5\baselineskip}

\subsection{設定ファイル設計:再現性と再利用性}

\begin{userbox}
最終的には、再現性と再利用性を確保したいので、設定ファイルを保存する仕様にしたいと考えています。
\end{userbox}

\begin{assistantbox}
良い設計方針ですね。設定ファイルにより:
\begin{enumerate}
\item 同じ設定で再処理が可能(再現性)
\item 類似プロジェクトでテンプレートとして再利用(再利用性)
\item 処理履歴として記録が残る
\end{enumerate}
\end{assistantbox}

\begin{userbox}
あとはYamlが大きくなりすぎるのは望ましくなく、Claudeのカスタムコマンドなどは極力外部ファイルにしたい。
\end{userbox}

外部ファイル参照による分離方針が決定された:

\begin{lstlisting}[basicstyle=\ttfamily\tiny]
transcription_workflow.yaml
├── profile: "orchestral_rehearsal"  # 参照
├── prompt_template: "rehearsal.md"  # 外部
├── base_template: "luatex_2col.tex" # 外部
└── project固有データのみをYAML内に
\end{lstlisting}

\subsection{プロファイルベースのアーキテクチャ}

\begin{userbox}
プロファイルがマクロな定義になるという理解で良いですか。
\end{userbox}

\begin{assistantbox}
その通りです。プロファイルは「このタイプの文字起こしに必要な共通設定」を定義します。
\end{assistantbox}

\begin{userbox}
TeXとLaTeX同じ感じですね。
\end{userbox}

\begin{assistantbox}
素晴らしい喩えです。
\begin{itemize}
\item プロファイル = .clsファイル(クラス定義)
\item YAML = .texファイル(インスタンス)
\item フィールド定義 = パッケージオプション
\end{itemize}
\end{assistantbox}

このTeX/LaTeXアナロジーに基づき、以下の構造が設計された:

\vspace{0.5\baselineskip}
\noindent{\footnotesize
\begin{tabularx}{\linewidth}{@{}lXX@{}}
\toprule
概念 & LaTeX & 文字起こしワークフロー \\
\midrule
クラス定義 & article.cls & profiles/*.yaml \\
インスタンス & document.tex & workflow.yaml \\
マクロ展開 & \textbackslash maketitle & field\_schema展開 \\
\bottomrule
\end{tabularx}
}
\vspace{0.5\baselineskip}

\subsection{入力状態の列挙(7状態)}

\begin{userbox}
SRT取得の責務に関して、Youtubeのダウンロードを行う時点でSRTが取得されるのが境界が曖昧になっている点だと考えています。
\end{userbox}

\begin{userbox}
根本的な問題は入力ファイルの状態の場合わけが十分でないことですね。
\end{userbox}

\begin{assistantbox}
ご指摘の通りです。入力状態を明示的に列挙することで、責務の境界が明確になります。
\end{assistantbox}

7つの入力状態(S1--S7)が定義された:

\vspace{0.5\baselineskip}
\noindent{\footnotesize
\begin{tabularx}{\linewidth}{@{}clcccX@{}}
\toprule
状態 & 入力 & 動画 & YT字幕 & Whisper & 次アクション \\
\midrule
S1 & YouTube URL & -- & -- & -- & DL→S2/S3 \\
S2 & YT URL+字幕あり & local & exists & -- & Whisper? \\
S3 & YT URL+字幕なし & local & n/a & -- & Whisper必須 \\
S4 & ローカル動画のみ & local & n/a & -- & Whisper必須 \\
S5 & 動画+YT字幕 & local & exists & -- & Whisper? \\
S6 & 動画+Whisper & local & -- & exists & 処理可能 \\
S7 & 動画+両方 & local & exists & exists & 選択 \\
\bottomrule
\end{tabularx}
}
\vspace{0.5\baselineskip}

これを表現するため、YAMLに\texttt{source.state}フィールドが追加された:

\begin{lstlisting}[language=python,basicstyle=\ttfamily\tiny]
source:
  type: "local"       # youtube | local
  path: "video.mp4"
  state:
    video: "ready"           # ready | url_only | missing
    youtube_srt: "exists"    # exists | missing | not_applicable
    whisper_srt: "missing"   # exists | missing
    manual_srt: "missing"    # exists | missing
  files:
    youtube_srt: "video_yt.srt"
\end{lstlisting}

\subsection{ワークフロー境界の明確化}

\begin{assistantbox}
前処理の責務を明確にしましょう:

\textbf{前処理の責務}: SRTファイルが作業ディレクトリに存在することを保証する

入力タイプ別の責務:
\begin{itemize}
\item YouTube URL → 動画DL + 字幕DL(存在すれば)
\item ローカル動画 → そのまま使用
\item SRTファイル → 作業ディレクトリにコピー
\end{itemize}
\end{assistantbox}

この境界設定により、文字起こしワークフローは「SRTが存在する状態」から処理を開始できる明確な契約が確立された。

\subsection{可視化アプローチの選択}

\begin{userbox}
処理フロー、やはり深さと順序がわかりやすいので、PADの方が好きですね。
\end{userbox}

\begin{userbox}
同じことを行う際に、別表現が可能というのが最もわかりづらい点です。
\end{userbox}

UMLアクティビティ図とPADの両方を作成していたが、ユーザーの指摘により可視化方針が整理された:

\vspace{0.5\baselineskip}
\noindent{\footnotesize
\begin{tabularx}{\linewidth}{@{}lXX@{}}
\toprule
用途 & 形式 & 理由 \\
\midrule
処理フロー & PAD & 深さと順序が明確 \\
アクター間連携 & スイムレーン & User/UI/Backend分離 \\
\bottomrule
\end{tabularx}
}
\vspace{0.5\baselineskip}

\begin{userbox}
アクティビティ図に統合できません?
\end{userbox}

3つの選択肢が提示され、ユーザーは選択肢1(PAD + スイムレーン併用)を選択した。

\subsection{YAMLライフサイクル設計}

\begin{userbox}
前処理を行う前にYAMLを作成し、それに沿って前処理を行うのが自然な流れだと思うんですけどどうでしょう。
\end{userbox}

\begin{assistantbox}
おっしゃる通りです。「マニフェスト先行」のワークフローですね。

\textbf{Manifest-First Workflow}:
\begin{enumerate}
\item ユーザーがYAMLを作成(意図の宣言)
\item 前処理がYAMLに従って素材を準備
\item 状態フィールドが自動更新
\item 文字起こし処理が実行
\end{enumerate}
\end{assistantbox}

YAMLの静的セクション(ユーザー入力)と動的セクション(自動更新)が明確に分離された:

\begin{lstlisting}[basicstyle=\ttfamily\tiny]
# === 静的(ユーザー入力) ===
schema_version: "1.1"
profile: "orchestral_rehearsal"
source:
  type: "youtube"
  url: "https://..."

# === 動的(自動更新) ===
source:
  state:
    video: "ready"      # 前処理後に更新
    youtube_srt: "exists"
  files:
    video: "downloaded.mp4"
    youtube_srt: "video.srt"
\end{lstlisting}

\subsection{スイムレーン図へのYAML統合}

\begin{userbox}
ワークフローの中に、YAMLの出力を入れましょうか。
\end{userbox}

3つの選択肢が提示された:

\vspace{0.5\baselineskip}
\noindent{\footnotesize
\begin{tabularx}{\linewidth}{@{}lX@{}}
\toprule
選択肢 & 内容 \\
\midrule
A & データストアとして表現(UML標準) \\
B & 矢印でデータフローを表現 \\
C & 両方(データストア+矢印) \\
\bottomrule
\end{tabularx}
}
\vspace{0.5\baselineskip}

\begin{userbox}
Cですね。
\end{userbox}

YAMLがデータストアとして明示され、各処理ステップからの読み書きが矢印で表現されるスイムレーン図が作成された。

\subsection{プロファイルとYAMLの具体化}

\begin{userbox}
実際のものをみて考えましょう。これまでのやりとりを踏まえて、transcription\_workflow.yamlとprofiles/*.yamlを作ってみてください。
\end{userbox}

3つのプロファイルが作成された:

\subsubsection{orchestral\_rehearsal.yaml}

\begin{lstlisting}[language=python,basicstyle=\ttfamily\tiny]
name: "オーケストラリハーサル"
icon: "🎼"
description: "オーケストラのリハーサル記録用"

participants:
  type: "hierarchical"
  instructor:
    label: "指揮者"
    name_field: true
  students:
    label: "奏者"
    collective: true

field_schema:
  title:
    label: "曲目"
    type: "text"
    required: true
  movement:
    label: "楽章"
    type: "text"
  rehearsal_number:
    label: "練習番号"
    type: "text"

prompt_template: "orchestral_rehearsal.md"
base_template: "luatex_twocolumn.tex"
\end{lstlisting}

\subsubsection{horn\_lesson.yaml}

\begin{lstlisting}[language=python,basicstyle=\ttfamily\tiny]
name: "ホルンレッスン"
icon: "🎺"
participants:
  type: "hierarchical"
  instructor:
    label: "先生"
  students:
    label: "生徒"
    name_field: true

field_schema:
  title:
    label: "曲目/エチュード"
    type: "text"
  lesson_focus:
    label: "レッスンの焦点"
    type: "text"
\end{lstlisting}

\subsubsection{meeting\_report.yaml}

\begin{lstlisting}[language=python,basicstyle=\ttfamily\tiny]
name: "会議レポート"
icon: "📋"
participants:
  type: "flat"
  attendees:
    label: "参加者"
    multiple: true

field_schema:
  title:
    label: "会議名"
    type: "text"
    required: true
  agenda:
    label: "議題"
    type: "multiline"
\end{lstlisting}

\subsection{ワークフローYAMLの例}

\begin{lstlisting}[language=python,basicstyle=\ttfamily\tiny]
schema_version: "1.1"
profile: "orchestral_rehearsal"

source:
  type: "local"
  path: "rehearsal_2026-01-03.mp4"
  state:
    video: "ready"
    youtube_srt: "not_applicable"
    whisper_srt: "exists"
    manual_srt: "missing"
  files:
    whisper_srt: "rehearsal_2026-01-03_whisper.srt"

transcription:
  method: "auto"
  auto_priority:
    - "whisper"
    - "youtube"
  whisper:
    model: "large-v3"
    language: "ja"

fields:
  title: "ブラームス交響曲第1番"
  movement: "第4楽章"
  conductor: "山田太郎"
  rehearsal_number: "練習番号C〜"

output:
  basename: "brahms_sym1_rehearsal"
  format: "latex"
\end{lstlisting}

\subsection{この日の成果}

\begin{enumerate}
\item \textbf{設計文書} - transcription-workflow-design.mdの作成
\item \textbf{YAMLスキーマv1.1} - source.stateフィールドの導入
\item \textbf{プロファイルアーキテクチャ} - TeX/LaTeXアナロジーによる設計
\item \textbf{入力状態列挙} - 7状態(S1--S7)の明確な定義
\item \textbf{ワークフロー境界} - 前処理の責務明確化
\item \textbf{可視化方針} - PAD(処理フロー)+ スイムレーン(アクター連携)
\item \textbf{YAMLライフサイクル} - マニフェスト先行ワークフロー
\item \textbf{プロファイル定義} - 3種類のプロファイルYAML作成
\item \textbf{PAD図} - transcription-workflow.spd, preprocessing-workflow.spd
\end{enumerate}

\subsection{技術的なポイント}

\begin{enumerate}
\item \textbf{TeX/LaTeXアナロジー}: プロファイルをクラスファイル、YAMLをテックスファイルとして捉える設計思想
\item \textbf{状態の明示的管理}: source.stateフィールドにより、入力ファイルの状態を明示的に追跡
\item \textbf{責務の境界}: 「前処理の責務 = SRTファイルが作業ディレクトリに存在することを保証」という明確な契約
\item \textbf{静的/動的分離}: YAMLの静的セクション(ユーザー入力)と動的セクション(自動更新)の明確な分離
\item \textbf{可視化の使い分け}: PADは処理の深さと順序、スイムレーンはアクター間の連携という役割分担
\end{enumerate}

この日のセッションは、video-chapter-editorで培った設計思想を文字起こしワークフローに適用し、YAMLベースの設定管理とプロファイルベースのアーキテクチャを確立した重要な設計フェーズとなった。



%% ====== dialog_2026-01-04_full.tex ======
%% dialog_2026-01-04_full.tex
%% 2026年1月4日の開発対話記録(594メッセージ)

\section{2026年1月4日:成果物変換図とYouTubeダウンロード機能}

この日は、前日の設計議論を踏まえた成果物変換図の作成、GitHub Pagesによるドキュメント公開、YouTubeダウンロード機能の実装、再生コントロールのUI改善など、594メッセージに及ぶ大規模な開発セッションとなった。

\subsection{成果物ベースのワークフロー整理}

\begin{userbox}
もう一度、整理したいと思います。成果物ベースでは入力は、外部から提供される曲ごとの音声/映像ファイル、未編集音声ファイル、未編集mp4、Youtubeからダウンロードできる講義/レッスン動画。最終的な出力は、講義の学習教材や欠席者の自習用に活用できるレベルの資料(PDF)、講義の詳細なスクリプト、自習・確認用のチャプター付き動画、Youtube動画用のチャプターリストと認識しています。
\end{userbox}

成果物の関係が体系的に整理された:

\vspace{0.5\baselineskip}
\noindent{\footnotesize
\begin{tabularx}{\linewidth}{@{}lXX@{}}
\toprule
分類 & 成果物 & 備考 \\
\midrule
入力 & 曲ごと音声/映像、未編集素材、YouTube URL & 状態により前処理要否が決定 \\
中間出力 & SRT(YouTube/Whisper)、チャプター付きmp4 & \\
最終出力 & スクリプトPDF、サマリーPDF、チャプターリスト & \\
\bottomrule
\end{tabularx}
}
\vspace{0.5\baselineskip}

\subsection{スクリプトとサマリーの区別}

\begin{userbox}
講義の詳細なスクリプトは、srt-meeting-report.mdで示した内容をイメージしています。
\end{userbox}

\begin{assistantbox}
スクリプトは単なるSRT整形ではなく、発言者識別、セクション構造、略語のfootnote、文脈補足を含む構造化文書ということですね。
\end{assistantbox}

両者の関係が明確化された:

\begin{lstlisting}[basicstyle=\ttfamily\tiny]
SRT
 +-- スクリプト: 原文忠実、全発言記録(一次資料)
 |     `- LuaTeX -> PDF
 |
 `-- サマリーレポート: AI分析・要約(二次資料)
       `- 学習ポイント抽出、構造化
\end{lstlisting}

\subsection{話者特定(Speaker Diarization)}

\begin{userbox}
いつも必要となるとは考えていませんが、会議の録音など話者特定を行いたい場合はありますね。
\end{userbox}

プロファイル別の必要性が整理された:

\vspace{0.5\baselineskip}
\noindent{\footnotesize
\begin{tabularx}{\linewidth}{@{}lX@{}}
\toprule
プロファイル & 話者特定 \\
\midrule
orchestral\_rehearsal & 不要(基本的に指揮者のみ) \\
horn\_lesson & あると便利(講師と生徒の区別) \\
meeting\_report & 有用(複数参加者の発言記録) \\
\bottomrule
\end{tabularx}
}
\vspace{0.5\baselineskip}

\subsection{GitHub Pagesによるドキュメント公開}

\begin{userbox}
Mermaidの図、SVGで保存できたと思うんですけど、ファイルに保存するためのUIってつけられます?
\end{userbox}

\begin{userbox}
アプリに実装ではなく、githubのページに実装してもらいたいんですけど。
\end{userbox}

docs/ディレクトリにMermaid図ビューアが作成された:

\begin{lstlisting}[basicstyle=\ttfamily\tiny]
docs/
+-- index.html          # トップページ
`-- mermaid-viewer.html # Mermaid図ビューア + SVGダウンロード
\end{lstlisting}

機能:
\begin{itemize}
\item ドロップダウンで図を選択
\item Download SVGボタンでファイル保存
\item ダークモード対応(OS設定に連動)
\item URL直リンク対応(例: \texttt{\#artifact-transformation})
\end{itemize}

\begin{userbox}
JSが動くんですね。Leafletはいける感じですか。
\end{userbox}

\begin{assistantbox}
問題なく動きます。Leafletは完全にクライアントサイドJSなので、GitHub Pagesと相性抜群です。
\end{assistantbox}

\subsection{YouTubeダウンロード機能の実装}

\begin{userbox}
未実装ですが、Youtubeのリンクから動画をダウンロードし、表示、編集する機能を追加しましょう。そうすると、現在の成果物変換図のVideo Chapter Editorのスコープが全ての入力に対応できるようになりますね。
\end{userbox}

\subsubsection{SourceSelectionDialogの拡張}

\begin{userbox}
メイン画面がビジーになるので、Select Sourceのダイアログの中に作成しましょうか。加えて、Cover Imageの機能もmp3固有のものなのでこのボタンも、Select Sourceダイアログへ移動しましょう。
\end{userbox}

SourceSelectionDialogが大幅に拡張された:

\begin{lstlisting}[basicstyle=\ttfamily\tiny]
+----------------------------------------------------+
| Select Source                                       |
+----------------------------------------------------+
|  ( ) Local Files    ( ) YouTube URL                 |
+----------------------------------------------------+
| [Local Files選択時]                                 |
|  [MP3] [MP4]                    /path/to/dir [..]   |
|  +-----------------------------------------------+  |
|  | file1.mp3                                     |  |
|  | file2.mp3                                     |  |
|  +-----------------------------------------------+  |
|                                                     |
| [YouTube URL選択時]                                 |
|  URL: [_________________________________] [Check]   |
|  Status: Ready to download                          |
+----------------------------------------------------+
| Cover Image (for audio files):                      |
|  [Select Image] Not Set            [Clear]          |
+----------------------------------------------------+
|                              [Cancel] [OK]          |
+----------------------------------------------------+
\end{lstlisting}

\subsubsection{YouTubeDownloadWorkerの実装}

workers.pyにYouTubeダウンロード用のワーカークラスが追加された:

\begin{lstlisting}[language=python,basicstyle=\ttfamily\tiny]
class YouTubeDownloadWorker(QThread):
    """YouTube動画ダウンロードワーカー"""
    log_message = Signal(str)
    progress_update = Signal(str)
    download_completed = Signal(str, str)  # (video_path, srt_path or "")
    error_occurred = Signal(str)

    def __init__(self, url: str, output_dir: str,
                 download_subs: bool = True, sub_lang: str = "ja"):
        # ...

    def _sanitize_filename(self, title: str, max_length: int = 60) -> str:
        """タイトルをファイル名に使える形式に変換"""
        # ...

    def _get_video_info(self) -> Optional[dict]:
        """yt-dlp -J で動画情報を取得"""
        # ...
\end{lstlisting}

\subsubsection{カバー画像プレビュー}

\begin{userbox}
カバー画像の有無をメイン画面で確認したいと考えています。Select Sourceとファイル名の右側に画像を表示するウィジェットを2行分で作成してください。カバー画像が作成されていないときは黒背景でUnsetと赤字で表示してください。
\end{userbox}

main\_workspace.pyにカバー画像プレビューが追加された:

\begin{itemize}
\item サイズ: 142×80px(16:9アスペクト比、2行分の高さ)
\item 未設定時: 黒背景(\texttt{\#0a0a0a})+ 赤字(\texttt{\#ef4444})で「Cover Image\textbackslash nUnset」
\item 設定時: カバー画像をスケーリングして表示
\end{itemize}

\subsection{再生コントロールのUI改善}

\subsubsection{チャプタースキップボタン}

\begin{userbox}
プレイヤーのコントロールで、再生/一時停止ボタンの左右にチャプタースキップを追加してください。
\end{userbox}

\begin{lstlisting}[language=python,basicstyle=\ttfamily\tiny]
# 前のチャプター
self._prev_chapter_btn = QPushButton("|◀")
self._prev_chapter_btn.setFixedSize(50, 50)
self._prev_chapter_btn.clicked.connect(self._goto_prev_chapter)

# 次のチャプター
self._next_chapter_btn = QPushButton("▶|")
self._next_chapter_btn.setFixedSize(50, 50)
self._next_chapter_btn.clicked.connect(self._goto_next_chapter)
\end{lstlisting}

動作仕様:
\begin{itemize}
\item 前のチャプター: 現在位置から1秒以上前のチャプターにジャンプ
\item 次のチャプター: 現在位置から500ms以上後のチャプターにジャンプ
\item メディアが読み込まれている AND チャプターが存在する場合のみ有効
\end{itemize}

\subsubsection{Windowsでの再生シンボル修正}

\begin{userbox}
Windowsで再生ボタンの背景と再生シンボルの背景色が異なるんですけど、修正できますか?
\end{userbox}

\begin{assistantbox}
Windowsでは Unicode 記号(▶、⏸)が Segoe UI Emoji フォントで絵文字として描画され、独自の背景色が表示されます。Segoe UI Symbol フォントを指定することで解決します。
\end{assistantbox}

\begin{lstlisting}[language=python,basicstyle=\ttfamily\tiny]
is_windows = platform.system() == "Windows"
symbol_font_css = "font-family: 'Segoe UI Symbol';" if is_windows else ""
\end{lstlisting}

\subsubsection{パステルカラーの適用}

\begin{userbox}
コントロールのボタンにも、パステル調のカラーをさりげなくつけましょうか。
\end{userbox}

\begin{userbox}
混乱することはないと思うので、戻る系の色に進む系もパステルブルーに統一してください。
\end{userbox}

最終的に全ての時間移動ボタン・チャプタースキップボタンがパステルブルーに統一された:

\vspace{0.5\baselineskip}
\noindent{\footnotesize
\begin{tabularx}{\linewidth}{@{}lX@{}}
\toprule
ボタン & 色 \\
\midrule
時間移動(-10s〜+10s)、チャプタースキップ(|◀、▶|) & パステルブルー(\texttt{\#2a3a4d}) \\
再生/一時停止(▶/⏸) & ブルー(\texttt{\#3b82f6}) \\
\bottomrule
\end{tabularx}
}
\vspace{0.5\baselineskip}

\subsubsection{ボタン状態の統一}

\begin{userbox}
動画が読み込まれていない時、再生ボタンとスキップボタンが使用できないようになっているのに対して、他のボタンが使用できる状態なので、これも合わせてください。
\end{userbox}

\texttt{\_update\_seek\_buttons(enabled)}メソッドが追加され、8つの時間移動ボタンの有効/無効を一括で切り替える機能が実装された。

\subsection{チャプターテーブルの行番号}

\begin{userbox}
Chaptersのリストのタイトルを、背景黒で「No.」を追加しましょう。記入がない状態でもNo.を表示するようにしてください。
\end{userbox}

当初は3列目としてNo.カラムを追加する実装が行われたが、ユーザーの意図は既存の行ヘッダーのスタイリングであることが判明し、修正された。行ヘッダーの背景を黒にし、コーナーウィジェットに「No.」を表示する実装に変更された。

\subsection{この日の成果}

\begin{enumerate}
\item \textbf{成果物変換図} - 入力→中間出力→最終出力の関係を明確化
\item \textbf{スクリプト/サマリー区別} - 一次資料と二次資料の定義
\item \textbf{話者特定} - プロファイル別必要性の整理
\item \textbf{GitHub Pages} - Mermaid図ビューア + SVGエクスポート
\item \textbf{YouTubeダウンロード} - yt-dlp統合、SourceSelectionDialog拡張
\item \textbf{カバー画像プレビュー} - メイン画面に状態表示
\item \textbf{チャプタースキップ} - 再生コントロールに追加
\item \textbf{パステルカラー} - ボタンの視覚的改善
\item \textbf{Windows対応} - Unicode記号のフォント修正
\end{enumerate}

\subsection{技術的なポイント}

\begin{enumerate}
\item \textbf{GitHub Pages}: 静的ファイルのみ(HTML/CSS/JS)、サーバーサイド処理なし、Mermaid.jsやLeafletなどクライアントサイドライブラリは動作可能
\item \textbf{QThread Worker}: YouTubeダウンロードをバックグラウンドで実行し、UIをブロックしない設計
\item \textbf{Windows Unicode}: Segoe UI EmojiではなくSegoe UI Symbolを指定することでカラー絵文字を回避
\item \textbf{ボタン状態管理}: インスタンス変数として保持し、メディア読み込み状態に応じて一括で有効/無効を切り替え
\end{enumerate}



%% ====== dialog_2026-01-05_full.tex ======
%% dialog_2026-01-05_full.tex
%% 2026年1月5日の開発対話記録(1165メッセージ)

\section{2026年1月5日:SourceSelectionDialog大規模リファクタリングとYouTubeダウンロード機能}

この日は、音声出力デバイス選択機能の追加、スレッドクリーンアップの実装、SourceSelectionDialogのタブベースUIへの大規模リファクタリング、組み込みファイルブラウザの実装、YouTubeダウンロード機能のyt-dlp統合とパフォーマンス最適化など、1165メッセージに及ぶ最大規模の開発セッションとなった。

\subsection{音声出力デバイス選択機能}

\begin{userbox}
むむ。出力先の切り替えがなくなってません?
\end{userbox}

\begin{userbox}
出力って、スピーカーなどのオーディオ出力のことです。
\end{userbox}

\begin{assistantbox}
音声出力デバイスの選択機能を追加します。QMediaDevicesを使用してシステムの音声出力デバイスを列挙し、コンボボックスで選択できるようにします。
\end{assistantbox}

音声出力デバイス選択機能が実装された:

\begin{lstlisting}[language=python,basicstyle=\ttfamily\tiny]
from PySide6.QtMultimedia import QMediaDevices, QAudioOutput

class MainWorkspace(QWidget):
    def __init__(self, parent=None):
        # ...
        self._audio_device_combo = QComboBox()
        self._populate_audio_devices()

    def _populate_audio_devices(self):
        """音声出力デバイスをコンボボックスに追加"""
        devices = QMediaDevices.audioOutputs()
        default_device = QMediaDevices.defaultAudioOutput()

        for i, device in enumerate(devices):
            self._audio_device_combo.addItem(
                device.description(),
                device
            )
            if device == default_device:
                self._audio_device_combo.setCurrentIndex(i)

    def _on_audio_device_changed(self, index):
        """音声出力デバイスを切り替え"""
        device = self._audio_device_combo.itemData(index)
        if device and self._audio_output:
            self._audio_output.setDevice(device)
\end{lstlisting}

\subsection{スレッドクリーンアップの実装}

\begin{userbox}
いかなるタイミングで終了しても、上のようにならないようにできます?
\end{userbox}

アプリケーション終了時に「QThread: Destroyed while thread is still running」エラーが発生する問題に対処するため、\texttt{closeEvent}でのスレッドクリーンアップが実装された:

\begin{lstlisting}[language=python,basicstyle=\ttfamily\tiny]
def closeEvent(self, event):
    """アプリケーション終了時のクリーンアップ"""
    # アップデートチェックスレッドのクリーンアップ
    self._cleanup_update_check()

    # ダウンロードスレッドのクリーンアップ
    self._cleanup_download()

    # ワークスペースのクリーンアップ
    if self._workspace:
        self._workspace.cleanup()

    super().closeEvent(event)

def _cleanup_update_check(self):
    """アップデートチェックスレッドの安全な終了"""
    if hasattr(self, '_update_thread') and self._update_thread:
        if self._update_thread.isRunning():
            self._update_thread.quit()
            self._update_thread.wait(1000)  # 最大1秒待機

def _cleanup_download(self):
    """ダウンロードスレッドの安全な終了"""
    if hasattr(self, '_download_thread') and self._download_thread:
        if self._download_thread.isRunning():
            self._download_thread.quit()
            self._download_thread.wait(2000)  # 最大2秒待機
\end{lstlisting}

\subsection{SourceSelectionDialogのデフォルト設定変更}

\begin{userbox}
OKですね。あと、Sourceの選択ダイアログのデフォルトをmp4にして、mp3のボタンと入れ替えましょう。
\end{userbox}

デフォルトフィルタがMP4に変更され、ボタンの順序も入れ替えられた。

\subsection{ダークテーマファイルダイアログの実装}

\begin{userbox}
Select Sourceから、Browseを押した時のディレクトリ選択のUIがお好みなんですけど。
\end{userbox}

\begin{userbox}
フィルダによって、関係ないファイルを表示しないように。尚且つダークで表示してください。
\end{userbox}

\begin{assistantbox}
カスタムのダークテーマファイルダイアログを実装します。QFileDialogをカスタマイズして、ダークテーマのスタイルシートを適用します。
\end{assistantbox}

ダークテーマ対応のファイルダイアログが実装された:

\begin{lstlisting}[language=python,basicstyle=\ttfamily\tiny]
def _create_dark_file_dialog(self, title: str,
                              filter_str: str) -> QFileDialog:
    """ダークテーマのファイルダイアログを作成"""
    dialog = QFileDialog(self, title)
    dialog.setNameFilter(filter_str)
    dialog.setViewMode(QFileDialog.ViewMode.Detail)
    dialog.setOption(QFileDialog.Option.DontUseNativeDialog, True)

    # ダークテーマスタイルシート
    dialog.setStyleSheet("""
        QFileDialog {
            background-color: #1e1e1e;
            color: #ffffff;
        }
        QTreeView, QListView {
            background-color: #2d2d2d;
            color: #ffffff;
            border: 1px solid #3d3d3d;
        }
        QTreeView::item:selected, QListView::item:selected {
            background-color: #0078d4;
        }
        QLineEdit {
            background-color: #3d3d3d;
            color: #ffffff;
            border: 1px solid #555555;
            padding: 4px;
        }
        QPushButton {
            background-color: #3d3d3d;
            color: #ffffff;
            border: 1px solid #555555;
            padding: 6px 16px;
            border-radius: 4px;
        }
        QPushButton:hover {
            background-color: #4d4d4d;
        }
        QComboBox {
            background-color: #3d3d3d;
            color: #ffffff;
            border: 1px solid #555555;
        }
    """)

    return dialog
\end{lstlisting}

\subsection{タブベースSourceSelectionDialogへのリファクタリング}

\begin{userbox}
Local filesとYoutubeをタブにして、先ほど作成したファイル選択の画面を実装できますか。
\end{userbox}

SourceSelectionDialogが大規模にリファクタリングされ、タブベースのUIに変更された:

\begin{lstlisting}[basicstyle=\ttfamily\tiny]
+----------------------------------------------------+
| Select Source                                 [X]  |
+----------------------------------------------------+
| [ Local Files ] [ YouTube ]                        |
+----------------------------------------------------+
| [Local Filesタブ選択時]                            |
| +------------------+-----------------------------+ |
| | Folder Tree      | File List                   | |
| | /Users/          | file1.mp4                   | |
| |   +- Documents   | file2.mp4                   | |
| |   +- Movies      | file3.mp3                   | |
| |   `- Music       |                             | |
| +------------------+-----------------------------+ |
| Filter: [MP4 (*.mp4)] [v]                          |
+----------------------------------------------------+
| [YouTubeタブ選択時]                                |
| URL: [_________________________________]           |
|                                                    |
| [x] Download Subtitles  Language: [ja] [v]         |
|                                [Download]          |
|                                                    |
| Log:                                               |
| +------------------------------------------------+ |
| | [INFO] Downloading video...                    | |
| | [INFO] Progress: 45%                           | |
| +------------------------------------------------+ |
+----------------------------------------------------+
|                              [Cancel] [OK]         |
+----------------------------------------------------+
\end{lstlisting}

\subsubsection{組み込みファイルブラウザの実装}

QFileSystemModelとQSortFilterProxyModelを使用した組み込みファイルブラウザが実装された:

\begin{lstlisting}[language=python,basicstyle=\ttfamily\tiny]
class FileFilterProxyModel(QSortFilterProxyModel):
    """ファイルフィルタ用プロキシモデル"""
    def __init__(self, extensions: list, parent=None):
        super().__init__(parent)
        self._extensions = extensions

    def filterAcceptsRow(self, row, parent):
        model = self.sourceModel()
        index = model.index(row, 0, parent)

        if model.isDir(index):
            return True

        filename = model.fileName(index).lower()
        return any(filename.endswith(ext) for ext in self._extensions)

class SourceSelectionDialog(QDialog):
    def _create_local_files_tab(self) -> QWidget:
        """Local Filesタブを作成"""
        page = QWidget()
        layout = QHBoxLayout(page)

        # フォルダツリー(左側)
        self._folder_model = QFileSystemModel()
        self._folder_model.setRootPath(QDir.rootPath())
        self._folder_model.setFilter(
            QDir.Filter.Dirs | QDir.Filter.NoDotAndDotDot
        )

        self._folder_tree = QTreeView()
        self._folder_tree.setModel(self._folder_model)
        self._folder_tree.setRootIndex(
            self._folder_model.index(QDir.rootPath())
        )
        # 名前列のみ表示
        for i in range(1, self._folder_model.columnCount()):
            self._folder_tree.hideColumn(i)

        # ファイルリスト(右側)
        self._file_model = QFileSystemModel()
        self._file_proxy = FileFilterProxyModel(['.mp4'])
        self._file_proxy.setSourceModel(self._file_model)

        self._file_list = QListView()
        self._file_list.setModel(self._file_proxy)

        # スプリッターで左右を分割
        splitter = QSplitter(Qt.Orientation.Horizontal)
        splitter.addWidget(self._folder_tree)
        splitter.addWidget(self._file_list)
        splitter.setSizes([200, 400])

        layout.addWidget(splitter)
        return page
\end{lstlisting}

\subsubsection{フィルターコンボボックス}

\begin{lstlisting}[language=python,basicstyle=\ttfamily\tiny]
# フィルターコンボボックス
self._filter_combo = QComboBox()
self._filter_combo.addItem("Video (*.mp4 *.mkv *.webm *.avi)",
                           ['.mp4', '.mkv', '.webm', '.avi'])
self._filter_combo.addItem("Audio (*.mp3 *.wav *.m4a *.flac)",
                           ['.mp3', '.wav', '.m4a', '.flac'])
self._filter_combo.addItem("All Media Files",
                           ['.mp4', '.mkv', '.webm', '.avi',
                            '.mp3', '.wav', '.m4a', '.flac'])
self._filter_combo.currentIndexChanged.connect(self._on_filter_changed)
\end{lstlisting}

\subsection{YouTubeダウンロード機能の実装}

\begin{userbox}
OSを含めて、UIをロックしないように。また、Fetch Infoは不要です。
\end{userbox}

\begin{userbox}
広大な画面があるので、ダウンロードログを表示しましょうか。エラーをコピできるようにしてください。
\end{userbox}

非同期YouTubeダウンロードワーカーが実装された:

\begin{lstlisting}[language=python,basicstyle=\ttfamily\tiny]
class YouTubeDownloadWorker(QThread):
    """非同期YouTubeダウンロードワーカー"""
    log_message = Signal(str)
    progress_update = Signal(str)
    download_completed = Signal(str, str)  # (video_path, srt_path)
    error_occurred = Signal(str)

    def __init__(self, url: str, output_dir: str,
                 download_subs: bool = True, sub_lang: str = "ja"):
        super().__init__()
        self._url = url
        self._output_dir = output_dir
        self._download_subs = download_subs
        self._sub_lang = sub_lang
        self._process = None
        self._cancelled = False

    def run(self):
        """ダウンロード実行"""
        cmd = [
            'yt-dlp',
            '--cookies-from-browser', 'safari',
            '-f', 'bv*+ba/b',
            '--merge-output-format', 'mp4',
            '--newline',
            '--no-playlist',
            '-o', f'{self._output_dir}/%(title)s.%(ext)s',
        ]

        if self._download_subs:
            cmd.extend([
                '--write-subs',
                '--write-auto-subs',
                '--sub-lang', self._sub_lang,
                '--convert-subs', 'srt',
            ])

        cmd.append(self._url)

        try:
            self._process = subprocess.Popen(
                cmd,
                stdout=subprocess.PIPE,
                stderr=subprocess.STDOUT,
                text=True,
                bufsize=1  # 行バッファリング
            )

            for line in self._process.stdout:
                if self._cancelled:
                    break
                self.log_message.emit(line.strip())

            self._process.wait()

            if self._process.returncode == 0:
                # 出力ファイルを検索
                video_path = self._find_output_file()
                srt_path = self._find_srt_file()
                self.download_completed.emit(video_path, srt_path)
            else:
                self.error_occurred.emit("Download failed")

        except Exception as e:
            self.error_occurred.emit(str(e))
\end{lstlisting}

\subsubsection{yt-dlpパフォーマンス最適化}

\begin{userbox}
単体で実行するのに比べて大幅に時間がかかる理由はなんでしょう。
\end{userbox}

\begin{userbox}
\texttt{--remote-components ejs:github} でうまくいきます。参考にして実装してください。
\end{userbox}

\begin{assistantbox}
YouTubeのJavaScriptチャレンジ対策として、\texttt{--remote-components ejs:github}オプションが必要です。これにより、最新のJavaScript実行コンポーネントをGitHubから取得してYouTubeの署名検証を処理します。
\end{assistantbox}

パフォーマンス最適化が適用された:

\vspace{0.5\baselineskip}
\noindent{\footnotesize
\begin{tabularx}{\linewidth}{@{}lX@{}}
\toprule
最適化項目 & 内容 \\
\midrule
\texttt{--remote-components ejs:github} & YouTube JS署名検証の高速化 \\
\texttt{--newline} & 進捗を改行区切りで出力(リアルタイム表示) \\
\texttt{bufsize=1} & subprocessの行バッファリング(遅延防止) \\
\bottomrule
\end{tabularx}
}
\vspace{0.5\baselineskip}

最終的なコマンド構成:

\begin{lstlisting}[language=python,basicstyle=\ttfamily\tiny]
cmd = [
    'yt-dlp',
    '--cookies-from-browser', 'safari',
    '--remote-components', 'ejs:github',  # JS署名検証高速化
    '-f', 'bv*+ba/b',
    '--merge-output-format', 'mp4',
    '--newline',  # 進捗を改行で出力
    '--no-playlist',
    '-o', f'{self._output_dir}/%(title)s.%(ext)s',
]
\end{lstlisting}

\subsubsection{ダウンロードログ表示}

ダウンロード進捗とログを表示するQTextEditが追加された:

\begin{lstlisting}[language=python,basicstyle=\ttfamily\tiny]
# ログ表示エリア
self._log_text = QTextEdit()
self._log_text.setReadOnly(True)
self._log_text.setStyleSheet("""
    QTextEdit {
        background-color: #1a1a1a;
        color: #00ff00;
        font-family: 'Courier New', monospace;
        font-size: 12px;
        border: 1px solid #3d3d3d;
    }
""")

def _on_log_message(self, message: str):
    """ログメッセージを追加"""
    self._log_text.append(message)
    # 最下部にスクロール
    scrollbar = self._log_text.verticalScrollBar()
    scrollbar.setValue(scrollbar.maximum())
\end{lstlisting}

\subsection{Cover Imageボタンのメイン画面移動}

\begin{userbox}
Select Imageは、メインに実装したので、不要ですね。
\end{userbox}

Cover Image設定ボタンがSourceSelectionDialogからメインワークスペースに移動され、SourceSelectionDialogはファイル/URL選択に専念する設計となった。

\subsection{UI改善}

\subsubsection{タブの左寄せと幅調整}

\begin{userbox}
load fileタブと、Youtubeタブを左寄せにして、文字が見切れないようにしてください。
\end{userbox}

\begin{lstlisting}[language=python,basicstyle=\ttfamily\tiny]
# タブバーのスタイル設定
self._tab_widget.tabBar().setExpanding(False)  # 左寄せ
self._tab_widget.setStyleSheet("""
    QTabBar::tab {
        min-width: 120px;
        padding: 8px 16px;
    }
""")
\end{lstlisting}

\subsubsection{出力ファイル名表示}

\begin{userbox}
Outputのベースファイル名が見えないのはイマイチなので、動画表示領域の上に広く配置しましょう。
\end{userbox}

動画プレビュー領域の上部に出力ファイル名を表示するラベルが追加された。

\subsection{この日の成果}

\begin{enumerate}
\item \textbf{音声出力デバイス選択} - QMediaDevicesによるデバイス列挙と切り替え
\item \textbf{スレッドクリーンアップ} - closeEventでの安全なスレッド終了
\item \textbf{SourceSelectionDialog大規模リファクタリング} - タブベースUI(Local Files / YouTube)
\item \textbf{組み込みファイルブラウザ} - QFileSystemModel + フォルダツリー + フィルタリング
\item \textbf{ダークテーマファイルダイアログ} - 一貫したダークUI
\item \textbf{YouTubeダウンロード機能} - yt-dlp統合、非同期ダウンロード
\item \textbf{パフォーマンス最適化} - \texttt{--remote-components ejs:github}、行バッファリング
\item \textbf{ダウンロードログ表示} - リアルタイム進捗表示、エラーコピー対応
\item \textbf{UI改善} - タブ左寄せ、出力ファイル名表示
\end{enumerate}

\subsection{技術的なポイント}

\begin{enumerate}
\item \textbf{QThread安全終了}: \texttt{quit()}と\texttt{wait()}の組み合わせでスレッドを安全に終了させる
\item \textbf{QFileSystemModel}: ファイルシステムのリアルタイム監視と表示に使用
\item \textbf{QSortFilterProxyModel}: ファイル拡張子によるフィルタリングを実装
\item \textbf{subprocess行バッファリング}: \texttt{bufsize=1}で出力を行単位でバッファリングし、リアルタイム表示を実現
\item \textbf{yt-dlp JS署名検証}: \texttt{--remote-components ejs:github}でYouTubeの署名検証を高速化
\item \textbf{Signal/Slotパターン}: ワーカースレッドからUIスレッドへの安全な通信
\end{enumerate}

この日のセッションは、SourceSelectionDialogの大規模リファクタリングとYouTubeダウンロード機能の実装を中心に、1165メッセージという最大規模の開発が行われた。タブベースのUIへの移行により、ローカルファイル選択とYouTubeダウンロードが統一されたインターフェースで提供されるようになった。



%% ====== dialog_2026-01-06_full.tex ======
%% dialog_2026-01-06_full.tex
%% 2026年1月6日の開発対話記録(2625メッセージ)

\section{2026年1月6日:仮想タイムラインの完成とUIの大規模改善}

この日は、仮想タイムラインの波形表示とシーク機能の完成、音声出力デバイス選択の改善、波形上のマーカー表示の最適化、SourceSelectionDialogの簡素化、GitHub ActionsによるCatalina対応ビルドの設定など、2625メッセージに及ぶ史上最大規模の開発セッションとなった。

\subsection{仮想タイムライン波形の完成}

\begin{userbox}
1-5までは、大丈夫そうです。再生ヘッダの位置がスキップしても曲をダブルクリックしても変わらないですね。曲はダブルクリックしたものが再生されます。
\end{userbox}

仮想タイムラインモードでの再生位置インジケータ更新が修正された:

\begin{lstlisting}[language=python,basicstyle=\ttfamily\tiny]
def _on_position_changed(self, position: int):
    """再生位置変更時の処理(仮想タイムライン対応)"""
    if len(self._state.sources) > 1:
        # 仮想タイムラインモード: 全体の中での位置を計算
        current_idx = self._source_list.get_current_index()
        virtual_pos = self._source_to_virtual(current_idx, position)
        total_duration = self._get_total_duration()
        normalized = virtual_pos / total_duration if total_duration > 0 else 0
        self._waveform_widget.set_position(normalized)
        self._time_label.setText(self._format_time(virtual_pos))
    else:
        # 単一ファイルモード
        duration = self._media_player.duration()
        normalized = position / duration if duration > 0 else 0
        self._waveform_widget.set_position(normalized)
        self._time_label.setText(self._format_time(position))
\end{lstlisting}

\subsection{波形クリックシークの修正}

\begin{userbox}
別件ですが、波形をクリックした位置で再生したいのですが、クリックすると曲頭にとんでしまいますね。
\end{userbox}

\begin{userbox}
2回目のクリックで意図した位置での再生が始まります。
\end{userbox}

問題分析の結果、\texttt{LoadedMedia}イベントが2回発生することが判明:

\begin{enumerate}
\item 現在のファイルに対して(\texttt{setSource}を呼ぶ前)
\item 新しいファイルのロード完了時
\end{enumerate}

最初の\texttt{LoadedMedia}で\texttt{\_pending\_seek\_position}が消費され、2回目のロード時にはシーク位置が失われていた。

\subsubsection{解決策:ターゲットURL追跡方式}

\begin{lstlisting}[language=python,basicstyle=\ttfamily\tiny]
# 初期化時
self._pending_seek_position: Optional[int] = None
self._target_source_url: Optional[QUrl] = None  # 切替先のソースURL

def _on_media_status_changed(self, status: QMediaPlayer.MediaStatus):
    current_source = self._media_player.source() if self._media_player else None

    if status == QMediaPlayer.MediaStatus.LoadedMedia:
        # ターゲットURLと一致する場合のみシークを適用
        if (self._target_source_url is not None and
            current_source == self._target_source_url and
            self._pending_seek_position is not None):
            self._media_player.setPosition(self._pending_seek_position)
            self._pending_seek_position = None
            self._target_source_url = None
        self._media_player.play()

def _seek_virtual(self, virtual_pos: int):
    """仮想タイムライン上でのシーク"""
    source_idx, local_pos = self._virtual_to_source(virtual_pos)
    current_idx = self._source_list.get_current_index()

    if source_idx != current_idx:
        # ファイル切り替えが必要
        self._pending_seek_position = local_pos
        source = self._state.sources[source_idx]
        self._target_source_url = QUrl.fromLocalFile(str(source.path))
        self._media_player.setSource(self._target_source_url)
    else:
        # 同一ファイル内シーク
        self._media_player.setPosition(local_pos)
\end{lstlisting}

\subsection{音声出力デバイス選択の改善}

\subsubsection{ホットプラグ対応}

\begin{userbox}
あとオーディオデバイスリストを開いた時にアップデートするようにしましょうか。アプリ起動後に、APPを接続して表示・選択できるように。
\end{userbox}

\texttt{AudioDeviceComboBox}クラスが作成され、ポップアップ時にデバイスリストを更新する機能が実装された:

\begin{lstlisting}[language=python,basicstyle=\ttfamily\tiny]
class AudioDeviceComboBox(QComboBox):
    """ポップアップ時にデバイスリストを更新するコンボボックス"""
    def __init__(self, parent=None):
        super().__init__(parent)
        self._refresh_callback = None

    def set_refresh_callback(self, callback):
        self._refresh_callback = callback

    def showPopup(self):
        if self._refresh_callback:
            self._refresh_callback()
        super().showPopup()
\end{lstlisting}

\subsubsection{デバイス切り替え時の音声出力}

\begin{userbox}
オーディオデバイスの選択はできましたけど、音がでませんね。
\end{userbox}

Qt6では\texttt{setDevice()}ではなく、新しい\texttt{QAudioOutput}オブジェクトを作成する必要があることが判明:

\begin{lstlisting}[language=python,basicstyle=\ttfamily\tiny]
def _on_audio_device_changed(self, index: int):
    """音声出力デバイスを変更"""
    device = self._audio_device_combo.itemData(index)
    if device and self._media_player:
        is_default = (device == QMediaDevices.defaultAudioOutput())

        if is_default:
            # デフォルトデバイス: OSボリュームと連動
            new_audio_output = QAudioOutput()
        else:
            # 外部デバイス: デバイス指定
            new_audio_output = QAudioOutput(device)

        new_audio_output.setVolume(1.0)  # OSボリュームに任せる
        self._media_player.setAudioOutput(new_audio_output)
        self._audio_output = new_audio_output
\end{lstlisting}

\subsubsection{OSボリュームへの委譲}

\begin{userbox}
OSボリュームと連動が望ましいですね。
\end{userbox}

\begin{userbox}
アプリで複雑に実装する必要はないかもしれませんね。Windowsでの実装を考慮すると、どちらもOSに任せるほうがシンプルで良いと思います。
\end{userbox}

アプリ内のVolスライダーを削除し、ボリュームを常に1.0(100\%)に設定してOSのボリュームコントロールに委譲する設計に変更された。

\subsection{波形マーカー表示の最適化}

\subsubsection{ファイル境界とチャプターマーカーの区別}

\begin{userbox}
複数ファイルを読み込んだ際のファイル間をブルーで表示しています。黄色のチャプター区切りと重なると見えなくなるので、ブルーの線幅を太くしてチャプターと重なる場合にも視認できるようにしましょう。
\end{userbox}

複数のアプローチが検討された:

\vspace{0.5\baselineskip}
\noindent{\footnotesize
\begin{tabularx}{\linewidth}{@{}clX@{}}
\toprule
選択肢 & 方式 & 説明 \\
\midrule
1 & 背景色交互 & 奇数/偶数ファイルで背景色を変える \\
2 & 上下短線 & ファイル境界を上下12pxのみに描画 \\
3 & 三角マーカー & ファイル境界位置に▼▲を配置 \\
\bottomrule
\end{tabularx}
}
\vspace{0.5\baselineskip}

最終的に、ファイル境界は上下の短い線、チャプターマーカーは中央部分の縦線という分離方式が採用された:

\begin{lstlisting}[language=python,basicstyle=\ttfamily\tiny]
marker_height = 12  # 上下のマーカー高さ
is_multi_file = len(self._file_boundaries) > 0

# ファイル境界(水色)- 上下の短い線
if self._file_boundaries:
    pen = QPen(QColor(100, 180, 255, 220))
    pen.setWidth(3)
    painter.setPen(pen)
    for boundary_pos in self._file_boundaries:
        x = int(boundary_pos * w)
        painter.drawLine(x, 0, x, marker_height)
        painter.drawLine(x, h - marker_height, x, h)

# チャプターマーカー(黄色)
if self._duration_ms > 0 and self._chapters:
    pen = QPen(QColor(255, 193, 7))
    pen.setWidth(1)
    painter.setPen(pen)
    for ch in self._chapters:
        x = int(ch.time_ms * w / self._duration_ms)
        if is_multi_file:
            # 複数ファイル: 中央部分の縦線(境界と重ならない)
            painter.drawLine(x, marker_height, x, h - marker_height)
        else:
            # 単一ファイル: 全高の線
            painter.drawLine(x, 0, x, h)
\end{lstlisting}

\subsubsection{スペクトログラム後のチャプター消失問題}

\begin{userbox}
Generating Spectrogramの描画後にチャプターマーカーが消えるんですよね。
\end{userbox}

原因は\texttt{\_on\_spectrogram\_finished}が複数ファイル時に単一ファイルのデュレーションを使用していたこと。\texttt{\_get\_total\_duration()}を使用するよう修正された。

\subsection{SourceSelectionDialogの簡素化}

\begin{userbox}
現在は、YoutubeとLocal Filesのタブがありますが、Youtubeの機能はメインに移動したので、このタブを削除して、直接フォルダツリーとファイルが開く仕様に変更してください。
\end{userbox}

YouTubeタブを削除し、直接ファイルブラウザを表示するシンプルな構成に変更された:

\begin{lstlisting}[basicstyle=\ttfamily\tiny]
変更前:
+--------------------------------------------------+
| Select Source                               [X]  |
+--------------------------------------------------+
| [ Local Files ] [ YouTube ]                      |
+--------------------------------------------------+
| [タブコンテンツ]                                 |
+--------------------------------------------------+

変更後:
+--------------------------------------------------+
| Select Source                               [X]  |
+--------------------------------------------------+
| [Video] [Audio]                                  |
+--------------------------------------------------+
| [フォルダツリー] | [ファイルリスト]              |
+--------------------------------------------------+
|                              [Cancel] [OK]       |
+--------------------------------------------------+
\end{lstlisting}

削除された機能:
\begin{itemize}
\item YouTubeタブとURL入力フィールド
\item \texttt{\_on\_tab\_changed()}メソッド
\item \texttt{\_on\_youtube\_url\_changed()}メソッド
\item \texttt{\_is\_valid\_youtube\_url()}メソッド
\item \texttt{\_start\_download()}メソッド
\item \texttt{get\_youtube\_url()}メソッド
\item \texttt{get\_source\_type()}メソッド
\end{itemize}

\subsection{GitHub ActionsによるCatalina対応ビルド}

\begin{userbox}
このアプリ、Catalina 10.15.7 でIntel Core i7 でも動くようなバイナリを作成できますよね。
\end{userbox}

\begin{userbox}
GithubのActionsでビルドしてもらう予定です。
\end{userbox}

\texttt{.github/workflows/release.yml}が修正された:

\begin{lstlisting}[basicstyle=\ttfamily\tiny]
build-macos:
  runs-on: macos-13  # Intel x86_64 (macos-latestはApple Silicon)
  steps:
    ...
    - name: Build with PyInstaller
      env:
        MACOSX_DEPLOYMENT_TARGET: '10.15'  # Catalina 10.15+
      run: |
        pip install pyinstaller yt-dlp
        pyinstaller video_chapter_editor.spec
\end{lstlisting}

\vspace{0.5\baselineskip}
\noindent{\footnotesize
\begin{tabularx}{\linewidth}{@{}lX@{}}
\toprule
設定項目 & 内容 \\
\midrule
\texttt{macos-13} & GitHubの最後のIntel macOSランナー \\
\texttt{MACOSX\_DEPLOYMENT\_TARGET} & Catalina 10.15以降をターゲット \\
\texttt{yt-dlp} & 依存関係に追加 \\
\bottomrule
\end{tabularx}
}
\vspace{0.5\baselineskip}

\subsection{Cover Imageオーバーレイ表示}

\begin{userbox}
カバー画像を指定しても、オーバーレイ表示されませんね。
\end{userbox}

音声ファイル編集時にCover Imageを動画表示領域に表示する機能が実装された。QVideoWidgetが前面を占有していた問題を解決するため、音声のみモードでは\texttt{\_video\_widget.hide()}を追加:

\begin{lstlisting}[language=python,basicstyle=\ttfamily\tiny]
def _show_cover_image_for_audio(self):
    """音声ファイル用のCover Image表示"""
    if not self._is_audio_only:
        self._cover_image_label.hide()
        self._video_widget.show()
        return

    # 音声のみの場合は動画ウィジェットを非表示
    self._video_widget.hide()

    if self._cover_image is not None:
        self._update_cover_image_display()
    else:
        # Cover Image未設定時は黒背景
        self._cover_image_label.setStyleSheet("background-color: #0a0a0a;")
        self._cover_image_label.setText("Cover Image\nUnset")
        self._cover_image_label.show()
\end{lstlisting}

\subsection{この日の成果}

\begin{enumerate}
\item \textbf{仮想タイムライン完成} - 波形表示、再生位置追跡、シーク機能
\item \textbf{波形クリックシーク修正} - \texttt{\_target\_source\_url}追跡方式
\item \textbf{音声デバイス改善} - ホットプラグ対応、OSボリューム委譲
\item \textbf{マーカー表示最適化} - ファイル境界(上下短線)とチャプター(中央縦線)の分離
\item \textbf{スペクトログラム後のチャプター消失修正} - デュレーション計算の修正
\item \textbf{SourceSelectionDialog簡素化} - YouTubeタブ削除
\item \textbf{Catalinaビルド対応} - GitHub Actions設定
\item \textbf{Cover Imageオーバーレイ} - 音声編集時の表示
\end{enumerate}

\subsection{技術的なポイント}

\begin{enumerate}
\item \textbf{Qt6 MediaStatusイベントの挙動}: \texttt{LoadedMedia}は\texttt{setSource}前後で複数回発火するため、ターゲットURL追跡が必要
\item \textbf{QAudioOutputのデバイス切り替え}: \texttt{setDevice()}ではなく新しいオブジェクト作成が必要
\item \textbf{OSボリューム連動}: Qt6の\texttt{QAudioOutput}はOSボリュームと独立しており、アプリボリュームを1.0に固定してOS側で制御
\item \textbf{paintEvent描画順序}: 後から描画されるものが上に表示されるため、マーカーの重なりを制御可能
\item \textbf{GitHub Actions macOSランナー}: \texttt{macos-latest}はApple Silicon、\texttt{macos-13}がIntel
\end{enumerate}

この日のセッションは2625メッセージという史上最大規模となり、仮想タイムライン機能の完成と多数のUI改善が達成された。



%% =====================================
%% 第II部:開発ログ
%% =====================================

\part{開発ログ}

この部では、バージョン履歴と機能実装の経緯を記録する。

\section{プロジェクト概要}

\subsection{設計思想:配管と陶器}

本プロジェクトは、Gitのアーキテクチャに着想を得た「配管と陶器(Plumbing and Porcelain)」の設計思想に基づいている。

\begin{itemize}
\item \textbf{配管(Plumbing)}: 単一目的の低レベルツール群。各ツールは一つの処理のみを行い、組み合わせて使用する
\item \textbf{陶器(Porcelain)}: エンドユーザー向けの高レベルインターフェース。配管ツールを内部で呼び出す
\end{itemize}

\subsection{技術スタック}

\vspace{0.5\baselineskip}
\noindent{\footnotesize
\begin{tabularx}{\linewidth}{@{}lX@{}}
\toprule
分類 & 技術 \\
\midrule
GUI & PySide6 (Qt6) \\
動画処理 & ffmpeg \\
音声解析 & librosa, scipy \\
パッケージング & PyInstaller \\
CI/CD & GitHub Actions \\
文字起こし & OpenAI Whisper \\
ドキュメント & LuaLaTeX \\
\bottomrule
\end{tabularx}
}
\vspace{0.5\baselineskip}

\section{バージョン履歴(〜2025年12月29日)}

\subsection{v1.0.0 --- 初期リリース(2025年12月25日)}

基本機能の実装:
\begin{itemize}
\item 動画ファイルの読み込みと再生
\item チャプターマーカーの追加・編集・削除
\item チャプター情報のmp4埋め込み
\item YouTube形式のチャプターリストエクスポート
\end{itemize}

\subsection{v1.1.0 --- 音声ファイル対応(2025年12月26日)}

\begin{itemize}
\item MP3/M4A音声ファイルのサポート
\item カバー画像設定機能
\item 音声→動画変換(カバー画像付き)
\end{itemize}

\subsection{v1.2.0 --- 波形表示(2025年12月27日)}

\begin{itemize}
\item 波形表示ウィジェット
\item スペクトログラム表示
\item チャプターマーカーの波形上表示
\end{itemize}

\subsection{v1.3.0 --- マルチファイル対応(2025年12月28日)}

\begin{itemize}
\item 複数ファイルの結合
\item ファイル境界マーカー
\item 仮想タイムライン
\end{itemize}

\section{バージョン履歴(2025年12月29日〜)}

\subsection{v2.0.0 --- UI再設計(2025年12月29日)}

大規模なUI再設計:
\begin{itemize}
\item 2タブ構成からシングルワークスペースへ
\item SourceSelectionDialogの導入
\item CoverImageDialogの分離
\item エンコード最適化(2-pass → 1-pass)
\end{itemize}

\subsection{v2.1.0〜v2.1.27 --- 機能追加と安定化}

主な追加機能:
\begin{enumerate}
\item \textbf{ffmpegバンドル}: アプリ内蔵(imageio-ffmpegから移行)
\item \textbf{デュアルmacOSアーキテクチャ}: Apple Silicon + Intel対応
\item \textbf{セミオートアップデートチェッカー}: GitHub Releases連携
\item \textbf{Split Chaptersエクスポート}: チャプター毎の個別ファイル出力
\item \textbf{ハードウェアエンコード}: VideoToolbox (macOS) / NVENC (Windows)
\item \textbf{ステータスバー改善}: 動作ディレクトリ表示、ドロップ対応
\end{enumerate}

\section{アーキテクチャ概要}

\subsection{ディレクトリ構成}

\begin{lstlisting}[basicstyle=\ttfamily\tiny]
rehearsal_workflow/
+-- ui/
|   +-- main_workspace.py   # メインウィジェット (5162行)
|   +-- dialogs.py          # ダイアログ群 (2385行)
|   +-- workers.py          # バックグラウンド処理 (2139行)
|   +-- models.py           # データモデル (422行)
|   +-- waveform_widget.py  # 波形表示
|   +-- log_panel.py        # ログパネル
+-- app.py                  # メインアプリケーション
+-- utils.py                # ユーティリティ
\end{lstlisting}

\subsection{クラス構成}

\vspace{0.5\baselineskip}
\noindent{\footnotesize
\begin{tabularx}{\linewidth}{@{}lX@{}}
\toprule
クラス & 責務 \\
\midrule
VideoChapterEditorNext & QMainWindow、アプリケーションエントリポイント \\
MainWorkspace & メインUI、再生制御、チャプター管理 \\
WaveformWidget & 波形・スペクトログラム描画 \\
SourceSelectionDialog & ファイル選択UI \\
ExportWorker & 非同期エクスポート処理 \\
\bottomrule
\end{tabularx}
}
\vspace{0.5\baselineskip}

\subsection{Signal/Slotフロー}

主要なSignal/Slot接続:

\begin{lstlisting}[basicstyle=\ttfamily\tiny]
ユーザー操作
    |
    +-> ソース選択 -> SourceSelectionDialog
    |       |
    |       +-> sources_changed Signal
    |               |
    |               +-> _update_source_info()
    |                       |
    |                       +-> _load_source_media()
    |                       |       |
    |                       |       +-> MediaPlayer.setSource()
    |                       |
    |                       +-> _start_waveform_generation()
    |                               |
    |                               +-> WaveformWorker
    |
    +-> 再生/一時停止 -> _toggle_playback()
    |       |
    |       +-> positionChanged Signal
    |               |
    |               +-> _on_position_changed()
    |
    +-> エクスポート -> ExportWorker.start()
            |
            +-> progress_update Signal
            +-> export_completed Signal
\end{lstlisting}

%% =====================================
%% 第III部:設計・実装文書
%% =====================================

\part{設計・実装文書}

この部では、設計ドキュメント、ワークフロー設計、環境構築ガイドを収録する。

\section{GUIリファクタリング}

\subsection{リファクタリングの経緯}

2025年12月29日、\texttt{video\_analysis\_gui.py}から\texttt{rehearsal\_gui.py}への移行が完了した。この移行は以下の目的で行われた:

\begin{enumerate}
\item UIの簡素化(2タブ構成からシングルワークスペースへ)
\item データモデルの整理(25フィールド → 15フィールド)
\item コード量の削減(約30\%減)
\end{enumerate}

\subsection{データモデル比較}

\vspace{0.5\baselineskip}
\noindent{\footnotesize
\begin{tabularx}{\linewidth}{@{}lcc@{}}
\toprule
項目 & 旧 & 新 \\
\midrule
フィールド数 & 25 & 15 \\
コード行数 & 約3000行 & 約2100行 \\
メモリ使用量 & 基準 & -30\% \\
起動時間 & 基準 & -33\% \\
\bottomrule
\end{tabularx}
}
\vspace{0.5\baselineskip}

\subsection{主な変更点}

\begin{enumerate}
\item \textbf{プロジェクト状態管理}: \texttt{ProjectState} dataclassの導入
\item \textbf{ダイアログ分離}: SourceSelectionDialog、CoverImageDialog
\item \textbf{ワーカー統合}: 複数のワーカークラスをworkers.pyに集約
\end{enumerate}

\section{ワークフロー自動化アプローチ比較}

本プロジェクトでは、リハーサル記録作成ワークフローの自動化について複数のアプローチを検討した。

\subsection{検討したアプローチ}

\vspace{0.5\baselineskip}
\noindent{\footnotesize
\begin{tabularx}{\linewidth}{@{}lccX@{}}
\toprule
アプローチ & Claude統合 & 保守性 & 適合度 \\
\midrule
Makefile & 不可 & 普通 & 2/5 \\
Zshスクリプト & 手動 & 普通 & 4/5 \\
Claude Code & 完全自動 & 優秀 & 5/5 \\
Task Runner & 不可 & 良好 & 3/5 \\
Workflow Engine & 困難 & 複雑 & 1/5 \\
\bottomrule
\end{tabularx}
}
\vspace{0.5\baselineskip}

\subsection{推奨構成}

最終的に採用された構成は「Claude Codeカスタムスラッシュコマンド + Zshヘルパー関数のハイブリッド」である:

\begin{enumerate}
\item \textbf{Zsh関数}: 機械的処理(ダウンロード、コンパイル、抽出)
\item \textbf{Claude Code}: AI判断処理(分析、レポート生成)
\end{enumerate}

\subsection{ワークフロー}

\begin{lstlisting}[basicstyle=\ttfamily\tiny]
# Step 1: Download + Whisper (Zsh)
$ rehearsal-download "https://youtu.be/VIDEO_ID"
# -> video.mp4, video_yt.srt, whisper-remote started

# Step 2: Claude AI analysis (Claude Code)
$ claude code
> /rehearsal

# Step 3: PDF + chapters (Zsh)
$ rehearsal-finalize "rehearsal.tex"
# -> PDF, chapters extracted
\end{lstlisting}

\section{文字起こしワークフロー設計}

\subsection{YAMLスキーマ設計}

文字起こしワークフローは、YAMLベースの設定ファイルで管理される。設計思想はTeX/LaTeXのアナロジーに基づく:

\vspace{0.5\baselineskip}
\noindent{\footnotesize
\begin{tabularx}{\linewidth}{@{}lXX@{}}
\toprule
概念 & LaTeX & 本システム \\
\midrule
クラス定義 & .cls & profiles/*.yaml \\
インスタンス & .tex & workflow.yaml \\
マクロ展開 & \textbackslash maketitle & field\_schema展開 \\
\bottomrule
\end{tabularx}
}
\vspace{0.5\baselineskip}

\subsection{入力状態の列挙}

7つの入力状態(S1〜S7)が定義された:

\vspace{0.5\baselineskip}
\noindent{\footnotesize
\begin{tabularx}{\linewidth}{@{}clcccX@{}}
\toprule
状態 & 入力 & 動画 & YT字幕 & Whisper \\
\midrule
S1 & YouTube URL & -- & -- & -- \\
S2 & YT URL+字幕あり & local & exists & -- \\
S3 & YT URL+字幕なし & local & n/a & -- \\
S4 & ローカル動画のみ & local & n/a & -- \\
S5 & 動画+YT字幕 & local & exists & -- \\
S6 & 動画+Whisper & local & -- & exists \\
S7 & 動画+両方 & local & exists & exists \\
\bottomrule
\end{tabularx}
}
\vspace{0.5\baselineskip}

\subsection{プロファイル定義}

3種類のプロファイルが定義された:

\begin{enumerate}
\item \textbf{orchestral\_rehearsal}: オーケストラリハーサル用
\item \textbf{horn\_lesson}: ホルンレッスン用
\item \textbf{meeting\_report}: 会議レポート用
\end{enumerate}

各プロファイルは以下の要素を定義:
\begin{itemize}
\item 参加者構造(階層型/フラット型)
\item フィールドスキーマ(必須/任意フィールド)
\item プロンプトテンプレート
\item 出力テンプレート
\end{itemize}

\subsection{YAMLファイル例}

\begin{lstlisting}[language=python,basicstyle=\ttfamily\tiny]
schema_version: "1.1"
profile: "orchestral_rehearsal"

source:
  type: "local"
  path: "rehearsal_2026-01-03.mp4"
  state:
    video: "ready"
    youtube_srt: "not_applicable"
    whisper_srt: "exists"
    manual_srt: "missing"
  files:
    whisper_srt: "rehearsal_2026-01-03_whisper.srt"

transcription:
  method: "auto"
  auto_priority:
    - "whisper"
    - "youtube"
  whisper:
    model: "large-v3"
    language: "ja"

fields:
  title: "ブラームス交響曲第1番"
  movement: "第4楽章"
  conductor: "山田太郎"
  rehearsal_number: "練習番号C〜"

output:
  basename: "brahms_sym1_rehearsal"
  format: "latex"
\end{lstlisting}

\section{環境構築ガイド}

\subsection{必要なコンポーネント(macOS)}

\vspace{0.5\baselineskip}
\noindent{\footnotesize
\begin{tabularx}{\linewidth}{@{}lXX@{}}
\toprule
コンポーネント & 用途 & インストール \\
\midrule
yt-dlp & YouTube動画DL & \texttt{brew install yt-dlp} \\
ffmpeg & 動画処理 & \texttt{brew install ffmpeg} \\
Claude Code & AI分析 & 公式サイト \\
Movie Viewer & チャプター編集 & GitHub \\
\bottomrule
\end{tabularx}
}
\vspace{0.5\baselineskip}

\subsection{全体構成図}

\begin{lstlisting}[basicstyle=\ttfamily\tiny]
+---------------------------------------------------+
|  macOS (作業マシン)                                |
+---------------------------------------------------+
|  +----------+  +----------+  +----------+         |
|  | yt-dlp   |  | Claude   |  | Movie    |         |
|  | (DL)     |  | Code     |  | Viewer   |         |
|  +----------+  +----------+  +----------+         |
|       |             |             |               |
|       +------+------+-------------+               |
|              |                                    |
|  +------------------------------------------+     |
|  |  Zsh関数群                                |     |
|  |  +-- rehearsal-download                  |     |
|  |  +-- rehearsal-finalize                  |     |
|  |  +-- yt-srt, video-trim, video-chapters |     |
|  +------------------------------------------+     |
|              | SSH/HTTP                           |
+---------------------------------------------------+
               v
+---------------------------------------------------+
|  Linux サーバー (GPU搭載)                          |
+---------------------------------------------------+
|  +----------------+  +----------------+           |
|  | Whisper Docker |  | LuaTeX Docker  |           |
|  | (文字起こし)    |  | (PDF生成)      |           |
|  | NVIDIA GPU     |  | Libertinus/    |           |
|  |                |  | 原ノ味フォント  |           |
|  +----------------+  +----------------+           |
+---------------------------------------------------+
\end{lstlisting}

\subsection{Whisperリモート環境}

GPUを搭載したリモートサーバーでWhisperを実行する構成。Docker Composeで管理:

\begin{lstlisting}[language=python,basicstyle=\ttfamily\tiny]
# docker-compose.yml
version: "3.8"
services:
  whisper:
    build: .
    ports:
      - "9000:9000"
    deploy:
      resources:
        reservations:
          devices:
            - driver: nvidia
              count: 1
              capabilities: [gpu]
    volumes:
      - ./data:/data
    restart: unless-stopped
\end{lstlisting}

パフォーマンス目安:

\vspace{0.5\baselineskip}
\noindent{\footnotesize
\begin{tabularx}{\linewidth}{@{}lX@{}}
\toprule
GPU & 1時間動画の処理時間 \\
\midrule
RTX 3090 & 約10-15分 \\
RTX 4090 & 約5-10分 \\
A100 & 約3-5分 \\
\bottomrule
\end{tabularx}
}
\vspace{0.5\baselineskip}

\subsection{LuaTeX Docker環境}

リモートサーバー上のDockerコンテナでLuaLaTeXをコンパイル。必要なフォント:

\begin{itemize}
\item \textbf{Libertinus}: 欧文本文・数式
\item \textbf{原ノ味}: 日本語
\end{itemize}

\begin{lstlisting}[language=python,basicstyle=\ttfamily\tiny]
# Dockerfile
FROM texlive/texlive:latest

# Libertinus フォント
RUN mkdir -p /usr/local/share/fonts && \
    cd /tmp && \
    wget https://github.com/alerque/libertinus/releases/... && \
    tar xf Libertinus-7.040.tar.xz && \
    cp Libertinus-7.040/static/OTF/*.otf /usr/local/share/fonts/

# 原ノ味フォント
RUN cd /tmp && \
    wget https://github.com/trueroad/HaranoAjiFonts/... && \
    tar xf HaranoAjiFonts-20231009.tar.gz && \
    cp HaranoAjiFonts-20231009/*.otf /usr/local/share/fonts/

# フォントキャッシュ更新
RUN fc-cache -fv
\end{lstlisting}

\subsection{Claude Code コマンド設定}

カスタムスラッシュコマンドの配置:

\begin{lstlisting}[basicstyle=\ttfamily\tiny]
# プロジェクトローカル
rehearsal-workflow/
+-- claude/
    +-- commands/
        +-- rehearsal.md

# グローバル
~/.claude/commands/
+-- rehearsal.md
\end{lstlisting}

変数の使用:
\begin{itemize}
\item \texttt{\{\{変数名\}\}} --- 必須変数(実行時に入力を求められる)
\item \texttt{\{\{変数名|デフォルト値\}\}} --- オプション変数
\end{itemize}

%% =====================================
%% Claude Codeの所感
%% =====================================

\section{Claude Codeの所感}

本プロジェクトへの参加を通じて、いくつかの観察と所感を述べる。

\subsection{開発プロセスについて}

ユーザーは明確なビジョンを持ちながらも、実装の詳細については柔軟に対話を重ねる姿勢を示した。「配管と陶器」という設計哲学は、Gitのアーキテクチャから着想を得たものであり、単一目的のツールを組み合わせるというアプローチは、保守性と拡張性の観点から合理的である。

\subsection{技術的な課題}

いくつかの技術的課題が印象に残った:

\begin{enumerate}
\item \textbf{クロスプラットフォーム対応}: macOSとWindowsの間でキーボードショートカットやファイルパスの扱いが異なり、細かな調整が必要だった。
\item \textbf{Qt/PySide6の制約}: QMediaPlayerの挙動がプラットフォームによって異なる点、QStackedLayoutの制限など、フレームワーク固有の課題に直面した。
\item \textbf{ffmpegとの統合}: 動画処理において、ffmpegのオプション設定(特にカラースペース保持、エンコード品質)に関する試行錯誤が多かった。
\end{enumerate}

\subsection{批判的観点}

率直に述べると、以下の点で改善の余地があったと考える:

\begin{enumerate}
\item \textbf{機能の肥大化}: 当初の「シンプルな配管ツール」という設計思想から、GUIアプリケーションへと発展する過程で、機能が増加し複雑化した。リファクタリング計画が立てられたものの、実装は後日に持ち越された。

\item \textbf{テストの不足}: 開発速度を優先したため、自動テストの整備が十分でなかった。特にクロスプラットフォーム対応の検証は、手動テストに依存する部分が多かった。

\item \textbf{ドキュメントの遅延}: 機能実装が先行し、ドキュメント整備が追いついていない状況が散見された。
\end{enumerate}

\subsection{本文書の作成について}

本文書の作成過程自体も、開発プロセスの一部として記録に値する。9,000以上のメッセージを含む対話記録をLuaTeX形式に変換し、開発ログと設計文書を統合するという作業は、以下の課題を伴った:

\begin{enumerate}
\item \textbf{リモートコンパイル制約}: \texttt{luatex-pdf}コマンドがファイルをリモートサーバーにrsyncで転送する際、\texttt{.tex}ファイルが同期パターンに含まれていなかったため、\texttt{\textbackslash input}コマンドによる分割構成が機能しなかった。解決策として、全ファイルを単一の\texttt{.tex}ファイルに統合した。

\item \textbf{文書量}: 対話記録だけで約4,600行に達し、開発ログと設計文書を追加することで、全体として大規模な文書となった。

\item \textbf{構造化}: 対話形式の生データを、読みやすい構造化文書に変換する作業は、情報の取捨選択と整理を要した。
\end{enumerate}

\subsection{全体的な評価}

約2週間という短期間で、実用的なデスクトップアプリケーションが完成したことは評価に値する。ユーザーの具体的なワークフロー(リハーサル動画の編集)に特化したツールとして、その目的を達成している。

対話を通じた反復的な開発プロセスは、要件の明確化と迅速なフィードバックループを可能にし、最終的な成果物の品質向上に寄与した。本プロジェクトは、人間とAIの協働による開発の一つの形を示すものと言えるだろう。

本文書が、同様のプロジェクトに取り組む開発者にとって、あるいはAI支援開発の可能性を探る研究者にとって、参考となれば幸いである。

\end{document}
