%% dialog_2025-12-24_full.tex
%% 2025年12月24日の開発対話記録(658メッセージ)

\section{2025年12月24日:PADレンダラーのスタイル改善}

この日は、PadAlignedRendererのカラム整列機能とテキスト折り返し機能の実装、およびオリジナルPADtoolsスタイルへの復元作業が行われた。658メッセージに及ぶ長時間の反復的な改善作業となった。

\subsection{カラム整列機能の要望}

\begin{userbox}
一つづつの箱が、深さに応じて縦が揃うように改良できないものですかね。
\end{userbox}

\begin{assistantbox}
PADtoolsにはカラム揃え機能がありません。現在の実装では各ボックスが子を相対位置で配置するため、深さごとの縦揃えができていません。

解決策として、2パスレンダリングを実装する必要があります:
\begin{enumerate}
\item \textbf{パス1}: ツリーを走査して各深さの最大幅を計算
\item \textbf{パス2}: 固定カラム幅で配置
\end{enumerate}

PADtoolsをフォークせずに、カスタムレンダラーで実装してみます。
\end{assistantbox}

\texttt{PadAlignedRenderer.java}が作成され、2パスレンダリングアルゴリズムが実装された。コンパイル時にPADtoolsのAPIとの不整合(43個のエラー)が発生したが、JARファイルを逆コンパイルして正しいメソッド名を特定することで解決した。

主なAPI修正点:
\begin{itemize}
\item \texttt{model.getChild()} → \texttt{model.getTopNode()}
\item \texttt{node.getCall()} → \texttt{node.getText()}
\item \texttt{ifNode.getElseChild()} → \texttt{ifNode.getFalseNode()}
\item \texttt{wc.getChild()} → \texttt{wc.getChildNode()}
\end{itemize}

\subsection{テキスト折り返し機能の追加}

\begin{userbox}
ですね。その際に、内容の記述を適切に改行してあまり右に伸びないようにしたいですね。
\end{userbox}

\begin{assistantbox}
テキストの折り返し機能を追加します。長いテキストが右に伸びすぎないよう、最大幅を設定して自動改行させます。
\end{assistantbox}

以下の機能が実装された:
\begin{itemize}
\item \texttt{BOX\_MAX\_WIDTH = 180}の定数を追加
\item 論理的な区切り点(→、、・、空白)での折り返し
\item 動的なボックス高さの計算
\end{itemize}

\begin{lstlisting}[language=java,basicstyle=\ttfamily\tiny]
private List<String> wrapText(String text, double maxWidth,
                               FontMetrics metrics) {
    List<String> lines = new ArrayList<>();
    if (metrics.stringWidth(text) <= maxWidth) {
        lines.add(text);
        return lines;
    }
    // Break at logical points
    String[] segments = text.split("(?<=[->|、・ ])|(?=[->|、・ ])");
    // ... wrapping logic
}
\end{lstlisting}

\subsection{オリジナルスタイルへの復元}

\begin{userbox}
あら、随分記法が変わりましたね。元の雰囲気を生かしつつ改良できませんか
\end{userbox}

オリジナルPADtoolsの出力と比較した結果、以下の違いが特定された:

\vspace{0.5\baselineskip}
\noindent{\footnotesize
\begin{tabularx}{\linewidth}{@{}lYY@{}}
\toprule
要素 & 元のスタイル & 現在の実装 \\
\midrule
端子 & 楕円 & 角丸四角 \\
呼出 & 両端に二重線 & 背景色付き四角 \\
コメント & 括弧のみ & 点線枠 \\
選択 & 矢印型 & 四角+文字 \\
色 & 白黒のみ & 色付き \\
\bottomrule
\end{tabularx}
}
\vspace{0.5\baselineskip}

これらの要素を順次修正し、オリジナルスタイルを復元した。

\subsection{線の接続パターンの修正}

\begin{userbox}
線が繋がっていたりいなかったり。忠実に再現できませんかね。
\end{userbox}

オリジナルPADtoolsの線描画パターンを分析した結果:
\begin{enumerate}
\item 左側の縦線(スパイン)が全要素を接続
\item 呼出ボックスの右側から子要素への縦線
\item 選択分岐の縦線がペナントの左端として共有
\end{enumerate}

\subsection{条件分岐(Switch/If)の改善}

\begin{userbox}
少しずつ直しましょう。条件分岐のボックスを同様に作成してください。
\end{userbox}

条件分岐の形状がシェブロン(両端が尖る)からペナント(左が平坦、右が尖る)に修正された。

\begin{userbox}
条件分岐をもっと修正しましょう。まず、左端から旗のようになってペナント形状が縦に繋がっています。
\end{userbox}

\begin{assistantbox}
オリジナルとの違いがわかりました:

\textbf{オリジナルPADtools:}
\begin{itemize}
\item 縦線がペナントの左端として共有されている
\item 条件テキストは縦線の左側に配置
\item ペナントは縦線から始まって右に伸びる
\end{itemize}

\textbf{現在の実装:}
\begin{itemize}
\item 縦線がxの位置にある
\item ペナントはx + conditionWidthから始まる(縦線と離れている)
\end{itemize}

修正します。
\end{assistantbox}

\begin{userbox}
音声のみ、動画のみの間に線はありませんね。
\end{userbox}

ケース間の水平線が削除された。

\begin{userbox}
おお線画なくなりましたけど、最も上と下は残してください。
\end{userbox}

ブロック最上部と最下部の水平線のみ残す形に修正された。

\begin{userbox}
両方あり、音声のみの上下に線を引いてくださいな
\end{userbox}

各ケースの上下に水平線が追加された。

\begin{userbox}
水平線が、入りませんね。
\end{userbox}

水平線が\texttt{arrowX}までしか描画されていなかったため、\texttt{arrowTipX}(矢印の先端位置)まで延長するよう修正された。

\subsection{シンプルな方針への転換}

長時間の反復的な修正作業の後、ユーザーはシンプルな方針への転換を提案した。

\begin{userbox}
うーん。難しいですね。改行なしにして、水平方向のみの移動で深さを揃えるようにしましょうか。
\end{userbox}

\begin{assistantbox}
そうですね。シンプルな方針に変更しましょう。

\textbf{新しい方針:}
\begin{enumerate}
\item \textbf{テキスト折り返しなし} - 1行で表示
\item \textbf{水平方向のみで深さを表現} - 同じ深さのノードは同じX座標
\item \textbf{縦方向は単純に上から下へ流れる}
\end{enumerate}

これにより:
\begin{itemize}
\item ペナント形状の複雑な水平線問題が解消
\item レイアウトが予測しやすくなる
\item オリジナルPADtoolsに近いシンプルな構造
\end{itemize}
\end{assistantbox}

\begin{userbox}
ですねー
\end{userbox}

テキスト折り返し機能が削除され、水平方向のみで深さを表現するシンプルなレイアウトに変更された。

\subsection{この日のコミット履歴}

この日は多くのコミットが行われた:

\begin{enumerate}
\item \texttt{e0c0383} - Add column-aligned PAD renderer with text wrapping
\item \texttt{5dc1e95} - Restore original PADtools visual style with column alignment
\item \texttt{8fc8051} - Fix vertical connecting lines to match original PADtools style
\item \texttt{4636c96} - Change condition branch shape to pennant
\item \texttt{8c60a11} - Fix condition branch layout: vertical line connects pennants
\item \texttt{fe10cad} - Improve Switch/If rendering: vertically connected pennants
\item \texttt{586462a} - Remove horizontal separator lines between Switch/If cases
\item \texttt{73e8a58} - Add top and bottom border lines for Switch/If blocks
\item \texttt{596038c} - Add horizontal separator lines between each case in Switch/If
\end{enumerate}

\subsection{この日の教訓}

この日の開発で得られた教訓は以下の通りである:

\begin{enumerate}
\item \textbf{既存ライブラリのAPI調査の重要性}: JARファイルの逆コンパイルにより正しいメソッド名を特定できた
\item \textbf{反復的な改善の限界}: 細かな視覚的調整を繰り返すよりも、シンプルな方針に転換する方が効果的な場合がある
\item \textbf{オリジナルとの比較の重要性}: 常にオリジナル出力と比較しながら作業を進めることで、差異を明確に把握できた
\end{enumerate}

