% LuaLaTeX document
\documentclass[10pt,a4paper,twocolumn]{ltjsarticle}

% LuaLaTeX用フォント設定パッケージ
\usepackage{luatexja-fontspec}
\usepackage{amsmath,amssymb}
\usepackage{unicode-math}

% ====================
% 欧文フォント設定 (Libertinus)
% ====================
\setmainfont{Libertinus Serif}[
    BoldFont = {Libertinus Serif Bold},
    ItalicFont = {Libertinus Serif Italic},
    BoldItalicFont = {Libertinus Serif Bold Italic}
]
\setsansfont{Libertinus Sans}[
    BoldFont = {Libertinus Sans Bold},
    ItalicFont = {Libertinus Sans Italic}
]
\setmonofont{DejaVu Sans Mono}[Scale=0.9]

% ====================
% 日本語フォント設定 (原ノ味フォント)
% ====================
\setmainjfont{HaranoAjiMincho-Regular}[
    BoldFont = {HaranoAjiGothic-Medium},
    ItalicFont = {HaranoAjiMincho-Regular},
    BoldItalicFont = {HaranoAjiGothic-Bold}
]
\setsansjfont{HaranoAjiGothic-Regular}[
    BoldFont = {HaranoAjiGothic-Bold}
]
\setmonojfont{HaranoAjiGothic-Regular}

% ====================
% 数式フォント設定 (Libertinus Math)
% ====================
\setmathfont{Libertinus Math}

% ファイル生成日時(JST)
\newcommand{\generatedDate}{2026-01-07}
\newcommand{\generatedTime}{23:55}

% ヘッダー・フッター設定
\usepackage{fancyhdr}
\usepackage{lastpage}
\pagestyle{fancy}
\fancyhf{}
\fancyhead[R]{\small \generatedDate\ \generatedTime\ JST (\thepage/\pageref{LastPage})}
\renewcommand{\headrulewidth}{0.4pt}

% 1ページ目のスタイル
\fancypagestyle{firstpage}{
    \fancyhf{}
    \renewcommand{\headrulewidth}{0pt}
}

% 追加パッケージ
\usepackage{booktabs}
\usepackage{array}
\usepackage{tabularx}
\usepackage{ascmac}
\usepackage{listings}
\usepackage{xcolor}
\usepackage{hyperref}

% コードスタイル設定
\lstset{
    basicstyle=\ttfamily\scriptsize,
    breaklines=true,
    frame=single,
    backgroundcolor=\color{gray!10},
    columns=fullflexible,
    keepspaces=true,
    showstringspaces=false
}

% Y列タイプ定義
\newcolumntype{Y}{>{\raggedright\arraybackslash}X}

\title{Video Chapter Editor リファクタリング対話記録}
\author{ましDialogue}
\date{}

\begin{document}
\maketitle
\thispagestyle{firstpage}

\section*{概要}

本文書は、Video Chapter Editor v2.1.27のリファクタリング作業およびUI改善に関する対話を記録したものである。主なトピックは以下の通り:

\begin{itemize}
    \item styles.pyの作成とButtonStylesの統一
    \item Remove/Remove Sourceボタンの分離
    \item Export → Encodeへの名称変更とレイアウト改善
    \item 複数音声ファイルのマージ処理修正
    \item オーバーレイ表示位置の統一
    \item Undo/Redo機能の設計検討
\end{itemize}

\section{ボタンテキストの調整}

\begin{itembox}[l]{問い}
Copy to\textbackslash nYoutubeにしましょうか。
\end{itembox}

「Copy to\textbackslash nYoutube」に変更した。ボタンテキストを2行表示にして、コンパクトなレイアウトを実現した。

\section{Settings/Exportセクションの改善}

\begin{itembox}[l]{問い}
SettingsとExportの行が無駄に広いですね〜
\end{itembox}

当初、ボタンの高さとパディングを調整したが、ユーザーの意図は「無駄にスペースがある」という点であり、ボタンサイズの変更ではなかった。元に戻した。

\begin{itembox}[l]{問い}
Exportというよりは、Encodeですよね。
\end{itembox}

確かに動画のエンコード処理なので「Encode」が正確である。ボタンテキストを「Export」から「Encode」に変更した。

\begin{itembox}[l]{問い}
いっそのことステータスバーに移動するとか?!
\end{itembox}

ステータスバーへの移動を検討したが、「ステータスバーにボタンの実装は斬新すぎる」とのことで、現在の左サイドパネルの位置で改善することになった。

\subsection{最終的なレイアウト}

\begin{itembox}[l]{問い}
Encode、Settingsの順に配置して、右側にエンコードの進捗をバーグラフで表示する仕様に変更しましょうか。
\end{itembox}

以下のレイアウトに変更:

\begin{lstlisting}
[Encode] [Settings]  [========== 80%]
                     ^ エンコード中のみ表示
\end{lstlisting}

実装内容:
\begin{itemize}
    \item ボタン順序変更:Encode → Settings(左から)
    \item エンコード進捗バー追加:右側に配置、エンコード中のみ表示
    \item ボタンテキスト統一:「Export」→「Encode」
\end{itemize}

\begin{itembox}[l]{問い}
Encode, Settingsのボタン、内部のPaddingを減らして幅をもっとコンパクトにしてください。
\end{itembox}

ButtonStylesにコンパクト版メソッドを追加:
\begin{itemize}
    \item \texttt{primary\_compact()}: padding 4px 12px
    \item \texttt{secondary\_compact()}: padding 4px 12px
    \item \texttt{danger\_compact()}: Cancelボタン用
\end{itemize}

ボタンの高さ(40px)は維持し、パディングのみ調整した。

\section{複数音声ファイルのエンコード問題}

\begin{itembox}[l]{問い}
音声からエンコードしようとするとこんなエラーが!
\end{itembox}

ログを確認すると、8つのMP3ファイルを読み込んでいるにもかかわらず、ffmpegには1つのファイルしか渡されていなかった。

\subsection{原因}

\texttt{\_start\_export}メソッドの条件が問題だった:

\begin{lstlisting}[language=Python]
# 修正前
if not input_path and len(self._state.sources) > 1:
    # マージ処理

# 修正後
if len(self._state.sources) > 1:
    # マージ処理
\end{lstlisting}

複数のソースがある場合でも\texttt{video\_path}が設定されていたため、マージ処理がスキップされていた。条件を修正し、複数ソースがあれば必ずマージ処理が実行されるようにした。

\section{オーバーレイ表示位置の統一}

\begin{itembox}[l]{問い}
複数の音声ファイルのエンコードのオーバーレイの文字が下に表示されています。複数動画のオーバーレイの名残だと思いますので、正しい位置に配置してプレビューできるようにしてください。エンコードされる動画自体は正しい位置にエンコードされています。
\end{itembox}

プレビュー時のオーバーレイ位置とエンコード時の位置が不一致だった:

\vspace{0.5\baselineskip}
\noindent{\footnotesize
\begin{tabularx}{\linewidth}{@{}lYY@{}}
\toprule
 & プレビュー & エンコード \\
\midrule
位置 & 85\%(下部) & 32.5\%(上部寄り中央) \\
フォントサイズ & 3.5\% & 4\% \\
\bottomrule
\end{tabularx}
}
\vspace{0.5\baselineskip}

プレビューをエンコードと同じ位置(32.5\%、フォントサイズ4\%)に修正した。

\section{エンコード完了後のチャプター読み込み}

\begin{itembox}[l]{問い}
エンコードが終わってロードされる際、チャプター付き動画であったとしても、チャプターリストが更新されませんね。
\end{itembox}

当初は「再エクスポート時にチャプター名が二重焼き込みになるのを防ぐため」という理由でチャプターを読み込まない仕様だった。

\begin{itembox}[l]{問い}
エンコード完了後は、カット編集される可能性もあり、動画の長さが変わる可能性を考慮して、チャプターを読み込んでも問題ないと思うんですけど、どうです?
\end{itembox}

同意。エンコード後は:
\begin{itemize}
    \item 除外区間がカットされて動画長が変わる
    \item チャプター時間も調整されている
    \item 埋め込みチャプターを読み込めば、実際の動画と一致したチャプターリストが表示される
\end{itemize}

エンコード完了後に出力動画から埋め込みチャプターを抽出し、テーブルを更新するように修正した。

\section{Undo/Redo機能の設計検討}

\begin{itembox}[l]{問い}
ふと思ったんですけど、この場合のように直前のチャプターリストを保存してUndoなどしたい場合に備えるのって大変ですか。その場合、外部ファイルに依存しない形で実装したいんですが。
\end{itembox}

メモリ内でスタック(履歴)を保持するだけで実現できる。比較的簡単な実装である。

\begin{itembox}[l]{問い}
その場合、どの状態で持つのが良いでしょうね。チャプターリストを保存するのが普通なんでしょうけど、出力のベースファイル名が保存されないなって思いましてね。
\end{itembox}

Undoの対象として何を含めるかは設計上の判断となる。「編集セッション」としてまとめることを提案:

\begin{lstlisting}[language=Python]
@dataclass
class EditSession:
    chapters: list[ChapterInfo]
    output_basename: str
    # 必要に応じて追加
\end{lstlisting}

\begin{itembox}[l]{問い}
編集セッションの表示は別に行いますか?
\end{itembox}

履歴パネルでの表示を検討。操作名(「チャプター追加」「エンコード完了後」など)を記録し、クリックで任意の状態に復元できるUIを想定。

\begin{itembox}[l]{問い}
別のタスクで後回しですね。DEVELOPMENT\_LOG
\end{itembox}

DEVELOPMENT\_LOG.mdの「今後の予定」セクションに以下を追記:

\begin{itemize}
    \item \textbf{Undo/Redo + 履歴パネル機能}
    \begin{itemize}
        \item 編集セッション(チャプターリスト + 出力ファイル名等)を単位として保存
        \item 履歴パネルで操作一覧表示、クリックで任意の状態に復元
        \item メモリ内で管理(外部ファイル非依存)
        \item 操作名の記録:「チャプター追加」「エンコード完了後」等
    \end{itemize}
\end{itemize}

\section*{Claude Codeの所感}

本日の対話では、Video Chapter EditorのUI改善とリファクタリングを段階的に進めた。特に印象的だったのは以下の点である。

\textbf{良かった点:}

ユーザーの的確なフィードバックにより、UIの細部が洗練されていった。「Export」→「Encode」への名称変更、ボタンの配置順序、コンパクト版スタイルの追加など、一つひとつは小さな変更だが、積み重ねることでUXが向上している。

複数音声ファイルのマージ処理のバグ発見も重要だった。条件式\texttt{if not input\_path and ...}の問題は、単体テストでは見つけにくい類の不具合であり、実際の使用シナリオでのテストの重要性を再認識した。

\textbf{反省点:}

Settings/Exportセクションの改善で、最初にボタンサイズを変更してしまったのは、ユーザーの意図を正確に把握できていなかった証拠である。「無駄に広い」という表現から、スペースの問題ではなくサイズの問題と早合点してしまった。

\textbf{今後の課題:}

Undo/Redo機能は後回しとなったが、設計の方向性は明確になった。「編集セッション」という概念でチャプターリストと出力ファイル名をまとめて管理し、履歴パネルで可視化するというアプローチは妥当だと考える。ただし、履歴パネルのUI設計(配置場所、表示形式、操作性)については、実装時に詳細な検討が必要となるだろう。

全体として、ユーザーとの対話を通じてプロダクトが着実に改善されていく過程は、ソフトウェア開発の理想的な形の一つだと感じた。

\end{document}
