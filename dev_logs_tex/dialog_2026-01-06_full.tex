%% dialog_2026-01-06_full.tex
%% 2026年1月6日の開発対話記録(2625メッセージ)

\section{2026年1月6日:仮想タイムラインの完成とUIの大規模改善}

この日は、仮想タイムラインの波形表示とシーク機能の完成、音声出力デバイス選択の改善、波形上のマーカー表示の最適化、SourceSelectionDialogの簡素化、GitHub ActionsによるCatalina対応ビルドの設定など、2625メッセージに及ぶ史上最大規模の開発セッションとなった。

\subsection{仮想タイムライン波形の完成}

\begin{userbox}
1-5までは、大丈夫そうです。再生ヘッダの位置がスキップしても曲をダブルクリックしても変わらないですね。曲はダブルクリックしたものが再生されます。
\end{userbox}

仮想タイムラインモードでの再生位置インジケータ更新が修正された:

\begin{lstlisting}[language=python,basicstyle=\ttfamily\tiny]
def _on_position_changed(self, position: int):
    """再生位置変更時の処理(仮想タイムライン対応)"""
    if len(self._state.sources) > 1:
        # 仮想タイムラインモード: 全体の中での位置を計算
        current_idx = self._source_list.get_current_index()
        virtual_pos = self._source_to_virtual(current_idx, position)
        total_duration = self._get_total_duration()
        normalized = virtual_pos / total_duration if total_duration > 0 else 0
        self._waveform_widget.set_position(normalized)
        self._time_label.setText(self._format_time(virtual_pos))
    else:
        # 単一ファイルモード
        duration = self._media_player.duration()
        normalized = position / duration if duration > 0 else 0
        self._waveform_widget.set_position(normalized)
        self._time_label.setText(self._format_time(position))
\end{lstlisting}

\subsection{波形クリックシークの修正}

\begin{userbox}
別件ですが、波形をクリックした位置で再生したいのですが、クリックすると曲頭にとんでしまいますね。
\end{userbox}

\begin{userbox}
2回目のクリックで意図した位置での再生が始まります。
\end{userbox}

問題分析の結果、\texttt{LoadedMedia}イベントが2回発生することが判明:

\begin{enumerate}
\item 現在のファイルに対して(\texttt{setSource}を呼ぶ前)
\item 新しいファイルのロード完了時
\end{enumerate}

最初の\texttt{LoadedMedia}で\texttt{\_pending\_seek\_position}が消費され、2回目のロード時にはシーク位置が失われていた。

\subsubsection{解決策:ターゲットURL追跡方式}

\begin{lstlisting}[language=python,basicstyle=\ttfamily\tiny]
# 初期化時
self._pending_seek_position: Optional[int] = None
self._target_source_url: Optional[QUrl] = None  # 切替先のソースURL

def _on_media_status_changed(self, status: QMediaPlayer.MediaStatus):
    current_source = self._media_player.source() if self._media_player else None

    if status == QMediaPlayer.MediaStatus.LoadedMedia:
        # ターゲットURLと一致する場合のみシークを適用
        if (self._target_source_url is not None and
            current_source == self._target_source_url and
            self._pending_seek_position is not None):
            self._media_player.setPosition(self._pending_seek_position)
            self._pending_seek_position = None
            self._target_source_url = None
        self._media_player.play()

def _seek_virtual(self, virtual_pos: int):
    """仮想タイムライン上でのシーク"""
    source_idx, local_pos = self._virtual_to_source(virtual_pos)
    current_idx = self._source_list.get_current_index()

    if source_idx != current_idx:
        # ファイル切り替えが必要
        self._pending_seek_position = local_pos
        source = self._state.sources[source_idx]
        self._target_source_url = QUrl.fromLocalFile(str(source.path))
        self._media_player.setSource(self._target_source_url)
    else:
        # 同一ファイル内シーク
        self._media_player.setPosition(local_pos)
\end{lstlisting}

\subsection{音声出力デバイス選択の改善}

\subsubsection{ホットプラグ対応}

\begin{userbox}
あとオーディオデバイスリストを開いた時にアップデートするようにしましょうか。アプリ起動後に、APPを接続して表示・選択できるように。
\end{userbox}

\texttt{AudioDeviceComboBox}クラスが作成され、ポップアップ時にデバイスリストを更新する機能が実装された:

\begin{lstlisting}[language=python,basicstyle=\ttfamily\tiny]
class AudioDeviceComboBox(QComboBox):
    """ポップアップ時にデバイスリストを更新するコンボボックス"""
    def __init__(self, parent=None):
        super().__init__(parent)
        self._refresh_callback = None

    def set_refresh_callback(self, callback):
        self._refresh_callback = callback

    def showPopup(self):
        if self._refresh_callback:
            self._refresh_callback()
        super().showPopup()
\end{lstlisting}

\subsubsection{デバイス切り替え時の音声出力}

\begin{userbox}
オーディオデバイスの選択はできましたけど、音がでませんね。
\end{userbox}

Qt6では\texttt{setDevice()}ではなく、新しい\texttt{QAudioOutput}オブジェクトを作成する必要があることが判明:

\begin{lstlisting}[language=python,basicstyle=\ttfamily\tiny]
def _on_audio_device_changed(self, index: int):
    """音声出力デバイスを変更"""
    device = self._audio_device_combo.itemData(index)
    if device and self._media_player:
        is_default = (device == QMediaDevices.defaultAudioOutput())

        if is_default:
            # デフォルトデバイス: OSボリュームと連動
            new_audio_output = QAudioOutput()
        else:
            # 外部デバイス: デバイス指定
            new_audio_output = QAudioOutput(device)

        new_audio_output.setVolume(1.0)  # OSボリュームに任せる
        self._media_player.setAudioOutput(new_audio_output)
        self._audio_output = new_audio_output
\end{lstlisting}

\subsubsection{OSボリュームへの委譲}

\begin{userbox}
OSボリュームと連動が望ましいですね。
\end{userbox}

\begin{userbox}
アプリで複雑に実装する必要はないかもしれませんね。Windowsでの実装を考慮すると、どちらもOSに任せるほうがシンプルで良いと思います。
\end{userbox}

アプリ内のVolスライダーを削除し、ボリュームを常に1.0(100\%)に設定してOSのボリュームコントロールに委譲する設計に変更された。

\subsection{波形マーカー表示の最適化}

\subsubsection{ファイル境界とチャプターマーカーの区別}

\begin{userbox}
複数ファイルを読み込んだ際のファイル間をブルーで表示しています。黄色のチャプター区切りと重なると見えなくなるので、ブルーの線幅を太くしてチャプターと重なる場合にも視認できるようにしましょう。
\end{userbox}

複数のアプローチが検討された:

\vspace{0.5\baselineskip}
\noindent{\footnotesize
\begin{tabularx}{\linewidth}{@{}clX@{}}
\toprule
選択肢 & 方式 & 説明 \\
\midrule
1 & 背景色交互 & 奇数/偶数ファイルで背景色を変える \\
2 & 上下短線 & ファイル境界を上下12pxのみに描画 \\
3 & 三角マーカー & ファイル境界位置に▼▲を配置 \\
\bottomrule
\end{tabularx}
}
\vspace{0.5\baselineskip}

最終的に、ファイル境界は上下の短い線、チャプターマーカーは中央部分の縦線という分離方式が採用された:

\begin{lstlisting}[language=python,basicstyle=\ttfamily\tiny]
marker_height = 12  # 上下のマーカー高さ
is_multi_file = len(self._file_boundaries) > 0

# ファイル境界(水色)- 上下の短い線
if self._file_boundaries:
    pen = QPen(QColor(100, 180, 255, 220))
    pen.setWidth(3)
    painter.setPen(pen)
    for boundary_pos in self._file_boundaries:
        x = int(boundary_pos * w)
        painter.drawLine(x, 0, x, marker_height)
        painter.drawLine(x, h - marker_height, x, h)

# チャプターマーカー(黄色)
if self._duration_ms > 0 and self._chapters:
    pen = QPen(QColor(255, 193, 7))
    pen.setWidth(1)
    painter.setPen(pen)
    for ch in self._chapters:
        x = int(ch.time_ms * w / self._duration_ms)
        if is_multi_file:
            # 複数ファイル: 中央部分の縦線(境界と重ならない)
            painter.drawLine(x, marker_height, x, h - marker_height)
        else:
            # 単一ファイル: 全高の線
            painter.drawLine(x, 0, x, h)
\end{lstlisting}

\subsubsection{スペクトログラム後のチャプター消失問題}

\begin{userbox}
Generating Spectrogramの描画後にチャプターマーカーが消えるんですよね。
\end{userbox}

原因は\texttt{\_on\_spectrogram\_finished}が複数ファイル時に単一ファイルのデュレーションを使用していたこと。\texttt{\_get\_total\_duration()}を使用するよう修正された。

\subsection{SourceSelectionDialogの簡素化}

\begin{userbox}
現在は、YoutubeとLocal Filesのタブがありますが、Youtubeの機能はメインに移動したので、このタブを削除して、直接フォルダツリーとファイルが開く仕様に変更してください。
\end{userbox}

YouTubeタブを削除し、直接ファイルブラウザを表示するシンプルな構成に変更された:

\begin{lstlisting}[basicstyle=\ttfamily\tiny]
変更前:
+--------------------------------------------------+
| Select Source                               [X]  |
+--------------------------------------------------+
| [ Local Files ] [ YouTube ]                      |
+--------------------------------------------------+
| [タブコンテンツ]                                 |
+--------------------------------------------------+

変更後:
+--------------------------------------------------+
| Select Source                               [X]  |
+--------------------------------------------------+
| [Video] [Audio]                                  |
+--------------------------------------------------+
| [フォルダツリー] | [ファイルリスト]              |
+--------------------------------------------------+
|                              [Cancel] [OK]       |
+--------------------------------------------------+
\end{lstlisting}

削除された機能:
\begin{itemize}
\item YouTubeタブとURL入力フィールド
\item \texttt{\_on\_tab\_changed()}メソッド
\item \texttt{\_on\_youtube\_url\_changed()}メソッド
\item \texttt{\_is\_valid\_youtube\_url()}メソッド
\item \texttt{\_start\_download()}メソッド
\item \texttt{get\_youtube\_url()}メソッド
\item \texttt{get\_source\_type()}メソッド
\end{itemize}

\subsection{GitHub ActionsによるCatalina対応ビルド}

\begin{userbox}
このアプリ、Catalina 10.15.7 でIntel Core i7 でも動くようなバイナリを作成できますよね。
\end{userbox}

\begin{userbox}
GithubのActionsでビルドしてもらう予定です。
\end{userbox}

\texttt{.github/workflows/release.yml}が修正された:

\begin{lstlisting}[basicstyle=\ttfamily\tiny]
build-macos:
  runs-on: macos-13  # Intel x86_64 (macos-latestはApple Silicon)
  steps:
    ...
    - name: Build with PyInstaller
      env:
        MACOSX_DEPLOYMENT_TARGET: '10.15'  # Catalina 10.15+
      run: |
        pip install pyinstaller yt-dlp
        pyinstaller video_chapter_editor.spec
\end{lstlisting}

\vspace{0.5\baselineskip}
\noindent{\footnotesize
\begin{tabularx}{\linewidth}{@{}lX@{}}
\toprule
設定項目 & 内容 \\
\midrule
\texttt{macos-13} & GitHubの最後のIntel macOSランナー \\
\texttt{MACOSX\_DEPLOYMENT\_TARGET} & Catalina 10.15以降をターゲット \\
\texttt{yt-dlp} & 依存関係に追加 \\
\bottomrule
\end{tabularx}
}
\vspace{0.5\baselineskip}

\subsection{Cover Imageオーバーレイ表示}

\begin{userbox}
カバー画像を指定しても、オーバーレイ表示されませんね。
\end{userbox}

音声ファイル編集時にCover Imageを動画表示領域に表示する機能が実装された。QVideoWidgetが前面を占有していた問題を解決するため、音声のみモードでは\texttt{\_video\_widget.hide()}を追加:

\begin{lstlisting}[language=python,basicstyle=\ttfamily\tiny]
def _show_cover_image_for_audio(self):
    """音声ファイル用のCover Image表示"""
    if not self._is_audio_only:
        self._cover_image_label.hide()
        self._video_widget.show()
        return

    # 音声のみの場合は動画ウィジェットを非表示
    self._video_widget.hide()

    if self._cover_image is not None:
        self._update_cover_image_display()
    else:
        # Cover Image未設定時は黒背景
        self._cover_image_label.setStyleSheet("background-color: #0a0a0a;")
        self._cover_image_label.setText("Cover Image\nUnset")
        self._cover_image_label.show()
\end{lstlisting}

\subsection{この日の成果}

\begin{enumerate}
\item \textbf{仮想タイムライン完成} - 波形表示、再生位置追跡、シーク機能
\item \textbf{波形クリックシーク修正} - \texttt{\_target\_source\_url}追跡方式
\item \textbf{音声デバイス改善} - ホットプラグ対応、OSボリューム委譲
\item \textbf{マーカー表示最適化} - ファイル境界(上下短線)とチャプター(中央縦線)の分離
\item \textbf{スペクトログラム後のチャプター消失修正} - デュレーション計算の修正
\item \textbf{SourceSelectionDialog簡素化} - YouTubeタブ削除
\item \textbf{Catalinaビルド対応} - GitHub Actions設定
\item \textbf{Cover Imageオーバーレイ} - 音声編集時の表示
\end{enumerate}

\subsection{技術的なポイント}

\begin{enumerate}
\item \textbf{Qt6 MediaStatusイベントの挙動}: \texttt{LoadedMedia}は\texttt{setSource}前後で複数回発火するため、ターゲットURL追跡が必要
\item \textbf{QAudioOutputのデバイス切り替え}: \texttt{setDevice()}ではなく新しいオブジェクト作成が必要
\item \textbf{OSボリューム連動}: Qt6の\texttt{QAudioOutput}はOSボリュームと独立しており、アプリボリュームを1.0に固定してOS側で制御
\item \textbf{paintEvent描画順序}: 後から描画されるものが上に表示されるため、マーカーの重なりを制御可能
\item \textbf{GitHub Actions macOSランナー}: \texttt{macos-latest}はApple Silicon、\texttt{macos-13}がIntel
\end{enumerate}

この日のセッションは2625メッセージという史上最大規模となり、仮想タイムライン機能の完成と多数のUI改善が達成された。

