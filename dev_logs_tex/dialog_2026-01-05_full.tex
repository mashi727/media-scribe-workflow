%% dialog_2026-01-05_full.tex
%% 2026年1月5日の開発対話記録(1165メッセージ)

\section{2026年1月5日:SourceSelectionDialog大規模リファクタリングとYouTubeダウンロード機能}

この日は、音声出力デバイス選択機能の追加、スレッドクリーンアップの実装、SourceSelectionDialogのタブベースUIへの大規模リファクタリング、組み込みファイルブラウザの実装、YouTubeダウンロード機能のyt-dlp統合とパフォーマンス最適化など、1165メッセージに及ぶ最大規模の開発セッションとなった。

\subsection{音声出力デバイス選択機能}

\begin{userbox}
むむ。出力先の切り替えがなくなってません?
\end{userbox}

\begin{userbox}
出力って、スピーカーなどのオーディオ出力のことです。
\end{userbox}

\begin{assistantbox}
音声出力デバイスの選択機能を追加します。QMediaDevicesを使用してシステムの音声出力デバイスを列挙し、コンボボックスで選択できるようにします。
\end{assistantbox}

音声出力デバイス選択機能が実装された:

\begin{lstlisting}[language=python,basicstyle=\ttfamily\tiny]
from PySide6.QtMultimedia import QMediaDevices, QAudioOutput

class MainWorkspace(QWidget):
    def __init__(self, parent=None):
        # ...
        self._audio_device_combo = QComboBox()
        self._populate_audio_devices()

    def _populate_audio_devices(self):
        """音声出力デバイスをコンボボックスに追加"""
        devices = QMediaDevices.audioOutputs()
        default_device = QMediaDevices.defaultAudioOutput()

        for i, device in enumerate(devices):
            self._audio_device_combo.addItem(
                device.description(),
                device
            )
            if device == default_device:
                self._audio_device_combo.setCurrentIndex(i)

    def _on_audio_device_changed(self, index):
        """音声出力デバイスを切り替え"""
        device = self._audio_device_combo.itemData(index)
        if device and self._audio_output:
            self._audio_output.setDevice(device)
\end{lstlisting}

\subsection{スレッドクリーンアップの実装}

\begin{userbox}
いかなるタイミングで終了しても、上のようにならないようにできます?
\end{userbox}

アプリケーション終了時に「QThread: Destroyed while thread is still running」エラーが発生する問題に対処するため、\texttt{closeEvent}でのスレッドクリーンアップが実装された:

\begin{lstlisting}[language=python,basicstyle=\ttfamily\tiny]
def closeEvent(self, event):
    """アプリケーション終了時のクリーンアップ"""
    # アップデートチェックスレッドのクリーンアップ
    self._cleanup_update_check()

    # ダウンロードスレッドのクリーンアップ
    self._cleanup_download()

    # ワークスペースのクリーンアップ
    if self._workspace:
        self._workspace.cleanup()

    super().closeEvent(event)

def _cleanup_update_check(self):
    """アップデートチェックスレッドの安全な終了"""
    if hasattr(self, '_update_thread') and self._update_thread:
        if self._update_thread.isRunning():
            self._update_thread.quit()
            self._update_thread.wait(1000)  # 最大1秒待機

def _cleanup_download(self):
    """ダウンロードスレッドの安全な終了"""
    if hasattr(self, '_download_thread') and self._download_thread:
        if self._download_thread.isRunning():
            self._download_thread.quit()
            self._download_thread.wait(2000)  # 最大2秒待機
\end{lstlisting}

\subsection{SourceSelectionDialogのデフォルト設定変更}

\begin{userbox}
OKですね。あと、Sourceの選択ダイアログのデフォルトをmp4にして、mp3のボタンと入れ替えましょう。
\end{userbox}

デフォルトフィルタがMP4に変更され、ボタンの順序も入れ替えられた。

\subsection{ダークテーマファイルダイアログの実装}

\begin{userbox}
Select Sourceから、Browseを押した時のディレクトリ選択のUIがお好みなんですけど。
\end{userbox}

\begin{userbox}
フィルダによって、関係ないファイルを表示しないように。尚且つダークで表示してください。
\end{userbox}

\begin{assistantbox}
カスタムのダークテーマファイルダイアログを実装します。QFileDialogをカスタマイズして、ダークテーマのスタイルシートを適用します。
\end{assistantbox}

ダークテーマ対応のファイルダイアログが実装された:

\begin{lstlisting}[language=python,basicstyle=\ttfamily\tiny]
def _create_dark_file_dialog(self, title: str,
                              filter_str: str) -> QFileDialog:
    """ダークテーマのファイルダイアログを作成"""
    dialog = QFileDialog(self, title)
    dialog.setNameFilter(filter_str)
    dialog.setViewMode(QFileDialog.ViewMode.Detail)
    dialog.setOption(QFileDialog.Option.DontUseNativeDialog, True)

    # ダークテーマスタイルシート
    dialog.setStyleSheet("""
        QFileDialog {
            background-color: #1e1e1e;
            color: #ffffff;
        }
        QTreeView, QListView {
            background-color: #2d2d2d;
            color: #ffffff;
            border: 1px solid #3d3d3d;
        }
        QTreeView::item:selected, QListView::item:selected {
            background-color: #0078d4;
        }
        QLineEdit {
            background-color: #3d3d3d;
            color: #ffffff;
            border: 1px solid #555555;
            padding: 4px;
        }
        QPushButton {
            background-color: #3d3d3d;
            color: #ffffff;
            border: 1px solid #555555;
            padding: 6px 16px;
            border-radius: 4px;
        }
        QPushButton:hover {
            background-color: #4d4d4d;
        }
        QComboBox {
            background-color: #3d3d3d;
            color: #ffffff;
            border: 1px solid #555555;
        }
    """)

    return dialog
\end{lstlisting}

\subsection{タブベースSourceSelectionDialogへのリファクタリング}

\begin{userbox}
Local filesとYoutubeをタブにして、先ほど作成したファイル選択の画面を実装できますか。
\end{userbox}

SourceSelectionDialogが大規模にリファクタリングされ、タブベースのUIに変更された:

\begin{lstlisting}[basicstyle=\ttfamily\tiny]
+----------------------------------------------------+
| Select Source                                 [X]  |
+----------------------------------------------------+
| [ Local Files ] [ YouTube ]                        |
+----------------------------------------------------+
| [Local Filesタブ選択時]                            |
| +------------------+-----------------------------+ |
| | Folder Tree      | File List                   | |
| | /Users/          | file1.mp4                   | |
| |   +- Documents   | file2.mp4                   | |
| |   +- Movies      | file3.mp3                   | |
| |   `- Music       |                             | |
| +------------------+-----------------------------+ |
| Filter: [MP4 (*.mp4)] [v]                          |
+----------------------------------------------------+
| [YouTubeタブ選択時]                                |
| URL: [_________________________________]           |
|                                                    |
| [x] Download Subtitles  Language: [ja] [v]         |
|                                [Download]          |
|                                                    |
| Log:                                               |
| +------------------------------------------------+ |
| | [INFO] Downloading video...                    | |
| | [INFO] Progress: 45%                           | |
| +------------------------------------------------+ |
+----------------------------------------------------+
|                              [Cancel] [OK]         |
+----------------------------------------------------+
\end{lstlisting}

\subsubsection{組み込みファイルブラウザの実装}

QFileSystemModelとQSortFilterProxyModelを使用した組み込みファイルブラウザが実装された:

\begin{lstlisting}[language=python,basicstyle=\ttfamily\tiny]
class FileFilterProxyModel(QSortFilterProxyModel):
    """ファイルフィルタ用プロキシモデル"""
    def __init__(self, extensions: list, parent=None):
        super().__init__(parent)
        self._extensions = extensions

    def filterAcceptsRow(self, row, parent):
        model = self.sourceModel()
        index = model.index(row, 0, parent)

        if model.isDir(index):
            return True

        filename = model.fileName(index).lower()
        return any(filename.endswith(ext) for ext in self._extensions)

class SourceSelectionDialog(QDialog):
    def _create_local_files_tab(self) -> QWidget:
        """Local Filesタブを作成"""
        page = QWidget()
        layout = QHBoxLayout(page)

        # フォルダツリー(左側)
        self._folder_model = QFileSystemModel()
        self._folder_model.setRootPath(QDir.rootPath())
        self._folder_model.setFilter(
            QDir.Filter.Dirs | QDir.Filter.NoDotAndDotDot
        )

        self._folder_tree = QTreeView()
        self._folder_tree.setModel(self._folder_model)
        self._folder_tree.setRootIndex(
            self._folder_model.index(QDir.rootPath())
        )
        # 名前列のみ表示
        for i in range(1, self._folder_model.columnCount()):
            self._folder_tree.hideColumn(i)

        # ファイルリスト(右側)
        self._file_model = QFileSystemModel()
        self._file_proxy = FileFilterProxyModel(['.mp4'])
        self._file_proxy.setSourceModel(self._file_model)

        self._file_list = QListView()
        self._file_list.setModel(self._file_proxy)

        # スプリッターで左右を分割
        splitter = QSplitter(Qt.Orientation.Horizontal)
        splitter.addWidget(self._folder_tree)
        splitter.addWidget(self._file_list)
        splitter.setSizes([200, 400])

        layout.addWidget(splitter)
        return page
\end{lstlisting}

\subsubsection{フィルターコンボボックス}

\begin{lstlisting}[language=python,basicstyle=\ttfamily\tiny]
# フィルターコンボボックス
self._filter_combo = QComboBox()
self._filter_combo.addItem("Video (*.mp4 *.mkv *.webm *.avi)",
                           ['.mp4', '.mkv', '.webm', '.avi'])
self._filter_combo.addItem("Audio (*.mp3 *.wav *.m4a *.flac)",
                           ['.mp3', '.wav', '.m4a', '.flac'])
self._filter_combo.addItem("All Media Files",
                           ['.mp4', '.mkv', '.webm', '.avi',
                            '.mp3', '.wav', '.m4a', '.flac'])
self._filter_combo.currentIndexChanged.connect(self._on_filter_changed)
\end{lstlisting}

\subsection{YouTubeダウンロード機能の実装}

\begin{userbox}
OSを含めて、UIをロックしないように。また、Fetch Infoは不要です。
\end{userbox}

\begin{userbox}
広大な画面があるので、ダウンロードログを表示しましょうか。エラーをコピできるようにしてください。
\end{userbox}

非同期YouTubeダウンロードワーカーが実装された:

\begin{lstlisting}[language=python,basicstyle=\ttfamily\tiny]
class YouTubeDownloadWorker(QThread):
    """非同期YouTubeダウンロードワーカー"""
    log_message = Signal(str)
    progress_update = Signal(str)
    download_completed = Signal(str, str)  # (video_path, srt_path)
    error_occurred = Signal(str)

    def __init__(self, url: str, output_dir: str,
                 download_subs: bool = True, sub_lang: str = "ja"):
        super().__init__()
        self._url = url
        self._output_dir = output_dir
        self._download_subs = download_subs
        self._sub_lang = sub_lang
        self._process = None
        self._cancelled = False

    def run(self):
        """ダウンロード実行"""
        cmd = [
            'yt-dlp',
            '--cookies-from-browser', 'safari',
            '-f', 'bv*+ba/b',
            '--merge-output-format', 'mp4',
            '--newline',
            '--no-playlist',
            '-o', f'{self._output_dir}/%(title)s.%(ext)s',
        ]

        if self._download_subs:
            cmd.extend([
                '--write-subs',
                '--write-auto-subs',
                '--sub-lang', self._sub_lang,
                '--convert-subs', 'srt',
            ])

        cmd.append(self._url)

        try:
            self._process = subprocess.Popen(
                cmd,
                stdout=subprocess.PIPE,
                stderr=subprocess.STDOUT,
                text=True,
                bufsize=1  # 行バッファリング
            )

            for line in self._process.stdout:
                if self._cancelled:
                    break
                self.log_message.emit(line.strip())

            self._process.wait()

            if self._process.returncode == 0:
                # 出力ファイルを検索
                video_path = self._find_output_file()
                srt_path = self._find_srt_file()
                self.download_completed.emit(video_path, srt_path)
            else:
                self.error_occurred.emit("Download failed")

        except Exception as e:
            self.error_occurred.emit(str(e))
\end{lstlisting}

\subsubsection{yt-dlpパフォーマンス最適化}

\begin{userbox}
単体で実行するのに比べて大幅に時間がかかる理由はなんでしょう。
\end{userbox}

\begin{userbox}
\texttt{--remote-components ejs:github} でうまくいきます。参考にして実装してください。
\end{userbox}

\begin{assistantbox}
YouTubeのJavaScriptチャレンジ対策として、\texttt{--remote-components ejs:github}オプションが必要です。これにより、最新のJavaScript実行コンポーネントをGitHubから取得してYouTubeの署名検証を処理します。
\end{assistantbox}

パフォーマンス最適化が適用された:

\vspace{0.5\baselineskip}
\noindent{\footnotesize
\begin{tabularx}{\linewidth}{@{}lX@{}}
\toprule
最適化項目 & 内容 \\
\midrule
\texttt{--remote-components ejs:github} & YouTube JS署名検証の高速化 \\
\texttt{--newline} & 進捗を改行区切りで出力(リアルタイム表示) \\
\texttt{bufsize=1} & subprocessの行バッファリング(遅延防止) \\
\bottomrule
\end{tabularx}
}
\vspace{0.5\baselineskip}

最終的なコマンド構成:

\begin{lstlisting}[language=python,basicstyle=\ttfamily\tiny]
cmd = [
    'yt-dlp',
    '--cookies-from-browser', 'safari',
    '--remote-components', 'ejs:github',  # JS署名検証高速化
    '-f', 'bv*+ba/b',
    '--merge-output-format', 'mp4',
    '--newline',  # 進捗を改行で出力
    '--no-playlist',
    '-o', f'{self._output_dir}/%(title)s.%(ext)s',
]
\end{lstlisting}

\subsubsection{ダウンロードログ表示}

ダウンロード進捗とログを表示するQTextEditが追加された:

\begin{lstlisting}[language=python,basicstyle=\ttfamily\tiny]
# ログ表示エリア
self._log_text = QTextEdit()
self._log_text.setReadOnly(True)
self._log_text.setStyleSheet("""
    QTextEdit {
        background-color: #1a1a1a;
        color: #00ff00;
        font-family: 'Courier New', monospace;
        font-size: 12px;
        border: 1px solid #3d3d3d;
    }
""")

def _on_log_message(self, message: str):
    """ログメッセージを追加"""
    self._log_text.append(message)
    # 最下部にスクロール
    scrollbar = self._log_text.verticalScrollBar()
    scrollbar.setValue(scrollbar.maximum())
\end{lstlisting}

\subsection{Cover Imageボタンのメイン画面移動}

\begin{userbox}
Select Imageは、メインに実装したので、不要ですね。
\end{userbox}

Cover Image設定ボタンがSourceSelectionDialogからメインワークスペースに移動され、SourceSelectionDialogはファイル/URL選択に専念する設計となった。

\subsection{UI改善}

\subsubsection{タブの左寄せと幅調整}

\begin{userbox}
load fileタブと、Youtubeタブを左寄せにして、文字が見切れないようにしてください。
\end{userbox}

\begin{lstlisting}[language=python,basicstyle=\ttfamily\tiny]
# タブバーのスタイル設定
self._tab_widget.tabBar().setExpanding(False)  # 左寄せ
self._tab_widget.setStyleSheet("""
    QTabBar::tab {
        min-width: 120px;
        padding: 8px 16px;
    }
""")
\end{lstlisting}

\subsubsection{出力ファイル名表示}

\begin{userbox}
Outputのベースファイル名が見えないのはイマイチなので、動画表示領域の上に広く配置しましょう。
\end{userbox}

動画プレビュー領域の上部に出力ファイル名を表示するラベルが追加された。

\subsection{この日の成果}

\begin{enumerate}
\item \textbf{音声出力デバイス選択} - QMediaDevicesによるデバイス列挙と切り替え
\item \textbf{スレッドクリーンアップ} - closeEventでの安全なスレッド終了
\item \textbf{SourceSelectionDialog大規模リファクタリング} - タブベースUI(Local Files / YouTube)
\item \textbf{組み込みファイルブラウザ} - QFileSystemModel + フォルダツリー + フィルタリング
\item \textbf{ダークテーマファイルダイアログ} - 一貫したダークUI
\item \textbf{YouTubeダウンロード機能} - yt-dlp統合、非同期ダウンロード
\item \textbf{パフォーマンス最適化} - \texttt{--remote-components ejs:github}、行バッファリング
\item \textbf{ダウンロードログ表示} - リアルタイム進捗表示、エラーコピー対応
\item \textbf{UI改善} - タブ左寄せ、出力ファイル名表示
\end{enumerate}

\subsection{技術的なポイント}

\begin{enumerate}
\item \textbf{QThread安全終了}: \texttt{quit()}と\texttt{wait()}の組み合わせでスレッドを安全に終了させる
\item \textbf{QFileSystemModel}: ファイルシステムのリアルタイム監視と表示に使用
\item \textbf{QSortFilterProxyModel}: ファイル拡張子によるフィルタリングを実装
\item \textbf{subprocess行バッファリング}: \texttt{bufsize=1}で出力を行単位でバッファリングし、リアルタイム表示を実現
\item \textbf{yt-dlp JS署名検証}: \texttt{--remote-components ejs:github}でYouTubeの署名検証を高速化
\item \textbf{Signal/Slotパターン}: ワーカースレッドからUIスレッドへの安全な通信
\end{enumerate}

この日のセッションは、SourceSelectionDialogの大規模リファクタリングとYouTubeダウンロード機能の実装を中心に、1165メッセージという最大規模の開発が行われた。タブベースのUIへの移行により、ローカルファイル選択とYouTubeダウンロードが統一されたインターフェースで提供されるようになった。

