%% dialog_2026-01-03_full.tex
%% 2026年1月3日の開発対話記録(284メッセージ)

\section{2026年1月3日:文字起こしワークフローUI設計とYAMLスキーマ}

この日は、文字起こしワークフローのUI設計に着手し、YAMLベースの設定ファイル設計、プロファイルベースのアーキテクチャ、入力状態の列挙、ワークフロー境界の明確化、そしてPADおよびスイムレーン図による可視化まで、284メッセージに及ぶ大規模な設計議論が行われた。

\subsection{文字起こしワークフローUI設計の開始}

\begin{userbox}
次に、文字起こしのUI作成に移行したいと思います。
\end{userbox}

\begin{assistantbox}
report\_workflow.pyをベースに、transcription\_workflow.pyを作成します。
\end{assistantbox}

\begin{userbox}
前処理のワークフローと同様な設計思想にて実装したいと思います。まず、機能を整理しましょう。
\end{userbox}

video-chapter-editorと同様の設計思想として、以下の機能が整理された:

\vspace{0.5\baselineskip}
\noindent{\footnotesize
\begin{tabularx}{\linewidth}{@{}lX@{}}
\toprule
機能カテゴリ & 内容 \\
\midrule
入力 & 動画ファイル、SRTファイル、YouTube URL \\
SRT取得 & YouTube字幕、Whisper文字起こし \\
プロファイル & オーケストラリハーサル、ホルンレッスン、会議 \\
出力 & LaTeX文書、Markdown \\
\bottomrule
\end{tabularx}
}
\vspace{0.5\baselineskip}

\subsection{設定ファイル設計:再現性と再利用性}

\begin{userbox}
最終的には、再現性と再利用性を確保したいので、設定ファイルを保存する仕様にしたいと考えています。
\end{userbox}

\begin{assistantbox}
良い設計方針ですね。設定ファイルにより:
\begin{enumerate}
\item 同じ設定で再処理が可能(再現性)
\item 類似プロジェクトでテンプレートとして再利用(再利用性)
\item 処理履歴として記録が残る
\end{enumerate}
\end{assistantbox}

\begin{userbox}
あとはYamlが大きくなりすぎるのは望ましくなく、Claudeのカスタムコマンドなどは極力外部ファイルにしたい。
\end{userbox}

外部ファイル参照による分離方針が決定された:

\begin{lstlisting}[basicstyle=\ttfamily\tiny]
transcription_workflow.yaml
├── profile: "orchestral_rehearsal"  # 参照
├── prompt_template: "rehearsal.md"  # 外部
├── base_template: "luatex_2col.tex" # 外部
└── project固有データのみをYAML内に
\end{lstlisting}

\subsection{プロファイルベースのアーキテクチャ}

\begin{userbox}
プロファイルがマクロな定義になるという理解で良いですか。
\end{userbox}

\begin{assistantbox}
その通りです。プロファイルは「このタイプの文字起こしに必要な共通設定」を定義します。
\end{assistantbox}

\begin{userbox}
TeXとLaTeX同じ感じですね。
\end{userbox}

\begin{assistantbox}
素晴らしい喩えです。
\begin{itemize}
\item プロファイル = .clsファイル(クラス定義)
\item YAML = .texファイル(インスタンス)
\item フィールド定義 = パッケージオプション
\end{itemize}
\end{assistantbox}

このTeX/LaTeXアナロジーに基づき、以下の構造が設計された:

\vspace{0.5\baselineskip}
\noindent{\footnotesize
\begin{tabularx}{\linewidth}{@{}lXX@{}}
\toprule
概念 & LaTeX & 文字起こしワークフロー \\
\midrule
クラス定義 & article.cls & profiles/*.yaml \\
インスタンス & document.tex & workflow.yaml \\
マクロ展開 & \textbackslash maketitle & field\_schema展開 \\
\bottomrule
\end{tabularx}
}
\vspace{0.5\baselineskip}

\subsection{入力状態の列挙(7状態)}

\begin{userbox}
SRT取得の責務に関して、Youtubeのダウンロードを行う時点でSRTが取得されるのが境界が曖昧になっている点だと考えています。
\end{userbox}

\begin{userbox}
根本的な問題は入力ファイルの状態の場合わけが十分でないことですね。
\end{userbox}

\begin{assistantbox}
ご指摘の通りです。入力状態を明示的に列挙することで、責務の境界が明確になります。
\end{assistantbox}

7つの入力状態(S1--S7)が定義された:

\vspace{0.5\baselineskip}
\noindent{\footnotesize
\begin{tabularx}{\linewidth}{@{}clcccX@{}}
\toprule
状態 & 入力 & 動画 & YT字幕 & Whisper & 次アクション \\
\midrule
S1 & YouTube URL & -- & -- & -- & DL→S2/S3 \\
S2 & YT URL+字幕あり & local & exists & -- & Whisper? \\
S3 & YT URL+字幕なし & local & n/a & -- & Whisper必須 \\
S4 & ローカル動画のみ & local & n/a & -- & Whisper必須 \\
S5 & 動画+YT字幕 & local & exists & -- & Whisper? \\
S6 & 動画+Whisper & local & -- & exists & 処理可能 \\
S7 & 動画+両方 & local & exists & exists & 選択 \\
\bottomrule
\end{tabularx}
}
\vspace{0.5\baselineskip}

これを表現するため、YAMLに\texttt{source.state}フィールドが追加された:

\begin{lstlisting}[language=python,basicstyle=\ttfamily\tiny]
source:
  type: "local"       # youtube | local
  path: "video.mp4"
  state:
    video: "ready"           # ready | url_only | missing
    youtube_srt: "exists"    # exists | missing | not_applicable
    whisper_srt: "missing"   # exists | missing
    manual_srt: "missing"    # exists | missing
  files:
    youtube_srt: "video_yt.srt"
\end{lstlisting}

\subsection{ワークフロー境界の明確化}

\begin{assistantbox}
前処理の責務を明確にしましょう:

\textbf{前処理の責務}: SRTファイルが作業ディレクトリに存在することを保証する

入力タイプ別の責務:
\begin{itemize}
\item YouTube URL → 動画DL + 字幕DL(存在すれば)
\item ローカル動画 → そのまま使用
\item SRTファイル → 作業ディレクトリにコピー
\end{itemize}
\end{assistantbox}

この境界設定により、文字起こしワークフローは「SRTが存在する状態」から処理を開始できる明確な契約が確立された。

\subsection{可視化アプローチの選択}

\begin{userbox}
処理フロー、やはり深さと順序がわかりやすいので、PADの方が好きですね。
\end{userbox}

\begin{userbox}
同じことを行う際に、別表現が可能というのが最もわかりづらい点です。
\end{userbox}

UMLアクティビティ図とPADの両方を作成していたが、ユーザーの指摘により可視化方針が整理された:

\vspace{0.5\baselineskip}
\noindent{\footnotesize
\begin{tabularx}{\linewidth}{@{}lXX@{}}
\toprule
用途 & 形式 & 理由 \\
\midrule
処理フロー & PAD & 深さと順序が明確 \\
アクター間連携 & スイムレーン & User/UI/Backend分離 \\
\bottomrule
\end{tabularx}
}
\vspace{0.5\baselineskip}

\begin{userbox}
アクティビティ図に統合できません?
\end{userbox}

3つの選択肢が提示され、ユーザーは選択肢1(PAD + スイムレーン併用)を選択した。

\subsection{YAMLライフサイクル設計}

\begin{userbox}
前処理を行う前にYAMLを作成し、それに沿って前処理を行うのが自然な流れだと思うんですけどどうでしょう。
\end{userbox}

\begin{assistantbox}
おっしゃる通りです。「マニフェスト先行」のワークフローですね。

\textbf{Manifest-First Workflow}:
\begin{enumerate}
\item ユーザーがYAMLを作成(意図の宣言)
\item 前処理がYAMLに従って素材を準備
\item 状態フィールドが自動更新
\item 文字起こし処理が実行
\end{enumerate}
\end{assistantbox}

YAMLの静的セクション(ユーザー入力)と動的セクション(自動更新)が明確に分離された:

\begin{lstlisting}[basicstyle=\ttfamily\tiny]
# === 静的(ユーザー入力) ===
schema_version: "1.1"
profile: "orchestral_rehearsal"
source:
  type: "youtube"
  url: "https://..."

# === 動的(自動更新) ===
source:
  state:
    video: "ready"      # 前処理後に更新
    youtube_srt: "exists"
  files:
    video: "downloaded.mp4"
    youtube_srt: "video.srt"
\end{lstlisting}

\subsection{スイムレーン図へのYAML統合}

\begin{userbox}
ワークフローの中に、YAMLの出力を入れましょうか。
\end{userbox}

3つの選択肢が提示された:

\vspace{0.5\baselineskip}
\noindent{\footnotesize
\begin{tabularx}{\linewidth}{@{}lX@{}}
\toprule
選択肢 & 内容 \\
\midrule
A & データストアとして表現(UML標準) \\
B & 矢印でデータフローを表現 \\
C & 両方(データストア+矢印) \\
\bottomrule
\end{tabularx}
}
\vspace{0.5\baselineskip}

\begin{userbox}
Cですね。
\end{userbox}

YAMLがデータストアとして明示され、各処理ステップからの読み書きが矢印で表現されるスイムレーン図が作成された。

\subsection{プロファイルとYAMLの具体化}

\begin{userbox}
実際のものをみて考えましょう。これまでのやりとりを踏まえて、transcription\_workflow.yamlとprofiles/*.yamlを作ってみてください。
\end{userbox}

3つのプロファイルが作成された:

\subsubsection{orchestral\_rehearsal.yaml}

\begin{lstlisting}[language=python,basicstyle=\ttfamily\tiny]
name: "オーケストラリハーサル"
icon: "🎼"
description: "オーケストラのリハーサル記録用"

participants:
  type: "hierarchical"
  instructor:
    label: "指揮者"
    name_field: true
  students:
    label: "奏者"
    collective: true

field_schema:
  title:
    label: "曲目"
    type: "text"
    required: true
  movement:
    label: "楽章"
    type: "text"
  rehearsal_number:
    label: "練習番号"
    type: "text"

prompt_template: "orchestral_rehearsal.md"
base_template: "luatex_twocolumn.tex"
\end{lstlisting}

\subsubsection{horn\_lesson.yaml}

\begin{lstlisting}[language=python,basicstyle=\ttfamily\tiny]
name: "ホルンレッスン"
icon: "🎺"
participants:
  type: "hierarchical"
  instructor:
    label: "先生"
  students:
    label: "生徒"
    name_field: true

field_schema:
  title:
    label: "曲目/エチュード"
    type: "text"
  lesson_focus:
    label: "レッスンの焦点"
    type: "text"
\end{lstlisting}

\subsubsection{meeting\_report.yaml}

\begin{lstlisting}[language=python,basicstyle=\ttfamily\tiny]
name: "会議レポート"
icon: "📋"
participants:
  type: "flat"
  attendees:
    label: "参加者"
    multiple: true

field_schema:
  title:
    label: "会議名"
    type: "text"
    required: true
  agenda:
    label: "議題"
    type: "multiline"
\end{lstlisting}

\subsection{ワークフローYAMLの例}

\begin{lstlisting}[language=python,basicstyle=\ttfamily\tiny]
schema_version: "1.1"
profile: "orchestral_rehearsal"

source:
  type: "local"
  path: "rehearsal_2026-01-03.mp4"
  state:
    video: "ready"
    youtube_srt: "not_applicable"
    whisper_srt: "exists"
    manual_srt: "missing"
  files:
    whisper_srt: "rehearsal_2026-01-03_whisper.srt"

transcription:
  method: "auto"
  auto_priority:
    - "whisper"
    - "youtube"
  whisper:
    model: "large-v3"
    language: "ja"

fields:
  title: "ブラームス交響曲第1番"
  movement: "第4楽章"
  conductor: "山田太郎"
  rehearsal_number: "練習番号C〜"

output:
  basename: "brahms_sym1_rehearsal"
  format: "latex"
\end{lstlisting}

\subsection{この日の成果}

\begin{enumerate}
\item \textbf{設計文書} - transcription-workflow-design.mdの作成
\item \textbf{YAMLスキーマv1.1} - source.stateフィールドの導入
\item \textbf{プロファイルアーキテクチャ} - TeX/LaTeXアナロジーによる設計
\item \textbf{入力状態列挙} - 7状態(S1--S7)の明確な定義
\item \textbf{ワークフロー境界} - 前処理の責務明確化
\item \textbf{可視化方針} - PAD(処理フロー)+ スイムレーン(アクター連携)
\item \textbf{YAMLライフサイクル} - マニフェスト先行ワークフロー
\item \textbf{プロファイル定義} - 3種類のプロファイルYAML作成
\item \textbf{PAD図} - transcription-workflow.spd, preprocessing-workflow.spd
\end{enumerate}

\subsection{技術的なポイント}

\begin{enumerate}
\item \textbf{TeX/LaTeXアナロジー}: プロファイルをクラスファイル、YAMLをテックスファイルとして捉える設計思想
\item \textbf{状態の明示的管理}: source.stateフィールドにより、入力ファイルの状態を明示的に追跡
\item \textbf{責務の境界}: 「前処理の責務 = SRTファイルが作業ディレクトリに存在することを保証」という明確な契約
\item \textbf{静的/動的分離}: YAMLの静的セクション(ユーザー入力)と動的セクション(自動更新)の明確な分離
\item \textbf{可視化の使い分け}: PADは処理の深さと順序、スイムレーンはアクター間の連携という役割分担
\end{enumerate}

この日のセッションは、video-chapter-editorで培った設計思想を文字起こしワークフローに適用し、YAMLベースの設定管理とプロファイルベースのアーキテクチャを確立した重要な設計フェーズとなった。

