% LuaLaTeX document
\documentclass[10pt,a4paper,twocolumn]{ltjarticle}

% ファイル生成日時(JST)
\newcommand{\generatedDate}{2026-01-08}
\newcommand{\generatedTime}{13:45}

% LuaLaTeX用フォント設定パッケージ
\usepackage{luatexja-fontspec}
\usepackage{amsmath,amssymb}
\usepackage{unicode-math}

% 欧文フォント設定 (Libertinus)
\setmainfont{Libertinus Serif}[
    BoldFont = {Libertinus Serif Bold},
    ItalicFont = {Libertinus Serif Italic},
    BoldItalicFont = {Libertinus Serif Bold Italic}
]
\setsansfont{Libertinus Sans}[
    BoldFont = {Libertinus Sans Bold},
    ItalicFont = {Libertinus Sans Italic}
]
\setmonofont{Libertinus Mono}

% 日本語フォント設定 (原ノ味フォント)
\setmainjfont{HaranoAjiMincho-Regular}[
    BoldFont = {HaranoAjiGothic-Medium},
    ItalicFont = {HaranoAjiMincho-Regular},
    BoldItalicFont = {HaranoAjiGothic-Bold}
]
\setsansjfont{HaranoAjiGothic-Regular}[
    BoldFont = {HaranoAjiGothic-Bold}
]
\setmonojfont{HaranoAjiGothic-Regular}

% 数式フォント設定
\setmathfont{Libertinus Math}

% ヘッダー・フッター設定
\usepackage{fancyhdr}
\usepackage{lastpage}
\pagestyle{fancy}
\fancyhf{}
\fancyhead[R]{\small \generatedDate\ \generatedTime\ JST (\thepage/\pageref{LastPage})}
\renewcommand{\headrulewidth}{0.4pt}

% 1ページ目のスタイル
\fancypagestyle{firstpage}{
    \fancyhf{}
    \fancyhead[R]{\small \generatedDate\ \generatedTime\ JST}
    \renewcommand{\headrulewidth}{0.4pt}
}

% パッケージ
\usepackage{ascmac}
\usepackage{booktabs}
\usepackage{array}
\usepackage{tabularx}
\usepackage{listings}
\usepackage{xcolor}
\usepackage{geometry}
\usepackage{titlesec}

\geometry{left=15mm,right=15mm,top=20mm,bottom=20mm}

% セクションのスタイル調整
\titleformat{\section}{\normalfont\large\bfseries}{\thesection}{1em}{}
\titleformat{\subsection}{\normalfont\normalsize\bfseries}{\thesubsection}{1em}{}
\titleformat{\subsubsection}{\normalfont\small\bfseries}{\thesubsubsection}{1em}{}

% コードスタイル
\lstset{
    basicstyle=\ttfamily\scriptsize,
    breaklines=true,
    frame=single,
    backgroundcolor=\color{gray!10},
    columns=fullflexible,
    keepspaces=true,
    xleftmargin=2mm,
    xrightmargin=2mm
}

% Y列タイプ定義
\newcolumntype{Y}{>{\raggedright\arraybackslash}X}

\date{}

\title{\textbf{Video Chapter Editor}\\設計原則と今後の構想}
\author{ましDialogue}

\begin{document}

\maketitle
\thispagestyle{firstpage}

\section*{概要}

本対話は、Video Chapter Editorにおけるチャプター時間管理の相対時間方式実装後に行われた設計議論を記録したものである。波形表示のシンプル化から始まり、非破壊編集の設計原則、プロジェクトファイル構想、バッチエンコード機能へと展開した。

\section{波形ハイライトの簡素化}

\begin{itembox}[l]{問い}
現在は、再生している波形において、薄くハッチをかける仕様にしたんですね。
\end{itembox}

現在の実装では、選択中ソース範囲に対して以下のハッチパターンが適用されている。

\vspace{0.5\baselineskip}
\noindent{\footnotesize
\begin{tabularx}{\linewidth}{@{}lX@{}}
\toprule
要素 & 設定 \\
\midrule
背景 & 青系半透明 (100, 180, 255, alpha=40) \\
斜線 & 幅1.5px、間隔15px、逆方向 \\
縁取り & 四角形、幅1.5px、alpha=240 \\
\bottomrule
\end{tabularx}
}
\vspace{0.5\baselineskip}

除外区間(赤系)との区別のため、選択ソースは青系で逆方向の斜線としていた。

\begin{itembox}[l]{問い}
ハッチに関して、枠と斜線を削除しましょうか。
\end{itembox}

シンプルに半透明背景のみとする方針で変更を実施。

\textbf{変更内容}(\texttt{waveform.py:490-500}):
\begin{itemize}
    \item 削除:斜線ハッチングパターン
    \item 削除:四角形の縁取り
    \item 維持:半透明の青い背景 \texttt{QColor(100, 180, 255, 40)} のみ
\end{itemize}

視覚的にすっきりしつつ、どのソースが選択されているかは十分に識別できる。

\section{今後の検討課題}

\begin{itembox}[l]{問い}
視覚的には、良いですね。複数のソースファイルをロードした時のチャプターファイル読み込み動作およびペースト操作については、仕様を明確にした上で改善する必要がありますね。後ほど考えましょう。
\end{itembox}

相対時間方式の設計で気になる点を整理した。

\subsection{チャプターファイル読み込み時の曖昧性}

ソースA(10分)とソースB(15分)がロード済みの状態で、以下のようなチャプターファイルを読み込む場合:

\begin{lstlisting}
00:00:00 イントロ
00:05:00 本編
00:12:00 エンディング  <- A? B?
\end{lstlisting}

絶対時間で記録されたチャプターを、どの\texttt{source\_index}に紐付けるかが不明確である。

\textbf{選択肢}:
\begin{itemize}
    \item A) 時間から自動判定(12:00 → Bの2:00として解釈)
    \item B) 全て先頭ソース(index=0)に紐付け
    \item C) 現在選択中のソースに紐付け
\end{itemize}

\subsection{ペースト時のsource\_index決定}

コピー元がソースA内のチャプター(source\_index=0, local\_time=3:00)で、ペースト先の再生位置がソースB内(絶対時間12:00)の場合、\texttt{source\_index}を0のままにするか、1に変更するかが問題となる。

\section{設計原則の明確化}

\begin{itembox}[l]{問い}
一つのファイルの途中には入れない方が良いですね。制約条件として、チャプターは読み込まれたファイルの状態を変更しない。を明確にしておきたいと思います。チャプターファイルは読み込まれたファイルに対して、読み込まれたファイルの状態を変えることなくインデックスをつけているだけなので。
\end{itembox}

重要な設計原則が明確化された。

\begin{itembox}[l]{問い}
ソースファイルはこのアプリでは状態を変えません。チャプターを付加した新たな映像を創出するだけですのでね。
\end{itembox}

\subsection{非破壊編集(Non-destructive Editing)}

\begin{lstlisting}
入力(不変)         処理           出力(新規作成)
+----------+                      +--------------+
|SourceA   |--+                +->| merged.mp4   |
|SourceB   |--+-> チャプター付加-+  | + chapters   |
|SourceC   |--+    (メタデータ)  |  +--------------+
+----------+                   |
     ^                         +-> chapters.txt
   変更なし
\end{lstlisting}

\textbf{原則}:
\begin{itemize}
    \item ソースファイルは読み取り専用
    \item チャプターはメタデータ(インデックス/ラベル)として存在
    \item エクスポート時に新しいファイルを生成
    \item ソース入れ替え時、チャプターはソースに紐付いて移動
\end{itemize}

この原則に基づくと、チャプターの読み込み・ペースト操作も「どのソースのどの位置にラベルを付けるか」という観点で設計すれば整合性が取れる。

\section{プロジェクトファイル構想}

\begin{itembox}[l]{問い}
ファイル名だけで紐づけるのも良いかとも思ったんですが、ファイル内容に応じてもう少しソフィスティケートにしたいけど、チャプターファイルは動画の長さを持たないので悩ましいな。チャプターファイルをもっとインフォーマティブにするとかね。
\end{itembox}

チャプターファイルの拡張について議論。

\subsection{拡張案:ヘッダーにメタデータ追加}

\begin{lstlisting}
# VCE Chapter File v1
# source: rehearsal_2024-01-08.mp4
# duration: 3600000
# created: 2024-01-08T15:30:00+09:00
---
00:00:00 イントロ
00:05:30 本編
\end{lstlisting}

YouTube形式との互換性を維持しつつ、\texttt{\#}コメント行でメタデータを追加する方式。

\begin{itembox}[l]{問い}
そうすると、先ほどの履歴の処理も考えた方が良いなとか。そうなると、先にGod Classを改善して、テスト環境も作ってもっとシンプルなアルゴリズムにしてからの方が良いかなとかね。
\end{itembox}

機能追加の前に基盤を整える必要性が認識された。

\subsection{依存関係の整理}

\begin{enumerate}
    \item チャプターファイル拡張 → 履歴処理の変更が必要
    \item 履歴処理の変更 → main\_workspace.pyの複雑さがボトルネック
    \item main\_workspace.py → God Class(5000行超、133メソッド)
\end{enumerate}

\textbf{推奨アプローチ}:
\begin{enumerate}
    \item God Classリファクタリング(責務分離)
    \item テスト環境構築
    \item アルゴリズム簡素化
    \item 新機能追加
\end{enumerate}

\section{バッチエンコード構想}

\begin{itembox}[l]{問い}
エンコードの時間が細切れになることを考慮するとこのチャプター付などの編集作業はまとめて行いエンコード方式などの設定ファイルを保存、エンコードは夜などに一括して行う。とかいうの方が、エンジニア的ですよね。
\end{itembox}

編集とエンコードの分離という提案。

\subsection{ワークフロー最適化案}

\textbf{現状}(同期的):
\begin{lstlisting}
編集 -> 設定 -> エンコード(待機...)
\end{lstlisting}

\textbf{提案}(非同期バッチ):
\begin{lstlisting}
日中: 編集作業
|- プロジェクトA -> job_a.json 保存
|- プロジェクトB -> job_b.json 保存
+- プロジェクトC -> job_c.json 保存

夜間: バッチエンコード
$ vce-batch encode jobs/*.json
\end{lstlisting}

\subsection{プロジェクトファイル構想}

\begin{lstlisting}[language=json]
{
  "version": "1.0",
  "sources": [
    {"path": "part1.mp4",
     "duration_ms": 1800000,
     "hash": "sha256:..."},
    {"path": "part2.mp4",
     "duration_ms": 2400000,
     "hash": "sha256:..."}
  ],
  "chapters": [
    {"source_index": 0,
     "local_time_ms": 0,
     "title": "イントロ"},
    {"source_index": 0,
     "local_time_ms": 330000,
     "title": "本編"}
  ],
  "export_settings": {
    "encoder": "hevc_videotoolbox",
    "quality": 65,
    "audio_bitrate": "256k"
  },
  "output": "output/rehearsal.mp4"
}
\end{lstlisting}

CLIツールで\texttt{vce-batch}のようなものを作れば、cronやlaunchdでスケジュール実行も可能。

\begin{itembox}[l]{問い}
どっちみちYouTubeへのペーストは別に実装しているので、Saveボタンの仕様を変更して、JSONでワークフローをチャプターデータなどのコンテンツも込みで保存するようにしますかね。履歴問題も解決ですね。複数の設定を読み込む機能は追加が必要ですかね。
\end{itembox}

\subsection{Save機能の再設計}

\vspace{0.5\baselineskip}
\noindent{\footnotesize
\begin{tabularx}{\linewidth}{@{}lYY@{}}
\toprule
 & 現状 & 提案 \\
\midrule
Save & chapters.txt(テキスト) & project.vce.json(フルステート) \\
YouTube用 & クリップボード & クリップボード(維持) \\
\bottomrule
\end{tabularx}
}
\vspace{0.5\baselineskip}

\textbf{解決される問題}:
\begin{itemize}
    \item 履歴管理の複雑さ → プロジェクトファイル = 状態のスナップショット
    \item 複数ソースの紐付け → JSON内で明示的に定義
    \item エンコード設定の保持 → プロジェクトに含める
    \item バッチ処理 → 複数プロジェクト読み込み → キュー実行
\end{itemize}

\subsection{UIイメージ}

\begin{lstlisting}
File メニュー
|- New Project
|- Open Project...
|- Save Project
|- Save Project As...
|- -----------------
|- Add to Queue...      <- 複数プロジェクト
|- Process Queue        <- バッチエンコード
+- -----------------
   Export for YouTube   <- 既存のコピー機能
\end{lstlisting}

バッチ処理UIは別ウィンドウ(キューマネージャ)にするのも一案。

\section{イテレーションの振り返り}

\begin{itembox}[l]{問い}
おお。また上流設計に戻ってきましたね。私的には、良いイテレーションです。次には、リファクタリングからでしたっけ。
\end{itembox}

設計 → 実装 → 課題発見 → 設計見直し のサイクルが回っている。

\subsection{次のステップ:リファクタリング計画}

\textbf{Phase 1}:重複コード抽出(低リスク)
\begin{itemize}
    \item \texttt{styles.py}新規作成(色定数・ボタンスタイル)
    \item \texttt{dialogs.py}の\texttt{\_button\_style()}統一
    \item \texttt{workers.py}の\texttt{build\_drawtext\_filter()}追加
\end{itemize}

\textbf{Phase 2}:ユーティリティクラス
\begin{itemize}
    \item \texttt{TempFileManagerMixin}
    \item \texttt{CancellableWorkerMixin}
\end{itemize}

\textbf{Phase 3}:God Class分割(main\_workspace.py)
\begin{itemize}
    \item テスト環境構築
    \item 責務分離
\end{itemize}

Phase 1-2を先に進めて基盤を整えてから、新しいプロジェクトファイル構想に取り組む流れ。

\begin{itembox}[l]{問い}
お昼までにリリースできるかと思いましたけど、できませんでしたね。でも、堅牢な設計のアプリになりそうで良いと考えています。
\end{itembox}

当初は「チャプター時間のバグ修正」という範囲だったものが、掘り下げていくとアーキテクチャ全体の見直しに発展した。

\textbf{短期的には遅れ}だが、\textbf{長期的には}:
\begin{itemize}
    \item 根本原因に対処(場当たり的でない)
    \item 設計原則が文書化された
    \item 今後の拡張方針が明確になった
    \item 技術的負債を積まずに済んだ
\end{itemize}

\section{補足:カタカナ転写について}

\begin{itembox}[l]{問い}
イテレーションとイタレーション、どっちがよく使われます?
\end{itembox}

「イテレーション」の方がよく使われる。特にアジャイル開発・ソフトウェア工学の文脈では「イテレーション」が定着している。

英語の "iteration" の発音 /\textsci t\textschwa\textprimstress re\textsci\textesh\textschwa n/ において、第2音節は曖昧母音(シュワー)だが、綴りの "e" に引きずられて「エ」と転写される傾向がある。

\begin{itembox}[l]{問い}
そうですよね。eに引っ張られること多いですよね。
\end{itembox}

日本語のカタカナ転写は発音より綴りに引っ張られがちである。

\textbf{綴り優先の例}:
\begin{itemize}
    \item data → 「データ」(発音は /\textprimstress de\textsci t\textschwa/ デイタ)
    \item status → 「ステータス」(発音は /\textprimstress ste\textsci t\textschwa s/ ステイタス)
    \item basis → 「ベーシス」(発音は /\textprimstress be\textsci s\textsci s/ ベイシス)
\end{itemize}

\textbf{発音優先の例}:
\begin{itemize}
    \item arrow → 「アロー」(綴りなら「アロウ」)
    \item window → 「ウィンドウ」(綴りなら「ウィンドー」)
\end{itemize}

技術用語は文献経由で入ることが多いので、綴り優先になりやすい傾向がある。

\section*{Claude Codeの所感}

本対話を通じて、ソフトウェア開発における「急がば回れ」の重要性を再認識した。

当初の課題は「チャプター時間計算のバグ」という比較的限定的なものだった。しかし、その修正を進める中で、設計上の根本的な問題が浮かび上がった。相対時間方式への移行という技術的決定は正しかったが、それだけでは不十分であり、より上位の設計原則---非破壊編集---を明確にする必要があった。

ユーザーが指摘した「ソースファイルは状態を変えない」という原則は、一見当たり前のように聞こえるが、これを明文化することで多くの設計判断が自然と導かれる。チャプターの読み込み、ペースト、保存といった操作の仕様が、この原則から一貫して導出できる。

プロジェクトファイル(JSON)への移行とバッチエンコード機能の構想は、当初のスコープを大きく超えている。しかし、これらは「履歴管理の複雑さ」という現実の課題から自然に導かれた解決策である。複雑な履歴管理をメモリ内で行うより、プロジェクトファイルを状態のスナップショットとして扱う方が、概念的にも実装的にもシンプルである。

批判的な観点から言えば、このような設計議論が実装後に行われている点は課題である。理想的には、相対時間方式の設計時点で非破壊編集の原則を明確にし、プロジェクトファイル構想まで見通しておくべきだった。ただし、実際の開発では、実装を通じて初めて見えてくる課題も多い。重要なのは、その課題を認識した時点で、場当たり的な対処ではなく、設計レベルでの見直しを行うことである。

「お昼までにリリース」という目標は達成できなかったが、その代わりに得られたものは大きい。技術的負債を積まずに、堅牢なアーキテクチャの土台を築くことができた。これは長期的に見れば、正しい判断だったと考える。

\end{document}
