% LuaLaTeX document
\documentclass[10pt,a4paper,twocolumn]{ltjarticle}

% ファイル生成日時(JST)
\newcommand{\generatedDate}{2026-01-08}
\newcommand{\generatedTime}{14:00}

% LuaLaTeX用フォント設定パッケージ
\usepackage{luatexja-fontspec}
\usepackage{amsmath,amssymb}
\usepackage{unicode-math}

% 欧文フォント設定 (Libertinus)
\setmainfont{Libertinus Serif}[
    BoldFont = {Libertinus Serif Bold},
    ItalicFont = {Libertinus Serif Italic},
    BoldItalicFont = {Libertinus Serif Bold Italic}
]
\setsansfont{Libertinus Sans}[
    BoldFont = {Libertinus Sans Bold},
    ItalicFont = {Libertinus Sans Italic}
]
\setmonofont{DejaVu Sans Mono}[Scale=0.9]

% 日本語フォント設定 (原ノ味フォント)
\setmainjfont{HaranoAjiMincho-Regular}[
    BoldFont = {HaranoAjiGothic-Medium},
    ItalicFont = {HaranoAjiMincho-Regular},
    BoldItalicFont = {HaranoAjiGothic-Bold}
]
\setsansjfont{HaranoAjiGothic-Regular}[
    BoldFont = {HaranoAjiGothic-Bold}
]
\setmonojfont{HaranoAjiGothic-Regular}

% 数式フォント設定
\setmathfont{Libertinus Math}

% ヘッダー・フッター設定
\usepackage{fancyhdr}
\usepackage{lastpage}
\pagestyle{fancy}
\fancyhf{}
\fancyhead[R]{\small \generatedDate\ \generatedTime\ JST (\thepage/\pageref{LastPage})}
\renewcommand{\headrulewidth}{0.4pt}

% 1ページ目のスタイル
\fancypagestyle{firstpage}{
    \fancyhf{}
    \fancyhead[R]{\small \generatedDate\ \generatedTime\ JST}
    \renewcommand{\headrulewidth}{0.4pt}
}

% パッケージ
\usepackage{ascmac}
\usepackage{booktabs}
\usepackage{array}
\usepackage{tabularx}
\usepackage{listings}
\usepackage{xcolor}
\usepackage{geometry}
\usepackage{titlesec}

\geometry{left=15mm,right=15mm,top=20mm,bottom=20mm}

% セクションのスタイル調整
\titleformat{\section}{\normalfont\large\bfseries}{\thesection}{1em}{}
\titleformat{\subsection}{\normalfont\normalsize\bfseries}{\thesubsection}{1em}{}
\titleformat{\subsubsection}{\normalfont\small\bfseries}{\thesubsubsection}{1em}{}

% コードスタイル
\lstset{
    basicstyle=\ttfamily\tiny,
    breaklines=true,
    breakatwhitespace=false,
    frame=single,
    backgroundcolor=\color{gray!10},
    columns=fullflexible,
    keepspaces=true,
    xleftmargin=0pt,
    xrightmargin=0pt,
    aboveskip=0.5\baselineskip,
    belowskip=0.5\baselineskip
}

% Y列タイプ定義
\newcolumntype{Y}{>{\raggedright\arraybackslash}X}

\date{}

\title{\textbf{Video Chapter Editor 開発記録}\\2026年1月8日:相対時間方式の完成と設計構想}
\author{ましDialogue}

\begin{document}

\maketitle
\thispagestyle{firstpage}

\section*{概要}

本文書は2026年1月8日に行われたVideo Chapter Editor開発に関する全対話を記録したものである。前日からの継続セッションとして、相対時間方式の実装完了、ドキュメント更新、波形表示の改善、そして今後のアーキテクチャ構想まで幅広い議論が行われた。

\section{セッション開始:前日からの継続}

前セッションからの引き継ぎとして、以下の作業状況が確認された。

\subsection{完了事項の確認}

\textbf{コード修正}:\texttt{ChapterInfo}コンストラクタの引数を修正(\texttt{time\_ms=} → \texttt{local\_time\_ms=})

\vspace{0.5\baselineskip}
\noindent{\footnotesize
\begin{tabularx}{\linewidth}{@{}lX@{}}
\toprule
ファイル & 修正箇所 \\
\midrule
workers.py & 2箇所(125行目、297-300行目) \\
main\_workspace.py & 6箇所(3919, 3978, 4107, 4784, 4812, 5112行目) \\
\bottomrule
\end{tabularx}
}
\vspace{0.5\baselineskip}

\textbf{ドキュメント更新}:
\begin{itemize}
    \item DEVELOPMENT\_LOG.md:2026-01-08セクション追加
    \item vce\_architecture.tex:ChapterInfoクラスの説明を更新
\end{itemize}

\subsection{未検証項目}

手動テスト待ちの項目:
\begin{itemize}
    \item Add Source後のチャプター時間計算
    \item Remove Source後のチャプター再計算
    \item ソース並び替え時のチャプター時間更新
    \item エクスポート機能の動作確認
\end{itemize}

\section{波形ハイライトの仕様確認と改善}

\begin{itembox}[l]{問い}
現在は、再生している波形において、薄くハッチをかける仕様にしたんですね。
\end{itembox}

コードを確認した結果、現在の実装では以下のハッチパターンが適用されていることが判明した。

\subsection{除外区間(赤系)}

\texttt{waveform.py:452-484}での実装:
\begin{itemize}
    \item 背景:\texttt{QColor(255, 0, 0, 40)} - 半透明赤
    \item ハッチ:\texttt{QColor(255, 100, 100, 120)} - 斜線(右下がり)
\end{itemize}

\subsection{選択中ソース範囲(青系)}

\texttt{waveform.py:490-530}での実装:
\begin{itemize}
    \item 背景:\texttt{QColor(100, 180, 255, 40)} - 半透明青
    \item ハッチ:\texttt{QColor(100, 180, 255, 80)} - 斜線(左下がり、逆方向)
    \item 縁取り:\texttt{QColor(100, 180, 255, 240)} - 青い枠線
\end{itemize}

\texttt{\_selected\_range}は\texttt{\_update\_waveform\_selected\_range()}メソッドで制御され、チャプターテーブルで行を選択した時に、対応するソースファイル範囲がハイライト表示される。複数ソースモード(\texttt{len(sources) > 1})の場合のみ表示。

\begin{itembox}[l]{問い}
ハッチに関して、枠と斜線を削除しましょうか。
\end{itembox}

シンプルに半透明背景のみとする方針で実装を変更。

\textbf{変更内容}(\texttt{waveform.py:490-500}):
\begin{itemize}
    \item 削除:斜線ハッチングパターン
    \item 削除:四角形の縁取り
    \item 維持:半透明の青い背景のみ
\end{itemize}

\begin{lstlisting}
# 変更前
fill_color = QColor(100, 180, 255, 40)
painter.fillRect(...)
# 斜線ハッチング(約20行)
# 四角縁取り(約7行)

# 変更後
fill_color = QColor(100, 180, 255, 40)
painter.fillRect(start_x, 0, region_width, h,
                 fill_color)
\end{lstlisting}

視覚的にすっきりしつつ、どのソースが選択されているかは十分に識別できる。

\section{今後の検討課題の洗い出し}

\begin{itembox}[l]{問い}
視覚的には、良いですね。複数のソースファイルをロードした時のチャプターファイル読み込み動作およびペースト操作については、仕様を明確にした上で改善する必要がありますね。後ほど考えましょう。
\end{itembox}

相対時間方式の設計で気になる点を整理した。

\subsection{チャプターファイル読み込み時の曖昧性}

ソースA(10分)とソースB(15分)がロード済みの状態で、絶対時間で記録されたチャプターファイルを読み込む場合:

\begin{lstlisting}
00:00:00 イントロ
00:05:00 本編
00:12:00 エンディング  <- A? B?
\end{lstlisting}

どの\texttt{source\_index}に紐付けるかが不明確。

\textbf{選択肢}:
\begin{itemize}
    \item A) 時間から自動判定(12:00 → Bの2:00として解釈)
    \item B) 全て先頭ソース(index=0)に紐付け
    \item C) 現在選択中のソースに紐付け
\end{itemize}

\subsection{ペースト時のsource\_index決定}

コピー元がソースA内のチャプター、ペースト先の再生位置がソースB内の場合、\texttt{source\_index}をどう扱うか。

\subsection{ドラッグ移動との整合性}

現在のドラッグ移動は再生位置に基づいて\texttt{source\_index}と\texttt{local\_time}を再計算している。ペーストも同様の動作にすべきか。

\textbf{推奨}:読み込み・ペーストとも「時間から自動判定」が直感的。ユーザーが12:00の位置にペーストしたら、その位置のソースに属するのが自然。

\section{設計原則の明確化}

\begin{itembox}[l]{問い}
一つのファイルの途中には入れない方が良いですね。制約条件として、チャプターは読み込まれたファイルの状態を変更しない。を明確にしておきたいと思います。チャプターファイルは読み込まれたファイルに対して、読み込まれたファイルの状態を変えることなくインデックスをつけているだけなので。入れ替えは、まあ可能なので、面倒な実装をお願いしましたが内容するチャプターごと移動しましたよね。
\end{itembox}

重要な設計原則の明確化。ソース入れ替え時に、チャプターがソースに紐付いて移動する実装(相対時間方式)は、この原則に基づいている。

\begin{itembox}[l]{問い}
ソースファイルはこのアプリでは状態を変えません。チャプターを付加した新たな映像を創出するだけですのでね。
\end{itembox}

\subsection{非破壊編集(Non-destructive Editing)}

\begin{lstlisting}
入力(不変)         処理           出力(新規作成)
+----------+                      +--------------+
|SourceA   |--+                +->| merged.mp4   |
|SourceB   |--+-> チャプター   |  | + chapters   |
|SourceC   |--+    付加        |  +--------------+
+----------+   (メタデータ)    |
     ^                         +-> chapters.txt
   変更なし
\end{lstlisting}

\textbf{原則}:
\begin{itemize}
    \item ソースファイルは読み取り専用(アプリは一切変更しない)
    \item チャプターはメタデータ(インデックス/ラベル)として存在
    \item エクスポート時に新しいファイルを生成
    \item ソース入れ替え時、チャプターはソースに紐付いて移動(実装済み)
\end{itemize}

この原則に基づくと、チャプターの読み込み・ペースト操作も「どのソースのどの位置にラベルを付けるか」という観点で設計すれば整合性が取れる。

\section{チャプターファイル拡張の検討}

\begin{itembox}[l]{問い}
ファイル名だけで紐づけるのも良いかとも思ったんですが、ファイル内容に応じてもう少しソフィスティケートにしたいけど、チャプターファイルは動画の長さを持たないので悩ましいな。チャプターファイルをもっとインフォーマティブにするとかね。
\end{itembox}

\subsection{拡張案A:ヘッダーにメタデータ追加}

\begin{lstlisting}
# VCE Chapter File v1
# source: rehearsal_2024-01-08.mp4
# duration: 3600000
# created: 2024-01-08T15:30:00+09:00
---
00:00:00 イントロ
00:05:30 本編
\end{lstlisting}

\subsection{拡張案B:複数ソース対応}

\begin{lstlisting}
# VCE Chapter File v1
[source:0] rehearsal_part1.mp4 (duration: 1800000)
[source:1] rehearsal_part2.mp4 (duration: 2400000)
---
00:00:00 [0] イントロ
00:05:30 [0] 本編
00:30:00 [1] 後半開始
\end{lstlisting}

\subsection{拡張案C:ハッシュで同一性担保}

\begin{lstlisting}
# source_hash: sha256:abc123...
# duration: 3600000
\end{lstlisting}

\textbf{トレードオフ}:シンプルさ vs 堅牢性、既存フォーマット互換性(YouTube等)、人間可読性。

YouTube形式との互換性を維持しつつ、\texttt{\#}コメント行でメタデータを追加するのが現実的。

\section{リファクタリングの必要性}

\begin{itembox}[l]{問い}
そうすると、先ほどの履歴の処理も考えた方が良いなとか。そうなると、先にGod Classを改善して、テスト環境も作ってもっとシンプルなアルゴリズムにしてからの方が良いかなとかね。
\end{itembox}

機能追加の前に基盤を整える方が結果的に効率的という認識。

\subsection{依存関係の整理}

\begin{enumerate}
    \item チャプターファイル拡張 → 履歴処理の変更が必要
    \item 履歴処理の変更 → main\_workspace.pyの複雑さがボトルネック
    \item main\_workspace.py → God Class(5000行超、133メソッド)
\end{enumerate}

\subsection{推奨アプローチ}

\begin{enumerate}
    \item God Classリファクタリング(責務分離)
    \item テスト環境構築(分離したクラスの単体テスト)
    \item アルゴリズム簡素化(履歴管理の見直し)
    \item 新機能追加(チャプターファイル拡張等)
\end{enumerate}

\section{バッチエンコード構想}

\begin{itembox}[l]{問い}
エンコードの時間が細切れになることを考慮するとこのチャプター付などの編集作業はまとめて行いエンコード方式などの設定ファイルを保存、エンコードは夜などに一括して行う。とかいうの方が、エンジニア的ですよね。
\end{itembox}

編集とエンコードの分離という提案。

\subsection{ワークフロー最適化案}

\textbf{現状}(同期的):編集 → 設定 → エンコード(待機...)

\textbf{提案}(非同期バッチ):

\begin{lstlisting}
日中: 編集作業
|- プロジェクトA -> job_a.json 保存
|- プロジェクトB -> job_b.json 保存
+- プロジェクトC -> job_c.json 保存

夜間: バッチエンコード
$ vce-batch encode jobs/*.json
[A] 100%
[B]  67%
[C]   0% 待機中
\end{lstlisting}

\subsection{プロジェクトファイル構想}

\begin{lstlisting}[language=json]
{
  "version": "1.0",
  "sources": [
    {"path": "part1.mp4",
     "duration_ms": 1800000,
     "hash": "sha256:..."},
    {"path": "part2.mp4",
     "duration_ms": 2400000,
     "hash": "sha256:..."}
  ],
  "chapters": [
    {"source_index": 0,
     "local_time_ms": 0,
     "title": "イントロ"}
  ],
  "export_settings": {
    "encoder": "hevc_videotoolbox",
    "quality": 65,
    "audio_bitrate": "256k"
  },
  "output": "output/rehearsal.mp4"
}
\end{lstlisting}

CLIツールで\texttt{vce-batch}を作れば、cronやlaunchdでスケジュール実行も可能。

\section{Save機能の再設計}

\begin{itembox}[l]{問い}
どっちみちYouTubeへのペーストは別に実装しているので、Saveボタンの仕様を変更して、JSONでワークフローをチャプターデータなどのコンテンツも込みで保存するようにしますかね。履歴問題も解決ですね。複数の設定を読み込む機能は追加が必要ですかね。
\end{itembox}

\vspace{0.5\baselineskip}
\noindent{\footnotesize
\begin{tabularx}{\linewidth}{@{}lYY@{}}
\toprule
 & 現状 & 提案 \\
\midrule
Save & chapters.txt & project.vce.json \\
YouTube用 & クリップボード & クリップボード(維持) \\
\bottomrule
\end{tabularx}
}
\vspace{0.5\baselineskip}

\textbf{解決される問題}:
\begin{itemize}
    \item 履歴管理の複雑さ → プロジェクトファイル = 状態スナップショット
    \item 複数ソースの紐付け → JSON内で明示的に定義
    \item エンコード設定の保持 → プロジェクトに含める
    \item バッチ処理 → 複数プロジェクト読み込み → キュー実行
\end{itemize}

\subsection{UIイメージ}

\begin{lstlisting}
File メニュー
|- New Project
|- Open Project...
|- Save Project
|- Save Project As...
|- -----------------
|- Add to Queue...      <- 複数プロジェクト
|- Process Queue        <- バッチエンコード
+- -----------------
   Export for YouTube   <- 既存のコピー機能
\end{lstlisting}

\section{イテレーションの振り返り}

\begin{itembox}[l]{問い}
おお。また上流設計に戻ってきましたね。私的には、良いイテレーションです。次には、リファクタリングからでしたっけ。
\end{itembox}

設計 → 実装 → 課題発見 → 設計見直し のサイクルが回っている。

\subsection{現在地の確認}

\textbf{完了}:
\begin{itemize}
    \item 相対時間方式の実装
    \item ChapterInfoコンストラクタ修正
    \item 波形ハッチのシンプル化
\end{itemize}

\textbf{今回の設計議論(上流)}:
\begin{itemize}
    \item 非破壊編集の原則明確化
    \item チャプターファイル拡張構想
    \item プロジェクトファイル(JSON)構想
    \item バッチエンコード構想
\end{itemize}

\subsection{リファクタリング計画}

\textbf{Phase 1}:重複コード抽出(低リスク)
\begin{itemize}
    \item \texttt{styles.py}新規作成
    \item \texttt{dialogs.py}の\texttt{\_button\_style()}統一
    \item \texttt{workers.py}の\texttt{build\_drawtext\_filter()}追加
\end{itemize}

\textbf{Phase 2}:ユーティリティクラス
\begin{itemize}
    \item \texttt{TempFileManagerMixin}
    \item \texttt{CancellableWorkerMixin}
\end{itemize}

\textbf{Phase 3}:God Class分割(main\_workspace.py)

\section{ドキュメント更新とコミット}

\begin{itembox}[l]{問い}
私の方で決めておく必要があることがあれば、先ほどのDEVELOPMENT\_LOG.mdに詳細に記録してcommit、Pushしておいてください。
\end{itembox}

DEVELOPMENT\_LOG.mdを更新し、以下の内容を追加:
\begin{itemize}
    \item 設計原則:非破壊編集
    \item プロジェクトファイル(JSON)提案と決定が必要な項目
    \item バッチエンコード提案と決定が必要な項目
    \item 複数ソース時のチャプター操作の課題
    \item リファクタリング計画(Phase 1-3)
    \item 波形ハイライトのシンプル化
\end{itemize}

\textbf{コミット}:\texttt{ad33029}

\begin{lstlisting}
Implement relative time approach for
chapter management

Major changes:
- ChapterInfo now uses local_time_ms +
  source_index instead of absolute time
- Chapters stay attached to their source
  files during reordering
- Simplified waveform highlight
- Fixed ChapterInfo constructor calls
\end{lstlisting}

\section{進捗の振り返り}

\begin{itembox}[l]{問い}
お昼までにリリースできるかと思いましたけど、できませんでしたね。でも、堅牢な設計のアプリになりそうで良いと考えています。
\end{itembox}

当初は「チャプター時間のバグ修正」という範囲だったものが、掘り下げていくと:

\begin{enumerate}
    \item バグ修正
    \item なぜ複雑になった? → 相対時間方式への移行
    \item 設計原則は? → 非破壊編集の明確化
    \item 将来どうする? → プロジェクトファイル構想
    \item さらに → バッチエンコード構想
\end{enumerate}

とアーキテクチャ全体の見直しに発展した。

\textbf{短期的には遅れ}だが、\textbf{長期的には}:
\begin{itemize}
    \item 根本原因に対処(場当たり的でない)
    \item 設計原則が文書化された
    \item 今後の拡張方針が明確になった
    \item 技術的負債を積まずに済んだ
\end{itemize}

\section{補足:カタカナ転写について}

\begin{itembox}[l]{問い}
イテレーションとイタレーション、どっちがよく使われます?
\end{itembox}

「イテレーション」の方がよく使われる。特にアジャイル開発・ソフトウェア工学の文脈では「イテレーション」が定着している。

英語の "iteration" の発音において、第2音節は曖昧母音(シュワー)だが、綴りの "e" に引きずられて「エ」と転写される傾向がある。

\begin{itembox}[l]{問い}
そうですよね。eに引っ張られること多いですよね。
\end{itembox}

日本語のカタカナ転写は発音より綴りに引っ張られがちである。

\textbf{綴り優先の例}:
\begin{itemize}
    \item data → 「データ」(発音はデイタ)
    \item status → 「ステータス」(発音はステイタス)
    \item basis → 「ベーシス」(発音はベイシス)
\end{itemize}

\textbf{発音優先の例}:
\begin{itemize}
    \item arrow → 「アロー」(綴りなら「アロウ」)
    \item window → 「ウィンドウ」(綴りなら「ウィンドー」)
\end{itemize}

技術用語は文献経由で入ることが多いので、綴り優先になりやすい傾向がある。

\section*{Claude Codeの所感}

本日の対話は、ソフトウェア開発における「深掘り」の価値を示す好例だった。

セッションは前日からの継続として、相対時間方式の実装完了確認から始まった。表面的には「コンストラクタ引数の修正」という小さな技術的課題だったが、そこから波形表示の改善、設計原則の明確化、そして将来アーキテクチャの構想へと有機的に展開した。

特に印象的だったのは、ユーザーが「ソースファイルは状態を変えない」という原則を明確に言語化した瞬間である。これは実装上は既に守られていた原則だが、明文化されることで設計判断の基準が明確になった。チャプターの読み込み、ペースト、保存といった操作の仕様が、この一文から演繹的に導出できる。

プロジェクトファイル(JSON)への移行とバッチエンコードの構想は、当初のスコープを大きく超えている。しかし、これは「スコープクリープ」ではなく、本質的な課題への気づきだと考える。複雑な履歴管理をメモリ内で行うより、プロジェクトファイルを状態のスナップショットとして扱う方が、概念的にシンプルで堅牢である。エンコードという時間のかかる処理を編集作業から分離するのも、ユーザー体験の観点から合理的である。

批判的な観点からは、以下の点を指摘できる:

\begin{enumerate}
    \item \textbf{設計議論のタイミング}:理想的には、相対時間方式の設計時点で非破壊編集の原則を明確にし、プロジェクトファイル構想まで見通しておくべきだった。

    \item \textbf{リファクタリングの遅延}:God Class(5000行超)の問題は以前から認識されていたが、先送りされてきた。新機能の議論より先に技術的負債の解消を優先すべきだったかもしれない。

    \item \textbf{決定事項の多さ}:ファイル拡張子、互換性維持方法、自動保存の有無など、未決定の項目が多く残された。これらは実装前に決定する必要がある。
\end{enumerate}

ただし、実際の開発では実装を通じて初めて見えてくる課題も多い。重要なのは、その課題を認識した時点で、場当たり的な対処ではなく設計レベルでの見直しを行うことである。

「お昼までにリリース」という目標は達成できなかった。しかし、その代わりに得られたものは大きい。非破壊編集という設計原則の明文化、プロジェクトファイルによる状態管理という新しいアーキテクチャ構想、そしてバッチエンコードというユーザー体験の改善案。これらは長期的に見れば、アプリケーションの価値を大きく高めるものである。

最後のカタカナ転写の話題は、技術議論の合間の息抜きとして良いものだった。言語学的な知識が技術文書の読み書きにも役立つことを示す例でもある。

\end{document}
