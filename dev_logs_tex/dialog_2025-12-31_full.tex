%% dialog_2025-12-31_full.tex
%% 2025年12月31日の開発対話記録(509メッセージ)

\section{2025年12月31日:ドラッグ&ドロップ実装とセミオートアップデーター}

この日は、UI操作の統一化から始まり、ドラッグ&ドロップ機能の本格実装、セミオートアップデーターの追加、Split Chapters機能の実装まで、v2.1.9からv2.1.25までの17回のリリースが行われた密度の高い開発セッションとなった。

\subsection{矢印キー挙動の統一(v2.1.9)}

\begin{userbox}
Enter(Return)で、チャプターの編集モードに入った際に、カーソルが入力済み文字の最後尾に入ります。その状態で、上矢印の挙動が、Macですと行の先頭に移動できるんですけど、Windowsだと上のセルに移動します。感覚的には、Macの方がイメージ通りですので、Windowsでも同じ挙動になるようにしてください。
\end{userbox}

\begin{assistantbox}
チャプター編集時の矢印キーの挙動を統一します。セル編集中の上下矢印キーを処理するコードを確認・修正します。
\end{assistantbox}

編集モード中のキーイベント処理が実装された:

\begin{lstlisting}[language=python,basicstyle=\ttfamily\tiny]
if self._table.state() == QAbstractItemView.State.EditingState:
    if key == Qt.Key.Key_Up:
        editor = self._table.findChild(QLineEdit)
        if editor:
            editor.setCursorPosition(0)
            return True
    elif key == Qt.Key.Key_Down:
        editor = self._table.findChild(QLineEdit)
        if editor:
            editor.setCursorPosition(len(editor.text()))
            return True
\end{lstlisting}

\subsection{メニューバーフォントサイズ統一(v2.1.10)}

\begin{userbox}
メニューバーのフォントサイズを、Window内のフォントサイズと同じにしてください。
\end{userbox}

QMenuBarとQMenuに\texttt{font-size: 16px}を追加し、ウィンドウ内UIと統一された。

\subsection{ドラッグ&ドロップ機能の実装}

\subsubsection{仕様策定}

\begin{userbox}
動画や音楽ファイルのドロップに対応するようにできますか。
\end{userbox}

\begin{userbox}
ドロップの仕様を決めたいと思います。決める必要がある事項を質問してください。
\end{userbox}

以下の仕様が決定された:

\vspace{0.5\baselineskip}
\noindent{\footnotesize
\begin{tabularx}{\linewidth}{@{}lX@{}}
\toprule
項目 & 仕様 \\
\midrule
ドロップ領域 & 動画プレビュー領域のみ \\
複数ファイル & 音声は結合リストに追加、動画は最初の1つのみ \\
上書き動作 & 確認なしで置き換え \\
フォルダ & 作業ディレクトリとして設定 \\
\bottomrule
\end{tabularx}
}
\vspace{0.5\baselineskip}

\subsubsection{DropVideoFrameクラス(v2.1.11)}

最初の実装として、QFrameを拡張したDropVideoFrameクラスが作成された:

\begin{lstlisting}[language=python,basicstyle=\ttfamily\tiny]
class DropVideoFrame(QFrame):
    files_dropped = Signal(list)
    folder_dropped = Signal(str)

    def __init__(self, parent=None):
        super().__init__(parent)
        self.setAcceptDrops(True)
        self._drag_active = False

    def dragEnterEvent(self, event):
        # Accept video/audio files and folders
        ...

    def dropEvent(self, event):
        # Emit files_dropped or folder_dropped signal
        ...
\end{lstlisting}

\subsubsection{チャプターペースト機能(v2.1.12-13)}

\begin{userbox}
続いて、チャプターのリストにペーストができる仕様にしてください。Youtubeのように.mmmがない場合は.000でパディングする仕様にしてください。
\end{userbox}

\begin{userbox}
video-chapter-editorと同じように、必ずチャプタの先頭に0:00:00.000を入れるようにしてください。
\end{userbox}

\texttt{paste\_chapters()}メソッドが追加され、YouTube形式(HH:MM:SS)とフル形式(HH:MM:SS.mmm)の両方に対応した。また、先頭に0:00:00.000がない場合は自動追加する機能も実装された。

\subsubsection{ドロップ時のバグ修正シリーズ(v2.1.14-17)}

\begin{userbox}
複数のmp3をドロップした時の挙動はどうなりましたっけ
\end{userbox}

確認の結果、\texttt{\_on\_source\_changed()}が未定義であることが判明し、\texttt{\_load\_source\_media()}に修正された(v2.1.15)。

\begin{userbox}
動画などをドロップした際の作業ディレクトリは、元の動画などのファイルが保存されているフォルダにしてください。
\end{userbox}

ファイルドロップ時に親フォルダを作業ディレクトリとして設定する機能が追加された(v2.1.16)。

\begin{userbox}
動画を一つドロップすると、Select Sourceでmp4を読み込んだ時と同じ挙動になりますよね。今は、ならないんですけど。
\end{userbox}

\texttt{\_prepare\_for\_new\_source()}ヘルパーメソッドを抽出し、Select Sourceダイアログとドロップ時の挙動を統一した(v2.1.17)。

\subsubsection{About/Helpダイアログ(v2.1.18)}

\begin{userbox}
バージョンの確認ができるようにしてください。あと、ヘルプに何も入ってませんね。
\end{userbox}

Help > About ダイアログにバージョン番号、機能一覧、GitHubリンクを追加。Help > Keyboard Shortcutsに操作説明を追加した。

\subsubsection{QVideoWidgetによるドロップイベントブロック問題}

\begin{userbox}
動画ファイルをドロップしても何も起きないですね。どうしたもんでしょう。
\end{userbox}

\begin{assistantbox}
QVideoWidgetがDropVideoFrameの上を覆っていたため、ドロップイベントが親フレームに届いていませんでした。
\end{assistantbox}

最初にイベントフィルタを使用した解決を試みたが機能せず(v2.1.19)、最終的に透明なDropOverlayウィジェットをQStackedLayoutのStackAllモードで配置する方式で解決した(v2.1.20):

\begin{lstlisting}[language=python,basicstyle=\ttfamily\tiny]
class DropOverlay(QWidget):
    files_dropped = Signal(list)
    folder_dropped = Signal(str)

    def __init__(self, parent=None):
        super().__init__(parent)
        self.setAcceptDrops(True)
        self.setAttribute(Qt.WidgetAttribute.WA_TranslucentBackground)
\end{lstlisting}

\subsubsection{ステータスバー更新問題(v2.1.21)}

\begin{userbox}
ドロップした際に、作業ディレクトリが切り替わらないですね。
\end{userbox}

デバッグ出力により、内部では\texttt{work\_dir}が正しく更新されているが、ステータスバーの表示が更新されていないことが判明:

\begin{userbox}
ステータスバーの表示が変わりません。
\end{userbox}

\texttt{work\_dir\_changed = Signal(object)}を追加し、app.pyで受け取ってステータスバーを更新する機能が実装された。

\subsection{セミオートアップデーター(v2.1.22)}

\begin{userbox}
Actionsで、アプリのアップデーターを準備することは可能ですか。
\end{userbox}

\begin{userbox}
オンラインでバージョンチェックして、アップデーターのダウンロードからインストールまでを行うことは可能ですか
\end{userbox}

3つのアプローチが提案された:

\vspace{0.5\baselineskip}
\noindent{\footnotesize
\begin{tabularx}{\linewidth}{@{}lX@{}}
\toprule
方式 & 概要 \\
\midrule
完全自動(Sparkle) & 署名必須、開発工数大 \\
半自動 & ダウンロード→Finder表示、すぐ実装可能 \\
通知のみ & リンク表示のみ、最もシンプル \\
\bottomrule
\end{tabularx}
}
\vspace{0.5\baselineskip}

\begin{userbox}
2ですかね。
\end{userbox}

\texttt{updater.py}モジュールが新規作成された:

\begin{lstlisting}[language=python,basicstyle=\ttfamily\tiny]
class UpdateChecker(QObject):
    update_available = Signal(str, str, str)  # version, url, notes
    check_finished = Signal()
    error = Signal(str)

    def run(self):
        # Checks GitHub API, compares versions, emits signal if update available

class UpdateDownloader(QObject):
    progress = Signal(int)  # 0-100
    finished = Signal(str)  # downloaded file path
    error = Signal(str)

def mount_and_open_dmg(dmg_path: str) -> bool:
    # macOS: hdiutil attach, open in Finder
\end{lstlisting}

アップデート手順:
\begin{enumerate}
\item 起動3秒後にGitHub Releases APIをバックグラウンドチェック
\item 新バージョンがあればステータスバーに通知表示
\item クリック→確認ダイアログ→ダウンロード開始
\item プログレスバーでダウンロード進捗を表示
\item macOS: DMGをマウントしてFinderで開く / Windows: ZIPをExplorerで開く
\end{enumerate}

\begin{userbox}
差分ではないんですね。
\end{userbox}

\begin{assistantbox}
その通りです。現在の実装はフルダウンロード方式です。毎回約80MBのDMG/ZIPをダウンロード。差分アップデートにはbsdiff等のパッチ生成・適用ロジックが必要で、現状のサイズでは許容範囲と判断しました。
\end{assistantbox}

\subsection{Split Chapters機能(v2.1.23)}

\begin{userbox}
エンコード時に、チャプターごと個別のファイルとして出力するオプションを追加し、Exportの左にトグルを追加してください。
\end{userbox}

\texttt{SplitExportWorker}クラスがworkers.pyに追加された:

\begin{lstlisting}[language=python,basicstyle=\ttfamily\tiny]
def sanitize_filename(name: str) -> str:
    """ファイル名に使えない文字を置換"""
    invalid_chars = '<>:"/\\|?*'
    for char in invalid_chars:
        name = name.replace(char, '_')
    return name.strip()

class SplitExportWorker(QThread):
    progress = Signal(int)          # 全体進捗 (0-100)
    chapter_completed = Signal(str) # 各チャプター完了時
    finished = Signal(list)         # 出力ファイルリスト
    error = Signal(str)
\end{lstlisting}

動作仕様:
\begin{itemize}
\item 「Split Chapters」チェックボックスをONでExport
\item 各チャプターが個別ファイルとして出力
\item 出力ファイル名: \texttt{\{base\}\_01\_\{chapter\_title\}.mp4}
\item \texttt{--}で始まる除外チャプターは出力されない
\item Embed Chap有効時は各ファイルにタイトル焼き込み
\end{itemize}

\subsection{PyInstaller問題の修正(v2.1.24-25)}

\begin{userbox}
mac版を実行すると開けないんですけど。なんでしょう。
\end{userbox}

ターミナルから実行した結果、\texttt{No module named 'email'}エラーが発生:

\begin{lstlisting}[basicstyle=\ttfamily\tiny]
ModuleNotFoundError: No module named 'email'
[PYI-275:ERROR] Failed to execute script 'run_video_chapter_editor'
\end{lstlisting}

\begin{assistantbox}
原因が判明しました。PyInstallerの設定で\texttt{email}モジュールを除外していますが、\texttt{urllib.request}(アップデーターで使用)がそれに依存しています。
\end{assistantbox}

PyInstallerのexcludesから\texttt{email}, \texttt{html}, \texttt{http}を削除し、v2.1.24でリリースされた。

\begin{userbox}
Aboutのバージョン、間違ってません?
\end{userbox}

\texttt{VERSION}定数が更新されていなかったため、v2.1.25で修正された。

\subsection{この日の成果}

\begin{enumerate}
\item \textbf{UI操作統一} - 矢印キー挙動、メニューフォントサイズ
\item \textbf{ドラッグ&ドロップ} - DropOverlay方式で完全実装
\item \textbf{チャプターペースト} - YouTube形式対応、先頭自動追加
\item \textbf{セミオートアップデーター} - GitHub Releases連携
\item \textbf{Split Chapters} - チャプター個別ファイル出力
\item \textbf{17バージョンリリース} - v2.1.9〜v2.1.25
\end{enumerate}

\subsection{技術的なポイント}

\begin{enumerate}
\item \textbf{QVideoWidgetのイベント処理}: 内部ウィジェットがイベントをブロックするため、イベントフィルタではなくQStackedLayout.StackAllモードのオーバーレイで解決
\item \textbf{Signal/Slotパターン}: \texttt{work\_dir\_changed}シグナルでコンポーネント間通信を実現
\item \textbf{PyInstaller依存関係}: urllib.requestはemail, html, httpモジュールに依存しており、除外リストから削除が必要
\item \textbf{GitHub Releases API}: バージョン比較とダウンロードURLの取得に使用
\end{enumerate}

