%% dialog_2025-12-26_full.tex
%% 2025年12月26日の開発対話記録(1124メッセージ)

\section{2025年12月26日:prep\_gui.pyの本格的なUI/UX改善とエクスポート機能の実装}

この日は、prep\_gui.pyの大規模な改善作業が行われた。1124メッセージに及ぶ長時間のセッションで、UI/UXの改善、エクスポート機能の実装、チャプター除外機能の追加が行われた。

\subsection{CenteredFileDialogの実装}

\begin{userbox}
ダイアログは、必ずWindowの中心に配置して欲しいのでQtのダイアログをカスタムしてフィルタ機能も追加してください。
\end{userbox}

\begin{assistantbox}
CenteredFileDialogクラスを作成します。showEventで親ウィンドウの中心に配置し、拡張子フィルタも適用します。
\end{assistantbox}

以下の機能を持つカスタムファイルダイアログが実装された:

\begin{itemize}
\item 親ウィンドウの中心に配置
\item QFileSystemModel.setNameFilters()による拡張子フィルタ
\item setNameFilterDisables(False)で対象外ファイルを非表示
\end{itemize}

\subsection{MP3結合とMP4変換}

\begin{userbox}
mp3の結合は、どのように行われますか
\end{userbox}

\begin{assistantbox}
ffmpegのconcatデムクサを使用します。ファイルリストを作成し、\texttt{-f concat -safe 0 -i list.txt}で結合します。
\end{assistantbox}

ユーザーは既存のシェル関数を共有した:

\begin{lstlisting}[language=bash,basicstyle=\ttfamily\tiny]
mp3tomp4 () {
    local mp3_file="$1"
    local image_file="$2"
    local output_file="${mp3_file:r}.mp4"
    local font="/System/Library/Fonts/ヒラギノ角ゴシック W6.ttc"
    local drawtext="drawtext=fontfile='${font}':textfile='${tmpfile}'"
    drawtext="${drawtext}:fontsize=56:fontcolor=white:borderw=2"
    drawtext="${drawtext}:box=1:boxcolor=black@0.6:boxborderw=15"
    drawtext="${drawtext}:x=(w-text_w)/2:y=h*0.325-th/2"
    ffmpeg -loop 1 -i "$image_file" -i "$mp3_file" \
        -c:v libx264 -preset ultrafast -tune stillimage -crf 32 \
        -c:a aac -b:a 192k -vf "${drawtext},${vf_pad}" \
        -pix_fmt yuv420p -shortest -movflags +faststart "$output_file"
}
\end{lstlisting}

このスタイル(ヒラギノフォント、半透明黒背景ボックス、画面上部1/3の位置)が後のチャプター名焼き込み機能に踏襲された。

\subsection{エンコード進捗表示の実装}

\begin{userbox}
エンコードの進捗状況も表示するようにできますか。
\end{userbox}

\begin{assistantbox}
ffmpegの出力から\texttt{time=}行を解析して進捗を表示します。QProcessを使用してリアルタイムで読み取ります。
\end{assistantbox}

\begin{userbox}
加えて、エンコード中にUIをロックしないようにしてください。
\end{userbox}

MergeWorker(QThread)を使用したバックグラウンド処理が実装された。

\subsection{タブ構成の変更}

\begin{userbox}
タブ構成で、最終出力のタブはあまり設定することもないので、タブ2にマージして出力する仕様に変更しようと考えていますが、いかがでしょう。
\end{userbox}

\begin{assistantbox}
良いと思います。ExportTabをEditTabに統合し、書出オプション(チャプター埋込、タイトル埋込)と書出ボタンを追加します。
\end{assistantbox}

3タブ構成から2タブ構成に変更された:
\begin{itemize}
\item タブ1: MergeTab(MP3結合・MP4変換)
\item タブ2: EditTab(チャプター編集・書出)
\end{itemize}

\subsection{再生コントロールのレイアウト改善}

\begin{userbox}
メインの再生早送りなどのボタンを中心に持ってきてください。また、再生ボタンの色を明るめの緑色で統一してください。
\end{userbox}

\begin{userbox}
画像と波形のグラフの幅を厳密に揃えてください。また、ボタンの配置は、波形の中心に配置してください。
\end{userbox}

AspectRatioWidgetにinnerWidthChangedシグナルを追加し、映像の幅変更時に波形・コントロールエリアの幅を同期する機能が実装された。

\begin{userbox}
ボタンを映像に重ねてほしいわけではありません。波形表示の下にあるコントロールボタンを中央に配置してください。というお願いでした。
\end{userbox}

この対話では、一時的にVideoOverlayWidget(映像上に再生ボタンを重ねる)が実装されたが、ユーザーの意図とは異なることが判明し、削除された。

\subsection{エクスポート機能の実装}

\begin{userbox}
大丈夫です!いよいよ書出の実装を行いましょう。
\end{userbox}

ExportWorkerクラスが作成され、以下の機能が実装された:

\begin{itemize}
\item FFMETADATA1形式のメタデータファイル生成
\item チャプター情報の埋め込み(\texttt{-map\_chapters 1})
\item タイトルの埋め込み
\item ストリームコピー(再エンコードなし)による高速書出
\end{itemize}

\subsection{チャプター名の映像焼き込み}

\begin{userbox}
チャプターごとに、チャプター名を動画に表示させたいと考えています。
\end{userbox}

\begin{assistantbox}
ffmpegのdrawtextフィルターを使用します。既存のmp3tomp4スクリプトのスタイルを踏襲して実装します。
\end{assistantbox}

\begin{userbox}
映像に5秒間のみのやきこみではなく、再生時に、常に再生しているチャプター名を表示する仕様にしてください。
\end{userbox}

当初は各チャプター開始時に5秒間のみ表示する仕様だったが、チャプター全体で常時表示する仕様に変更された。

\begin{lstlisting}[language=python,basicstyle=\ttfamily\tiny]
def _create_drawtext_filter(self) -> str:
    for i, ch in enumerate(self.chapters):
        start_sec = ch.time_ms / 1000.0
        # 次のチャプターの開始時間まで、または動画終了まで表示
        if i + 1 < len(self.chapters):
            end_sec = self.chapters[i + 1].time_ms / 1000.0
        else:
            end_sec = self.total_duration_ms / 1000.0

        drawtext = (
            f"drawtext=fontfile='{self.FONT_PATH}'"
            f":textfile='{textfiles[i]}'"
            f":fontsize={self.FONT_SIZE}"
            f":fontcolor=white:borderw=2:bordercolor=black"
            f":box=1:boxcolor=black@0.6:boxborderw=15"
            f":x=(w-text_w)/2:y=h*0.325-th/2"
            f":enable='between(t,{start_sec:.3f},{end_sec:.3f})'"
        )
\end{lstlisting}

\subsection{YouTube用チャプターコピー機能}

\begin{userbox}
チャプターファイルの書き出しの際に、Youtube用のms表記のないものを書き出すチェックボックスを追加してください。
\end{userbox}

\begin{userbox}
ちょっと仕様を変更しましょう。Youtube用のチャプターはコピペするだけなので、ファイルに書き出しを行う必要がないので、チェックボックスも削除してコピーボタンを押すとYoutubeのチャプター用にmsが削除されたチャプターリストがコピーされるという仕様に変更してください。
\end{userbox}

ファイル書き出しからクリップボードコピーに仕様変更された。📋ボタンでYouTube用フォーマット(HH:MM:SS タイトル)がコピーされる。

\subsection{チャプター除外機能(--プレフィックス)}

\begin{userbox}
動画のチャプターで、先頭に--をつけたチャプターを除外する実装です。--をチャプター名に付したチャプターを保存する際には、--のチャプターを除いてチャプター名の焼き込みを行うとともに、除外したチャプターの時間を適切に調整してチャプターファイルに保存する仕様に変更したいのです。
\end{userbox}

\begin{assistantbox}
理解しました。例えば:

\textbf{元:}
\begin{enumerate}
\item 0:00:00 オープニング
\item 0:15:27 曲A
\item 0:30:00 --休憩(15分間カット)
\item 0:45:00 曲B
\item 1:00:00 --準備中(10分間カット)
\item 1:10:00 曲C
\end{enumerate}

\textbf{書出後:}
\begin{enumerate}
\item 0:00:00 オープニング
\item 0:15:27 曲A
\item 0:30:00 曲B(15分短縮)
\item 0:45:00 曲C(25分短縮)
\end{enumerate}
\end{assistantbox}

以下のメソッドが実装された:

\begin{itemize}
\item \texttt{\_process\_excluded\_chapters()}: 除外区間・保持区間・調整後チャプターを計算
\item \texttt{\_create\_trim\_concat\_filter()}: ffmpegのtrim→concatフィルター生成
\item 保持区間を結合し、調整後の時間でチャプター埋込・焼込
\end{itemize}

\subsection{進捗バーの実装}

\begin{userbox}
エンコードの進捗状況をどこかに表示したいですね。
\end{userbox}

QProgressBarが追加され、ffmpegの\texttt{time=HH:MM:SS.xx}出力を解析してパーセンテージを表示する機能が実装された。

\begin{lstlisting}[language=python,basicstyle=\ttfamily\tiny]
self.export_progress = QProgressBar()
self.export_progress.setStyleSheet("""
    QProgressBar {
        border: 1px solid #555;
        border-radius: 4px;
        background: #2a2a2a;
        height: 20px;
        text-align: center;
        color: #fff;
    }
    QProgressBar::chunk {
        background: qlineargradient(x1:0, y1:0, x2:1, y2:0,
            stop:0 #4CAF50, stop:1 #66BB6A);
        border-radius: 3px;
    }
""")
\end{lstlisting}

\subsection{この日の成果}

\begin{enumerate}
\item \textbf{CenteredFileDialog}の実装 - 親ウィンドウ中央配置、拡張子フィルタ
\item \textbf{MergeWorker}によるバックグラウンドエンコード
\item \textbf{タブ統合} - 3タブから2タブ構成に変更
\item \textbf{ExportWorker}の実装 - チャプター埋込、タイトル焼込
\item \textbf{チャプター名常時表示} - 各チャプター区間中の連続表示
\item \textbf{YouTube用チャプターコピー} - ワンクリックでクリップボードコピー
\item \textbf{チャプター除外機能} - \texttt{--}プレフィックスで区間カット・時間調整
\item \textbf{進捗バー} - ffmpeg出力解析によるリアルタイム表示
\end{enumerate}

\subsection{技術的なポイント}

\begin{enumerate}
\item \textbf{QSortFilterProxyModelの限界}: 当初カスタムプロキシモデルでフィルタを試みたが、QFileSystemModel.setNameFilters()の方が適切だった
\item \textbf{VideoOverlayWidgetの不要な実装}: ユーザーの意図を誤解し一時的に実装したが、不要と判明し削除
\item \textbf{drawtext + textfile}: 日本語文字のエスケープ問題を回避するため、textfileパラメータを使用
\item \textbf{ffmpeg filter\_complex}: 除外区間のカット処理には、trim→concat→drawtextの複合フィルターが必要
\end{enumerate}

この日は非常に多くの機能が実装され、prep\_gui.pyの行数が大幅に増加した。特にエクスポート機能とチャプター除外機能は複雑なffmpegフィルターチェーンを必要とし、実装に時間を要した。

